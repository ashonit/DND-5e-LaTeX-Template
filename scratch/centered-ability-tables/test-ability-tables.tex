\documentclass[fonts=wotc, bg=none, nooutline, letterpaper,twocolumn]{dndbook}

\begin{document}

\chapter{Ability Table Test}

\begin{DndMonster}[width=\linewidth]{Bugbear}
  \DndMonsterType{Medium humanoid (goblinoid), chaotic evil}

  \DndMonsterBasics[
    armor-class = {16 (hide armor, shield)},
    hit-points  = {27 (5d8 + 5)},
    speed       = {30 ft.},
  ]

  % This will show the centered ability score table
  \DndMonsterAbilityScores[
    str = 15,
    dex = 14,
    con = 13,
    int = 8,
    wis = 11,
    cha = 9,
  ]

  \DndMonsterDetails[
    skills = {Stealth +6, Survival +2},
    senses = {darkvision 60 ft., passive Perception 10},
    languages = {Common, Goblin},
    challenge = {1 (200 XP)},
  ]

  \DndMonsterSection{Traits}
  \DndMonsterAction{Brute.} A melee weapon deals one extra die of its damage when the bugbear hits with it (included in the attack).

  \DndMonsterAction{Surprise Attack.} If the bugbear surprises a creature and hits it with an attack during the first round of combat, the target takes an extra 7 (2d6) damage from the attack.

  \DndMonsterSection{Actions}
  
  \DndMonsterAttack[
    name=Morningstar,
    distance=melee,
    type=weapon,
    mod=+4,
    reach=5,
    dmg=\DndDice{2d8 + 2},
    dmg-type=piercing,
  ]
  
\end{DndMonster}

\end{document}