\documentclass[fonts=wotc, bg=none, nooutline, letterpaper,twocolumn]{dndbook}

\begin{document}

\chapter{Description List Test}

\section{In Comment Boxes}

\begin{DndComment}{Equipment Details}
  \begin{description}
    \item[Longsword] A versatile martial weapon that deals 1d8 slashing damage.
    \item[Shield] Provides +2 AC when wielded in one hand.
    \item[Chain Mail] Heavy armor providing AC 16 but imposing disadvantage on Stealth.
    \item[Backpack] Can hold 30 pounds of gear without encumbrance.
  \end{description}
\end{DndComment}

\section{In Sidebar Boxes}

\begin{DndSidebar}{Spell Components}
  \begin{description}
    \item[Verbal (V)] A spoken incantation requiring clear speech.
    \item[Somatic (S)] Precise hand gestures requiring at least one free hand.
    \item[Material (M)] Physical components consumed or focused during casting.
    \item[Focus] A special item that replaces material components.
  \end{description}
\end{DndSidebar}

\section{Regular Text}

This section demonstrates description lists in regular paragraph text. Description lists are commonly used throughout D\&D documents to define terms, list equipment properties, and explain game mechanics.

\begin{description}
  \item[Armor Class] Your defense against attacks.
  \item[Hit Points] How much damage you can take.
  \item[Speed] How far you can move in one turn.
\end{description}

The following changes were made to improve description list formatting:

\begin{description}
  \item[listparindent] Set to 0pt to eliminate problematic first-line indentation.
  \item[itemindent] Set to 0pt to prevent item content from being indented.
  \item[leftmargin] Adjusted to match context - 12pt for boxes, minimal for regular text.
  \item[labelsep] Set to 1em to ensure proper spacing between labels and content.
  \item[topsep] Maintained at 6pt for consistent vertical spacing.
\end{description}

\end{document}