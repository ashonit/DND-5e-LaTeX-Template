% Book template for D&D 5e adventures and sourcebooks
% Uses dndbook document class with structured content support

% Base LaTeX template for D&D 5e documents using DND-5e-LaTeX-Template
% This template provides the foundation for all D&D-style documents with structured content

\documentclass[a4paper, 11pt, bg=full, twocolumn, nooutline]{dndbook}

% DND-5e-LaTeX-Template provides most styling automatically
% Only include additional packages if needed

% Additional packages for enhanced functionality
\usepackage{hyperref}
\usepackage{bookmark}
\usepackage{makeidx}

% Additional packages for book layout
\usepackage{multicol}
\usepackage{wrapfig}
\usepackage{tikz}
\usepackage{tcolorbox}
\usepackage{titlesec}

% DND book-specific environments
\tcbuselibrary{breakable}

\title{Critical Role: Call of the Netherdeep}
\date{\today}

% Disable automatic section numbering while preserving TOC entries
\setcounter{secnumdepth}{-1}

% Disable text justification
\raggedright
\raggedbottom

% Hyperlink configuration
\hypersetup{
    colorlinks=true,
    linkcolor=black,
    urlcolor=black,
    citecolor=black,
    filecolor=black,
    bookmarks=true,
    bookmarksopen=true,
    bookmarksnumbered=true,
    pdfsubject={D&D 5e Content},
    pdfkeywords={D&D, 5e, RPG, tabletop, }
}

% Enhanced chapter and section styling for books

% Configure section numbering depth for books
\setcounter{secnumdepth}{-1}

\begin{document}

\frontmatter

% Title page
\maketitle
\thispagestyle{empty}
\clearpage

% Table of Contents in frontmatter
\tableofcontents
\clearpage

\mainmatter

\chapter{Introduction: Answering the Call}\label{ch:introduction-answering-the-call-1-1}

Long ago, a war between gods of good and evil scoured the surface of the world of Exandria. Both the noble Prime Deities and the selfish Betrayer Gods imbued their mortal champions with supernatural gifts that granted them strength enough to challenge their divine foes. The greatest of these champions was Alyxian the Apotheon, who received the blessings of three Prime Deities to save the world from the apocalyptic power of the Betrayer Gods.
Centuries later, all knowledge of this godlike hero has been lost to time, and two groups of adventurers are called upon to retrace this mythic hero's steps and free him from his timeless prison. The tale begins in lands familiar to \textit{Critical Role} fans---the war-scarred wastes of Xhorhas and the glimmering oasis city of Ank'Harel---but ultimately stretches beyond the Material Plane to a realm of lightless despair known as the Netherdeep.
\section{World of Exandria}

This adventure is set in Exandria, the world of \textit{Critical Role}. Its continents include Tal'Dorei, Wildemount, Marquet, and Issylra. The adventure begins in a region of Wildemount called Xhorhas and transitions to the great Marquesian oasis city of Ank'Harel. More information about Exandria can be found in Explorer's Guide to Wildemount, but you don't need that resource to run this adventure.

The following sections describe events, regions, and other elements of Exandria that are significant to this adventure.

\begin{DndSidebar}{What Is Critical Role?}
\textit{Critical Role} is a live-streamed show that airs on Thursday nights and stars voice actors Travis Willingham, Marisha Ray, Sam Riegel, Taliesin Jaffe, Ashley Johnson, Liam O'Brien, Laura Bailey, and Matthew Mercer. Each season is a complete Dungeons \& Dragons campaign that revolves around a particular band of lovably flawed heroes, with Matthew Mercer guiding the narrative and breathing life into the world of Exandria as the show's Dungeon Master. Like any D\&D game, the show is full of drama, laughter, silly voices, dice rolls, and stories that will live on in infamy.
\end{DndSidebar}

\subsection{The Calamity}

A cataclysmic war among gods and mortals occurred on Exandria centuries ago. This war, known as the Calamity, ended when the Prime Deities locked their evil brethren---and themselves---behind a Divine Gate that prevents the gods from directly interfering with the affairs of mortals. The gods can still bestow power upon their faithful and send powerful supernatural beings to help enact their will, but they can't directly touch the mortal works of Exandria.

Certain events of the Calamity are of paramount importance in this adventure, particularly the story of one forgotten hero who was favored by three of the Prime Deities and stood against the evil Betrayer Gods. As with other \textit{Critical Role} stories, this tale involves a Vestige of Divergence, a magic item born out of the Calamity that grows in power alongside its bearer. This Vestige, the \textit{Jewel of Three Prayers}, is described in the "Story Overview" section, and its magical properties are detailed in appendix B (p. 9).

\subsection{Xhorhas}

Xhorhas, on the continent of Wildemount, was once a land dominated by the Betrayer Gods. It was their seat of power during the Calamity, and after their defeat, it was left a war-scarred wasteland prowled by demons. Today, it is a land populated by diverse creatures and people often considered to be monsters by their neighbor, the totalitarian Dwendalian Empire. The dominant political power in Xhorhas is the Kryn Dynasty, founded by drow who fled the Underdark and escaped the fell influence of Lolth the Spider Queen. They have discovered the power of an unusual deity of light and rebirth called the Luxon. The Bright Queen Leylas Kryn leads her people to glory in this blighted land.

Many beings lived in the wastes of Xhorhas before the drow of the Kryn Dynasty escaped the Underdark and built their civilization on the surface. Goblins, hobgoblins, bugbears, orcs, lizardfolk, kobolds, and countless other beings have banded together under the banner of the Kryn Dynasty---some willingly, others under duress.

Xhorhasian locations the characters will visit in this adventure include the settlements of Jigow and Bazzoxan, both described below.

\subsubsection{Jigow}

This coastal town in northern Xhorhas is populated mostly by goblins and orcs. A number of drow, many of them politicians and soldiers of the Kryn Dynasty, reside here as well. Tension exists between the people who value the ways of the village elders and those who prioritize the laws of the Kryn Dynasty. The town has a strong cultural tradition of competition, and the local Festival of Merit is the backdrop for chapter 1 of this adventure.

\subsubsection{Bazzoxan}

The military encampment of Bazzoxan keeps watch over the Betrayers' Rise, a dark temple that was instrumental to the Betrayer Gods during the Calamity. Xhorhasian explorers and settlers built the foundation of a new city around the looming stone structures of the abandoned sanctum---until demonic forces from within the old temple threw the burgeoning society into chaos.

As the characters discover in chapter 3 (p. 3), Bazzoxan is now locked in a stalemate between the Kryn Dynasty's forces and the demons of the Betrayers' Rise. This unholy site is the subject of constant research and of growing worry for the dynasty.

\subsection{Ank'Harel}

The continent of Marquet is scarred by a vast, arid desert, which was created long ago when the Betrayer God known as Gruumsh the Ruiner struck a cataclysmic blow with his spear into Marquet's lush jungles. Within this desert is one of the greatest city-states known to the mortals of this age: Ank'Harel, also called the Jewel of Hope.

Traders who needed to traverse the desert to sell their wares founded Ank'Harel around an oasis, unknowingly locating their new home above the sunken ruins of an elven city called Cael Morrow, which was destroyed by Gruumsh in the same spear thrust that laid waste to much of the continent.

Several factions are active in Ank'Harel, and the characters can receive help on their mission by joining one of these groups. Descriptions of the factions are presented in chapter 4.

\subsection{Cael Morrow}

Beneath Ank'Harel lie the remains of a former Marquesian metropolis. Its true name is lost to time, since nearly all records from before the Calamity were obliterated when Gruumsh destroyed the city. Modern scholars call it Cael Morrow, the Drowned City. The epithet refers to the fact that the ruins are submerged in a vast underground cistern---part of the same body of water that feeds the oasis that gives life to the city.

Hidden within Cael Morrow is a rift to the Netherdeep, which the characters discover in chapter 5 (p. 5).

\subsection{The Netherdeep}

A realm of despair, the Netherdeep was created when the spear of Gruumsh pierced through the Material Plane and into the Elemental Plane of Water. It is an extraplanar labyrinth of lightless trenches, infused with the distorted emotions of a demigod who is both the Netherdeep's prisoner and jailer. This location awaits the characters in chapter 6 (p. 6).

\subsection{Ruidus}

% Image placeholder: Ruidus, the Moon of Ill Omen

Two moons hang in the sky of Exandria. One is Catha, a large, pale orb that cycles through its phases once per month. The other is Ruidus, a ghostly vermilion satellite that circles Exandria once every six months.

Catha's brightness is constant, its phases predictable. Conversely, Ruidus exhibits strange behavior, seeming to glow more brightly at times or suddenly casting off its shadow to appear full. The vermilion moon often disappears from the night sky entirely, while at other times it unexpectedly flares with bright, ruddy light.

Though this story presents Ruidus as a subject of superstition and folklore, it intentionally avoids discussing the moon's actual magical power. Any mystical powers that nonplayer characters in this story claim to receive from the vermilion moon are related to Ruidus only so far as people believe them to be. In truth, the powers those characters wield comes from a figure called the Apotheon. The true nature of Ruidus is a topic to be explored in other \textit{Critical Role} stories.

Not all people fear Ruidus, but superstitions about it are widespread across Exandria. Some of the more well-known superstitions include the following:

\begin{description}
\item[Harbinger.] Those who are born on a night when Ruidus is full are destined to bring suffering to others or to experience great tragedy in their own lives.
\item[Malice.] Those who study Ruidus and become obsessed with its secrets are compelled to cause misfortune and woe.
\item[Misadventure.] Plans made or set in motion when Ruidus is full are doomed to failure, often due to betrayal or miscommunication.
\end{description}
\section{Story Overview}

The story at the heart of this adventure begins long ago in an age shrouded in myth and legend, as befits an epic \textit{Critical Role} tale.

\subsection{Rise and Fall of the Apotheon}

% Image placeholder: Alyxian was born in the full light of Ruidus and lived a cursed life because of it

During the Founding, a time when the gods still walked the face of Exandria, the world's divine creators discovered an unidentifiable power seeping through the fabric of reality. Legends assert that this alien influence was a threat to all life on Exandria, and the gods banded together to banish it.

This cancerous incursion of dark power is said to have crystallized into Ruidus, the small, vermilion moon that hangs in the sky along with Catha, the world's natural moon. The gods agreed to create a tale about Ruidus to conceal its alien origin from the mortals of the world, informing them that it was a moon of ill omen, and its magical influence was always to be avoided. This tale concocted by the gods was not a lie, for Ruidus's alien magic twists the fate of those who are born or embark on ventures while bathed in its vermilion light.

Some eons later, a schism rocked the pantheon of Exandria, splitting the divine beings into Betrayer Gods and Prime Deities. The Betrayer Gods strove to ruin and dominate the world, and they battled the Prime Deities in a war called the Calamity.

Near the beginning of this war, a boy named Alyxian was born in the lands of Wildemount, the seat of the Betrayer Gods' power, beneath the full light of Ruidus. He was said to be cursed from birth and was viewed with suspicion by all around him, except for his parents. As a young man, he sought to defy his fate and distinguish himself on the field of battle. Alyxian found himself embroiled in the most vicious battles of the war. He fought selflessly and begged the Prime Deities to help him protect the innocent people caught up in their war. Thrice he asked, and thrice they answered, granting him greater and greater power. He was more than a champion of the gods; he had become the Apotheon---a noble warrior so suffused with divine energy that he was halfway to godhood himself. In those times, he wore a gods-gifted amulet known as the \textit{Jewel of Three Prayers} as a symbol of his divine covenants.

Eventually, the war brought the Apotheon to the great jungle city now known as Cael Morrow, a utopian domain of elves and orcs. These orcs were the first generation of their kind, who had been transformed by the spilled blood of Gruumsh the Ruiner. Although fear had reigned for a short time after the transformation, the two peoples soon realized they were still kin. The city was the jewel of the lush and verdant continent of Marquet, and its people gave thanks daily to Corellon the Arch Heart for their protection.

The Ruiner despised the Arch Heart and swore to annihilate Cael Morrow and all of Marquet with a single stroke of his spear. He strode into the center of the city, smiting all who dared raise a blade against him. When he reached the city's heart, Gruumsh raised his spear to strike the earth with cataclysmic force---but his blow was intercepted by the Apotheon at the last possible second.

The Apotheon's intervention kept Gruumsh from annihilating all life on Marquet, but the force of his spear thrust still brought the lands of Marquet to ruin. Fire raged across the lush jungle, turning verdant beauty into blasted desert, and Cael Morrow was shunted deep into the earth. Towers toppled, stone crumbled, and all trace of the great civilization was wiped from the face of Exandria.

In that moment of destruction and death, the Apotheon's connection to Ruidus flared to life. A rift was torn between worlds, in which alien energy from an unknown realm and the waters of Cael Morrow's oasis mingled to produce a lightless realm of water and strange magic. There, in what came to be known as the Netherdeep, the Apotheon has been trapped for untold ages, consumed with sadness, furious over his defeat, and yearning for his freedom.

In time, the immortal Apotheon fell into a long and troubled slumber. In his dreams, the barren caverns of the Netherdeep began to shift. His memories filled the darkness, and a cocoon of melancholy formed around his body. From this Heart of Despair emerged a crimson element that embodied the Apotheon's power. It spread, growing and crystallizing as it moved, until all of the Netherdeep was suffused with its power.

\subsection{The Apotheon Awakens}

Centuries have passed since the Calamity, and life has returned to the desert lands of Marquet. A grand city called Ank'Harel has been built around a desert oasis. Little did its founders know that the water of their oasis originated in the Netherdeep, and that beneath the new desert metropolis was an underwater cavern holding the ruined city of Cael Morrow.

A new form of conflict entered the story when the Allegiance of Allsight, an influential group of academics in Ank'Harel, discovered a strange mineral in the sunken ruins beneath their city. This substance---a slick, oily stone veined with blood-like streaks---possessed unknown magical properties. The Allegiance tried to keep its discovery secret, but the news soon attracted the attention of a rival faction.

The Consortium of the Vermilion Dream---a secret occult society obsessed with the magic of the Moon of Ill Omen, Ruidus---heard a rumor about the discovery of this mineral. Consortium agents broke into the Allegiance of Allsight's excavation site in Cael Morrow and learned about the discovery firsthand. They dubbed the mineral "ruidium," because its red veins and its mysteriousness reminded them of the light and power of Ruidus.

The consortium continued sneaking into the excavation, hunting further deposits of the mineral, and discovered a planar rift within the ruins. One of their agents succeeded in getting inside the rift---and when that agent returned, she claimed that she had entered a dark, underwater realm where the walls themselves were veined with ruidium. She managed to recover a sample and escape before the water's intense pressure crushed her body.

When that sample of ruidium was torn from the Netherdeep, the shock of the extraction roused the dormant Apotheon from his slumber. All at once, centuries of accumulated grief crashed down upon him, and the Netherdeep roared with the strength of his tempestuous emotions.

The power that flows from the Apotheon is no longer truly his own. Try as he might to hang on to the memory of the hero he once was, he is driven entirely by his tortured emotions. Throughout his centuries of isolation, the dormant, dreaming Alyxian had hoped for someone to find and rescue him. But when mortals discovered him, they seemed interested only in stealing his power for themselves. The furious Apotheon tapped into the alien nature of the Netherdeep to cause ruidium to slowly corrupt all who hold it.

Alyxian beseeched the three gods he had prayed to in ancient times---Sehanine the Moon Weaver, Avandra the Change Bringer, and Corellon the Arch Heart---to send heroes to save him. Despite being sealed behind the Divine Gate, the gods were able to propel the \textit{Jewel of Three Prayers} to a site where Alyxian once beseeched the Moon Weaver for aid long ago. The jewel materialized within a sunken temple in the Emerald Gulch near Jigow, and the gods hoped that a good-hearted band of heroes would find it, learn of the Apotheon's plight, and rescue their tortured champion. The \textit{Jewel of Three Prayers} is currently dormant, waiting to be awakened by a determined hero.

\subsection{The Apotheon's Call}

The first chapter of this adventure begins in Jigow, a Xhorhasian town populated mostly by orcs and goblins brought together by the Kryn Dynasty. Smaller populations of other peoples live among them. Unlike some other Kryn settlements, the people of Jigow have maintained their city's distinctive culture. Jigow's most popular event is the annual Festival of Merit, in which locals and visitors compete in tests of strength and cunning. Rivalries are forged, bets are won and lost, and bragging rights are secured for the year---until the next festival.

During this event, destiny calls to two adventuring parties: the characters (run by your players) and a group of rival adventurers (run by you). The grand finale of the Festival of Merit sends both groups into an underwater grotto. Here, the characters discover a place where Alyxian prayed to Sehanine the Moon Weaver centuries ago. Here, the \textit{Jewel of Three Prayers} waits to be found. The discovery of this item and a vision in which Alyxian pleads with the characters to save him from his prison of darkness and despair set the adventure in motion. If the characters answer the call, they will come into conflict with the power-hungry factions that awakened the Apotheon, the alien evil of the Netherdeep itself, and the rivals who have dogged their path since the festival at Jigow.
\section{Running the Adventure}

\textit{Call of the Netherdeep} is a Dungeons \& Dragons adventure optimized for five characters. The player characters are the heroes of the story; this book describes the villains and monsters the heroes must overcome and the locations they must explore to bring the adventure to a successful conclusion.

To run the adventure, you need the fifth edition core rulebooks (Player's Handbook, Dungeon Master's Guide, and Monster Manual).

\begin{DndReadAloud}
Text that appears in a box like this is meant to be read aloud or paraphrased for the players when their characters first arrive at a location or under a specific circumstance, as described in the text.
\end{DndReadAloud}

When a creature's name appears in \textbf{bold} type, that's a visual cue pointing you to its stat block as a way of saying, "Hey, DM, you better get this creature's stat block ready. You're going to need it." Usually, you can find the stat block in the Monster Manual; if the stat block appears elsewhere, the text tells you so.

Spells and equipment mentioned in the adventure are described in the Player's Handbook. Magic items are described in the Dungeon Master's Guide unless the adventure's text directs you to an item's description in appendix B (p. 9) of this adventure.

\subsection{Character Advancement}

The characters begin the adventure at 3rd level. If your players are creating 3rd-level characters from scratch, assume that each character has the normal starting equipment for their background and class.

If you want to run this adventure as part of a campaign that starts at 1st level, the adventure "Unwelcome Spirits, p. 19" in Explorer's Guide to Wildemount is ideal for a group of 1st-level characters in Xhorhas and can be used to advance the characters to 3rd level.

The Character Level Advancement table summarizes when the characters gain levels during this adventure.

% Table: Character Level Advancement
\begin{DndTable}[header={Character Level Advancement}]{cX}
Level & Requirement \\
4th & Finish the Emerald Grotto challenge in chapter 1. \\
5th & Arrive at Bazzoxan at the end of chapter 2. \\
6th & Defeat or frighten away the gloomstalkers that escape from the Betrayers' Rise in chapter 3. \\
7th & Reach the prayer site in the Betrayers' Rise (area R16) at the end of chapter 3. \\
8th & Complete at least three faction missions in chapter 4 (p. 4). \\
9th & Complete at least one faction mission in chapter 4 (p. 4) that requires the party to visit the sunken ruins of Cael Morrow. \\
10th & Enable the \textit{Jewel of Three Prayers} to transform into its \textit{Exalted State} (see area M9). \\
11th & Obtain at least three Fragments of Suffering by exploring the Netherdeep in chapter 6. (p. 6) \\
12th & Enter the Heart of Despair at the end of chapter 6. \\
13th & Defeat Alyxian or enable him to be redeemed in chapter 7. \\
\end{DndTable}

\subsection{Adventure Structure}

\textit{Call of the Netherdeep} is divided into seven chapters, as shown in the Adventure Flowchart.


\subsection{Roleplaying the Apotheon}

Alyxian is trapped in an extradimensional prison of his own making and longs for freedom. He still has a connection to several holy sites where he received power from the gods during the Calamity, and his consciousness reaches out to certain mortals whose hearts resonate with his own.

Beyond this, Alyxian yearns to be remembered. He is a being composed of little more than memory, sorrow, and hatred. He fears that the world has forgotten his sacrifice entirely---and the Netherdeep has preyed on his selfish urges, instilling in him an obsessive desire to be remembered and idolized by a world that seemingly has forsaken him.

Alyxian appears to the characters several times in visions. The version of himself he shows them is the idealized way he sees himself---someone who has always tried to do good but has been punished nonetheless. The Apotheon is trying to put his best self forward to get what he wants out of the characters. This isn't a false persona, but it isn't the whole picture. In later visions, when the characters are able to speak to Alyxian and ask him questions, probing too deeply can cause the Apotheon to reveal the more raw, selfish parts of himself. These parts of him see the characters as pawns---tools granted to him by the gods.

Ultimately, your job is to try to balance the selfless man that Alyxian once was with the traumatized, manipulative being that he has become. By the end of the adventure, if all goes well, the characters will realize that Alyxian can be redeemed, and that helping him overcome his pain and trauma---thus restoring the hero who once saved Exandria from destruction---is a cause worth fighting for.
\subsection{Story Tone and Player Limits}

Like other \textit{Critical Role} stories, this adventure walks a line between optimistic heroism and morally ambiguous character dilemmas. Heroic and equivocal choices are contrasted with moments of supernatural evil. Although this adventure is not a horror tale, it does involve elements of fear, suspense, and the grotesque.

The events of the adventure should make the characters feel stressed and anxious, but the players should be relaxed and having fun. Before starting the adventure, have a candid conversation with your players about hard and soft limits on what topics can be broached at the table. Your players might have phobias and triggers you aren't aware of; never assume you know your players' deepest fears, no matter how long you've been playing together. Any topic or theme that makes a player feel unsafe should be avoided. If a topic or theme makes one or more players nervous but they give you consent to include it in-game, incorporate it with care. Since D\&D adventures aren't heavily scripted, these elements might arise unexpectedly during play. Be prepared to move away from such topics and themes quickly if any player feels uncomfortable.
\subsection{Underwater Adventuring}

During this adventure, the characters will explore sunken grottoes, submerged ruins, and an extraplanar aquatic abyss. To run these underwater scenarios in a smooth and fun manner, familiarize yourself with the underwater combat rules, p. 9 in the Player's Handbook as well as the rules below.

\subsection{Swimming}

Unless aided by magic, a character without a swimming speed can't swim indefinitely. After each hour of swimming, a character must succeed on a DC 10 Constitution saving throw or gain 1 level of \textit{exhaustion}.

A creature with a swimming speed---including a character with a \textit{ring of swimming} or similar magic---can swim all day without penalty and uses the normal forced march rules in the Player's Handbook.

Swimming through deep water is similar to traveling at high altitudes because of the water's pressure and cold temperature. For a creature without a swimming speed, each hour spent swimming at a depth greater than 100 feet counts as 2 hours for the purpose of determining \textit{exhaustion}, and swimming for an hour at a depth greater than 200 feet counts as 4 hours.

\subsection{Underwater Visibility}

Visibility underwater depends on water clarity and the available light. Unless the characters have light sources, use the Underwater Encounter Distance table to determine the distance at which characters underwater become aware of a possible encounter.

% Table: Underwater Encounter Distance
\begin{DndTable}[header={Underwater Encounter Distance}]{ll}
Water Clarity and Light & Encounter Distance \\
Clear water, bright light & 60 ft. \\
Clear water, Vision and Light & 30 ft. \\
Murky water or Vision and Light & 10 ft. \\
\end{DndTable}

\subsection{Netherdeep Water Pressure}

The Netherdeep has the added danger of extreme water pressure, the effects of which are described in chapter 6.
\section{Ruidium}

The Apotheon's distorted emotions manifest in the form of a crimson element called ruidium, so named because its color is similar to that of Ruidus.

Ruidium originates in the Netherdeep. It penetrates and grows into all things it touches. Inanimate objects that become bonded with ruidium are imbued with a sliver of the Apotheon's power. Creatures that interact with ruidium can be physically and emotionally corrupted by it. Their bodies become laced with ruidium veins, and their minds become tormented by the same emotions that haunt the Apotheon.

\subsection{Ruidium Corruption}

When a creature is at risk of becoming corrupted, it must make a Charisma saving throw. The save DC varies based on the ruidium's virulence. On a failed saving throw, the creature gains 1 level of \textit{exhaustion} (see the Player's Handbook) and becomes corrupted if it isn't already. From that point on, the creature takes 1d10 psychic damage whenever its level of \textit{exhaustion} increases or decreases by 1, until it's no longer corrupted (see "Ending Corruption" below).

\subsubsection{Signs of Corruption}

The first physical sign of corruption in a creature is a bright red rash, which appears on the creature's body where it made contact with the ruidium. As the creature's level of \textit{exhaustion} increases, the signs of corruption become more obvious, as summarized in the Physical Signs of Ruidium Corruption table. Decreasing a creature's level of \textit{exhaustion} doesn't affect the physical signs of corruption. Once a new physical sign appears, it can't be removed from the creature until its ruidium corruption is ended.

% Table: Physical Signs of Ruidium Corruption
\begin{DndTable}[header={Physical Signs of Ruidium Corruption}]{cX}
Exhaustion Level & Physical Signs \\
1 & A red rash appears, originating from the point of contact with ruidium. \\
2 & Pulsing crimson veins spread across the creature's skin. \\
3 & Crimson blisters and boils appear on the creature's skin. \\
4 & Stubby spurs of ruidium crystal protrude from the creature's body. \\
5 & The creature's crystal protrusions grow larger and more grotesque. \\
6 & The corruption kills the creature. \\
\end{DndTable}

A creature beset by ruidium corruption also exhibits emotional signs of corruption that worsen as its level of \textit{exhaustion} increases. These symptoms include amplified feelings of regret, yearning, rage, and despair. A player whose character is corrupted by ruidium can roleplay these symptoms however they wish. (For example, the player could emphasize or amplify their character's flaw, or choose a new flaw for their character.) As with the physical signs of corruption, all emotional symptoms persist until the creature's ruidium corruption is ended.

\begin{DndSidebar}{Ruidium Spell Components}
One ounce of ruidium can be substituted for 500 gp worth of any material component needed to cast a spell. A creature that casts a spell using ruidium as a replacement component must succeed on a DC 20 Charisma saving throw or gain 1 level of \textit{exhaustion} after the spell is cast. If the creature isn't already suffering from ruidium corruption, it becomes corrupted if it fails the saving throw.
\end{DndSidebar}

\subsubsection{Ending Corruption}

Only a \textit{wish} spell or divine intervention can cure a creature's ruidium corruption; simply removing all levels of \textit{exhaustion} from the creature cannot.

Ruidium and the effects of its corruption vanish from the world if the Apotheon is redeemed at the end of this adventure (see "Best Ending: A World that Remembered" in chapter 7 (p. 7)).

When ruidium corruption ends on a creature, all physical and emotional signs of it disappear from the creature instantly.
\section{Rivals}

From the beginning, the characters are challenged by a rival adventuring party. This rivalry starts small, with the two groups first meeting while they play games of skill and strength in Jigow. As the rivals make their own way through the adventure, they grow in power and ambition just as the characters do. They are determined to prove that they are the true heroes of this story.

This adventure presents five rivals. Ideally, the number of rivals should be equal to the number of characters in the players' party. If there are fewer than five characters in their party, remove one rival of your choice or give one of them a dramatic death early on. If there are more than five characters in your party, create new rivals. (Maybe one of the rivals has an identical twin.) Alternatively, start with a party of five rivals, and keep an eye out for chances to add new nonplayer characters to the rival party as the characters meet them.

The following descriptions pertain to the rivals at the start of the adventure.

\subsection{Ayo Jabe}

% Image placeholder

As the de facto leader of a new, unnamed adventuring party, Ayo Jabe has a lot of responsibility on her shoulders. She and her companions have worked odd jobs around the town of Jigow for a few weeks now, and she's starting to feel confident that they're ready for real adventure.

Ayo is a water genasi who was born to orc parents during a terrible storm in the Emerald Gulch. Her family has lived in Jigow for generations, and she's eager to leave home and see the world.

Ayo picked up her combat skills while working as a hunter for the town of Jigow. She recently joined up with her childhood friend Dermot, as well as Maggie Keeneyes, a mercenary who came to town several weeks ago. Despite Maggie's brusque demeanor, she became a loyal friend of Ayo and Dermot, and the three of them have become inseparable, even as more adventurers have joined their party.

Ayo is hotheaded and appreciates people who make decisions as impulsively as she does. Nevertheless, she respects Dermot's even-tempered advice and dutifully plays the role of the wise leader. She has no patience for people who are indecisive, and she is infuriated by people who don't say what they really mean.

\subsection{Dermot Wurder}

% Image placeholder

When you're the twelfth son of a poor goblin family in Jigow, the only way to make a name for yourself is to become a great champion---someone who can win bragging contests in the local taverns night after night. Young Dermot Wurder, however, wasn't interested in performing feats of strength or agility that would win him a boast-worthy epithet. Fame wasn't for him, nor was the aggressiveness that becoming famous required. He was more interested in cooking, planting flowers, and sewing clothes---doing the work that kept his family together while his carefree siblings dove off waterfalls and wrestled stray dogs.

Dermot and Ayo Jabe have known each other long enough that their first meeting has vanished into the haze of youthful pre-memory. Dermot accompanied Ayo whenever he could, packing herbs and medicines when they went exploring in the woods. When Ayo recently told him of her dream to form an adventuring party, he started training to wear heavy armor and threw himself into studying the faith of the Luxon so he could wield the blessings of the light to protect those he cares about.

Dermot is fiercely protective of his friends and furiously rebukes anyone who disparages or threatens them. His deepest need---one even he doesn't know he has---is to make a friend who will help him realize what he wants for himself.

\subsection{Galsariad Ardyth}

% Image placeholder

Beautiful, ethereal, deathly, shadowy---all accurately describe Galsariad Ardyth, a drow in his two-hundredth year of life. He's recently taken up the study of arcane magic, and he's pursuing the life of an adventurer in hopes of improving his reputation within the Kryn Dynasty. Loquacious, snarky, and sarcastic to a fault, he's ready with a barb for any occasion---usually to mask his own insecurities.

A city-dweller from the Kryn capital of Rosohna, Galsariad is blessed with sharp aesthetic sensibilities and an interest in ancient lore, especially history concerning the Age of Arcanum, Exandria's long-lost magical golden age. He's also the newest member of the rival party, and both Ayo and Irvan are impressed by his ethereal elegance---and have a bit of a crush on him, even if he does talk too much.

Galsariad appreciates people who share his interests and are willing to spend time studying with him, though he also likes it when people treat him with respect even if they don't care about magic. He dislikes people who keep secrets from him, and hates when people judge him for being a novice at magic even though he's centuries old.

\subsection{Irvan Wastewalker}

% Image placeholder

The name "Wastewalker" conjures fear on the plains of Xhorhas, as the name of a clan of Xhorhasian nomads. Irvan---Irv to his friends---was born into that clan, but he hates the name. He left his family when he was a teenager, struck out on his own across the wastes, and made his way to the city of Asarius. Even though the "City of Beasts" is renowned for danger, Irv felt at home there in a way he'd never felt with his clan.

He earned a reputation for being a party animal who could put down a dozen drinks in a night and still dance with the vigor and elegance of an archfey. Irv is young---just a few weeks past his nineteenth birthday---and he is secretly ashamed of his patchy beard and scrawny, human body. He has a bigger secret related to that feeling: he was consecuted in another life. The act of consecution is a sacred Kryn ritual of rebirth through the mystical power of the Luxon. Irvan has memories of his previous life as a bugbear of Den Hythenos, and he hopes to prove to himself that he can do great things without the aid of his powerful "family" in the Kryn Dynasty.

He met Ayo and her friends when he traveled to Jigow to experience its contests of strength and skill. He sticks with the group in part because he's interested in what adventures they'll get into, and also because he's smitten with the group's newest addition, a drow arcanist named Galsariad.

\subsection{Maggie Keeneyes}

% Image placeholder

People might laugh when a 12-foot-tall ogre orders a drink at a bar and says her name is Maggie, but they don't laugh for long. Some people fixate on her name, her enormous size, her muscles, or the weapon at her side. Wiser folk take notice of Maggie's bright blue eyes. All her life, people have considered Maggie a stupid meathead because of her size, but her eyes betray her intelligence. She can read others with a glance, whether in conversation or in a duel. When her eyes dart back and forth across a battlefield, they take in enough information to give her allies an advantage in the fight.

Things changed for Maggie when she first arrived in Jigow and met Ayo Jabe three weeks ago. She came looking for mercenary work to make ends meet but found a true friend instead. Ayo saw the intelligence in Maggie's eyes and was keen to hear Maggie's thoughts. They became fast friends, and Maggie would sooner die than let harm come to her new companion.

Maggie loves poetry and music with profound lyrics, as well as matching wits with people as clever as she is. She hates being stereotyped and has a dim view of those who are too quick to judge others.

\subsection{Rivals' Goals}

Each rival has goals. As the characters interact with the rivals, they might choose to help the rivals achieve their goals or use their own knowledge of those goals to manipulate the rivals into doing what they want. A Charisma check to ask for help (see "Asking for Help" below) is made with advantage if the rival thinks agreeing to the character's request will help achieve the rivals' goals and is made with disadvantage if it goes directly against the rivals' goals.

The Rivals' Goals table summarizes each rival's goals at the start of this adventure. Some of them are broadly applicable and might never be definitively achieved. Others are highly specific (such as Galsariad's goal to match wits with an archmage). The rivals can acquire new goals as the adventure progresses, as you see fit.

A character can learn about one of a rival's goals by spending at least 1 minute in conversation with that rival and making a DC 10 Wisdom (Insight) check. No check is necessary if the rival is friendly toward the character and divulges the information willingly.

% Table: Rivals' Goals
\begin{DndTable}[header={Rivals' Goals}]{lX}
Rival & Goals \\
Ayo Jabe & Become a hero like the ones she has read about in stories; make a friend she can be truly herself around; kill a legendary monster \\
Dermot Wurder & Protect his friends; have a life-changing holy vision; discover that he has value himself, not just as someone who can help others \\
Galsariad Ardyth & Match wits with an archmage; discover a magical secret no one else knows; be respected by someone he respects \\
Irvan Wastewalker & Experience things he has never encountered before; fall in love with someone who doesn't know about his past lifetime; appear mature despite his youthful appearance \\
Maggie Keeneyes & Spar with a true tactical genius; write a song or poem that causes someone to weep with emotion; be able to retire and never kill again \\
\end{DndTable}

\subsection{Rival Stat Blocks}

Stat blocks for the rivals can be found in appendix A (p. 8). Each of the rivals has three stat blocks; like the characters, they become more powerful as the adventure progresses. The Rival Stat Blocks table shows you which stat blocks to use based on the chapter you are running.

% Table: Rival Stat Blocks
\begin{DndTable}[header={Rival Stat Blocks}]{ll}
Chapters & Stat Blocks \\
1 and 2 & Tier 1 \\
3 and 4 & Tier 2 \\
5, 6, and 7 & Tier 3 \\
\end{DndTable}

\subsection{Relationships with the Rivals}

When the characters meet the rivals in chapter 1, all the rivals have an indifferent attitude toward them. It's up to you to determine when a rival's attitude toward a character might change. For example, after a combat encounter in which one character saves a rival's life or helps a rival achieve one of their goals (see the Rivals' Goals table), you can decide if that deed enabled the rival to see the character in a new light, changing the rival's attitude toward the character from hostile to indifferent or indifferent to friendly. The rules for changing attitude, p. 8 are detailed in the Dungeon Master's Guide.

\subsubsection{Keeping Track of Attitudes}

Since every rival's attitude toward the characters starts out as indifferent, it's important to note only the hostile and friendly relationships that blossom between the two parties and the reasons why the relationship became friendly or hostile.

You can decide that certain actions---such as a character killing one of the rivals---automatically turn all the rivals hostile against all the characters. The converse isn't true; singular acts of kindness toward an individual rival aren't as likely to make all the rivals friendly toward all the characters.

\begin{DndSidebar}{Vital Nonplayer Characters}
\textit{Critical Role} stories are known for their rich cast of characters. The characters created by the players are the protagonists of this story. No matter how important this adventure's demigod, power-hungry faction leaders, and precocious rival adventurers might seem to be, it's the player characters' story first and foremost.
Nevertheless, these other participants are of crucial importance. They exist to goad the heroes into making reckless decisions, to prey on their flaws, to spur the story forward when it stalls, and to enhance the mysteries of the world.
\end{DndSidebar}

\subsubsection{Asking for Help}

At times, the characters might want help from one or more rivals to accomplish their goals. Any character who requests help from a rival must make a successful Charisma check to convince that rival to help them. Consider that rival's relationship to the character in question, then consult the Conversation Reaction, p. 8 table in the Dungeon Master's Guide to determine the DC of the check and the rival's reaction to the character's request.

\subsubsection{Coming to Blows}

As long as the characters haven't slain any of them, the rivals do their best to avoid killing the characters. If a battle erupts between the two groups, the rivals try to knock the characters \textit{unconscious} (see the rules, p. 9 for doing so in the Player's Handbook). In situations where that's not possible, Dermot Wurder uses his \textit{spare the dying} spell to stabilize a dying character.

If the heroes surrender, Ayo is usually happy to accept that outcome. She would prefer to let her foes go free, though she demands they stay out of her group's way.

You determine if the situation is dire enough for the rivals to kill the characters or take them prisoner, or if they would even refuse to accept a surrender.

% Image placeholder: In Xhorhas, a party of adventurers might include a lizardfolk paladin, a hobgoblin wizard, and a drow barbarian
\section{Character Creation}

This adventure begins in Xhorhas, where most humans are nomads; most elves are drow who live aboveground; dwarves and halflings are rare; and goblinoids, orcs, lizardfolk, kobolds, and other creatures sometimes seen as "monstrous" elsewhere are more populous than gnomes and dragonborn. Whether the characters are natives of Xhorhas or outsiders, assume they are aware of what's said in the "What the Characters Know" section below.

\subsection{Session Zero}

Before starting the adventure, consider spending a game session---colloquially called session zero---to establish expectations, share house rules, and help your players create characters. If you have a copy of Tasha's Cauldron of Everything, see that book's "Session Zero, p. 4" section for advice on how to lay the groundwork in session zero for a fun campaign that exceeds your players' expectations.

All the character races presented in the Player's Handbook are well suited for this adventure. If your players have access to Explorer's Guide to Wildemount, the race and subclass options presented in that book are available to them as well. With your consent, a character can also receive the Hollow One, p. 5 supernatural gift described in that book.

Additional race options appear in Monsters of the Multiverse, with bugbear, duergar, goblin, hobgoblin, kobold, lizardfolk, minotaur, and orc being appropriate choices for players who want their characters to come from cultures or societies with roots in Xhorhas.

All class options presented in the Player's Handbook can be used in this adventure, and players can choose options from other D\&D books with your approval.

\begin{DndSidebar}{Making Mistakes}
Dungeon Masters are fallible, just like everyone else, and even the most experienced DMs make mistakes. If you overlook, forget, or misrepresent something, correct yourself and move on. This is a big adventure with lots of interconnecting elements; no one expects you to internalize or memorize every aspect of it. As long as your players are having fun, everything will be just fine.
\end{DndSidebar}

\subsubsection{Character Back Stories}

The beginning of this adventure assumes the characters are familiar with each other. Are they using Jigow as a rest stop on their way elsewhere, are they residents of Jigow, or some combination of the two?

If the players are unfamiliar with Exandria and the lands of Xhorhas, they might find it daunting to create a back story set in this world. If you have Explorer's Guide to Wildemount, the Heroic Chronicle, p. 5 section in that book roots a character's history in the lands of Wildemount by helping the player determine their character's birth nation, home settlement, and relationships with family members, allies, and enemies. It also helps establish major events that happened to the character before the campaign began.

Regardless of whether you use that information, asking the players the following questions about their characters can help them come up with a strong back story:

\begin{itemize}
\item What's one small heroic deed you've already accomplished?
\item Do you give your trust easily, or do people have to earn your trust?
\item What's one thing you want more than anything else? How will being an adventurer help you achieve it?
\end{itemize}

\subsubsection{What the Characters Know}

Characters begin the adventure knowing the following facts, which you can share with the players as they create their characters.

\paragraph{Kryn Dynasty}

The Kryn Dynasty is the dominant nation in Xhorhas. It was founded by a drow queen named Leylas Kryn, who fled the Underdark and the tyrannical rule of Lolth the Spider Queen along with her disciples. The Bright Queen still rules the dynasty centuries later, and its cities contain more than just drow. Orcs, goblinoids, tieflings, humans, and many others call the cities of the dynasty their home. Countless more denizens of the dynasty are nomads who roam the wastes in clans, hunting mastodons and other Xhorhasian megafauna.

\paragraph{Jigow}

This coastal settlement is actually a string of villages that are home to a collection of folk from all over Xhorhas. Goblin and orc clans founded Jigow, which explains why the settlement is governed by two elders---a goblin and an orc. The Aurora Watch (the military arm of the Kryn Dynasty) maintains a presence here, under the command of a drow called Taskhand Durth Mirimm.

Townsfolk tend to be competitive, and friendly rivalries are commonplace. Most of Jigow's residents live in a central region called the Jumble.

\paragraph{The Luxon}

The official deity of the Kryn Dynasty, whose symbol appears on the nation's heraldry, is the Luxon. This mysterious divine entity of light and rebirth has granted its faithful several esoteric secrets, the greatest of which is consecution---the act of preparing one's soul for rebirth. Through consecution, some people within the Kryn Dynasty have lived many lifetimes, often in bodies different from the ones they were first born in. In the sequence of consecution, a drow might become a goblin, then be reborn as a bugbear, then an orc, and so on---all the while gaining greater knowledge about the world through their experiences. This process has no mechanical benefit, but players can make consecution and rebirth an interesting part of their characters' back stories.

If a follower of the Luxon who has undergone the ritual of consecution dies within 100 miles of a Luxon beacon, their soul is ensnared by it and reincarnated within the body of a random Humanoid newborn within 100 miles of the beacon.
\section{Pronunciation Guide}

The Pronunciations table shows how to pronounce many of the unusual names that appear in this adventure.

% Table: Pronunciations
\begin{DndTable}[header={Pronunciations}]{lll}
Name & Pronunciation & Description \\
\textbf{Aloysia Telfan} & ah-LOY-see-uh TELL-fin & Elf agent of the Consortium of the Vermilion Dream \\
Alyxian & ah-LICK-see-in & Tragic human hero who sacrificed himself to save others \\
Ank'Harel & AWNK-har-el & Majestic oasis city in Marquet \\
Apotheon & uh-PAW-thee-awn & Alyxian's title, denoting his godlike powers \\
\textbf{Aradrine the Owl} & AWR-uh-dreen & Goliath co-leader of the Consortium of the Vermilion Dream \\
Ayo Jabe & AYE-oh JAW-bay & Water genasi leader of the rival party \\
Bautha Dyrr & BOUGH-thuh DEER & Drow priest in Bazzoxan \\
Bazzoxan & BAZ-oh-zan & Military outpost in Xhorhas \\
Beltreath & BELL-treeth & Sword wraith commander in Cael Morrow \\
Bookkeeper Khime & KEEM & Orc member of the Allegiance of Allsight \\
Cael Morrow & KAYL MAW-row & Sunken ruin of an ancient city beneath Ank'Harel \\
Catha & KATH-uh & Exandria's larger, silvery moon \\
Colbu Kaz & COAL-boo KAZ & Goblin elder of Jigow \\
Dendarron the Sun Bear & den-DAR-run & Halfling co-leader of the Consortium of the Vermilion Dream \\
Dermot Wurder & DER-mott WER-der & Goblin priest of the Luxon and a member of the rival party \\
Exandria & ex-ANN-dree-uh & The world of Critical Role \\
Gaeya Iliera & GUY-yaw ill-ee-AIR-uh & Drow chef in the Aurora Watch \\
Galsariad Ardyth & gal-SAIR-ee-add AR-dith & Drow mage and a member of the rival party \\
Hadarai & HAD-ah-rye & Ghost of an orc priest of Corellon \\
Irvan Wastewalker & UR-vin & Human scoundrel and a member of the rival party \\
\textbf{Jamil A'alithiya} & juh-MEEL AYE-uh-lee-thee-uh & Human High Curator of the Archive of the Cobalt Soul in Ank'Harel \\
Jigow & JEE-gow (rhymes with cow) & Town in Xhorhas populated mostly by goblins and orcs \\
Kalym Telaarin & kal-EEM tel-AW-rin & Drow warrior in Bazzoxan \\
Khelkur the Gull & KEL-ker & Dwarf co-leader of the Consortium of the Vermilion Dream \\
Kryn Dynasty & KREEN & Governing body of Xhorhas \\
Larthul the Wolf & LAR-thool & Human co-leader of the Consortium of the Vermilion Dream \\
Luxon & LUCKS-awn & Mysterious god of light and rebirth \\
Marquet & mar-KET & Continent covered in deserts, mountains, and jungles \\
Naevyn Tasithar & NAY-vinn TASS-uh-thar & Drow scout in Bazzoxan \\
Nedosi Anay & nuh-DOE-see AH-nay & Half-orc security captain of the Luck's Run casino \\
\textbf{Perigee} & PAIR-uh-jee & Deva agent of Sehanine corrupted by ruidium \\
Postraeck & POST-rake & Human bandit living in the wastes of Xhorhas \\
\textbf{Prolix Yusaf} & PRO-licks & Tiefling agent of the Allegiance of Allsight \\
Rosohna & roh-SO-nuh & Capital of the Kryn Dynasty \\
Ruidium & roo-IDD-ee-um & Corrupting element born of the Netherdeep \\
Ruidus & ROO-idd-iss & Exandria's smaller, vermilion moon \\
Saqiri & sah-KEE-ree & Drow hunter from Alyxian's past \\
Theo Nathope & THEE-oh NAH-tho-pee & A young, innocent avatar of the Apotheon \\
Ushru & OO-shroo & Orc elder of Jigow \\
\textbf{Verin Thelyss} & VAIR-in THAY-liss & Drow military commander of Bazzoxan \\
Watcher Sylralei & SEEL-ruh-lay & Dwarf berserker of the Sentinels of Memory \\
Wildemount & WILD-mount (long i) & Continent covered in vegetation and snow \\
Xhorhas & ZJOR-hawss & Eastern realm of Wildemount, mostly marshes and grasslands \\
\textbf{Xot} & ZJOT & Goliath archaeologist of the Allegiance of Allsight \\
\end{DndTable}

\chapter{A Fateful Competition}\label{ch:a-fateful-competition-2-2}

Like many of the heroic legends that echo throughout the annals of Exandrian history, this tale begins in peace---or what passes for peace in Jigow at festival time. Goblin children run through the streets, screaming with mirth, while orcs, humans, drow, and people of countless other folk clink tankards, hurl hatchets at targets, stuff their faces with meat pies, and plunge into the frigid waters of the Ifolon River---all in the name of friendly competition.
The characters are in Jigow during the city's annual Festival of Merit. Though they are the protagonists of this story, today they are but a small handful of excited festivalgoers amid a merry throng. Many of the Xhorhasians present are famed warriors, skilled athletes, cunning con artists, and talented arcanists. Among those taking part in the festivities are the characters' rivals, another group of adventurers whose fates are entwined with the characters.
At the festival's end, the characters and their rivals are chosen to dive into the waters of the Emerald Gulch in search of a prize. Little do they know that this plunge will start them on an adventure greater than any of them could possibly imagine.
% Content: Unknown (dict)
\section{Running This Chapter}

This chapter has two parts: the Festival of Merit and the Emerald Grotto. Each part can be played in a single 3- to 4-hour session. Be sure to review the information about the rival nonplayer characters in the "Rivals" section of the introduction (p. 0) before running this chapter. The rivals' interactions with the characters in this chapter will lay the groundwork for relationships that could last throughout the adventure.

The first half of the chapter is a lighthearted and free-form exploration of Jigow and the Festival of Merit. Familiarize yourself with the map of Jigow and the locations keyed to it so that you can effectively guide the characters as they explore the settlement and partake in the festival games.

The second part is a dangerous but straightforward dungeon delve into the Emerald Grotto near Jigow. The characters must overcome a series of challenges to reach the treasure at the end of the dungeon before the rivals do.

\subsection{Rival Impressions}

Whenever a rival is present for an event in this chapter, a "Rival Impressions" section explains how that individual might react to the characters' actions. These suggestions will help you develop the relationships between the rivals and the characters. The evolution of a fierce rivalry into a friendship---or even a romance---is an exciting way to enhance the ongoing story of this adventure. Conversely, a friendship crumbling because of a breakdown in communication or a clash of two unbending ideals is a tragedy that echoes the darker tones of this adventure.

A character can try to change a rival's attitude through roleplaying, which you can adjudicate based on the social interaction rules, p. 8 in the Dungeon Master's Guide. You can also improvise and adjust the rivals' attitudes at your discretion, based on how you feel the rivals would react in any given situation.
\section{Jigow}

Jigow is located in northern Xhorhas on the banks of the Emerald Gulch and the Ifolon River. An amalgamation of coastal villages that were originally settled by several nomadic clans of orcs and goblins, Jigow is now home to humans and other folk as well, and the city gained a significant drow population after Jigow became a part of the Kryn Dynasty.

The villages and townships that make up Jigow are loosely divided into three major areas: the Meatwaters, the main dock area on the shores of the Ifolon River; the Wetwalks, a collection of houses on stilts closest to the wetlands and marshes; and the Jumble, the most densely populated region of the city, where houses are built among giant mangrove trees or on the backs of horizonback tortoises and used as traveling homes.

Jigow has a council of elders who help its people resolve internal conflicts and a military presence in the form of the Aurora Watch---the soldiers of the Kryn Dynasty. The dynasty also has a political liaison in Jigow: a drow named Durth Mirimm (see area J9). Serving as an intermediary between Jigow's elders and the court of the Bright Queen Leylas Kryn, Durth cares more about what Jigow can do for the dynasty than how he can improve the lives of Jigow's citizens.

The Jumble is the heart of Jigow and home to a colorful patchwork of local cultures. The traditions of ancient Xhorhasian peoples, many of them orcs and goblins, live on in the peaceful chaos of the Jumble. The twisting, unplanned streets give the district its name---a medley of homes, shops, workshops, athletic fields, amphitheaters, shrines, taverns, stables, and public spaces for lively gatherings.

Jigow was built from the sweat of its founders' brows, the strength of their muscles, and the music of their voices. In Jigow, any chance to show off one's physical, social, or intellectual prowess is a welcome one, and there's no opportunity better than the annual festival. On this day, friendly and not-so-friendly contests between residents are settled. Every public space is filled with festival games, food stalls, or other amenities for the merrymakers, and plenty of shops and private residences have flung their doors wide to join in the revelry.
\section{Adventure Start}

This story begins as the characters enter the Jumble. Read or paraphrase the boxed text below to set the scene:

\begin{DndReadAloud}
You have entered the Jumble, a large district of tangled roads and single-story buildings in the town of Jigow. Throngs of people, most of them orcs and goblins, move through the streets, laughing, singing, running, and sightseeing. All are enthralled by the raucous sights and sounds of the town's Festival of Merit.
You hear snippets of conversations as people pass by: a goblin mother telling her children not to go near the baby horizonback tortoises, a drow guard in shining insectile armor complaining to his partner that his gauntlet was crushed by a hulking orc while arm-wrestling, and a pair of young orcs in swimwear hollering as they rush toward the banks of the Ifolon River.
All around you, colorful signs and banners point toward festival booths surrounded by cheering people. On this street alone, you can see a meat-pie eating contest near a shop mounted on the back of a massive tortoise, and on the other side of the road, a banner emblazoned with the words "Riddles and Rhymes: Unbeatable Riddles!" That banner points toward a three-story temple in the center of the Jumble. The town is yours to explore---where do you want to go?
\end{DndReadAloud}

\subsection{Festival of Merit}

Seven festival games and challenges are described in this chapter, as follows:

\begin{description}
\item[J1: Best Pies in the Jumble.] The Unbroken Tusk inn boasts the best meat pies east of the Ashkeeper Peaks. Its owner has offered a prize to the person who can eat the most pies.
\item[J2: One-Shot Solution.] A maze has been constructed in the Jumble. Competitors win a prize if they navigate the maze in one try without coming upon a dead end.
\item[J3: The Ifolon Plunge.] Some distance offshore in the Ifolon River, a rusted spear is lodged into a weathered piece of a pier that has long since rotted away. It is a popular spot for Jigow's most talented swimmers to test their speed and skill.
\item[J4: Call to Arms.] One of the Aurora Watch soldiers stationed in Jigow has challenged anyone and everyone to best her in an arm-wrestling contest. So far, she remains undefeated.
\item[J5: Wetwalks Paddywhack.] Members of the Gakthash and Uvuroh goblin clans time their rice paddy harvests to coincide with the Festival of Merit every year. This event turns the mundane act of harvesting into a spirited competition.
\item[J6: Herding the Horizonbacks.] This year's brood of young horizonback tortoises must be relocated from the Wetwalks to the Jumble.
\item[J7: Riddles and Rhymes.] One of Jigow's elders is fond of making up riddles. He has come up with several new ones for the festival this year.
\end{description}

Each of these challenges is described in greater detail in its own section. Except where otherwise noted, any number of characters can participate in an event.

The characters might want to look for a particular type of game: a contest of athleticism, a test of puzzle-solving, or a trial of endurance. In that case, they must spend a half-hour strolling through the streets in search of the type of challenge they're looking for.

\subsubsection{Medals of Merit Cards}

A character who wins a festival contest earns Medals of Merit. Each medal is described in appendix B (p. 9), and its description also appears on Medals of Merit in appendix C (p. 10). You can hand a Medals of Merit to each player whose character won the medal.

\subsection{Locations in Jigow}

The following locations are keyed to the map of Jigow. They include places where Festival of Merit contests are staged (areas J1 through J7) and other important locations that the characters might visit (areas J8 through J10).



\subsubsection{J1: Best Pies in the Jumble}

This contest tests the participants' perseverance and endurance. Read:

\begin{DndReadAloud}
The savory scents of meat, pastry, and spices fill the air around a three-story building mounted on the back of a gigantic tortoise. A cooking stand and a festival stage with a long table are set up at the tortoise's feet, where an orc stands over a massive oven. She bustles from the oven to the stage and back, placing delicious-looking hand pies at each seat and stacking more of them on a cooling rack nearby.
People are already gathered on stage, including a scrawny young human with a mop of brown hair and a scruffy beard. As you approach, the orc calls out in a melodic baritone, "Come to sample the best meat pies this side of the Wastes, yes? I sell 'em for meals up there, but I'm running pie-eating contests down here all day, if you've got an orc-sized stomach."
\end{DndReadAloud}

\textbf{Agathe Silverspoon} is a chaotic good \textbf{orc} who moves from counter to counter, deftly slicing a variety of cooked and spiced meats, then folding them into piecrusts that she places into the oven behind her.

Agathe gestures toward the raised platform next to her cooking station, where three people are already seated: a cocky male \textbf{drow} flexing his muscles, pointing at people in the gathering crowd and winking; a female halfling (use the \textbf{thug} stat block), sitting very still with a smirk on her face; and a young male human, scanning the crowd as if looking for someone. This individual is \textbf{Irvan Wastewalker (tier 1)}, whose stat block appears in appendix A (p. 8). These contestants have the following modifiers to their Constitution checks and saving throws, which come into play during the pie-eating contest:

\begin{description}
\item[Male Drow:] con 10
\item[Female Halfling:] con 14
\item[Irvan Wastewalker:] con 12 on checks, con 3 on saving throws
\end{description}

\paragraph{Putting Away the Pies}

It costs 5 sp to enter this contest. On Agathe's signal, everyone seated at the platform can begin eating. The pies taste great, the meat seasoned to perfection and the crusts light and fluffy. Of course, there's not much chance to savor the flavor of the pies when the goal is to put away as many of them as possible.

All contestants can easily eat their first pie, requiring no check. Eating a second pie requires a successful DC 8 Constitution check. For every pie after the second, the DC increases by 2. The person who eats the most pies before failing a check is the winner (ties are possible). A contestant whose check fails is too full to continue eating and must make a DC 15 Constitution saving throw. On a failed save, the contestant becomes \textit{poisoned} for 1 hour and feels queasy.

\paragraph{Meeting Irvan}

Irvan sits next to one of the characters during the contest. If they speak to him before the contest begins, he merely wishes the character good luck. He talks more after the contest.

\paragraph{Rival Impressions}

Win or lose, Irvan is friendly toward characters who competed in the contest. His friendliness evaporates quickly if a character insults his scruffy and adolescent appearance.

As a new member of his adventuring party, Irvan isn't keen on revealing that he's part of a group. But he does invite any of the characters that he's friendly with to join him for the festival's closing ceremony at sunset.

\paragraph{Treasure}

Each winner of this contest receives a tin \textit{medal of the meat pie} (see appendix B (p. 9)) shaped like---you guessed it---a meat pie.

Win or lose, a character can't compete in this contest again until they complete a long rest.

% Image placeholder: {@creature Agathe Silverspoon|CRCotN} makes the best meat pies in Jigow

\subsubsection{J2: One-Shot Solution}

This game tests one's memory and navigation skills. To begin, read:

\begin{DndReadAloud}
Near the entrance to the festival grounds, a maze has been constructed out of four-foot-tall walls of wooden planks lashed together with ropes and propped up between sandbags. Banners are hung over the openings to the maze, signifying the entrance and the exit. The passageways of the maze are five feet wide. Off to the side, a tan-feathered aarakocra sits on a chair thirty feet above the ground, giving him a vantage point where he can see the whole area.
A small crowd is cheering on and heckling the people inside. Near the entrance, an ogre in leather armor is encouraging a reluctant goblin in a breastplate to give the maze a try.
\end{DndReadAloud}

The way to win is to study the maze from the outside, and then follow the only correct path to the exit in one shot. If a creature reaches a dead end, the judge---a lawful neutral \textbf{aarakocra} named Sharpwatch---swoops down and plucks them from the maze.

\textbf{Maggie Keeneyes (tier 1)}, whose stat block appears in appendix A (p. 8), has figured out the solution to the maze, but the ogre's size keeps her from entering the maze. She wants her friend, a goblin named \textbf{Dermot Wurder (tier 1)}, whose stat block also appears in appendix A (p. 8), to go in her place. Dermot, however, is convinced that even with her instructions, he will come to a dead end.

\paragraph{Studying the Maze}

A character can study the maze from the outside and make a DC 14 Intelligence (Investigation) check. If this check succeeds, the character has advantage on the check made to navigate the maze.

\paragraph{Entering the Maze}

It costs 3 sp to participate in this challenge. When a character enters the maze, they must make a DC 14 Wisdom (Survival) check to try to find the correct path. On a success, the character traverses the maze in one shot and emerges victorious on the other side.

On a failed check, the character gets lost and ends up in one of the maze's dead ends. A lost character can make a DC 18 Dexterity (Stealth) check to try to backtrack quickly out of the dead end, avoiding Sharpwatch's notice on a success; the character can then make another Wisdom (Survival) check to navigate the maze. If either of these checks is a failure, Sharpwatch swoops down and plucks the character from the maze, depositing them outside.

A character can try the maze as many times as they would like, paying the 3 sp entrance fee each time.

\paragraph{Meeting Maggie and Dermot}

After the event, Maggie approaches a character who successfully navigated the maze and congratulates them. Dermot follows up, introducing himself and asking if the characters can help him with a small task:

\begin{DndReadAloud}
"It's nothing urgent," he says, "but if you find my friend Ayo, would you tell her Dermot's wondering what the plan is after the closing ceremony? She has dark hair and blue skin. She said she was going to compete in a contest by the river---I think."
\end{DndReadAloud}

% Image placeholder: {@item Medal of the Maze|CRCotN}

Ayo Jabe can be found competing in area J3.

\paragraph{Rival Impressions}

Dermot thinks the best of people and is friendly toward anyone who encourages him to walk the maze or who backs him up by saying he should sit it out. Both Dermot and Maggie are hostile toward characters who insult them---especially if anyone makes fun of the fact that Maggie is too big to navigate the maze or implies that she lacks insight because she's a musclebound ogre.

\paragraph{Treasure}

Characters who complete the maze are awarded a \textit{medal of the maze} (see appendix B (p. 9)) made of smooth driftwood, hung on a piece of rough twine. A labyrinthine pattern has been carved into the front of the medal.

\subsubsection{J3: The Ifolon Plunge}

This swimming race pits competitors against the surging current and sharks of the Ifolon River. Read:

\begin{DndReadAloud}
Several people are gathered on the piers, arguing among themselves and venting their ire at a spry goblin who responds to their anger with an amused expression. Suddenly, a brash voice cuts through the chatter, and a blue-skinned woman pushes her way through the crowd and stands protectively in front of the goblin, yelling, "She's not gonna say it again! If you're here with a team, only one member of your team can participate in the challenge! The rules are clear, and if you can't handle it, get off the pier!" After just a couple of seconds, small groups of people begin to quietly murmur among themselves.
\end{DndReadAloud}

The game is refereed by an impassive, lawful neutral \textbf{goblin} named Omo.

The blue-skinned individual is a water genasi named \textbf{Ayo Jabe (tier 1)}, whose stat block appears in appendix A (p. 8). After breaking up the argument on the docks, she returns to her companion, an elegant drow in wizard's robes. This is \textbf{Galsariad Ardyth (tier 1)}, whose stat block also appears in appendix A (p. 8).

Ayo and Galsariad exchange a few terse words before Ayo motions toward the pier. A character who eavesdrops on them can make a DC 14 Wisdom (Perception) check. On a success, the character learns that the drow offered to cast a \textit{longstrider} spell on the genasi, but she declined; she wants to win on her own.

\paragraph{Participating in the Challenge}

Entering this contest costs 5 sp. The goal of this challenge is to swim upstream through the rushing waters of the Ifolon to reach a rusted spear sticking out of a rotten wooden post, then bring the spear back to the dock.

Only one of the characters can participate in this challenge. If multiple characters from the party want to take part, anyone after the first must make a DC 14 Charisma (Deception) check. A character who succeeds on this check can compete; those who fail the check are ejected from the pier by Omo.

Other than Ayo, the contestants are a lanky male orc \textbf{scout}, a broad-shouldered female human \textbf{thug}, and a surly male \textbf{goblin}. These nonplayer characters have the following modifiers to their Strength (Athletics) checks, which come into play during the swimming contest:

\begin{description}
\item[Ayo Jabe:] athletics 4
\item[Male Orc:] athletics 0
\item[Female Human:] athletics 2
\item[Male Goblin:] athletics -1
\end{description}

\paragraph{Race for the Spear}

On Omo's mark, the contestants plunge into the water and begin swimming. Competitors roll initiative when they dive into the water, with all participants expected to use the Dash action every round. Any contestant who moves more than 30 feet on their turn must make a DC 13 Strength (Athletics) check at the end of that turn. On a failed check, the contestant gains 1 level of \textit{exhaustion} and is pushed 15 feet downstream as they are pummeled by the current.

The distance from the pier to the spear is 75 feet, and it takes an action to wrench the spear free. A character can pull the spear out of another contestant's hands by swimming to within 5 feet of them and succeeding on a Strength check with a DC equal to 10 + the other contestant's Strength modifier. Whoever returns the spear to Omo is the winner.

\paragraph{Danger from Below}

After the spear is pulled from the rotten post, two Ifolon striped sharks (use the \textbf{reef shark} stat block) emerge from the depths of the river. Characters who have a passive Wisdom (Perception) score of 13 or higher notice the water churning more severely as the sharks, riled up by the antics of the swimmers, rise to feed.

Roll initiative for the sharks. They attack contest participants indiscriminately, interested in scoring what they think is an easy meal. A shark tries to flee when it has been reduced to 11 hit points or fewer. Ayo doesn't waste time fighting the sharks; her only goal is to reach the finish line (with the spear, if she isn't already holding it). The other contestants in the water try to fend off the sharks.

\paragraph{Meeting Ayo and Galsariad}

Prior to the challenge, Ayo is focused on the race and politely brushes off any attempt at conversation. Once the characters are safe and sound on the pier after the race, she and Galsariad approach them.

\paragraph{Rival Impressions}

Ayo is impressed more by courage and determination than by success. If a character participated in the contest but failed to recover the spear, she claps that character on the shoulder with a crooked grin on her face and says, "I've never seen anyone give it their all like that before. 'Cept for myself, I mean." She goes on to tell the character that she has assembled an adventuring party, and that she'd like to introduce everyone after the festival's closing ceremony at sundown.

Ayo looks poorly on a character who broke the rules of the contest to participate. Galsariad, however, quietly smiles and inclines his head at that character, impressed by their cunning.

\paragraph{Treasure}

Omo awards the winner a \textit{medal of the conch} (see appendix B (p. 9)) made of sea glass that has been magically crafted into the shape of a conch shell.

\subsubsection{J4: Call to Arms}

This arm-wrestling competition tests a character's strength and determination. Read:

\begin{DndReadAloud}
You approach a brick building with a sign that reads, "Beefslab Butchers." A simple table and two chairs have been set up outside. Both chairs are currently occupied by people wearing the armor of the Aurora Watch, their elbows braced against the table and their hands locked in an arm-wrestling duel. Two dozen festivalgoers watch them silently, until---SLAM!---the occupant of the left chair pushes her opponent's arm to the table and the crowd erupts in whoops and cheers.
\end{DndReadAloud}

The victor in this contest, and the defending champion of the event, is Maryl Bronzefang, a lawful neutral, orc \textbf{gladiator}. A native of Jigow serving in Bazzoxan's Aurora Watch, she was given leave to return home for the festival. She is flanked by two compatriots, both female drow \textbf{veterans} clad in chitinous splint armor. A quiet male goblin \textbf{commoner}---Vars, the owner of the butcher shop---sits off to the side, managing the wagers made by onlookers.

After prodding Dermot to try the maze in area J2, Maggie Keeneyes can be found here, eager to compete.

Entering this contest is free, though bystanders can talk to Vars to place bets on the participants.

Maryl Bronzefang competes only against the best, and to challenge her, a character must first defeat one of her drow companions. Maryl returns to the table after at least one character has accomplished this, casting a challenging grin at the victor and inviting them to try their luck against her. The first character to defeat Maryl receives a challenge from Maggie. Otherwise, Maggie challenges Maryl after all other contestants fall by the wayside. These nonplayer characters have the following modifiers to their Strength checks, which come into play during the arm-wrestling contest:

\begin{description}
\item[Maggie Keeneyes:] str 18
\item[Maryl Bronzefang:] str 18
\item[Female Drow:] str 16
\end{description}

\paragraph{Contest Rules}

In any arm-wrestling match, have both contestants make a Strength (Athletics) check with a DC equal to 10 + the other contestant's Strength modifier. If one contestant succeeds and the other fails, the one who succeeded wins the match. Repeat the checks as needed until one contestant fails and the other succeeds.

A character who loses a match and wants to try again must first succeed on a DC 15 Charisma (Deception or Persuasion) check. On a failed check, the character is laughed away by the crowd. If a character loses to the same foe twice, that contestant declines any further challenges from that character.

\paragraph{Betting}

A character can place a bet with Vars by handing him 5 sp and declaring who they think will win the next match. A character whose choice wins the match receives 9 sp, and Vars pockets 1 sp.

\paragraph{Treasure}

Vars hands anyone who bests Maryl and Maggie a \textit{medal of muscle} (see appendix B (p. 9)) made of ebony and shaped like a flexing arm. If one stares at it for long enough, the muscles appear to bulge.

\subsubsection{J5: Wetwalks Paddywhack}

This game requires teamwork and skill with a blade. Read:

\begin{DndReadAloud}
On the eastern edge of the Jumble, the docks rise upward from the river to run alongside wooden shacks built on stilts, casting crooked shadows over the wetlands below. Some areas of the wetlands have been cordoned off to create rice paddies, which are now ready for harvest.
Blue and green banners wave over one end of the docks, where two elderly goblin farmers preside over the contest. "You know how it goes!" one hollers out in a reedy voice. "First team to harvest the rice from their section wins!"
\end{DndReadAloud}

This event is managed by two neutral good \textbf{goblins} named Beetle and Zag. Their farm is jointly run by the Gakthash and Uvuroh goblin clans, and every year, the goblin couple uses the festival competitions as a labor-saving way to harvest their crops.

After the pie-eating contest in area J1, Irvan can also be found relaxing here.

\paragraph{Harvesting the Rice}

A team of two is required to participate in this contest, and there is no entry fee. If a character needs a partner, Irvan grudgingly volunteers to be their teammate for the event. When a team enters the contest, they are asked to designate which is the cutter (who cuts the stalks down) and which is the gatherer (who gathers the stalks and prepares them for drying). Zag provides the cutter with a sickle before Beetle directs the teams to the paddies and asks them to await the signal to start.

Each team is responsible for a 20-foot-long, 5-foot-wide strip of a paddy. On Beetle's signal, the teams race to successfully harvest their strip. Aside from the characters, two other teams are competing in this race: an imposing female \textbf{orc} and her rambunctious son (who uses the \textbf{commoner} stat block); and a pair of \textbf{drow} siblings, who wear broad-brimmed straw hats and constantly bicker with each other. These teams have the following modifiers to their Strength (Athletics) and Dexterity (Sleight of Hand) checks, which come into play during the contest:

\paragraph{Orc Team}

Athletics athletics 3, Sleight of Hand sleight\_of\_hand 0

\paragraph{Drow Team}

Athletics athletics 0, Sleight of Hand sleight\_of\_hand 2

At the end of each minute, each team's cutter must make a DC 11 Strength (Athletics) check; on a success, the cutter clears a 5-foot-square section of their paddy. Once a section has been cleared, the team's gatherer must spend 1 minute trying to gather and bundle the shorn rice, doing so with a successful DC 11 Dexterity (Sleight of Hand) check; a failed check causes the team's cutter to make their next Strength (Athletics) check with disadvantage, as the gatherer's bundling work gets in the way. The first team to clear and bundle the rice from all four 5-foot squares of their strip of paddy wins (ties are possible).

\paragraph{Treasure}

Beetle and Zag award each member of the winning team a copper \textit{medal of the wetlands} (see appendix B (p. 9)) engraved with a sheaf of rice, hung on a necklace woven of reeds.

\subsubsection{J6: Herding the Horizonbacks}

This challenge tests the participants' acrobatic ability and skill with handling animals. Read:

\begin{DndReadAloud}
Festivalgoers are clustered around a paddock full of young horizonback tortoises that are shuffling around inside the enclosure. The fifteen-foot-tall tortoises are being tended by young herders, who are fitting them with makeshift bridles. These individuals are overseen by a tall, attractive orc who makes sure the bridles are attached correctly.
The orc waves at you and says, "Have you come to join the horizonback migration? There's a tortoise for each of you, and a medal to be won if you make it to the end in one piece. Last year we had to rebuild three houses after the festival!"
\end{DndReadAloud}

After the Ifolon Plunge in area J3, Ayo Jabe can be found relaxing here, watching the tortoises. She laughs at characters who refuse the challenge.

The paddock is maintained by a horizonback tortoise trainer named Adan, a lawful good, orc \textbf{scout}. Adan supervises the transfer of the young tortoises from the paddock to the open fields on the west side of the Jumble, where the tortoises will begin training to become mounts and mobile buildings for the tribes of Jigow.

A \textbf{young horizonback tortoise} uses the \textbf{ankylosaurus} stat block, with these changes:

\begin{itemize}
\item The tortoise can breathe air and water.
\item It understands Goblin but can't speak.
\item It has Bite attack that is identical to the ankylosaurus's Tail attack, except that it deals piercing damage.
\end{itemize}

There are as many tortoises available as there are characters who want to participate in this event. Attached to the front edge of each tortoise's shell is a makeshift seat big enough for a single rider.

\paragraph{Maneuvering the Herd}

Once all the contestants are seated on their tortoises, Adan climbs onto the back of a larger tortoise outside the paddock. With a yell and a hand signal, Adan begins to lead the group of tortoises out of the paddock.

To successfully maneuver a tortoise to the destination, a character must make three successful Wisdom (Animal Handling) checks. The DC for these checks begins at 12 and increases by 1 for each check thereafter. On a failed check, the tortoise tries to buck the character from its back. The character must succeed on a DC 13 Dexterity saving throw or fall from the tortoise and take 3 (1d6) bludgeoning damage. On a successful check, the character doesn't fall and can try to maneuver the tortoise again.

After a failed saving throw, a character can mount their tortoise by succeeding on a DC 15 Dexterity (Acrobatics) check. If this check fails, the belligerent tortoise makes a Bite attack against the character. Once mounted on the tortoise again, a character can continue trying to move the tortoise along by making another Wisdom (Animal Handling) check as described above.

If a tortoise is \textit{charmed} by an \textit{animal friendship} spell or a similar effect, Wisdom (Animal Handling) checks to guide it are made with advantage, and it won't buck its handler on a failed check.

\paragraph{Treasure}

Once a tortoise reaches its destination, Adan awards its rider a \textit{medal of the horizonback} (see appendix B (p. 9)).

No two medals awarded from this contest look the same. Each one appears to be made from some different part of a tortoise: a piece of shell, a claw, or a tooth. Adan assures the characters that these medals were made from tortoises that died naturally.

\subsubsection{J7: Riddles and Rhymes}

This event challenges a character's intellect with a series of riddles. Read:

\begin{DndReadAloud}
An elderly goblin in blue robes presides over a set of three tables. One table holds a wooden box, the second displays a glittering star map of the Exandrian sky, and on the third rests a row of colored bottles. A banner hanging over the setup reads, "Test your wit with me, against these riddles three."
\end{DndReadAloud}

Elder Colbu Kaz, a chaotic neutral, goblin \textbf{priest}, created these three puzzles. Each attempt to answer one riddle costs 2 sp. Answering at least two of the riddles correctly is enough to win the contest (see "Treasure" below). Various other challengers are trying to solve the riddles, including Galsariad Ardyth. This clever drow solves two of the riddles after a total of four guesses. If the characters solve each riddle on the first try, he is impressed (see "Meeting Galsariad" below).

\paragraph{Table 1: The Box's Riddle}

On this table are a locked wooden box and four metal keys. On a placard beside the box is written the following riddle in Common:

\begin{DndReadAloud}
Bronze, copper, silver, gold---Ancient ones from tales of old. Match the key to the box's lock; A mistaken choice begets a shock.
\end{DndReadAloud}

A character who inspects the keys can see that each is made of a different metal---one bronze, one copper, one silver, and one gold. Close examination of the box reveals that its lock is shaped like a dragon with a fluted crest, and the edges of the box are decorated with engravings of pearls and shells.

The lock's shape and sea-themed decorations are clues that suggest a bronze dragon---and thus, it's the bronze key that opens the lock. A character who inserts the incorrect key or tries to pick the lock must succeed on a DC 16 Dexterity saving throw or take 2 (1d4) lightning damage as the lock emits an electric shock.

A character can make a DC 10 Intelligence check to recognize that the metallic hues of the keys and the first two lines of the riddle are talking about not just metals but also types of metallic dragons. A character who succeeds on a DC 13 Intelligence (Arcana or History) check recognizes the shape of the dragon lock or the box's aquatic decorations as indicative of a bronze dragon; if a character saw someone get shocked by lightning after an incorrect guess, the DC for this check is reduced to 11.

\paragraph{Table 2: The Star Map's Riddle}

The star map resting on this table depicts the moons and constellations visible in the Exandrian night sky, accompanied by the following riddle in Common:

\begin{DndReadAloud}
Two birds sit in a speckled field, One silver and calm, one scarlet with woe. Nigh all year, the red one yields To silver's illusions, mischief, and shelt'ring glow.
\end{DndReadAloud}

To answer the riddle, a character must identify the "birds" of the riddle and point them out on the star map. The answer is the twin moons of Exandria, Catha and Ruidus.

A character who succeeds on a DC 13 Intelligence (History or Religion) check recalls the stories regarding Exandria's moons, recognizing the second line as referring to the superstitions surrounding Ruidus and the last line as referring to the divine portfolio of Sehanine the Moon Weaver. A character who succeeds on a DC 11 Wisdom (Perception) check determines that there is only one red object on the star map---Ruidus.

\paragraph{Table 3: The Bottles' Riddle}

On this table is a row of seven small bottles filled with colored liquids, and the following riddle in Common:

\begin{DndReadAloud}
Two of us are brewed from blight And always sat to purple's right. Three are juice, one burns like flame, And no two colors taste the same. Even flasks hold naught but pain, And shade of sky will leave a stain. Though tasty are those at each end, Neither is the winner's friend. A puzzle for the keen and wise, Drink the light to claim your prize.
\end{DndReadAloud}

The bottles, numbered from left to right, have the following contents:

\begin{description}
\item[Bottle 1.] This purple bottle is filled with berry juice.
\item[Bottle 2.] This green bottle is filled with blight brew. A character who sips from this bottle must succeed on a DC 10 Constitution saving throw or become \textit{poisoned} for 1 minute, spending the time retching.
\item[Bottle 3.] This sky-blue bottle is filled with berry juice.
\item[Bottle 4.] This white bottle is filled with overwhelmingly spicy pepper oil. A character who sips from this bottle must succeed on a DC 10 Constitution saving throw or take 1 fire damage from the spice.
\item[Bottle 5.] This purple bottle is filled with a liquid known in Jigow as glow potion. A character who drinks from this bottle feels warmth settle in their stomach as their skin emits dim silvery light out to a range of 5 feet for 1 minute.
\item[Bottle 6.] This green bottle is filled with blight brew, just like bottle 2.
\item[Bottle 7.] This red bottle is filled with apple juice.
\end{description}

A character must drink from the correct bottle to win, and anyone who attempts the challenge has only two chances to guess correctly. If someone guesses incorrectly twice, they fail this riddle. Bottle 5, containing the glow potion, is the correct bottle.

A character who succeeds on a DC 12 Intelligence check comes to one of the following conclusions:

\begin{itemize}
\item Bottles 2 and 6 are both directly to the right of a purple bottle and thus must contain blight brew.
\item Bottles 2, 4, and 6---the "even" ones---should be avoided, as they hold only "pain."
\item If the sky-blue bottle leaves a "stain," it must be harmless juice, but not the correct answer.
\item The first and last bottles must also be juice, as they are "tasty" but won't win the riddle.
\end{itemize}

Characters can make multiple Intelligence checks to gain multiple hints, but a character who fails a check can't make another until they drink from a bottle.

A \textit{detect magic} spell or similar effect also reveals the fifth bottle to be the only one containing magical liquid.

\paragraph{Meeting Galsariad}

If the characters didn't meet him at the Ifolon Plunge in area J3, Galsariad Ardyth approaches the characters after they're done with this challenge, win or lose.

\paragraph{Rival Impressions}

Unless the characters solve each riddle on the first try, Galsariad brusquely brushes past them, chuckling, "Better luck next time." Nonetheless, he is impressed by characters who display a ruthless drive to succeed. If this is the case, he inclines his head and says, "Your cunning is a cut above that of everyone else in this backwater. Do you plan on participating in the grand finale at sundown? I'd be interested in testing my wits against you."

\paragraph{Treasure}

If a character answers at least two of the riddles correctly, Elder Colbu Kaz awards that character a \textit{medal of wit} (see appendix B (p. 9)), which is sculpted and painted to look like the head of a fox.

\subsubsection{J8: Temple of the Luxon}

\begin{DndReadAloud}
The walls of this three-story temple are crafted from semitranslucent glass that seems to glow with its own faint light. The building, which is shaped like a dodecahedron, is dedicated to the Luxon, the First Radiance and patron deity of the Kryn Dynasty.
\end{DndReadAloud}

The temple is maintained by Belana Zolaed, a neutral good drow \textbf{priest} who also conducts worship ceremonies.

After the One-Shot Solution contest (see area J2), Dermot can be found here, reciting traditional prayers to the Luxon and committing them to memory.

\subsubsection{J9: Helter-Skelter}

Though official records call this wood-and-stone complex the Dynasty Outpost, the rebellious folk of Jigow have nicknamed it "Helter-Skelter" for the chaotic environment that the Kryn officers often descend into when dealing with the ornery youngsters---and not-so-youngsters---of the town.

The complex consists of two main structures: a low wooden barracks that houses the guards of the Aurora Watch stationed in Jigow, and a two-story stone dodecahedral building that is the residence of Durth Mirimm, a lawful neutral \textbf{drow elite warrior} who is the Taskhand of the Aurora Watch. Durth governs the settlement alongside the two clan elders, Colbu Kaz and Ushru.

\subsubsection{J10: Black Islands}

\begin{DndReadAloud}
Black basalt islands rise in jagged ridges from the waters of the Ifolon River.
\end{DndReadAloud}

Visitors and natives alike enjoy mingling, chatting, and watching the sun set over the river waters from the peaks of these islands. The smallest island contains the entrance to the Emerald Grotto.

\subsection{Final Contest}

Once the characters have participated in as many competitions as they want to, or at a time you feel is appropriate, the sun begins to set, and the sound of rhythmic drumming echoes throughout the Jumble, summoning the participants to the paddock (area J6), which has been cleared of horizonback tortoises. Elder Ushru, a neutral good orc \textbf{priest}, stands in the middle of the road atop a platform of wooden crates. When all the festival officials and a large throng of revelers are assembled, the drums die down, and the elder speaks:

\begin{DndReadAloud}
An elderly orc, dressed in deep blue robes, stands atop a platform of crates. He smiles at the crowd. "Mighty warriors, brilliant strategists, you have impressed us with your feats of strength, your incisive wit, and your sturdy bellies," he booms proudly. "But the main event, as you know, is yet to come. Only two teams will be chosen to compete in the final challenge---a race through the Emerald Grotto, in the depths of which the greatest prize awaits!"
\end{DndReadAloud}

Elder Ushru explains that he and Elder Colbu Kaz will choose two teams from among the festival's most successful contestants and invite them to engage in a race through a submerged grotto to retrieve a jeweled icon that has symbolized the competitive spirit of Jigow for generations: the Emerald Eye. The winners of the race receive 100 gp and are hailed as the city's Champions of Merit for the next year.

Elder Colbu Kaz joins Elder Ushru on the platform, where they consult briefly, pointing at figures in the crowd and whispering to each other. They eventually decide on two teams: the player characters and the rivals. Both groups are invited to follow the elders to the Emerald Grotto to compete in the grand finale.
\section{The Emerald Grotto}

The Black Islands sit in the mouth of the Ifolon River where it flows into the Emerald Gulch, a sea with a rocky shoreline. A cave in the smallest island offers access to the Emerald Grotto, a twisting network of underwater caves where the grand finale of the festival takes place.

\subsection{The Grand Finale Begins}

The characters and the rivals proceed with the elders to the southeastern edge of Jigow with what seems like the entire population in a procession behind them. The atmosphere is one of anticipation and excitement.

The characters and their rivals follow the elders as they wade through the shallows, from one island to another. When they reach the smallest island, Elder Colbu Kaz places several \textit{potions of water breathing} in front of the contestants---one for each of them. "Their magic lasts for one hour," he says.

Then, Elder Ushru halts the crowd and raises his arms to address all assembled:

% Image placeholder: {@item Potion of Water Breathing}

\begin{DndReadAloud}
Elder Ushru stands before the mouth of a cave and addresses you and your opponents. "You brave and valiant individuals have triumphed in many challenges this day; now, the final Test of Merit awaits you!" He pulls a palm-sized gold-and-emerald amulet from his robes, holds it aloft, and declares: "This medal's twin has been placed in the deepest cavern of the Emerald Grotto. Be the first to claim the Emerald Eye and return it to this place to be declared Jigow's Champions of Merit! Begin now!"
\end{DndReadAloud}

Have each character make a Dexterity check. If at least half the characters get a total of 13 or higher on the check, the characters enter the grotto first. Otherwise, the rivals enter first.

\subsection{Keeping Track of the Race}

The progress of the characters and their rivals as they move through the grotto is measured in rounds. You'll need to keep track of the number of rounds it takes for the party to traverse the locations in the grotto, as it affects the encounter in area E10.

The number of rounds it takes to traverse a location and clear its obstacles (if any) is detailed in the "Race Progression" section of each location's description.

If the rivals enter the grotto after the characters, they choose whichever path in area E3 the characters didn't take.

\subsection{Emerald Grotto Features}

The Emerald Grotto is a natural cave system. Unless otherwise noted, its features are as follows:

\begin{description}
\item[Ceilings.] Chambers are 15 feet high, and tunnels are 5 feet high.
\item[Light.] Unless otherwise specified, there is no natural light.
\item[Water.] Most of the Emerald Grotto is underwater (see "Underwater Adventuring" in the introduction (p. 0)).
\end{description}

\subsection{Emerald Grotto Locations}

The following locations are keyed to the Emerald Grotto map.



\subsubsection{E1: Grotto Entrance}

\begin{DndReadAloud}
The entrance to the grotto is a narrow cavern that leads to a murky pool of water.
\end{DndReadAloud}

Elder Colbu Kaz and Elder Ushru wait here for the contestants to return.

Though the narrow cavern looks like a dead end, characters who enter the pool of water, or who succeed on a DC 13 Wisdom (Perception) check, notice that the tunnel dips below sea level. The rest of the cavern beyond this location is underwater.

\paragraph{Race Progression}

It takes the characters 1 round to dive into the water and traverse this room.

\subsubsection{E2: Kelp Tangle}

\begin{DndReadAloud}
The light that trickles in from the grotto's entrance is blotted out by the thick patches of kelp that fill this underwater chamber. The tops of the kelp fronds brush the ceiling, but between the stalks you can spot a glimmer of light coming from farther ahead.
\end{DndReadAloud}

This is the first fully submerged chamber of the grotto. The characters must make a DC 13 Strength (Athletics) group check as they try to swim through the mass of kelp (see the Player's Handbook for the rules on group checks, p. 7). The result of this check determines the time it takes them to move through the chamber.

\paragraph{Race Progression}

If the group check is successful, it takes the characters 1 round to traverse the room; otherwise, it takes 2 rounds.

\subsubsection{E3: Cavern Fork}

\begin{DndReadAloud}
Silvery moonlight filters in through a gap in the cavern's stone ceiling. Ahead, the path through the grotto splits into two tunnels: one to the south and one to the east.
\end{DndReadAloud}

If the rival party entered the grotto first, they take the southern path at the fork (leading to areas E7, E8, and E9). Maggie Keeneyes lingers at the fork, waiting for the characters and then encouraging them to take the eastern path (leading to areas E4, E5, and E6). If the characters try to take the southern path instead, Maggie blocks their way and defends herself if attacked. If the characters take the eastern path, Maggie waits until they're gone and then follows the rest of her group.

\paragraph{Race Progression}

If the characters dawdled while choosing which path to take, it takes them 2 rounds to traverse the room; otherwise, it takes them 1 round.

\subsubsection{E4: Ghostgrass Patch}

\begin{DndReadAloud}
The walls, ceiling, and floor of this chamber are covered in bioluminescent algae, which emits a faint bluish glow that dimly lights the wide cavern. Two large, jagged stone pillars, also coated with algae, stand at opposite ends of the chamber. Surrounding the far pillar, in front of the exit from this cavern, is a thick patch of bone-white sea grass.
\end{DndReadAloud}

The pale plants blocking the exit to the cavern are ghostgrass, a carnivorous variety of sea grass. A creature that tries to swim past the ghostgrass must succeed on a DC 12 Dexterity saving throw or take 3 (1d6) necrotic damage and be \textit{restrained}. A \textit{restrained} creature takes the damage again at the start of each of its turns until it escapes.

A creature can use an action to try to free itself or another creature within its reach from the ghostgrass, doing so with a successful DC 12 Strength (Athletics) check. A character freed from the ghostgrass emerges on the other side of the patch at the cavern's exit.

\paragraph{Algae}

A character can harvest some of the algae from the cavern walls with a successful DC 11 Intelligence (Nature) or Wisdom (Survival) check. A harvested patch of algae can be used as a portable light source, shedding dim light in a 10-foot radius. Algae removed from the water dies and ceases to glow after 1 hour.

\paragraph{Race Progression}

It takes at least 2 rounds to traverse this chamber, plus the number of rounds it takes for characters who are caught in the ghostgrass to get free. If any characters stopped to harvest algae, add 1 round to the total.

\subsubsection{E5: Landslide}

\begin{DndReadAloud}
The tunnel opens into a cavernous space. Rays of moonlight shine through a wide crack in the ceiling, which sits ten feet above the water level. One side of the chamber is covered in the crumbled and muddy remains of a landslide.
As you swim into this area, ripples course through the water as more rocky debris begins to slide toward you.
\end{DndReadAloud}

Each character must make a DC 14 Dexterity saving throw, taking 10 (3d6) bludgeoning damage from the landslide on a failed save, or half as much damage on a successful one.

The landslides that plague this chamber have partially blocked the exit. To find the way out, a character must succeed on a DC 10 Intelligence (Investigation) check. If everyone making this check fails, the character who rolled the highest eventually locates the exit, but it takes more time. A character who succeeds on this check by 5 or more also spots a hidden cave at the top of the landslide.

\paragraph{Hidden Cave}

A character who explores this cave discovers a vial containing a \textit{potion of healing} and a soggy but usable \textit{spell scroll} of \textit{thunderwave}.

\paragraph{Race Progression}

It takes at least 2 rounds to traverse the room. If no character succeeded on the check to locate the exit, add 2 rounds to the total. If a character stopped to search the hidden cave, add 1 round to the total.

\subsubsection{E6: Quipper Den}

\begin{DndReadAloud}
A school of tiny fish, their bodies shimmering in the dim light, swirls in formation like an undulating silvery pillar.
\end{DndReadAloud}

A \textbf{swarm of quippers} swims in the middle of this cave. The swarm attacks characters who move within 10 feet of it or deal damage to it. The swarm pursues the characters into area E10 or back into E5 if it isn't killed.

\paragraph{Race Progression}

It takes at least 1 round to traverse this chamber. Keep track of how many rounds of combat the party spends fighting the swarm, and add that number of rounds to the total.

\subsubsection{E7: Octopus's Garden}

\begin{DndReadAloud}
After swimming through a small patch of sea grass, you emerge into a long cavern. The water here doesn't quite reach the ceiling, leaving a small pocket of air near the top. A slight crack in the ceiling allows moonlight to filter through. At the center of the cavern is a pit whose floor is thirty feet deeper than the rest of the cavern---and you catch the faint glint of metal at the bottom.
\end{DndReadAloud}

A \textbf{giant octopus} dwells at the bottom of the pit and attacks anyone who enters it. A character who enters the pit and succeeds on a DC 17 Wisdom (Perception) check notices the octopus at the bottom and is not surprised when it attacks. If reduced to 13 hit points or fewer, it uses Ink Cloud and moves 10 feet away, hoping the characters leave it alone.

\paragraph{Treasure}

Once the octopus is defeated, a character who searches the bottom of the pit finds 57 sp and a pouch containing \textit{dust of dryness} in the form of eight marble-sized pellets that have absorbed hundreds of gallons of water. When a pellet is smashed, the act releases a 15-foot cube of water.

\paragraph{Race Progression}

It takes at least 1 round to traverse this chamber. Keep track of how many rounds the party spends fighting the octopus and add that number to the total. If one or more characters take the time to search the pit, add 1 round to the total.

% Image placeholder: A giant octopus waits to ambush prey in the depths of the Emerald Grotto

\subsubsection{E8: Riptide Tunnel}

\begin{DndReadAloud}
This tunnel rises at a steep incline and contains a strong, opposing current. A heavy stone tablet covered in glyphs and wreathed in sea grass rests on the tunnel floor.
\end{DndReadAloud}

The tunnel is 30 feet long, and the current rushing through it makes swimming northward difficult. A character who wants to swim against the current must make a DC 15 Strength (Athletics) check. On a failed check, the character makes no progress on their turn.

\paragraph{Tablet}

A \textit{detect magic} spell reveals that the tablet radiates an aura of evocation magic. The tablet has AC 15, 10 hit points, and immunity to poison and psychic damage. Destroying the tablet causes the current to subside, allowing creatures to travel through the tunnel without resistance.

\paragraph{Race Progression}

Count the number of rounds it takes for the characters to reach area E9 by swimming against the current. If the characters destroy the tablet, they can swim the tunnel's length in 2 rounds.

\subsubsection{E9: Dead End?}

\begin{DndReadAloud}
The tunnel opens into a smooth-walled cavern that appears to be a dead end.
\end{DndReadAloud}

An exit tunnel is hidden behind some rocks on the east side of the chamber. A character can locate the exit in one of two ways:

\begin{itemize}
\item A character can spend 1 round search the east wall, then succeed on a DC 11 Wisdom (Perception) check.
\item A character can spend 1 round feeling the currents, finding the exit with a successful DC 11 Wisdom (Survival) check.
\end{itemize}

If no character succeeds on either check, the character who rolled highest eventually locates the exit, but it takes more time. A character who succeeds on either check by 5 or more also notices a cave concealed behind the kelp on the south side of the chamber.

\paragraph{Concealed Cave}

A character who searches this cave finds three \textit{potions of healing}, likely stashed here by a hopeful contestant planning to gain an advantage in this race.

\paragraph{Race Progression}

It takes at least 1 round to traverse the room and 1 round to find the eastern exit (or 2 rounds if no character succeeded on a check to locate the exit). If a character stopped to search the concealed cave, add 1 round to the total.

\subsubsection{E10: Moonshark Lair}

\begin{DndReadAloud}
Silvery light dances along the walls of this underwater cave. The water here is a little warmer, and a gentle current pulls you toward a crevice in the south wall that emits shimmering golden light. Prowling around the cave's pillars is a massive shark. A gold amulet hangs from a thick rope tied around the shark's body, and a silver spear protrudes from its side.
\end{DndReadAloud}

The amulet is the Emerald Eye---the goal of this race. The silver spear lodged in the shark's hide has been enchanted by the divine power of Sehanine the Moon Weaver. The shark uses the \textbf{giant shark} stat block. A character within 5 feet of the shark can use an action to try to dislodge the spear, doing so with a successful DC 13 Strength (Athletics) check. While the spear is lodged in the \textbf{shark}, apply these changes to the shark's stat block:

\begin{itemize}
\item The shark glows with silvery illumination, shedding bright light in a 10-foot radius and dim light for an additional 10 feet.
\item If the shark uses its action to make a Bite attack but misses, it can use a bonus action to swim up to 25 feet. This movement doesn't provoke opportunity attack.
\end{itemize}

The shark immediately attacks any creature that enters its chamber. Depending on how quickly the characters moved through the grotto, the encounter could go one of three ways:

\begin{description}
\item[Fast Pace (10 or Fewer Rounds).] The characters outpaced the rivals. The rivals don't appear until after the encounter with the shark is resolved.
\item[Moderate Pace (Between 11 and 15 Rounds).] The characters beat the rivals to the chamber, but not by much. The rivals enter the chamber on the third round of combat and roll initiative. Unless the rivals are hostile toward the characters, they aid the characters in the fight against the shark; otherwise, they focus on acquiring the amulet or try to sabotage the characters (though these attempts are never intentionally lethal).
\item[Slow Pace (16 or More Rounds).] The rivals get to the chamber ahead of the characters. When the characters arrive, the rivals have reduced the shark to 75 hit points. The characters roll initiative when they enter the chamber.
\end{description}

\paragraph{Emerald Eye}

A druid of Jigow cast \textit{animal friendship} on the shark earlier today and tied the Emerald Eye around its body, then made a speedy getaway. A character can cut the amulet free by attacking the rope around the shark instead of the shark itself. The rope has AC 17 and 5 hit points. It can be removed as an action if the shark is killed or \textit{incapacitated}. The amulet, which is made of gold and emeralds, is worth 500 gp---though no one in Jigow will buy it, since it must be returned the town elders once the contest is completed.

\paragraph{Moonshark's Defeat}

When the shark is reduced to 0 hit points, it thrashes about in its death throes and crashes into the stone pillar in the south part of the room, which tumbles into the south wall and cracks it open. Read:

\begin{DndReadAloud}
The dying shark slams into the stone pillar in the south end of the cavern. The pillar cracks under the force, teeters, then crashes down against the south wall. The wall fractures and collapses, revealing a passage awash with golden light.
\end{DndReadAloud}

The newly opened passage (area E11) leads to a hidden chamber (area E12).

\subsubsection{E11: Moon Weaver's Gateway}

\begin{DndReadAloud}
The cramped passage bends upward until the tunnel becomes vertical. Golden light radiates from the upper end of the shaft.
\end{DndReadAloud}

Characters who swim through this passage break the surface of the water and emerge in a glowing, air-filled chamber (area E12).

\subsubsection{E12: Prayer Site of Sehanine}

\begin{DndReadAloud}
The watery environment gives way to a cavern dotted with trees and pools of water. The walls are covered in vines of green ivy interspersed with colorful flowers. Part of the cavern is open to the sky.
In the middle of the cavern is a sphere of pale light surrounding a crystal pedestal, upon which rests a golden pendant attached to a fine golden chain.
\end{DndReadAloud}

The object on the crystal pedestal glitters like the sunrise. This is the \textit{Jewel of Three Prayers}, a Vestige of Divergence that will drive the course of the adventure from now on. The \textit{Jewel of Three Prayers} is currently in its Dormant State (see appendix B (p. 9) for the description of this magic item).

If the rivals are present, they are hesitant to touch the jewel, giving the characters the first opportunity to do so. If the characters are also hesitant, Ayo Jabe steels herself and strides forward to claim it (see "Disputing Destiny" below). Otherwise, as soon as one of the characters touches the jewel, the entire party experiences the following vision:

\begin{DndReadAloud}
The light around the pedestal fades as a spectral figure in the form of a male human rises from the amulet. He is dressed in leather armor and a tattered red cape and wears a shield. His face is framed by curly brown hair and bears a melancholic expression. He pleads, "I am imprisoned. Please help me."
Suddenly, the ground vanishes beneath your feet, and you fall, tumbling through a vortex of golden light. You fall deeper, deeper, deeper, then suddenly stop. The golden light subsides, and you find yourself suspended in a pitch-black expanse. You feel water buoying you. A vermilion light appears in the distance, illuminating the melancholy warrior, who is shackled to the ground by disgusting strands of a fleshy, crimson substance.
He raises his gaze to the heavens and, sobbing, chokes out, "Moon Weaver, I beg of you. Guide those with the power to save me to the site where first I prayed to you." He looks around, and his eyes settle on you. "Oh, gods, there you are! My name is Alyxian. I am lost in darkness. Long ago, I prayed to the Change Bringer in the heart of a temple of evil. I beg you, take my jewel and..."
You feel consciousness leaving you, as if the pressure of the water were crushing the life out of you. Alyxian's voice is the last thing you hear before you pass out. "Save me. Please."
\end{DndReadAloud}

If the rivals are present, they can hear the distorted, disembodied voice of Alyxian as if he were speaking underwater. When the vision ends, the characters fall \textit{unconscious} for 1 minute.

What happens next depends on the characters' relationship with the rivals:

\begin{description}
\item[Friendly or Indifferent Rivals.] The characters awaken to find the rivals attempting to rouse them. Though they didn't see the vision, the rivals sensed a surge of energy erupting from the jewel, then found the characters \textit{unconscious} in the cavern. If the characters retrieved the Emerald Eye, the rivals concede the competition to them. Otherwise, the rivals proudly hold up the amulet, claiming it as theirs, but congratulate the characters on a race well run.
\item[Hostile Rivals.] The characters awaken to discover that the rivals have absconded with both the Emerald Eye and the \textit{Jewel of Three Prayers}.
\end{description}

\paragraph{Disputing Destiny}

If Ayo or another one of the rivals touched the \textit{Jewel of Three Prayers}, they are the recipients of the vision instead. The rivals fall \textit{unconscious} for 1 minute, and it is up to the characters to decide what to do with them. In this scenario, which is examined further in chapter 2 (p. 2), the rivals are the "heroes" of the story, and the characters must decide if they want to aid their rivals or take back the jewel and claim their own heroic destiny.

% Image placeholder: A spectral figure not seen in ages sets the adventurers on a quest that could change the fate of the world
\section{Next Steps}

Whether or not the characters retrieved the Emerald Eye, they can backtrack through the Emerald Grotto and avoid any potential combat. They can also take the time to investigate places that they didn't check out on the way in.

\subsection{Characters Triumphant}

If the characters present the Emerald Eye to Elder Ushru, he declares them the winners of this year's Festival of Merit. The ecstatic crowd surrounds the characters and offers congratulations.

As the folks of Jigow begin to settle down and turn in for the night, Elder Ushru asks the characters to hand over the Emerald Eye, in exchange for the 100 gp reward and an invitation to stay the night at the Unbroken Tusk inn free of charge.

\subsection{Rivals Triumphant}

If the rivals present the Emerald Eye to Elder Ushru, he declares them the champions of this year's Festival of Merit, and the crowd acknowledges them with adulation.

While the townsfolk crowd around the rivals, Elder Ushru approaches the characters. Though they lost the race, the elder assures them that he chose their group to compete for a reason---that true merit is something proven not in a single day but rather over the course of an individual's life, one small act at a time. As a consolation prize, he offers them rooms at the Unbroken Tusk inn for the night, free of charge.

\subsection{What of the Jewel?}

Unless the characters take steps to conceal the \textit{Jewel of Three Prayers}, Elder Ushru takes notice of the item. Even if they do keep it from his view, Ushru senses a change in the characters' demeanor---something happened to them in the grotto, something no one in Jigow could have expected. This suspicion prompts him to speak to the characters in the morning---leading into the events of chapter 2 (p. 2).

\chapter{The Leave-Taking}\label{ch:the-leave-taking-3-3}

In this chapter, the characters learn more about the events that occurred in the Emerald Grotto and the mythic significance of the \textit{Jewel of Three Prayers}. Following this discovery, the characters depart Jigow and travel across Xhorhas toward Bazzoxan, a bleak military outpost guarding a fortress of evil.
Most of this chapter deals with the characters' journey across the wastes and the challenges and friendly strangers they encounter along the way. Whether the rivals accompany the characters or strike out on their own depends on the relationship between the two parties and the rivals' curiosity about the \textit{Jewel of Three Prayers}.
% Content: Unknown (dict)
\section{Running This Chapter}

The journey can begin in one of three ways, depending on how chapter 1 ended:

\begin{description}
\item[Heroic Quest.] In the Emerald Grotto, the characters obtained the \textit{Jewel of Three Prayers} and experienced the vision from the Apotheon. In this scenario, they are approached by Elder Ushru, who senses that the \textit{Jewel of Three Prayers} is a powerful magic item and suggests the characters undertake a journey to learn more about it.
\item[Ours by Right.] If the rivals experienced the vision and claimed the jewel, they take the role of the heroes of the story, with the characters as allies or as interlopers who want the jewel for themselves. If the characters meet with Elder Ushru, they learn that the rivals have embarked on a quest of monumental significance---perhaps compelling the characters to pursue them and claim the destiny that they believe is rightfully theirs.
\item[Warring Destinies.] If the characters got the jewel but grievously insulted the rivals (or even killed one or more of them), the rivals try to steal the jewel in the middle of the night and begin the journey to Bazzoxan themselves.
\end{description}

In any event, it takes the characters several days to reach their destination. Use the Xhorhas Encounters table (p. 2) later in the chapter to spice up the journey.

The trek is interrupted by a chance for the characters to rest at a caravan stop along the Emerald Loop---where the rivals might also be camping for the night. This midway encounter gives the characters a chance to make friends with the rivals or steal the jewel from them.
\section{After the Festival}

The events of this chapter begin at dawn on the day after the Festival of Merit. If the characters claimed the \textit{Jewel of Three Prayers} from the Emerald Grotto, start with "Heroic Quest." If the rivals claimed the jewel instead, start with "Ours by Right." If the characters acquired the jewel but made enemies of the rivals, start with "Warring Destinies."

\subsection{Heroic Quest}

Read the following to start this episode:

\begin{DndReadAloud}
Daybreak comes with a knock on the door of your room. The voice of Elder Ushru comes through the door. "Travelers? I wish to tell you something of grave import. Will you have a morning meal with me?" He pauses, and then sheepishly adds, "I do have the right room, don't I?"
\end{DndReadAloud}

Ushru invites the characters for breakfast at the Unbroken Tusk inn. At first he is jovial and conversational with the characters, but then his tone becomes deadly serious:

\begin{DndReadAloud}
"A vision came to me last night. A golden amulet rose to the surface of a pool of blood, and countless hands lunged upward out of the blood to grasp it. In turn, all the hands fell back into the pool until only one remained. The one whose hand claimed the amulet was a handsome man with curly hair and a somber expression. He turned to me and said, 'They must find me. The ones from the grotto.' Tell me, did anything happen to you in that place that would help me understand this vision?"
\end{DndReadAloud}

If the characters offer information, Ushru doesn't recognize the name Alyxian---no living creature remembers the name of the Apotheon. Nevertheless, if the characters tell Ushru the truth about what happened to them in the Emerald Grotto, he gravely encourages them to travel eastward to Bazzoxan, describing it as follows:

\begin{DndReadAloud}
"Bazzoxan is a grim town under the control of the Aurora Watch, not a pleasant place to visit. But if what you tell me is true, what you have found is a Vestige of Divergence, an enchanted relic from the time of the Calamity known as the \textit{Jewel of Three Prayers}. People do not stumble upon Vestiges without a reason. And there is no place in Xhorhas where the memory of the Calamity lingers more strongly than in Bazzoxan."
\end{DndReadAloud}

Elder Ushru urges the characters to be truthful because they might have embarked on a course of action that will change their lives. If they refuse to be forthcoming with information, he sighs, shrugs, and bids them good day. He says they're welcome to stay in Jigow as long as they want, if that's what they prefer to do.

\subsubsection{Friendly Rivalry}

If the characters show little interest in learning more about the jewel, perhaps their rivals don't share that attitude. If the rivals were present at the prayer site in the Emerald Grotto, they didn't experience the vision of the Apotheon; instead, they heard only a muffled voice, as if the speaker were underwater. Nevertheless, what they heard was enough to make them curious.

If the characters are on friendly terms with the rivals, the rivals meet up with them soon after the characters' breakfast with Elder Ushru.

\textbf{Ayo Jabe} doesn't mince words; she wants to know what they found in the grotto. If she gets the sense that the characters have stumbled onto something big, her eyes grow wide. She decides that she and her group want a piece of the action and proposes that they travel with the characters, saying that there's safety in numbers. A character who makes a successful DC 13 Wisdom (Insight) check realizes that she isn't hiding anything and wants nothing more than to be a part of a grand adventure.

The other rivals have their own opinions:

\begin{itemize}
\item \textbf{Dermot} is genuinely concerned for the characters' well-being. He mentions that he heard what sounded like a voice in the grotto and wants to know if the characters heard it too.
\item \textbf{Galsariad} is curious about what the characters learned in their vision and is eager for the characters to share any magical secrets they might have discovered.
\item \textbf{Irvan} has no interest in what happened in the grotto, but he is interested in learning what the jewel can do.
\item \textbf{Maggie} agrees with anyone who expresses concern for the characters but doesn't say much on her own. She has her suspicions that more happened than what they have related during the event in the grotto, but she won't push them if they don't want to reveal the full truth.
\end{itemize}

\subsection{Ours by Right}

If the rivals acquired the \textit{Jewel of Three Prayers} and experienced the vision of Alyxian, they are the ones approached by Elder Ushru in the morning, and they are the ones who are encouraged to travel to Bazzoxan. In this case, read the following to the players:

\begin{DndReadAloud}
You're eating breakfast at the Unbroken Tusk while locals chat around you. Through the cacophony, one voice catches your attention.
"Rumor has it they're going to Rosohna to sell it. Elder Ushru met with them and everything, kept whispering while pointing at a huge, shiny amulet on the table. He was talking about 'destiny' and other heroic-like words. I think they were the group who won the grand finale yesterday. The amulet looked plenty magical, but even if it isn't, it'd be worth a fortune. Yeah, they're traveling down the Emerald Loop by now."
\end{DndReadAloud}

Gossip about the rivals' meeting with Elder Ushru has spread quickly throughout Jigow. Everyone is talking about the expensive-looking, possibly magical amulet they found in the Emerald Grotto, and the characters are bound to catch word of it. People are saying that the jewel would probably sell for over 1,000 gold pieces---maybe twice that if it's magical, and twice that again if the sellers were to make the long, oversea journey to a trade hub like the desert metropolis of Ank'Harel.

Nothing is forcing the characters to chase down the rivals, but the thought of losing out on such a prize is enough to motivate most adventurers. If they aren't interested in chasing down the rivals, see the "Refusing the Call" sidebar for advice on how to continue this adventure.

\begin{DndSidebar}{Refusing the Call}
If the characters don't want to follow the plot or travel to Bazzoxan, you will need to create side encounters in Xhorhas that will guide them in that direction. The characters can find jobs in Jigow as caravan guards, mercenaries, or bounty hunters; you can use the Xhorhas Encounters table (p. 2) later in this chapter to determine how they are challenged.
\end{DndSidebar}

\subsection{Warring Destinies}

If the characters won the Festival of Merit but made enemies of the rival party, either by insulting them or by injuring or killing one or more of them, the rivals try to get revenge by stealing the \textit{Jewel of Three Prayers} from the characters while they sleep.

The rivals' plan is to gather outside the inn where the characters are staying. One rival then sneaks into the characters' room at the inn and searches for the jewel. If the thief doesn't return after an hour, the rivals travel to the Emerald Loop Caravan Stop (described later in this chapter) and wait up to seven days for their missing companion.

\subsubsection{Who's the Thief?}

The likely thief is Irvan, who sees the jewel not as a mystical item but rather a treasure. It could also be the shadowy Galsariad if the magical nature of the jewel is made explicit. If Ayo was pushed to the brink by the murder of one of her comrades, she could be the thief.

The thief wears a thick black cloak with a hood and mask. A character who sees the thief can use an action to try to recognize them through their disguise, doing so with a successful DC 15 Wisdom (Perception) check.

The attempted theft has three possible outcomes:

\begin{description}
\item[Caught Sleeping.] If none of the characters has a passive Wisdom (Perception) score of 15 or higher, the thief escapes with the jewel. The rival party heads to Rosohna. Along the way, they receive a vision from Alyxian that directs them to Bazzoxan.
\item[Slow to Act.] A character who has a passive Wisdom (Perception) score of 15 to 18 notices the thief after they've found the jewel and as they're leaving out the window. The thief notices the character on the way out and runs toward their rendezvous point (area J10). See "Pursuing the Thief" below if the characters give chase.
\item[Quick to Act.] A character who has a passive Wisdom (Perception) score of 19 or higher notices the thief while they're searching the room, but before they've found the jewel. This character can sneak up on the thief by making a successful DC 15 Dexterity (Stealth) check, catching the thief by surprise. Otherwise, the thief notices that they've been spotted and leaps out the nearest window. The thief then runs to rejoin their friends at the edge of town. See "Pursuing the Thief" below if the characters give chase.
\end{description}

\subsubsection{Pursuing the Thief}

A character who notices the thief can try to capture them. Doing so requires a successful DC 15 Dexterity check to catch up to the thief followed by a successful grapple check (see the Player's Handbook). If either check fails, the thief escapes.

If the character doesn't pursue the thief immediately and instead spends time rousing their allies, grabbing a weapon, or donning armor, the thief escapes.

\subsubsection{Questioning the Thief}

If the characters catch the thief, the rival refuses to give away their plan as long as their attitude toward the party remains hostile (see the rules for social interaction, p. 8 in the Dungeon Master's Guide). But if a character succeeds on a DC 18 Charisma (Persuasion) check, the rival shares the plan willingly. The rival also shares the plan under duress if a character threatens them with harm and succeeds on a DC 15 Charisma (Intimidation) check. You can grant advantage or impose disadvantage on either check if you think the thief would be especially responsive or resistant to persuasion or intimidation.

A rival captured by the party tries to escape at the earliest opportunity, perhaps when most or all of the characters are asleep or distracted. If the rival is bound with rope, the rival can escape their bonds with a successful DC 15 Dexterity (Sleight of Hand) check. If manacles are used instead, the DC is 20. If the check fails, the rival can try again after 1 hour.
\section{Road to Bazzoxan}

To reach Bazzoxan, the characters must follow the Emerald Loop south before turning east onto the Hallowed Path. These dirt roads are shown on the accompanying map of Xhorhas. Although the roads are well patrolled by the Kryn Dynasty, the wastes are still dangerous: wandering scavengers, fearsome megafauna, and plenty of other threats await foolish travelers who venture too far from the road.



\subsection{Travel Pace}

The first leg of this journey takes the characters through the wastes of Xhorhas, a stark landscape roamed by mastodons and packs of moorbounders. In the second leg, the characters enter the valley known as the Barbed Fields, where spines of stone ranging from 10 to 30 feet tall linger as eerie reminders of the Calamity. Between the two halves of the journey is a caravan stop where the characters can rest, buy items, and interact with friendly nonplayer characters.

How long the journey takes and the number of random encounters that occur depend on whether the characters are moving at a normal, fast, or slow pace (see the rules for travel pace, p. 8 in the Player's Handbook):

\begin{description}
\item[Normal Pace.] The journey takes 12 days, and the characters have six random encounters: four encounters prior to reaching the Emerald Loop Caravan Stop (where they arrive at the end of the eighth day) followed by two more encounters.
\item[Fast Pace.] The journey takes 9 days, and the characters have four random encounters: three encounters prior to reaching the Emerald Loop Caravan Stop (where they arrive at the end of the sixth day) followed by one more encounter.
\item[Slow Pace.] The journey takes 15 days, and the characters have eight random encounters: five encounters prior to reaching the Emerald Loop Caravan Stop (where they arrive at the end of the tenth day) followed by three more encounters.
\end{description}

\subsection{Random Encounters}

The Xhorhas Encounters table presents events that can take place during the trek to Bazzoxan. Unless specified, you decide what time of day each encounter occurs. Encounters marked with an asterisk (*) are unique and can occur only once, while the others can happen multiple times. If you roll an encounter that can't be repeated, roll again for a new result, or choose one you like.

In addition to random encounters, two encounters occur at specific points, as described later in the chapter:

\begin{description}
\item[Rivals' Reunion.] This encounter occurs if the rivals left Jigow ahead of the characters and the characters travel at a fast pace to catch up with them on the way to Bazzoxan.
\item[Emerald Loop Caravan Stop.] This encounter occurs at the point where the Emerald Loop meets the Hallowed Path---unless the characters choose not to rest at the caravan stop.
\end{description}

% Table: Xhorhas Encounters
\begin{DndTable}[header={Xhorhas Encounters}]{cX}
d8 & Encounter \\
1 & A Lucky Break* \\
2 & Aurora Watch Patrol \\
3 & Crashed Wagon* \\
4 & Demonic Carrion* \\
5 & Feast for the Eyes \\
6 & Ill Omen \\
7 & Moorbounder Mayhem* \\
8 & Roadside Raiders \\
\end{DndTable}

\subsubsection{A Lucky Break}

Just off the road ahead, the characters spot the corpse of a \textbf{gloomstalker} (see appendix A (p. 8)). Even in death, its corpse is covered in roiling, wispy shadows. The grass around the creature has been trampled flat.

A character who examines the gloomstalker corpse, which is riddled with broken arrow shafts, can make a DC 16 Wisdom (Medicine) or Intelligence (Investigation) check. On a success, they determine that the creature also has a serious burn along its underside, suggesting it was likely killed by archers and some sort of magic that dealt fire or radiant damage.

\paragraph{Irvan's Ring}

If the rivals are traveling ahead of the characters, they were involved in this battle. (Otherwise, the gloomstalker was killed by an Aurora Watch patrol.) A character who succeeds on a DC 18 Wisdom (Survival) check finds a line of footprints leading through the trampled battlefield and ending at a withered oak tree. Beneath the tree, in plain view, is a silver ring with a small sapphire. This ring belongs to Irvan Wastewalker, who lost it crawling away from the battle after he was wounded by the gloomstalker. The ring is worth 250 gp, and Irvan becomes friendly toward any character who returns it to him.

\subsubsection{Aurora Watch Patrol}

The Aurora Watch of the Kryn Dynasty patrols the roads in Xhorhas, protecting travelers from both the monstrous denizens of the wastes and unscrupulous individuals who might lurk along the road. Each time this encounter occurs, the characters meet a different patrol. Choose one of the following options:

\begin{description}
\item[Patrol at Rest.] The characters come across six lawful neutral Aurora Watch \textbf{veterans} (a mix of lawful good drow, humans, and orcs) who are resting at a campsite just off the road. Clearly in good spirits, they invite the characters to join them. Traveling with this patrol is a lawful good, drow \textbf{scout} who also serves as a cook. If the characters take a long rest with the patrol, the food they were given the night before grants them 1d10 temporary hit points, p. 9 when they awaken the next day.
\item[Patrol in Battle.] The characters see six lawful neutral Aurora Watch \textbf{veterans} (a mix of lawful good drow, humans, and orcs) fighting a \textbf{hezrou}. If the characters help the Aurora Watch slay the demon, the soldiers thank them, give them a total of 50 gp as a reward, and promise to commend them to Taskhand \textbf{Verin Thelyss} when the patrol returns to Bazzoxan. If the characters do nothing to help, the soldiers slay the demon, but only after one soldier is killed and two others are reduced to 0 hit points (one stable, the other dying).
\item[Lost Patrol.] On the road in front of the characters are the bodies of six Aurora Watch soldiers near the corpse of a hezrou that is slowly dissolving into black ichor. Two \textbf{shadow demons} and two \textbf{quasits} lurk beneath the soldiers' corpses, waiting for more prey to arrive. They fight to the death.
\end{description}

\subsubsection{Crashed Wagon}

The characters spot an overturned wagon in a ditch on one side of the road. The ox that pulled the wagon is nowhere to be found (it tore free of its harness and wandered away). Lying beneath the wagon are the bodies of three merchants---two drow and an orc---who died in the crash. Around the orc's broken neck is a small key that unlocks an iron strongbox in the wagon (see "Treasure" below).

Characters who investigate the interior of the wagon disturb three \textit{invisible} \textbf{will-o'-wisps}, which illuminate and attack. If any of the wisps are killed, the others decide the characters are too much of a threat and flee on their next turn. If all three wisps are defeated, the spirits of the slain travelers grant one character of your choice a \textit{charm of heroism} (see the rules for supernatural gifts, p. 7 in the Dungeon Master's Guide).

\paragraph{Treasure}

After the will-o'-wisps are dealt with, a character who searches the wagon can make a DC 15 Intelligence (Investigation) check. On a success, they find a locked iron box containing 50 gp and a \textit{potion of giant strength (hill)}. A character who lacks the proper key can use an action and \textit{thieves' tools} to try to pick the lock, doing so with a successful DC 15 Dexterity check.

% Image placeholder: Rival adventurers camp near the bones of a mastodon in the wastes of Xhorhas

\subsubsection{Demonic Carrion}

The characters encounter the corpse of an udaak---one of the enormous, ape-like demons that have prowled the wastes of Xhorhas since the Calamity. The demonic corpse, which is punctured with countless spears and axes, is slowly melting into a pool of black ichor. Surrounding the body are the corpses of two dozen humans wearing hide armor, as well as the bodies of the warriors' horses. What looks like a carrion bird circles the battlefield at a height of 300 feet.

The carrion bird is a \textbf{vrock} that has been feeding on the mortal flesh strewn about the battlefield. It swoops down to attack characters who disturb any of the corpses. In addition, three \textbf{quasits} hide in the udaak's fur and attack non-demons that come within 10 feet of the udaak's melting corpse.

A character who investigates the tracks around this battlefield can make a DC 14 Wisdom (Survival) check. On a success, the character sees the tracks of a horse that apparently escaped the carnage. The tracks lead in the direction of the Emerald Loop Caravan Stop.

\paragraph{Treasure}

A character who investigates the carnage for at least 10 minutes can make a DC 13 Intelligence (Investigation) check. On a success, they find a \textit{+1 dagger} embedded in the hide of the udaak. The dagger is inscribed with the name of its owner, Kierchaly Wastewalker. The characters might later learn that this warrior is among the dead, but his daughter and son survived, and they are now at the caravan stop.

\subsubsection{Feast for the Eyes}

The characters see a \textbf{gloomstalker} (see appendix A (p. 8)) antagonizing 1d3 \textbf{mammoths} that are moving slowly across the wastes, minding their own business. The creatures are 3d6 × 100 feet away from the characters when the encounter begins.

The gloomstalker shrieks as it descends upon a mammoth with claws bared, but its attacks have little effect. Frustrated, it ascends into the sky and spots the party. The gloomstalker lets out another shriek as it flies toward the characters, eager to feast on their flesh. The characters have at least a round or two before the gloomstalker gets close enough to harm them. Hungry and hostile, the gloomstalker fights to the death. Assume that the creature has recharged Shriek by the time it reaches the characters.

Each time the gloomstalker uses Shriek during the battle against the characters, 1d6 \textbf{skeletons} of dead human soldiers burst up out of the ground within 30 feet of the characters and attack them, acting on their own initiative count. Skeletons that aren't reduced to 0 hit points collapse into inanimate piles of bones after 10 minutes or when the gloomstalker is slain.

\subsubsection{Ill Omen}

While the party is camping at night, one of the characters sees Exandria's smaller moon flare with bright red light. The moon, which has appeared as a slim crescent throughout the night, suddenly becomes full. When this happens, each character becomes cursed for the next 24 hours. While cursed in this way, a character has disadvantage on ability checks and saving throws. The effect can be ended on a character by any magic that removes a curse.

The characters have another encounter immediately after they begin traveling the next day. Roll again on the Xhorhas Encounters table (p. 2), or choose an encounter you like.

\subsubsection{Moorbounder Mayhem}

The characters hear a wild yell. In the distance, they see a lawful neutral, tiefling \textbf{scout} named Justice on the back of a \textbf{moorbounder} (see appendix A (p. 8)) and hanging on for dear life as the creature snarls and bucks about. Two \textbf{bristled moorbounders} (see appendix A (p. 8)) are circling the imperiled tiefling. Justice can hang on for another minute at most, and the moment he touches the ground, the bristled moorbounders intend to tear him limb from limb.

Justice is a trained moorbounder rider, and the beast he is clinging to is Rice Pudding, his beloved steed. Rice Pudding was terrified by the other moorbounders that ambushed them along the road. If the characters scare off the attacking moorbounders, Justice and Rice Pudding will be saved. A character can use an action to try to scare off a moorbounder, doing so with a successful DC 21 Charisma (Intimidation) check.

Justice is a scout for a caravan that is heading toward the Emerald Loop Caravan Stop. He elects not to accompany the characters, but he does promise to give them a reward once they reach the caravan stop. (If the characters have this encounter after they pass the caravan stop, Justice's group is traveling west, in the opposite direction from the characters' route, and Justice instead rewards them with 5 gp per character.)

\subsubsection{Roadside Raiders}

On a rare sunny day in the wastes, characters who have a passive Wisdom (Perception) score of 15 or higher see a cloud of dust being kicked up about half a mile down the road---and it continues to head toward them. A few minutes later, the characters are surrounded by human bandits mounted on bedraggled horses. These bandits call themselves the Road Raiders. Their leader, a grizzled man named Postraeck, is a neutral evil \textbf{bandit captain} who rides an imposing black \textbf{warhorse}. Postraeck is better known as Six-Knives because of the six daggers he carries on his person at all times, each of which has a name (Boneshaver, Gleam, Grudge, Jabby, Pierce, and Tickles). His fellow riders are ten \textbf{bandits} (lawful evil humans) mounted on \textbf{riding horses}.

Six-Knives claims that he's a Kryn tax collector and that he needs each of the characters to cough up 25 gp for the "road 'n' infer-structure tax." No such tax exists, and the characters don't need to make any kind of check to glean that these hoodlums are not tax collectors. If the characters pay, Six-Knives tips his hat with a chuckle, and the Road Raiders gallop away.

If the characters refuse to pay, Six-Knives clicks his tongue and rears his horse. "Let's loosen their purse-strings, y'all!" he roars, ordering the bandits to attack. All of them, including Six-Knives, are cowards, and any bandit reduced to fewer than half their hit points tries to flee in their next turn.

Future occurrences of this encounter are with Six-Knives at the head of the group---or, if he's dead, one of his protégés, now a \textbf{bandit captain}---hoping to plunder more coin from the characters' pockets or exacting revenge for their earlier losses.
\section{Reunion with the Rivals}

% Image placeholder: The spires of the Barbed Fields reach as high as thirty feet into the sky

This encounter takes place only if the characters are trailing behind the rivals and traveling at a fast pace to catch up with them. The time of day when this encounter occurs is up to you. In the morning or around midday, the rivals are traveling. In the evening, the rivals are making camp. In the middle of the night, \textbf{Galsariad Ardyth (tier 1)} is on watch; see appendix A (p. 8) for his stat block.

This is an open-ended encounter that unfolds in a certain way depending on the characters' objective:

\begin{description}
\item[Jewel Theft.] If the characters intend to take the \textit{Jewel of Three Prayers} from the rivals, this will likely be a combat encounter or an attempted infiltration.
\item[No Quarrel.] If the characters aren't looking to quarrel, this encounter is a roleplaying opportunity during which the characters can interact with the rivals.
\end{description}

\subsection{Friendly or Indifferent Rivals}

If the rivals are friendly or indifferent toward the characters, they welcome the characters' arrival. If the party catches up to the rivals in the daytime, the rivals agree to travel together for a while. If this encounter occurs at night, the rivals invite the characters into their camp.

To make roleplaying the rivals easier, have one or two of them take center stage during this encounter. Let them ask the questions, do most of the talking, and try to get what they want from the characters. The other rivals might be cooking food or engaged in some other minor activity in the background.

If you think a rival has a chance to achieve one of their goals (see the Rivals' Goals table in the introduction (p. 0)), this is an opportunity for them to try to achieve that goal. For example, Maggie, who wants to spar with a tactical genius, might invite a character to take a seat at her Dragonchess board and have a game. While they play, she makes conversation about battle tactics---how she prefers to gang up on foes and overwhelm them, rather than engaging in a fair fight, since, in her words, "People who fight fair don't win fights." Determine the winner of the game by having each participant make an Intelligence check at the end of the conversation, with victory going to whoever rolls the highest.

\subsection{Hostile Rivals}

If the rivals are hostile, Ayo Jabe urges the characters to keep away. She doesn't want to fight but makes it clear that her group is prepared to defend themselves.

The characters can't sneak up on the rivals on the open road. If the rivals are camping, any character who tries to sneak into the camp must succeed on a DC 16 Dexterity (Stealth) check to avoid being detected by Galsariad. If Galsariad detects the intruders but is friendly or indifferent toward one or more of them, he is willing to quietly talk things out. If not, combat breaks out as he shouts to rouse his allies.

If the characters insist on fighting or are caught trying to steal from the rivals, the rivals attack.
\section{Emerald Loop Caravan Stop}

At the intersection of the Emerald Loop and the Hallowed Path stands a forest known as the Wandering Oak's Grove. A clearing inside the forest is used by travelers as a rest stop. Read or paraphrase the following as the characters arrive:

\begin{DndReadAloud}
A large stand of trees rises out of the barren wasteland. The leaves of dozens of towering oaks rustle in the breeze, and the scent of a bonfire wafts from a trail of smoke rising from somewhere ahead. The path through the trees is deeply rutted by the work of countless wagon wheels. A wooden sign posted at the head of the path bears the decorated shepherd's crook that is the holy symbol of Melora the Wild Mother, and the words "Emerald Loop Caravan Stop."
\end{DndReadAloud}

A character can make a DC 13 Intelligence (History) check to see if they know anything about the caravan stop. On a success, they recall that this is a safe and respected way point that has existed for over 150 years. It's said that some of the best hunters in the wastes stop here to sell their catches---and that the best mastodon kor'rundl (see area L5) in the Kryn Dynasty is sold here.

Characters who succeed on a DC 13 Intelligence (Religion) check know that a treant named Wandering Oak first seeded this forest, and that she later entrusted it to a pair of dryads of the Wild Mother who call themselves the Acorn Sisters. They permit some logging but harshly punish anyone who takes more than they allow.

\subsection{Caravan Stop Locations}

The following locations are keyed to the Emerald Loop Caravan Stop map.



\subsubsection{L1: Entrance to the Clearing}

Read or paraphrase the following as the characters enter the caravan stop:

\begin{DndReadAloud}
The forest pathway opens into a large clearing, nearly a hundred feet across. Music, laughter, and the savory aromas of fresh-cooked food and woodsmoke drift through the air. On the west side of the clearing stands a full-grown horizonback tortoise. On its back is the homestead of a family of goblins, who are making repairs to the structure. On the other side of the clearing, seven covered wagons encircle a bonfire, and a rope fence has been set up north of this encampment to contain the wagons' oxen.
\end{DndReadAloud}

\subsubsection{L2: Big Yuyo}

Big Yuyo is an adult \textbf{horizonback tortoise} (see appendix A (p. 8))---known as a "kinespaji" in Goblin---that is the home of the Kruzkrenner goblin family. The tortoise is named after a Xhorhasian vegetable called a yuyo---an ochre cucumber that grows in Rosohna's underground gardens.

The platforms and structures on Big Yuyo's back are made of wood, thatch, canvas, and rope. They sometimes need maintenance, and the family is busy repairing their home. They're currently on a trip from Rosohna to Asarius to meet up with the bugbear side of their family.

\paragraph{Goblin Family}

The Kruzkrenner family is made up of the following people. Some of them are working on repairs and spending time together on Big Yuyo's back, but some might be found elsewhere at the caravan stop:

\textbf{Papa Drazagorr} (lawful neutral, goblin \textbf{veteran}) served in the Aurora Watch when he was younger and still has an appreciation for military discipline. These days, he doesn't wear armor or carry weapons, except for a dagger.

\textbf{Pops Kelbadurn} (neutral good, goblin \textbf{guard}) served in the Asarius Town Guard and still has his old chain mail, spear, and shield stored in the command post (area L2f). He's easygoing and artistic.

\textbf{Auntie Jaller} (chaotic neutral, hobgoblin \textbf{druid}) is old and elegant-looking, thanks to her long silver hair and deep green robes. She calls herself an "old witch" and spends much of her time brewing potions and inventing new tea blends.

\textbf{Little Gothby} (lawful neutral, goblin noncombatant) is a dutiful 5-year-old. He doesn't talk to strangers but instead heads for one of his dads if he is approached.

\textbf{Little Chespa} (chaotic good, goblin noncombatant) is 3 years old. She can speak in full sentences, and can read and write as well as an adult. She likes to hop down off the tortoise and frolic through the clearing.

\paragraph{Big Yuyo Locations}

Locations keyed to Big Yuyo are described below:

\begin{description}
\item[L2a: Upladder.] This ladder leads up at the rear of Big Yuyo's shell to the platforms on the tortoise's back. Little Chespa yells, "Halt! Who goes there?" to anyone she doesn't know who tries to board Big Yuyo.
\item[L2b: Rear Hut.] The rear platform supports a hut where Auntie Jaller and the two kids sleep. It's filled with homemade straw dolls and other toys.
\item[L2c: Central Platform.] This central platform supports a 30-foot-tall watchtower.
\item[L2d: Archery Tower.] This elevated platform is designed to give an archer a clear view of the surroundings.
\item[L2e: Starboard Gazebo.] This platform holds a gazebo, where the family can relax.
\item[L2f:] Command Post. The command post holds Drazagorr and Kelbadurn's bed and their belongings, which include a set of goblin-sized \textit{chain mail}, a \textit{spear}, a \textit{shield}, two \textit{shortbows}, two \textit{quivers} containing 20 \textit{arrows} apiece, and a wooden chest that holds 100 gp (the family's savings) and a collection of sentimental keepsakes.
\end{description}

\subsubsection{L3: Ox Corral}

The seven oxen penned here are used to pull the covered wagons in area L5. They are usually docile, but a character who tries to spook them by making a sudden noise can make a DC 15 Charisma (Intimidation) check, causing them to rampage around the corral and cause a bit of chaos on a successful check. The ruckus creates enough of a distraction to allow one or more characters to search a location without being seen by others in the caravan stop.

\subsubsection{L4: Adventurers' Camp}

The rivals camp here if they arrived at the caravan stop ahead of the characters. If the characters are traveling ahead of the rivals or with them, this campsite is being used by an Aurora Watch patrol instead.

Meeting the rivals here provides an opportunity for the characters to interact with them if the Rivals' Reunion encounter has not occurred.

\subsubsection{L5: Bonfire}

This bonfire is the center of activity at the caravan stop. Gaeya Iliera, a neutral good, drow \textbf{scout} in a rugged outfit, is cooking some of the best mastodon kor'rundl to be found outside of Asarius. This savory dish is a hearty rice bowl topped with spicy grilled mastodon and smoky-flavored kor'run, a vegetable like a deep red squash grown in Rosohna's sunless gardens. A character who partakes of this food awakens the next morning with 1d10 temporary hit points, p. 9 that last for the next 24 hours.

Several other travelers are relaxing around the bonfire, playing games, sharing stories, and singing songs. Among those gathered here are the following individuals:

\begin{itemize}
\item Tyvak and Moghra (chaotic good, human \textbf{berserkers}) survived the battle with the udaak that the characters saw the aftermath of in the Demonic Carrion encounter. They are the daughter and son of Kierchaly Wastewalker and belong to the same clan as Irvan Wastewalker. If Tyvak sees that one of the characters has retrieved her father's \textit{+1 dagger} from that site, she offers to trade her \textit{+1 battleaxe} for it and thanks the characters with all her heart if the dagger is turned over.
\item Justice (lawful neutral, tiefling \textbf{scout}) and his \textbf{moorbounder} mount, Rice Pudding, are here if they survived the Moorbounder Mayhem encounter. If the characters helped him during that encounter, Justice gives them a pair of \textit{goggles of night} that the caravan he's working for had a hard time selling at their last stop.
\item The Acorn Sisters, Lanata and Robur, are \textbf{dryads}. As the caretakers of this forest, they're fine with bonfires as long as only dead wood is burned. It's customary to share food, drink, and song with them in appreciation for their hospitality. Characters who show them great respect by making an offering directly to them awaken in the morning with a \textit{charm of vitality} (see the rules for supernatural gifts, p. 7 in the Dungeon Master's Guide).
\end{itemize}

\paragraph{Shopping}

The caravanners parked around the bonfire---including Justice's employers, if he is present---have merchandise for sale. Two neutral \textbf{bugbear chiefs} guard the wagons, and six lawful neutral Aurora Watch \textbf{veterans} (a mix of drow, humans, and orcs) resting at the camp are prepared to apprehend any thieves and drag them to Bazzoxan in chains.

The Merchants' Wares table lists the items available for sale and the number of each item that the merchants have in stock.

% Table: Merchants' Wares
\begin{DndTable}[header={Merchants' Wares}]{Xrc}
Item & Price & Number \\
\textit{Backpack} & 2 gp & 1d6 \\
\textit{Blanket} & 5 sp & 2d6 \\
\textit{Healer's Kit} & 5 gp & 1d6 \\
\textit{Hunting Trap} & 5 gp & 1d6 \\
\textit{Oil Flask} & 1 sp & 2d6 \\
\textit{Parchment (One Sheet)} & 1 sp & 6d6 \\
\textit{Potion of Healing} & 50 gp & 1d4 \\
\textit{Rations (1 Day)} & 5 sp & 10d6 \\
\textit{Rope, Hempen (50 Feet)} & 1 gp & 1d6 \\
\textit{Quiver} of 20 \textit{Arrows} & 2 gp & 1d4 \\
\textit{Quiver} of 20 \textit{Crossbow Bolts} & 2 gp & 1d4 \\
\textit{Sack} & 1 cp & 2d6 \\
\textit{Tent, Two-person} & 2 gp & 1d4 \\
\end{DndTable}
\section{Arrival at Bazzoxan}

This chapter concludes as the characters reach the entrance to the fortress-town of Bazzoxan. As they approach the gate, read or paraphrase the following:

\begin{DndReadAloud}
Jutting from the rusty orange mountains of eastern Xhorhas are tall spires of black stone. This is the Betrayers' Rise---a foul fortress from times long past. Beneath the spires, a vast gate blocks all entry to the military outpost of Bazzoxan, which sits in the shadows of the Betrayers' Rise. Crumbling and desolate structures huddle between iron watchtowers, and soldiers of the Aurora Watch patrol the tops of the walls and the massive gate ahead of you. As you approach, a guard on the wall barks, "Identify!"
\end{DndReadAloud}

If the characters are traveling with members of the Aurora Watch (perhaps the soldiers they met in the Aurora Watch Patrol encounter or at the caravan stop), the soldiers vouch for them, responding with, "Friends of the Dynasty!" No further words need be said, and the entire group is granted access as the gate is opened.

If the characters are traveling alone, they find that the guards of Bazzoxan are eager to let them in. Upon scanning their appearance, a guard asks, "Are you mercenaries? We could use your strength of arms." If the characters are interested or want to discuss terms, the guards say they'll bring the party to the Taskhand---the local authority of this settlement---to explain the details. The gate opens, and the characters pass through into the grim town of Bazzoxan.

\chapter{Bazzoxan}\label{ch:bazzoxan-4-4}

An immense temple of dusky stone called the Betrayers' Rise casts a foreboding shadow over the town of Bazzoxan. The next leg of the adventure sends the characters deep into this ancient stronghold of evil. There, they learn more about Alyxian's past and where he's now imprisoned, thereby awakening new power within the \textit{Jewel of Three Prayers}.
Bazzoxan is a military outpost locked in a stalemate while trying to protect the rest of the Kryn Dynasty from the demons that surge from the depths of the Betrayers' Rise. Just as the characters enter the town, they find themselves facing an incursion. After the attack, they're welcomed to Bazzoxan by the drow commander of the town's defenders, Taskhand \textbf{Verin Thelyss}.
The characters can interact with the locals and investigate the town, where they meet agents of three factions from the distant continent of Marquet. The characters might ally with one of these faction agents, who all want to control the power hidden within the Betrayers' Rise---and the characters' rivals might do the same. The characters' journey through the temple eventually brings them to the place where Alyxian asked the god Avandra for her aid during the Calamity.
At this prayer site, the characters receive another vision, unlock new properties of the \textit{Jewel of Three Prayers}, and come face to face with their rivals once again. The outcome of this confrontation not only determines who possesses the jewel but also defines the tangled web of relationships between the characters, their rivals, and the Marquesian factions---just before the characters are swept up in a spell that transports them to Marquet.
% Content: Unknown (dict)
\section{Running This Chapter}

The first part of this chapter is an exploration sequence in which the characters interact with locals and visit locations in Bazzoxan to learn more about their vision in the Emerald Grotto and the \textit{Jewel of Three Prayers}. This section and the dungeon exploration that follows are intentionally open-ended and designed to reward inquisitive characters.

Because the characters can go anywhere in town, it's important for you to read both the "Bazzoxan Overview" and "Locations in Bazzoxan" sections to familiarize yourself with the town before running this chapter.

If the characters get stuck or frustrated, you can use friendly nonplayer characters such as \textbf{Verin Thelyss} and the faction agents to give them hints about how they could learn more. Ultimately, the characters need to find the prayer site inside the Betrayers' Rise.

Starting in this chapter, the characters' rivals use their tier 2 stat blocks (see appendix A (p. 8)).

\subsection{Who Has the Jewel?}

The path to the prayer site can be opened only by using the \textit{Jewel of Three Prayers}. Doing so allows access to a segment of the Betrayers' Rise that has never been explored by the Aurora Watch. If the characters don't have the jewel, they might need to take it from the rivals---by stealth, by force, or by barter---as described in area B7, later in the chapter.

In addition, the final confrontation in this chapter is a duel for the possession of the jewel. Regardless of whether the rivals or the characters end up with the jewel, the adventure continues on the far-off continent of Marquet. The "Next Steps" section of the chapter explains how to transition the characters from Bazzoxan to their new location.
\section{No Time for Pleasantries}

This chapter begins as two Aurora Watch \textbf{guards} guide the characters from the gate of Bazzoxan toward the Gatehold Barracks. Read or paraphrase the following as the characters enter the town:

\begin{DndReadAloud}
From around the corner of a crumbling building, five twitching, skinless masses of blinking eyes and slavering mouths wriggle into view. One of them is shredding the flesh of a hapless soldier in three of its amorphous maws. Trumpets blare an alarm, and several Aurora Watch soldiers hurry into the streets. One of them, a male drow, pauses to address you. "Newcomers? No time for pleasantries---draw your weapons or get to safety! Find me at the barracks after!"
\end{DndReadAloud}

% Image placeholder: A gibbering mouther threatens two Aurora Watch soldiers

Five \textbf{gibbering mouthers} have sneaked into Bazzoxan and are intent on consuming anyone in their path. Opposing them are members of the Aurora Watch led by Taskhand \textbf{Verin Thelyss} (see the accompanying stat block).

To the people of Bazzoxan, Verin is a beacon of hope amid the darkness that shrouds the town. But the weight of responsibility colors Verin's otherwise bright and youthful demeanor with melancholy. He cares about the town he has sworn to protect, but he is weary of the incessant threat posed by the denizens of the Betrayers' Rise.

The alarms sounding from watchtowers across the town indicate that similar attacks are happening in other areas. Civilians head for secure buildings while members of the Aurora Watch mobilize to engage the threat. Characters can tell from everyone's drawn faces and practiced movements that attacks like this occur often. The characters can either get involved in the fight or run for safety.

% Image placeholder: {@creature Verin Thelyss|CRCotN}

\subsection{Fight!}

If the characters choose to help the Aurora Watch, everyone rolls initiative. \textbf{Verin} and two orc \textbf{guards} of the Aurora Watch stay with the party, while the rest of the soldiers leave to secure other parts of the city.

\paragraph{Rivals}

If the rivals are accompanying the characters when they enter Bazzoxan, Verin directs the rivals to assist in defense of the infirmary (area B3). They're momentarily indignant at being ordered around, but they fall in line when Ayo echoes the command.

\subsection{Flight!}

If the characters flee, they must find a safe building in which to hide. It quickly becomes clear that most such places have already been barricaded shut by the residents holed up inside them. A character who wants to get into one of these safe houses must succeed on a DC 20 Charisma (Persuasion) check to convince its occupants to open the door. Alternatively, a character can make a DC 20 Wisdom (Perception) check to scan the area, finding an available safe house on a success. A character can make either check once per minute. Every time a character gets a failure on one of these checks, a \textbf{gibbering mouther} emerges from hiding and attacks that character.

\paragraph{Rivals}

If the rivals are with the characters when they enter Bazzoxan, Verin orders the rivals to stay and fight while the characters flee. Ayo Jabe doesn't like taking orders, but she falls in line---as do her companions.

\subsection{Aftermath}

The alarms stop after 5 minutes, signaling that the danger has been quelled. The uneasy calm of daily life in Bazzoxan resumes. The monsters that attacked other parts of town have been handled by Aurora Watch soldiers.

After the gibbering mouthers have been defeated, Aurora Watch soldiers begin carrying the wounded to the infirmary (area B3). One way or another, the rivals end up in the infirmary after the battle, and their condition depends on the actions of the characters:

\begin{itemize}
\item If the characters fought alongside Verin, the rivals are safe at the infirmary and eager for news.
\item If the characters fled with the civilians, leaving the rivals to fight alongside Verin, Ayo Jabe was gravely wounded in the battle. Maggie Keeneyes carried Ayo to the infirmary, where she is treated by Dermot Wurder. Irvan and Galsariad linger outside the infirmary, concerned about their group's future.
\end{itemize}

\subsubsection{Chatting with Verin}

If the characters are with Verin when the battle ends, he formally introduces himself as the leader of Bazzoxan and apologizes that their arrival was complicated by the gibbering mouther attack. He answers general questions about Bazzoxan (anything found in the "Bazzoxan Overview" section below) but soon excuses himself to check on the rest of the town. He leaves the characters to explore Bazzoxan at their leisure and invites them to meet with him later at the Gatehold Barracks (area B4).
\section{Bazzoxan Overview}

Bazzoxan hums with activity. People are cleaning up debris from the last attack, tending to the injured, performing last rites for the dead, repairing fortifications, and preparing for the next incursion. Gibbering mouthers---created by demons in the Betrayers' Rise---are common threats, as are dretches, quasits, and other lesser demons. On rare occasions, more powerful demons attack, or demonic energy animates the bodies of dead Aurora Watch soldiers.

Most of the town's inhabitants are soldiers who understand that their role here is not to defeat the horrors in the Betrayers' Rise, but to keep them in check for as long as possible. The town's civilian residents go about their lives as normally as they can, and few complain about their situation.

Details about locations in the town can be found in the "Locations in Bazzoxan" section later in the chapter.

% Image placeholder: Bazzoxan has an eerie beauty that is lost on most visitors and residents

\subsection{Geography}

Bazzoxan is built into the base of a rust-red mountain range north of the valley known as the Barbed Fields. The craggy, rocky landscape is mostly devoid of plant life, except for a few scraggly pines and patches of scrub.

Bazzoxan was once a thriving mining town, but it was ravaged when explorers awakened a portal to the Abyss in the depths of the Betrayers' Rise. Many of the town's buildings were destroyed when the first surge of demons attacked, and the town was hastily converted into a military outpost to hold the demons at bay. The only structures regularly tended to are the Aurora Watch barracks, most of which are buildings converted from their original purpose.

\subsubsection{Sacrifice Engines}

Among the most prominent features of Bazzoxan are the town's two sacrifice engines, each one a 30-foot-deep pit lined with rows of silver blades. The blades can be made to spin by pulling a lever near the edge of the pit. A creature in the pit while the blades are spinning takes 44 (8d10) slashing damage at the start of each of its turns.

In bygone days, servants of the Betrayer Gods threw mortals into these pits to appease their evil lords. Now, the Aurora Watch herds demons and other monsters into them to avoid prolonged battles.

\subsection{Populace}

The majority of Bazzoxan's inhabitants are members of the Aurora Watch. Also residing here are a handful of Kryn Dynasty arcanists attempting to find and seal the planar rifts that threaten the city. Dealing with these phenomena is far beyond the characters' abilities at present, and in any event they can't afford to linger in Bazzoxan, lest their rivals find the prayer site of Avandra before them. Nonetheless, this chapter provides an opportunity for you to foreshadow the devastation that the rift to the Netherdeep can cause for Ank'Harel later in the adventure.

The few civilians who remain in Bazzoxan support the soldiers as well as they can and are trained in the basics of combat in case the forces from the Betrayers' Rise begin to overwhelm the military. Most residents are drow, but some goblins, hobgoblins, bugbears, orcs, and other Xhorhasian folk live here as well.

Most residents of Bazzoxan avoid conversation, citing the need to tend to their duties if they are approached. The Bazzoxan Scenes table provides suggestions for small engagements the characters can experience while exploring the town.

% Table: Bazzoxan Scenes
\begin{DndTable}[header={Bazzoxan Scenes}]{cX}
d10 & Encounter \\
1 & A chaotic neutral, orc \textbf{berserker} tends a garden of root vegetables in one of the ruined buildings, aggressively pulling up weeds. \\
2 & A weary drow mother (neutral good \textbf{commoner}) sits in the marketplace, instructing her children on what to do if she dies. \\
3 & A tired drow merchant (neutral \textbf{commoner}) attempts to sell what's left of his inventory. \\
4 & An Aurora Watch soldier (lawful good \textbf{drow elite warrior}) breaks up a fight between two sobbing citizens (lawful good \textbf{commoners}). \\
5 & Ten Aurora Watch \textbf{guards} (lawful good drow, humans, and orcs) jog in formation, keeping tempo by singing about the ways they'll die. \\
6 & A neutral good \textbf{bugbear} fixes the wheelchair of a lawful neutral, drow \textbf{veteran} who has lost a leg. \\
7 & A teenage, lawful neutral, dwarf \textbf{acolyte} sits in a dark corner, weeping as she clutches a helm to her chest. \\
8 & A red-faced human (chaotic neutral \textbf{commoner}) rambles drunkenly about how the demons are coming to kill them all. \\
9 & Two members of the Aurora Watch, a lawful neutral \textbf{drow elite warrior} and a chaotic neutral \textbf{ogre}, bicker over which of them should take on a dangerous assignment. \\
10 & An old tiefling (lawful good \textbf{veteran}) sobs as he carries a small casket toward the crematorium (area B2). \\
\end{DndTable}

\subsection{Rivals in Bazzoxan}

After the encounter with the gibbering mouthers, the rivals lose some of their bluster. Though they try to keep up appearances, they are severely shaken by the experience. Galsariad is the most deeply affected, his already sarcastic demeanor becoming close to insulting toward friend and foe alike as his insecurities threaten to rise to the surface. Irvan becomes withdrawn, overwhelmed by the oppressive atmosphere in Bazzoxan and the morbid humor of the locals. Dermot throws himself fervently into following the Luxon's precepts, determined to do whatever it takes to protect his friends and rarely leaving Ayo's side. If he is persuaded to take a moment away from the group, he talks openly about his concerns. He recognizes that his friends have been traumatized and would like them to go home rather than continuing. No matter what, he intends to support them to the end.

If Ayo was injured in the encounter as a result of the characters' decision to run and hide, the rivals are hostile toward whomever they see as the characters' leader, and Maggie Keeneyes is hostile toward the entire party because of their cowardice. Ayo is shaken by the injuries, causing her to question her skill as a leader, but she bears no ill will toward the characters for putting their safety first.

\subsection{Locations in Bazzoxan}

The following locations are keyed to the map of Bazzoxan.



\subsubsection{B1: Gate of Bazzoxan}

The entrance to Bazzoxan is a wide gate that is opened or closed from atop its battlements by operating a series of wheels, chains, and pulleys.

\subsubsection{B2: Crematorium}

The crematorium is a simple stone building built against the mountainside.

When the characters approach the building, read or paraphrase the following:

\begin{DndReadAloud}
Thick, acrid smoke billows from the chimneys of this stone building. Corpses wrapped in beige cloth are stacked in a small yard northeast of the building.
A willowy man wearing a leather apron greets you at the entrance with surprising cheer. "New bodies for Bazzoxan, I see!" Behind him stands a giant of a man: muscular, blue-eyed, with short-cropped hair and a missing arm.
\end{DndReadAloud}

The two men are Reynard Allerton, a chaotic good \textbf{scout}, and his bigger, older brother, Sebastian Allerton, a neutral good \textbf{thug}. Their crematorium has two furnaces, each of which can hold up to four Medium bodies or one Large body at a time. The brothers are responsible for the proper disposal of corpses. Reynard seems to be the more gregarious of the siblings, although if the characters can coax Sebastian out of his silence with a joke or a kind remark, the older brother is also revealed to be a good-natured chatterbox.

When the characters arrive, the brothers are burning the latest batch of corpses. Reynard promptly asks the characters to help him and his brother with their task.

\paragraph{Corpse Disposal}

As the brothers (and any characters who decide to assist them) work through the stack of corpses, Reynard becomes increasingly reluctant to continue, asking the characters to tend to the bodies and making excuses for why he should tend the fire. A character who succeeds on a DC 11 Wisdom (Insight) check can tell that he's deliberately avoiding one of the large corpses near the bottom of the pile, out of disdain and horror.

Characters who are helping with the pile of bodies find the source of Reynard's disgust: a dead vrock whose body reeks of offal and blood. Reynard has been putting off the task of burning it because he finds the demon repulsive and because the corpse hasn't dissolved into a pool of demonic ichor as it should have, which suggests some sort of abnormality.

Before anyone can dispose of the vrock's remains, a commotion arises nearby:

\begin{DndReadAloud}
"Wait! Stop!" a voice shouts from down the street. A bookish tiefling jogs into view. A variety of archaeology tools hang at his waist along with a satchel holding his notebooks. "You can't burn that yet. I need to study it!" Reaching into the satchel, he pulls out a badge and holds it up, as if the badge alone explained his reason for being here.
\end{DndReadAloud}

The tiefling is \textbf{Prolix Yusaf}, a lawful neutral \textbf{scholarly agent} (see appendix A (p. 8)) from the Allegiance of Allsight. His badge displays the symbol of his faction: a scroll bearing a single, watchful eye.

Prolix introduces himself as an archaeologist from Ank'Harel, a city on the distant continent of Marquet. He nervously explains that because vrocks often swallow jewelry and other ornamental objects, he has reason to believe that the vrock might have a few valuable items from the Betrayers' Rise in its stomach---and he'd like to try to cut them out before the Allertons burn the corpse.

Lacking experience with dead bodies, Prolix turns to the characters for assistance. A character who wants to extract the contents of the vrock's stomach must first cleave through the vrock's hide. To do so, the character must succeed on a DC 15 Wisdom (Survival) check. A failed check triggers an eruption of spores from the vrock's corpse that fill a 10-foot-radius sphere centered on the corpse. All creatures in that area must succeed on a DC 14 Constitution saving throw or become \textit{poisoned}. A creature \textit{poisoned} in this way takes 5 (1d10) poison damage at the start of each of its turns. It can repeat the saving throw at the end of each of its turns, ending the effect on itself on a success.

\paragraph{Treasure}

The vrock's gullet contains a pair of gold-trimmed \textit{eyes of the eagle}, three gaudy rings worth 2 gp each, and a cube-shaped puzzle box made of onyx and measuring 3 inches on a side. A \textit{detect magic} spell reveals an aura of abjuration magic around the box, which is locked. Prolix grabs it before anyone else can do so, leaving the other items for the characters. As soon as the puzzle box is removed from the dead vrock's gullet, the enchantment on the vrock fades, and it begins to dissolve into a puddle of reeking ichor.

Anyone who tinkers with the puzzle box's locking mechanism for 1 minute can make a DC 20 Intelligence check, opening the box on a success. The box contains a folded-up \textit{spell scroll} of \textit{bestow curse}.

% Image placeholder: {@creature Aloysia Telfan|CRCotN} and {@creature Prolix Yusaf|CRCotN}

\paragraph{Roleplaying Prolix}

Once he has obtained the puzzle box, Prolix's demeanor turns from friendly to reserved. He claims that he knows how to unlock the puzzle box but insists on doing so in an isolated location to prevent the "danger" contained within from hurting anyone. A character who succeeds on a DC 10 Wisdom (Insight) check can determine that Prolix is lying and trying to sound important.

A character who presses Prolix for answers can make a DC 12 Charisma (Persuasion) check. If the check succeeds, Prolix gives one of the following answers; if the check succeeds by 5 or more, he becomes friendly toward the characters and shares everything he knows freely:

\paragraph{Why are you here}

"I'm an archaeologist, but I'm not here to perform archaeology, per se. I was sent by the Allegiance of Allsight to spy on a rival agent of the Consortium of the Vermilion Dream." Prolix says that the Allegiance doesn't know why the Consortium dispatched someone to Bazzoxan, but the reason can't be good.

\paragraph{Who is your rival}

"Aloysia. That's her name. She's looking for something in the Betrayers' Rise. She's not hard to spot. Tall, pale, dressed in red. Cold-blooded. Seems like she'd do anything to get what she wants. And what she wants is... well, that's what I'd like to know!" Prolix goes on to say that both he and Aloysia are staying at the Ready Room (area R7).

\paragraph{Do you know how to open the puzzle box or not}

"Of course not! But it came from inside the temple---it must hold something important." Prolix intends to keep the box and its contents from falling into Aloysia's hands.

If the characters let Prolix keep the puzzle box, he offers them a deal: if they find out what Aloysia is up to and stop her, he will give them 200 gp, as well as the box and whatever might be inside it.

\paragraph{Prolix and the Jewel}

If the characters have the \textit{Jewel of Three Prayers} and show it to Prolix, he says he doesn't know what the jewel represents, although he recognizes it as something both powerful and ancient. If the characters give him the opportunity, he draws a quick sketch of the item, jots down some notes, and invites the characters to return with him to the Crystal Chateau, his university in Ank'Harel, once their respective missions in Bazzoxan are complete. He suggests the university's libraries might have more information about the jewel's purpose.

If the characters seem eager to learn more about the jewel, Prolix mentions meeting a "real smartie" who is also staying at the Ready Room (area B7). This other scholar is a tiefling named Question, who Prolix believes is visiting Bazzoxan for research as well. Prolix relates having an animated discussion with Question the other night about key historical figures of the Calamity.

\subsubsection{B3: Infirmary}

Three buildings on the perimeter of the central square have doors marked with a symbol of a hand alight with magic, identifying them as places of healing. When the characters enter one of the buildings, read or paraphrase the following:

\begin{DndReadAloud}
The walls are lined with shelves that hold old books and medical supplies. The floor is crammed with narrow cots. There's a small office in the back, its door ajar, and the air reeks of medicinal poultices.
\end{DndReadAloud}

Inside, the characters find a lawful neutral, drow \textbf{priest} chatting with a young female drow, whose Aurora Watch armor and satchel are stacked nearby. Other wounded patients---most of them asleep---fill the infirmary beds.

When she sees the characters, the priest leaves her patient's side to greet them, introducing herself as Bautha Dyrr. She then asks them about their health. A character who succeeds on a DC 15 Intelligence (Religion) check identifies a pin in the shape of a black wing attached to Bautha's cloak as the holy symbol of Xalicas, a legendary solar who served as the right hand of Corellon the Arch Heart before falling in the Calamity. As a follower of Xalicas, Bautha disapproves of the reclamation efforts in Bazzoxan; though she supports the effort to hold back the forces of the Abyss, she believes the Kryn Dynasty's military presence here is creating needless casualties with little to show for it. Nevertheless, Bautha strives to heal and protect those in Bazzoxan, and she casts \textit{cure wounds} or \textit{lesser restoration} on victims for free, as long as she has spell slots to spare.

Bautha also has a \textit{spell scroll} of \textit{greater restoration}. She must succeed on a DC 15 Wisdom check to cast the spell from the scroll; otherwise, the scroll is destroyed with no effect. She will sell it to the characters for 500 gp.

Bautha is familiar with \textbf{Prolix Yusaf} and \textbf{Aloysia Telfan}, since both individuals have visited the infirmary for care in the past week. She sees them as reckless and naive, and she thinks that their obsession with researching the Betrayers' Rise can lead to no good end. A character who succeeds on a DC 13 Charisma (Persuasion) check gets her to reveal that Aloysia was injured when she tried to delve into the Betrayers' Rise alone---and now she's looking for mercenaries to explore it on her behalf.

\paragraph{Bautha and the Jewel}

If the characters show Bautha the \textit{Jewel of Three Prayers}, identify it by name, and ask her about it, she doesn't recognize it, but something about the name reminds her of an old tale. She thinks for a moment, then says:

\begin{DndReadAloud}
"Three prayers... Oh! You know, it must be that old story about the champion of three gods. Goodness gracious, I can't believe I remembered it. It's a fragment of a fragment of a legend, even in temples in Vasselheim and Ank'Harel." She clears her throat. "Long ago, a hero of the Calamity begged for aid from three different gods in his time of need. Avandra the Change Bringer was one of them, and I think Sehanine the Moon Weaver was another. And, uh, I'm afraid I don't remember anything else."
\end{DndReadAloud}

As the characters prepare to leave the infirmary, Bautha suddenly recalls one other detail:

\begin{DndReadAloud}
"This might be a fiction, but Avandra is said to have grieved for the hero---not because he had died, but because she feared her aid forced him to give up so much of himself that he would be robbed of a future."
\end{DndReadAloud}

\paragraph{Naevyn's Request}

As the characters make their way to the exit, the wounded member of the Aurora Watch calls for the characters to approach her cot. Her body is covered with wounds she received in the gibbering mouther attack.

The drow is Naevyn Tasithar, a neutral good \textbf{scout}. She weakly salutes the characters, then asks them to hand her her satchel. She pulls out a small tiger's eye carving of a rabbit with her name carved into the bottom. Naevyn explains that she was planning to visit the Wall of the Unforgotten (area B6) and place this trinket there. If asked why, she looks away and refuses to answer, but asks the characters to place it by the wall. Characters who speak to Kalym, a priest stationed at the wall, can learn Naevyn's reason.

\subsubsection{B4: Gatehold Barracks}

The Aurora Watch's command center is located in the Gatehold Barracks. The complex contains lodgings for Taskhand \textbf{Verin Thelyss} and his senior officers, as well as accommodations for regular soldiers. Like most of Bazzoxan's buildings, these are in a constant state of reconstruction; splintered boards and broken shingles are salvaged to be sorted and repurposed, and canvas is tacked up on walls to cover holes that have yet to be repaired.

When the characters enter this area for the first time, read or paraphrase the following:

\begin{DndReadAloud}
These barracks are filled with the sounds of chatter, voices shouting orders, and clanging metal. Armored Aurora Watch soldiers go about their business, performing drills and sharpening weapons.
\end{DndReadAloud}

The barracks are divided into four main sections: the war room, the mess hall, the training area, and the sleeping quarters. If Taskhand \textbf{Verin Thelyss} is expecting the characters, a soldier is waiting for them and leads them to the war room.

\paragraph{War Room}

The war room is a small, sparsely furnished office Verin uses to discuss battle strategy and political concerns with his officers. A circular table in the center of the room is covered with maps of Bazzoxan; updated barricade designs; and notes regarding various horrid creatures, their redoubts, and the rifts they spawn from.

Verin greets the characters with warmth and gentle concern. After the initial pleasantries, he politely asks the characters to state their intended business in Bazzoxan.

As the conversation unfolds, Verin shares the following information:

\begin{itemize}
\item An expedition from the Library of the Cobalt Soul is currently in the Betrayers' Rise with an Aurora Watch escort, researching battles of the Calamity that took place within its halls.
\item An agent of a foreign organization calling itself the Consortium of the Vermilion Dream is staying in Bazzoxan. This agent, an elf named Aloysia, has been reprimanded for trying to bribe Aurora Watch soldiers to protect her while she searches for an ancient holy site she claims is within the Betrayers' Rise.
\item An archaeologist named Prolix from the Allegiance of Allsight in Ank'Harel also recently arrived in Bazzoxan. He is ostensibly assisting Aurora Watch arcanists who are researching ways to seal the rift within the Betrayers' Rise. Verin suspects that the archaeologist has other motives, because he was seen earlier in the day lurking around the crematorium (area B2).
\end{itemize}

If the characters ask Verin to share his thoughts regarding Bazzoxan's connection to the Calamity, he dissembles. A character who presses Verin for more information can make a DC 14 Charisma (Persuasion) check. If the check succeeds, Verin admits he believes the Betrayers' Rise might hold the solution to breaking the stalemate that grips the town, and he has sent out scouting parties and arcane experts to confirm that idea. None of the expeditions have been fruitful so far, and Verin can't spare any more personnel to expand the effort.

If the characters tell Verin that they intend to journey to the Betrayers' Rise, he asks them to proceed with caution---and to retreat if they think they're getting in over their heads, since he can't afford to send soldiers to help them. He explains:

\begin{DndReadAloud}
"The Betrayers' Rise is... challenging to navigate. Its passages shift to confuse mapmakers and search parties. I think the power of the Abyss flows through it, warping the environment so no two groups follow the same path through it. Whether it's demon magic or not doesn't really matter; all I know is that if you get lost down there, my soldiers won't be able to find you."
\end{DndReadAloud}

\paragraph{Mess Hall}

The mess hall, on the west side of the area, is filled with tables and plain wooden chairs. Though it's meant to be for soldiers, food here is doled out to anyone who asks. The repast consists of gruel, plain vegetables, bread, tough meat, and smoked fish.

The soldiers gathered here are relaxing and sharing gossip about recent events in the town. Characters who loiter in the mess hall overhear the following pieces of information, which they can also learn by engaging any of the soldiers in conversation:

\begin{itemize}
\item One member of the Cobalt Soul expedition, a tiefling named Question, went with his group up to the gate of the Betrayers' Rise but elected not to enter the place. He's holed up at the Ready Room (area B7), ready to talk about his research with anyone who will listen.
\item An elf named Aloysia tried to bribe some soldiers into escorting her into the Betrayers' Rise. Some people believe that she wants only to ransack the temple, but others think her concern is religious and she is searching for long-lost lore about the Prime Deities.
\end{itemize}

\paragraph{Training Area}

One of the buildings in the center of the barracks is used as a training area, where new soldiers learn their trade under the watch of officers. The characters can engage with the soldiers in three ways:

\begin{description}
\item[Shooting Match.] If a character carries a ranged weapon, a grizzled orc \textbf{veteran} challenges that character to a friendly shooting match. The match consists of 3 rounds, during each of which a participant makes one attack with the ranged weapon of their choice at a target 40 feet away. The target has a bullseye surrounded by three concentric rings, each one corresponding to a point value. A hit inside the outer ring (AC 14) is worth 1 point, a hit inside the middle ring (AC 16) is worth 3 points, and a hit inside the inner ring (AC 18) is worth 5 points. A critical hit scores a bullseye, which is worth 10 points. Whoever has the most points at the end of the 3 rounds is the victor (ties are possible).
\item[Sparring Match.] A character can challenge an Aurora Watch soldier, or vice versa, to a sparring match with training weapons. This match can be run as a combat encounter, using the \textbf{veteran} stat block for the character's opponent. A creature reduced to 0 hit points by a training weapon falls \textit{unconscious} and is stable. The match ends when one participant falls \textit{unconscious} or concedes defeat.
\item[Wrestling Match.] A character can challenge an Aurora Watch soldier, or vice versa, to a wrestling match. Have a character make three DC 14 Strength (Athletics) or Dexterity (Acrobatics) checks in a row (character's choice of which skill to use). If at least two of these checks succeed, the character's opponent taps out; otherwise, the character loses the match.
\end{description}

\paragraph{Sleeping Quarters}

The sleeping quarters in the Gatehold Barracks contain small armories where the soldiers can store their weapons and equipment when they're off duty.

The characters aren't permitted in this location, and there is no treasure to be found here.

\subsubsection{B5: Dilapidated Temple}

What was once a flourishing, spiral-shaped temple has fallen into disrepair after those who tended to it were driven away or slain by demons. It has been transformed into an indoor farming plot to grow Xhorhasian vegetables that don't require sunlight.

When the characters enter the area, read or paraphrase the following:

\begin{DndReadAloud}
Glass crunches underfoot amid overgrown weeds. Collapsed walls have been replaced with rows of wooden shelves that hold earthenware pots. Numerous alcoves are being used as planters for stubborn flora. Just outside the far end of the enclosure, a smashed mosaic of stained glass sits in a splintered window frame next to a withered, skeletal tree.
\end{DndReadAloud}

A character who makes a successful DC 17 Intelligence (Religion) check can piece together that the mosaic once depicted a woman's face on a golden disk: the iconography of Avandra the Change Bringer.

\paragraph{Foghome the Gardener}

A neutral good, firbolg \textbf{priest} named \textbf{Foghome} moves through the temple at all hours of the day. For more information on firbolgs, see the accompanying "Firbolgs of Exandria" sidebar.

% Image placeholder: {@creature Foghome|CRCotN} the Firbolg

\textbf{Foghome} is here as a guest of Taskhand \textbf{Verin Thelyss} because he has offered to help the Aurora Watch create its own food source. \textbf{Foghome} can tell the characters several things:

\begin{itemize}
\item No vegetation survives for long in Bazzoxan. \textbf{Foghome} thinks this is the result of the corrupting influence of the Betrayers' Rise.
\item He is a follower of Melora the Wild Mother, but he recognizes the temple as a former holy place of Avandra the Change Bringer. He hopes to honor both deities by cultivating new life in the midst of the death and destruction that haunts this place.
\item While meditating the other night, he felt something surge within the earth nearby---a warm and kindly presence reminiscent of the Prime Deities. He suspects the presence is a sign that a relic of the Calamity has awakened within the Betrayers' Rise.
\end{itemize}

\begin{DndSidebar}{Firbolgs of Exandria}
Firbolgs are forest-dwelling Humanoids native to the Greying Wildlands of Wildemount. They are on the tall side of Medium, and their bodies are covered with thick fur ranging in color from tones of brown and ruddy red to cool gray and blue, and occasionally in hues of pink or green. They have floppy, pointed ears and broad, pink noses.
A firbolg nonplayer character encountered in Exandria might have one or both of the following traits:
\paragraph{Powerful Build}

The firbolg counts as one size larger when determining its carrying capacity and the weight it can push, drag, or lift.
\paragraph{Speech of Beast and Leaf}

The firbolg can communicate in a limited way with Beasts and Plants. These creatures understand the meaning of the firbolg's words, though the firbolg has no ability to understand their languages. In addition, the firbolg has advantage on Charisma checks it makes to influence a Beast or Plant.
\end{DndSidebar}

\paragraph{Foghome and the Jewel}

If asked about the \textit{Jewel of Three Prayers}, \textbf{Foghome} has no recollection of an item bearing that name. If he examines the jewel, he correctly senses three fragments of divine power within it, one awake and two slumbering. He also tells the characters that one of the fragments in the jewel is similar to the presence he felt within the earth while he was meditating.

\textbf{Foghome} invites the characters to visit him at the temple any time. He mentions that he meditates to commune with the Wild Mother in the evenings, and he would be happy to reach out to her for more answers on the characters' behalf if they decide to join him one night.

If the characters take him up on his offer, they receive a vision during \textbf{Foghome}'s meditation:

\begin{DndReadAloud}
You see a young human, curls plastered to his brow by sweat, sinking to his knees in the garden. His eyes betray a lack of sleep, and despite the softness of his features, he looks like he has experienced more pain than any one person should bear. "They're counting on me," he whispers. "They need me, but I'm exhausted and frightened. What do I do? Please. Someone tell me---what do I do?"
The moons race across the sky, soon replaced by the angry red sun that rises before a fateful battle. Five silhouettes approach the temple, and one claps the kneeling figure on the back. Their voices are muffled, as if coming at you through water, but you see them embrace, standing strong, before marching off to war.
\end{DndReadAloud}

When the vision concludes, the characters fall \textit{prone} on the ground, having momentarily passed out. \textbf{Foghome} doesn't see the vision, but he sees the characters fall \textit{unconscious} and then awaken moments later. If they admit to seeing the vision, he asks them to describe it before nodding slowly and explaining he has seen the vision before. He believes the central figure in the vision to be a hero of the Calamity and recounts the old stories of Bazzoxan's origin:

\begin{DndReadAloud}
"The Kryn capital of Rosohna was the Betrayer Gods' citadel of Ghor Dranas in the time before the Calamity, and the Betrayers' Rise was once a fortress protecting Ghor Dranas. In the darkest days of the Calamity, the Prime Deities and their most loyal champions laid siege to Ghor Dranas before capturing the Betrayers' Rise and routing the Betrayer Gods' forces there. Though storytellers have long since forgotten the names of many of those heroes, the force of their bonds still resonates across the land."
\end{DndReadAloud}

If the characters ask him to elaborate, \textbf{Foghome} draws their attention to several alcoves where the plants are markedly healthier than the rest of those within the temple. Engraved on the back walls of the alcoves are the following timeworn words in Common (parts of the messages that are no longer legible are indicated by dashes):

\begin{itemize}
\item "Friends forever. Don't------forget------"
\item "You owe me a drink in Marquet, A------"
\item "I am writing this------record of me stating------I will never, ever duel you again."
\item "This------written proof------promise to show you the mountains of my home one day."
\item "We love you. We will always be with you."
\end{itemize}

\subsubsection{B6: Wall of the Unforgotten}

\begin{DndReadAloud}
A stretch of stone wall is carved with names and wreathed in dried flowers, silver necklaces, and ropes weighted with good-luck charms. The ground near the wall is cluttered with trinkets left there as offerings. A drow of the Aurora Watch prays near a corner of the wall.
\end{DndReadAloud}

The Wall of the Unforgotten is a memorial site for the soldiers of the Aurora Watch. The wall is crowded with names, most scratched into the surface with knives.

The praying figure is a lawful neutral, drow \textbf{priest} of the Luxon named Kalym Telaarin. Kalym is 300 years old and has many scars on his otherwise handsome face. His left leg from the knee down is an intricate prosthetic carved from chitinous black rock and ringed with bands of pale purple light. As the characters approach, Kalym rises to his feet and says sternly, "Be respectful. They've earned their rest, and they deserve to have that rest honored."

If the characters show Kalym the carving they received from the young drow in the infirmary (area B3), his demeanor softens. He explains that it's part of a recent practice taken up by younger members of the Aurora Watch: when a soldier is not sure of living long enough to make friends that will hold a memorial for them, they arrange to have a keepsake of theirs placed by the wall in their honor.

Kalym directs the characters to place the scout's carving amid the items strewn at the base of the wall. After the characters do this, Kalym kneels and resumes praying. If the characters stay until he is done a couple minutes later, Kalym rises and gifts each willing character with a magical Mote of Possibility, which manifests as a flash of silver light as he touches each character's forehead in turn. A character who fails a saving throw or misses on an attack roll can use a reaction to expend their Mote of Possibility, allowing them to reroll the d20. The character must use the new roll. A Mote of Possibility vanishes from a character after 24 hours or when it is expended.

Kalym returns to duty after imparting his gifts. He has no useful information about the Betrayers' Rise, the \textit{Jewel of Three Prayers}, or the comings and goings of faction agents in Bazzoxan.

\subsubsection{B7: The Ready Room}

The Ready Room is Bazzoxan's only inn and one of the only structures in town that doesn't look like a military barracks:

\begin{DndReadAloud}
A large barn with a steeply pointed roof looms over the battered structures near the center of town. Over the double door hangs a sign that bears the tavern's name, The Ready Room, in Undercommon, Orc, Goblin, and Common.
\end{DndReadAloud}

The Ready Room is run by Delez and Prima Demona, a brother-and-sister team who are both lawful neutral \textbf{drow}. Delez runs the bar and the kitchen and handles most interactions with customers, while Prima manages the finances and inventory.

The ground floor contains piles of tools (lanterns, shovels, ladders, and the like) and a dozen wooden stalls that have been turned into curtained sleeping quarters, each one holding five wooden cots. A rope-and-pulley hoist functions like an elevator, allowing access to a spacious loft where the proprietors have equipped a bar and surrounded it with two dozen stools that are mostly occupied by patrons whenever the place is open for business. Behind the bar is a cooking space with a stove, where Derez prepares food for guests. A curtain behind the bar conceals a small office and storage room, where Prima spends much of her time.

A tankard of beer costs 1 cp, while a bowl of mushy rice and overcooked fish costs 2 cp. The use of a cot costs 2 sp per night. Characters can purchase any common adventuring equipment at the Ready Room, but all such items sell for double their normal price. Although Prima doesn't appreciate her prices being questioned, a character who makes a successful DC 17 Charisma (Persuasion) check can get her to lower the prices to only a 50 percent markup.

\paragraph{Rivals}

The rivals like to gather for food and drinks at the bar. This is a good place for the characters to sit and chat with them, especially if the two groups didn't interact much in chapter 2 (p. 2). It's also a chance to show how the rivals' travels have begun to weigh on them; there is a different quality to the tone of their boasts, as well as the way they address each other and the characters (see "Rivals in Bazzoxan" earlier in the chapter).

\paragraph{Question and Answers}

Question, a chaotic neutral, tiefling \textbf{monastic operative} (see appendix A (p. 8)), can be found lounging around the Ready Room at all hours. Unlike most others in Bazzoxan, Question almost always has a smile on her face.

The tiefling doesn't take notice of the characters unless they make contact first, at which point she performs an awkward little bow. After introducing herself, Question launches into an excited spiel:

\begin{DndReadAloud}
"Tell me you feel it, too," says the tiefling with a grin. "My comrades don't, but I do. The history surrounding this place. The romance of heroes fighting against impossible odds, sacrificing everything for the sake of defending the world. All done for people who would never know them, never think of them, never be able to thank them. Simply because it was the right thing to do. And ages later, we now see the Aurora Watch somehow doing the same. And they don't even know it!"
\end{DndReadAloud}

Question calls this phenomenon, related to the cyclical nature of history, "mythic resonance." She believes that acts of monumental courage leave an impression on the world that lingers after all memory of the deeds has faded. Question theorizes that this has occurred in Bazzoxan, and the ancient battles of the Calamity are still echoing here today.

\paragraph{What Question Knows}

Question is eager to show characters her sketchbook. She explains that her drawings are based on research she conducted in the archives of the Cobalt Soul in Ank'Harel, the city where she has lived for the past five years. On the pages of the sketchbook, characters can see the following drawings:

\begin{itemize}
\item A sketch of an island bathed in the moonlight of Catha and Ruidus, both full. The island is covered with trees, marble pillars, and crystal shards and resembles the final chamber in the Emerald Grotto.
\item A rough drawing of what appears to be a more elaborate version of the \textit{Jewel of Three Prayers}. It is larger and has three delicate spires inlaid with stones.
\item A rubbing of a symbol that depicts a long-haired woman peering into the distance. A character who makes a successful DC 12 Intelligence (Religion) check identifies this as the holy symbol of Avandra the Change Bringer. Question explains that this rubbing was taken from a wall carving in the Betrayers' Rise (see area R1 later in the chapter).
\item A folk ballad, written in Undercommon, that speaks of an unnamed "halfway god," a champion who was given so much power by three of the Prime Deities during the Calamity that he became nearly divine himself.
\end{itemize}

\paragraph{Question and the Jewel}

If asked about the \textit{Jewel of Three Prayers}, Question speaks of it with reverent awe, calling it a "lost Vestige." Though she doesn't know the relic's true nature, she believes the jewel was an ordinary amulet until it was filled with divine power during the Calamity.

If the characters show Question the jewel, the tiefling quivers with excitement and immediately requests a closer look. After a minute of inspection, she says that this object is the key to the mythic cycle that resonates throughout Bazzoxan. She begs the characters to contact her again if they learn more about the jewel during their stay in Bazzoxan.

\paragraph{Question's Allies}

Question doesn't have much to say about the other members of her expedition, dismissing them as the sort who prefer excitement over introspection. A character who succeeds on a DC 11 Wisdom (Insight) check discerns that Question is hiding something; if confronted, she admits that she is terrified of going inside the Betrayers' Rise. She explains that after entering the temple with the other expedition members, she heard voices calling to her, which nearly caused her to run away screaming. Instead, she feigned sudden illness and returned to town after promising to make the next trip inside. She expresses concern that her compatriots have yet to return, but she holds out hope that they'll make it back out.

\paragraph{Aloysia Telfan}

During their conversation with Question, characters who have a passive Wisdom (Perception) score of 12 or higher notice a gaunt figure in scarlet robes nearby, eavesdropping. This figure, \textbf{Aloysia Telfan}, is a lawful evil, elf \textbf{occult initiate} (see appendix A (p. 8)) who thirsts for knowledge and power. She approaches the characters, if they don't accost her first, and greets them politely.

Aloysia gets right to the point: she wants to find an ancient prayer site deep inside the Betrayers' Rise---and based on what she has learned of the characters from their conversation with Question, she surmises that such a mission would be beneficial to them as well. She offers to pay the characters 50 gp each to accompany her, though they can get her up to 100 gp each with a successful DC 13 Charisma (Persuasion) check if they haggle. If pressed to reveal what she's after, Aloysia explains that she's looking for a crimson mineral called ruidium that is said to be infused with the power of Ruidus. She says nothing about her organization, the Consortium of the Vermilion Dream, unless someone asks about it by name. In that case, she says that the Consortium is a group of folklore aficionados.

\paragraph{What Aloysia Knows}

Aloysia believes the following facts to be true:

\begin{itemize}
\item Fragmented historical records tell of an unidentified hero of the Calamity who hailed from the lands of Wildemount. He was born under the full moon of Ruidus and wields the vermilion moon's power. (Partially true. Alyxian was marked by Ruidus, but he creates ruidium on his own, not by drawing on the moon's power.)
\item This hero participated in climactic battles across the continents of Wildemount and Marquet---including the battle at the Betrayers' Rise. (True.)
\item The hero died in the battle at the Betrayers' Rise and was brought back to life as a vengeful revenant by Avandra the Change Bringer. (Mostly false. Alyxian nearly died but was healed and granted further power by Avandra.)
\item The hero wore a pendant that contained the power of three gods. This pendant is the key to acquiring the ruidium her organization seeks. (Partially true. The \textit{Jewel of Three Prayers} holds the power of three gods and, in its Exalted State, can be used to open the rift to the Netherdeep in Cael Morrow, as described in chapter 5 (p. 5).)
\end{itemize}

\paragraph{Roleplaying Aloysia}

Assuming the characters are cordial, the ambitious occultist steers the conversation to her interests at every turn, while maintaining a pleasant demeanor and answering just enough of the characters' questions to keep them from walking away.

If the characters turn down her offer, Aloysia's charm vanishes immediately, replaced by cold disdain. Additionally, if any character mentions having received a vision from the Apotheon or reveals that the characters possess the \textit{Jewel of Three Prayers}, Aloysia becomes fixated on obtaining the jewel for her faction.
\section{Betrayers' Rise}

Any time while in Bazzoxan, the characters can choose to travel north to the Betrayers' Rise, which looms over Bazzoxan.

A steep, winding staircase, wide enough for four Medium creatures to walk abreast, leads to a large gate built into the facade of the Betrayers' Rise. Read or paraphrase the following when the characters come close enough to see the details of the structure:

\begin{DndReadAloud}
The Betrayers' Rise is both ominous and beautiful. Its black walls are carved with intricate reliefs, many of them depicting scenes of torture. Lifelike grotesques of leering demons perch high above, peering down at you with contempt.
An immense double door forms the entrance. This gate is made of onyx inlaid with copper and silver. Guards of the Aurora Watch stand between you and the demonic fortress.
\end{DndReadAloud}

The Betrayer Gods used demons to defend this fortress-temple against the Prime Deities. The scenes depicted on the fortress walls depict the Prime Deities being tormented. The outer doors are extremely heavy but well balanced and require a combined Strength of 44 to open or close.

The Aurora Watch has 2d6 \textbf{veterans} (drow, humans, and orcs) stationed outside the Betrayers' Rise at any time. These guards ensure that the doors remain closed to prevent demons and other evil creatures from escaping. At least one character must succeed on a DC 15 Charisma (Deception, Intimidation, or Persuasion) check before the guards will allow one or more characters to get within 30 feet of the outer doors.

\subsection{Gloomstalker Escape}

As the characters examine the Betrayers' Rise and its entrance more closely, read:

\begin{DndReadAloud}
Shouting erupts from the soldiers just as the colossal doors fly open. Two shadowy, winged creatures burst through the opening, immediately take flight, and screech as they dive toward you, their ebon talons extended.
\end{DndReadAloud}

The escaped creatures are two \textbf{gloomstalkers} (see appendix A (p. 8)). The soldiers stationed outside can assist the characters in battling the gloomstalkers, or you can have them rush inside to help three gravely injured Aurora Watch soldiers (lawful good, orc \textbf{knights}) in area R1.

Each gloomstalker fights until it is reduced to half its hit points or fewer, then soars away. After seeing the characters in combat against the gloomstalkers, the Aurora Watch soldiers hold the party in high regard. From now on, the characters can freely enter the Betrayers' Rise whenever they like---including right away, if they so desire. After the characters have either entered the Betrayers' Rise or returned to town, the soldiers tend to their wounded and close the doors.

\subsection{Features of the Betrayers' Rise}

The interior of the Betrayers' Rise has the following features:

\begin{description}
\item[Black Stone.] Unless otherwise noted, all surfaces in the Betrayers' Rise are made of smooth, shiny black stone. The power that pervades the place prevents the architecture from being affected by magic such as \textit{stone shape}.
\item[Ceilings.] Unless otherwise noted, the vaulted ceilings of the temple's chambers are 50 feet high, and its hallways are 20 feet high.
\item[Light.] The interior of the Betrayers' Rise contains no light sources unless otherwise noted.
\item[Shifting Halls.] The interior of the Betrayers' Rise is different for each group that enters. The chaotic magic of the Abyss warps the configuration to stymie explorers and baffle cartographers. The characters experience a particular version of the temple---the one shown on the Betrayers' Rise map. If Aurora Watch soldiers or the rivals enter, they experience a different configuration.
\end{description}

\subsubsection{Expanding the Betrayers' Rise}

If you want to or need to, you can create new maps for the Betrayers' Rise, new encounters there, and other routes through it. The Betrayers' Rise Encounters table provides some suggestions for unsettling creatures that a group might encounter while delving into these new areas, and you can use the treasure tables, p. 7 in chapter 7, p. 7 of the Dungeon Master's Guide to generate treasure guarded by these creatures.

% Table: Betrayers' Rise Encounters
\begin{DndTable}[header={Betrayers' Rise Encounters}]{cX}
d10 & Encounter \\
1 & The characters enter a room lined with flickering candles, only to discover that their shadows have detached from them and transformed into \textbf{shadow demons} (one per character). \\
2 & Four \textbf{ochre jellies} burble along the walls of a hallway, looking for prey to devour. \\
3 & A \textbf{yochlol} in giant spider form skitters around in an antechamber draped in spider webs. \\
4 & A \textbf{yochlol} in drow form wears the insectile armor of the Aurora Watch. Claiming to be a lost Aurora Watch soldier, the demon tries to lure the characters into greater danger. \\
5 & Two \textit{invisible} \textbf{barlguras} ambush the characters. \\
6 & A passageway opens into what was once a coliseum and now contains three enraged \textbf{nightmares}. \\
7 & A gurgling, acidic river cuts through a cavern where three \textbf{ropers} stealthily await. \\
8 & Three \textbf{chasmes} emerge from their hive, which has overtaken what was once a chandelier. \\
9 & A \textbf{glabrezu} tempts all who come into its chamber with promises of riches. Anyone it can't bribe into serving it is attacked. \\
10 & Orange light emanates from the depths of a chasm, where a \textbf{balor} sleeps in molten lava. If it is attacked, the demon wakes. The characters, who have no chance of defeating this powerful demon in combat, must flee or die. \\
\end{DndTable}

\subsection{Retreat and Reinforcements}

To survive the trials of the Betrayers' Rise, the characters might need to retreat to Bazzoxan to regroup and rest. If the characters retreat from the Betrayers' Rise, new demons emerge from the Abyssal portals in the depths of the temple after 8 hours to populate empty areas. Roll a d6 for each area in the Betrayers' Rise that the characters explored on their previous trip. On a roll of 6, the area has become home to one or more new creatures, determined by rolling another d6 and consulting the Betrayers' Rise Reinforcements table.

% Table: Betrayers' Rise Reinforcements
\begin{DndTable}[header={Betrayers' Rise Reinforcements}]{cX}
d6 & Encounter \\
1 & 1d6 \textbf{quasits} nesting in the corpses of the area's previous inhabitants or crawling on the walls in centipede form \\
2 & 1 \textbf{chasme} clinging to the ceiling \\
3 & 1 \textit{invisible} \textbf{barlgura} standing guard \\
4 & 1d4 \textbf{shadow demons} lurking in the darkness \\
5 & 3d6 \textbf{manes} looking for a more powerful demon to lead them \\
6 & 1 \textbf{gloomstalker} (see appendix A (p. 8)) feeding on the corpse of a giant rat and hungry for more food \\
\end{DndTable}

\subsection{Locations in the Betrayers' Rise (R1-R8)}

The following locations are keyed to the Betrayers' Rise map.



\subsubsection{R1: Entrance}

When the characters enter the Betrayers' Rise, read:

\begin{DndReadAloud}
You stand on a bloodstained bridge that extends over a chasm of swirling silver mist. On the far side of the bridge is a closed double door of black stone inscribed with a circle containing the profile of a smiling woman. The symbol is gouged with deep marks, as if something with claws has tried, but failed, to scour it from the stone.
\end{DndReadAloud}

A character who succeeds on a DC 10 Intelligence (Religion) check recognizes the engraving as the symbol of Avandra the Change Bringer. If the characters have the \textit{Jewel of Three Prayers}, add:

\begin{DndReadAloud}
This double door begins to radiate faint crimson light. The light slowly pulses, as if beckoning you to touch it. The \textit{Jewel of Three Prayers} grows warm to the touch and pulses in sync with the glowing door.
\end{DndReadAloud}

The characters can't pass through the double door leading to area R2 unless they have the \textit{Jewel of Three Prayers}. The doors glow whenever the jewel comes within 20 feet of them. While the doors glow, they can be opened with a gentle touch; they can't be opened by any other method. If the characters don't have the jewel, they must turn back and try to acquire it from their rivals, or they can descend into the misty chasm beneath them (see "Misty Chasm" below) if they have the means to do so safely.

\paragraph{Outer Doors}

The Aurora Watch won't allow the outer doors to remain open. If the characters want to leave the Betrayers' Rise after the outer doors are closed, they must open the doors themselves.

\paragraph{Misty Chasm}

Beneath the bridge is an 80-foot-long vertical shaft that opens into a 1,000-foot-deep chasm (see area R15) shrouded in silvery mist. Any character who listens intently hears hissing voices emanating from inside the chasm. The voices say the following in Common:

\begin{DndReadAloud}
"Together. Yes, we can break them together. Come to us. Let us catch you. Let us have you. Come to us. We won't let you fall."
\end{DndReadAloud}

The shaft leading down to the chasm is too sheer to climb without the use of magic or climbing gear. Any character who falls into the chasm takes 70 (20d6) bludgeoning damage from the plunge.

\subsubsection{R2: Hall of Holes}

\begin{DndReadAloud}
The walls of this hallway are covered with carvings that depict a great battle involving mortals, celestials, and fiends. A faint whistling noise emerges from the walls, sounding almost like snoring.
\end{DndReadAloud}

A character who succeeds on a DC 15 Intelligence (History) check recognizes that the wall carvings depict the Battle of the Barbed Fields. This fight was a climactic battle of the Calamity, in which the devotees of the Prime Deities broke through the garrison at the Betrayers' Rise and reached the walls of Ghor Dranas. Prominently depicted in one scene is a proud, melancholy warrior with curly hair and carrying a spear and shield. By his side are two figures; a white-haired girl no more than twelve years old, and a young adult woman with hair that flows behind her, turning into a road upon which countless soldiers march. A character who makes a successful DC 10 Intelligence (Religion) check realizes that the latter two figures are common depictions of the gods Sehanine the Moon Weaver and Avandra the Change Bringer.

No check is needed to identify the source of the snoring sound: it comes from the walls. Close examination reveals that the walls have scores of 1-inch-diameter holes drilled into them. As the characters move along the corridor, eyeballs, teeth, and liquid flesh begin to ooze from the holes, pooling on the floor. After 1 round, this grotesque slurry forms into three \textbf{gibbering mouthers} that attack the characters. One mouther appears at each end of the hall, and the third appears in the middle of the hallway to the west.

\paragraph{Escape Route}

The double door leading to area R1 closes behind the characters unless it is held or wedged open. It glows red when the \textit{Jewel of Three Prayers} is within 20 feet of it. While the double door is glowing, it opens with a touch; otherwise, it can't be opened.

\subsubsection{R3: Vestibule}

\begin{DndReadAloud}
This chamber is empty except for a skull-shaped vase on a small, semicircular table resting against the back wall of an alcove in the north wall. Six narrower alcoves line the east and west walls. The northeast alcove seems to have once led to a hallway, but it has caved in.
\end{DndReadAloud}

A character can clear the rubble blocking the northeast hallway in 1 hour; reduce the time proportionately if other characters assist. If the rubble is cleared, characters can follow the tunnel beyond it that leads to area R8.

\paragraph{Northwest Alcove}

On the back wall of this alcove is scrawled the following message in Abyssal: "To me, touch your hand. To me, relinquish hope. To you, accept death. To you, invite hollowness." A character who places their hand on this wall feels it writhe beneath their touch. If the character succeeds on a DC 13 Wisdom (Insight) check, they understand that the wall seems to want to devour their body, but that the wall can consume only someone who has accepted that they'll die and doesn't struggle against that fate.

While touching the wall, a character can make a DC 15 Wisdom check to make peace with their own death. On a failed check, nothing happens, and the character must rest for at least 1 hour before they can attempt the check again. On a successful check, the wall churns and pulls the character through it, whereupon the character emerges in a narrow hallway (the southern part of area R4). Anyone or anything that character is holding is also pulled through the wall.

A character who succeeds on the check gains the following benefit and drawback:

\begin{DndSidebar}{}
\begin{description}
\item[Benefit.] The character has advantage on saving throws against being \textit{frightened} for the next 24 hours.
\item[Drawback.] The character has disadvantage on death saving throws for the next 24 hours.
\end{description}
\end{DndSidebar}

\paragraph{Secrets of the Skull}

The skull vase and the table in the north alcove are the keys to revealing the hidden entrance to the Basilica of Revelry (area R9). Close examination reveals that the vase contains the withered remains of a bouquet of lilies.

A character who has a passive Wisdom (Perception) score of 13 or higher notices jagged markings along the edge of the table, which make up the following phrase in Abyssal: "Cast that which you value into Oblivion." A character who can read the writing and succeeds on a DC 13 Intelligence (Religion) check recognizes that "Oblivion" refers to Tharizdun, the Betrayer God of emptiness and entropy, also known as the Chained Oblivion.

To open the entrance, a character must place something of value inside the vase, such as a bouquet of fresh flowers, a gemstone, coins totaling at least 25 gp, or anything else the character regards as valuable. Once a satisfactory offering has been placed within the skull-shaped vase, its eye sockets flash red, the offering decays to the point of being worthless, and the wall behind the table slides upward, revealing a long tunnel that eventually leads down to area R12.

\subsubsection{R4: Path of Emptiness}

Characters are \textit{deafened} while in this corridor. The only sound any of them can hear is the beating of their own heart.

\paragraph{One-Way Secret Door}

The wall that lies between area R4 and area R3 is a featureless surface when viewed from this side, offering no way of returning to area R3. This one-way passage can't be navigated, except with a \textit{passwall} spell or similar magic.

\subsubsection{R5: Flagellant's Path}

\begin{DndReadAloud}
You enter a chamber that reeks of death and incense. To the west is an open archway that leads into a small side room, and to the north is a hallway flanked by two statues of swollen, worm-bodied abominations, each with three arms and the hairless head of a howling, fang-toothed man wearing a crown of black spikes.
The hallway past the statues is lined with dozens of curved, quivering blades and ends at a double door. Mounted above the double doors is an eight-inch-tall canister made of translucent crystal.
\end{DndReadAloud}

A character who succeeds on a DC 13 Intelligence (Religion) check recognizes the statues as depictions of Torog the Crawling King, god of enslavement and torture. Characters who have a passive Wisdom (Perception) score of 12 or higher notice a phrase written in Undercommon on the floor between the two statues that reads: "Revel in the pain you inflict on others. Relish the pain you suffer yourself as an offering to the Crawling King."

\paragraph{Bladed Corridor}

The corridor north of the statues has twitching silver blades protruding from its walls. Whenever one or more creatures are in the hall, the floor tilts to one side and then the other so that creatures stumble into the blades. A creature that moves into the corridor or starts its turn there must make a DC 20 Dexterity saving throw, taking 17 (5d6) slashing damage on a failed saving throw, or half as much damage on a successful one.

The first time a character takes damage from the blades, read or paraphrase the following:

\begin{DndReadAloud}
A dim burgundy glow suffuses the corridor, illuminating every razor-sharp bit of metal as dark fluid begins to pour into the canister above the double door.
\end{DndReadAloud}

\paragraph{Opening the Double Door}

The double door at the north end of the corridor can be opened only after the crystal canister is filled with blood or demonic ichor. To meet this requirement, the bladed walls must deal a total of 70 slashing damage to creatures that have blood or demonic ichor in their bodies. A character within reach of the canister can pour blood into it; each ounce of blood poured into the canister reduces by 5 the amount of damage needed to open the doors.

If the crystal canister is destroyed, the characters have no way to disable the hall's trap. The canister is a Tiny object with AC 13, 5 hit points, and immunity to poison and psychic damage.

\paragraph{Trap Reset}

The corridor stops tilting and resets itself after 1 hour or when the double door is opened, whereupon all blood in the crystal canister disappears.

\subsubsection{R6: Cloister of Penitence}

Opening the door to this room releases three hostile \textbf{flameskulls}---the remains of torturers who served the Betrayer Gods long ago.

% Image placeholder: Torturers who served the Betrayer Gods linger on as flameskulls

\begin{DndReadAloud}
The floor of this small room is crusted with dried blood, and the walls are lined with hooks, spiked whips, and other implements of torture or self-flagellation. Slouched in a corner is a desiccated human body in acolyte's robes.
\end{DndReadAloud}

\paragraph{Treasure}

A character who searches the corpse finds a pouch containing 22 gp and four stoppered vials that each contain 1 ounce of blood. This blood can be poured into the canister at the north end of the Flagellant's Path (see area R5 for details).

\subsubsection{R7: Stairs Down}

Five-foot-wide stairs of black stone cling to the walls of a rectangular shaft. The stairs descend 100 feet to area R11.

As the characters make their descent, the sound of distant wailing (emanating from the statue in area R12) grows louder.

\subsubsection{R8: Spiders' Chancel}

\begin{DndReadAloud}
A haunting song echoes through this half-collapsed chamber, interspersed with a clicking sound that seems to reverberate in your bones. Thick spiderwebs, giving off a pale green glow, stretch from floor to ceiling. Crawling in the webs are two giant spiders and two creatures that look like drow but with the lower bodies of monstrous spiders. Four corpses cocooned in webs rest atop the debris near the west wall.
\end{DndReadAloud}

The two \textbf{driders} are former drow warriors of the Aurora Watch mutated by the demonic magic of Lolth the Spider Queen. The two \textbf{giant spiders} in this area have also been tainted, changing their type to Fiend.

The driders and the giant spiders attack intruders on sight, pursuing those that flee.

\paragraph{Hidden Singers}

A \textit{see invisibility} spell or similar magic reveals a mural composed of ghostly figures on the walls of the chamber. The figures all have their mouths open as though singing or screaming. One figure in the middle of the north wall holds a large book that glows faintly and is warm to the touch. Pressing on the book in the mural opens a secret door in the northeast corner of the room, which can also be found by examining that section of wall and succeeding on a DC 20 Intelligence (Investigation) check. Beyond the secret door is a small room lined with moldering, leather-bound books. Each tome contains barely legible scrawlings in Abyssal written by an unsteady hand, detailing the demon lords' alliances with the Betrayer Gods during the Calamity.

A character who spends 10 minutes searching the small chamber finds a book buried in the debris, opened to a page that shows a drawing of a stained-glass window. Written beneath the drawing is the following mantra:

\begin{DndReadAloud}
"First praise be to the Spider Queen, whose kingdom was forsaken. Next, exult the Ruiner, whose spear shook the green earth. In turn, pay homage to Oblivion and the ending of all things. And last, I give myself to the Crawling King."
\end{DndReadAloud}

A character who can read the writing and succeeds on a DC 15 Intelligence check realizes that this is a code or instruction. (It provides the correct sequence of sigils to touch to open the stained-glass mural in area R9.)

\paragraph{Treasure}

The corpses are the remains of an Aurora Watch expedition that was trapped in the Betrayers' Rise. A character who searches the bodies finds an onyx-hilted longsword that has become a \textit{sword of vengeance} haunted by the spirit of its former wielder.

\subsection{Locations in the Betrayers' Rise (R9-R16)}

\subsubsection{R9: Basilica of Revelry}

\begin{DndReadAloud}
Two rows of black stone pillars marbled with white veins support the vaulted ceiling of this hall, the floor of which is polished to a mirror-like sheen. Ornate brass braziers rest at the feet of four of the pillars, and faint music fills the hall.
\end{DndReadAloud}

The music has no readily identifiable source, and the four braziers are bolted to the floor. If one or more characters enter the room, read:

\begin{DndReadAloud}
The four braziers ignite, their bowls filling with pale flames that coalesce into dancing, humanoid forms. The glow from the flames glints off the north wall, where a towering mosaic of stained glass depicts a maelstrom of color, within which are nine symbols. The image shifts even as you stare at it.
\end{DndReadAloud}

Each \textbf{dancing flame} uses the \textbf{succubus}/\textbf{incubus} stat block, with these changes:

% Image placeholder: This stained-glass mosaic is the key to escaping the Basilica of Revelry

\begin{itemize}
\item The flame sheds bright light in a 20-foot radius and dim light for an additional 20 feet. If it leaves the hall or is reduced to 0 hit points, it is destroyed and can't re-form for 24 hours.
\item It lacks the Shapechanger trait and the Etherealness action.
\item Its Claw attack deals fire damage instead of slashing damage.
\end{itemize}

The dancing flames appear to beckon to characters who are closest to them. A character who succeeds on a DC 12 Wisdom (Insight) check intuits, based on the way they move in time with the music, that the flames are looking for dance partners.

A character who uses an action to dance within 5 feet of a flame creature can make a DC 12 Dexterity (Acrobatics) or Charisma (Performance) check. On a successful check, the dancing flame flickers brightly with joy and uses its reaction to fly to the stained-glass mosaic and touch the next symbol in the sequence needed to open it (see "Stained-Glass Mosaic" below). On a failed check, or if a character approaches within 5 feet of a flame but refuses to dance, the flame attacks.

\paragraph{Stained-Glass Mosaic}

The 10-foot-wide section of wall at the north end of the basilica resembles a tall, stained-glass window. Its imagery changes constantly, like the patterns in a kaleidoscope, but the nine symbols remain static. A character who succeeds on a DC 11 Intelligence (Religion) check recognizes these as the symbols of evil gods. The symbols and the gods they represent are summarized in the Symbols of the Gods table.

If four of the symbols are pressed in the correct order, as noted in the Symbol Sequence table, a door opens at the base of the wall, leading to a hallway (area R10). A creature that presses an incorrect symbol must make a DC 16 Wisdom saving throw, taking 28 (8d6) psychic damage on a failed save, or half as much damage on a successful one.

The stained-glass mosaic can be bypassed by breaking through it with a successful DC 12 Strength check or by destroying it. The mosaic has AC 11, 10 hit points, vulnerability to bludgeoning and thunder damage, and immunity to poison and psychic damage. Destroying the mosaic causes the flame creatures in the hall to attack. Moreover, each creature in the hall when the mosaic breaks must make a DC 20 Wisdom saving throw unless it is a Construct, a Fiend, or an Undead. On a failed save, the creature is cursed and can't regain hit points until the effect on that creature is ended by a \textit{remove curse} spell or similar magic.

If destroyed, the stained-glass window repairs itself after 1 hour.

% Table: Symbols of the Gods
\begin{DndTable}[header={Symbols of the Gods}]{Xc}
Symbol & God \\
Crown Made of Ram Horns & Asmodeus \\
Flail of Iron Chains & Bane \\
Bleeding Eye & Gruumsh \\
Spider & Lolth \\
Coiled Serpent & Zehir \\
Jagged Spiral & Tharizdun \\
Severed Hand with an Eye in the Palm & Vecna \\
Trio of Arms & Torog \\
Dragon's Claw & Tiamat \\
\end{DndTable}

% Table: Symbol Sequence
\begin{DndTable}[header={Symbol Sequence}]{cX}
Order & Symbol \\
1st & Spider (Lolth) \\
2nd & Bleeding Eye (Gruumsh) \\
3rd & Jagged Spiral (Tharizdun) \\
4th & Trio of Arms (Torog) \\
\end{DndTable}

\subsubsection{R10: Supplicant's Pit}

\begin{DndReadAloud}
The walls of this hallway are decorated with carvings of supine figures and chained bodies. The floor is covered by a nest of black chains.
The sound of distant wailing comes from a dark, circular hole in the floor at the north end of the hall.
\end{DndReadAloud}

A creature that enters this hall immediately feels weighed down and senses a terrible dread clawing at the back of its mind. The creature must make a DC 14 Wisdom saving throw and succeeds automatically if it is immune to the \textit{charmed} or \textit{frightened} condition. On a failed save, the creature falls \textit{prone} and its speed is reduced to 0 as it is forced to grovel before an unseen presence. The creature can repeat the saving throw at the end of each of its turns, ending the effect on itself on a success. Once a creature succeeds on the saving throw, it gains immunity to this effect for the next 24 hours.

\paragraph{Chains of Darkness}

When a creature falls \textit{prone} in this hall, the chains on the floor slither toward it, and the figures carved in the walls drone in low voices, "Imprison those who cannot resist you. Drag all life into darkness." The creature must succeed on a DC 15 Dexterity saving throw or be \textit{grappled} by the chains (escape DC 15). Until it escapes, the creature takes 5 (1d10) psychic damage at the start of each of its turns. Any creature that is not \textit{grappled} by the chains can use an action to try to free a chained creature within reach, doing so with a successful DC 15 Strength (Athletics) check. Whatever the result of the check, the creature making the attempt takes 11 (2d10) psychic damage.

\paragraph{Circular Hole}

The circular hole at the north end of the hall is the mouth of a shaft that descends 10 feet before opening in the ceiling of area R12. The distance from the floor of this hallway to the floor of area R12 is 100 feet. A character who falls down the hole can try to catch the outstretched hand of the northernmost stone arm in area R12, doing so with a successful DC 15 Dexterity saving throw and reducing the fall from 100 feet to 30 feet. The character can climb down the rest of the way with a successful DC 15 Strength (Athletics) check.

\subsubsection{R11: Stairs Up}

Five-foot-wide stairs of black stone cling to the walls of a rectangular shaft. The stairs ascend 100 feet to area R7.

From this area, the wailing in R12 is loud and impossible to ignore.

\subsubsection{R12: Threshold of the Excoriated}

\begin{DndReadAloud}
The awful wailing noise gets even louder as you approach this chamber. A beam of light from above illuminates a statue in the middle of the room. The statue, made of pale marble, is shaped like a man on his knees, arms pinioned behind his back, trussed up in chains that cut into his skin. Hooks line the corners of the figure's mouth, pulling his lips away from his teeth, and smaller barbs encircle his eyes, holding the eyelids open. All the while, hideous noise comes out of the statue's open mouth---yet still, its expression is one of ecstasy. A double door made of iron is set into the statue's chest; these surfaces have no handles.
A triangle, each side forty feet long, is carved into the floor around the statue, surrounding it. At each of the triangle's points, a stone arm juts up from the floor with its hand outstretched.
\end{DndReadAloud}

A \textit{detect magic} spell reveals an aura of abjuration magic emanating from the iron doors in the statue's chest. Similar auras are apparent in the palms of the stone arms that are within the spell's range.

\paragraph{Wailing Statue}

Any creature within 30 feet of the statue that can hear the wailing sound coming from it must make a successful DC 16 Charisma saving throw or become cursed by Torog's hunger. A creature that succeeds on this saving throw is immune to the effect of the wailing statue for 24 hours. The curse has the following effects on a creature:

\begin{itemize}
\item At the start of each of its turns, the creature must make a DC 16 Wisdom saving throw. On a failed save, the creature must use its action to harm another creature it can see, using a spell or an attack that deals the greatest amount of damage.
\item While the creature has fewer than half its hit points, its successful weapon attacks deal an extra 3 (1d6) psychic damage.
\end{itemize}

The creature can repeat the saving throw at the end of each of its turns, ending the effect on itself on a success. Otherwise, the curse of Torog remains in effect on the creature until ended by a \textit{remove curse} spell or similar magic, or until the creature enters area R16.

% Image placeholder: A giant wailing statue guards the path to the Prayer Site of Avandra, but learning its secret is no simple matter

\paragraph{Arms of Torog}

The northernmost of these three statues is 70 feet tall, the one to the east is 40 feet tall, and the one to the west is 20 feet tall. Each arm ends in an outstretched hand.

Characters who have a passive Wisdom (Perception) score of 13 or higher notice that each one of the arms bears a throbbing, purple-black pustule on its palm. Each pustule has AC 17, 25 hit points, and immunity to poison and psychic damage. When a pustule is reduced to 0 hit points, it is destroyed, and one side of the triangle engraved on the floor begins to glow white. Destroying all three pustules lights up the entire triangle and unlocks the doors embedded in the statue's chest. If the pustules are all destroyed, they re-form after 1 hour, at which point the doors close and become locked again unless they are held open or are opened from the inside.

\paragraph{Iron Doors}

The doors in the statue's chest open only if the three pustules on the palms of the stone arms have been destroyed. Beyond the doors is a narrow passage that descends to area R16.

\paragraph{South Hallway}

When one or more characters enter the hallway to the south, they hear a loud grinding noise. This is the sound of a stone slab rising from the floor to block the entrance to area R14. This slab has no handle or keyhole. Defeating the fiendish orcs in area R14 or using a \textit{knock} spell or similar magic causes the slab to sink back into the floor; it can't be moved or breached otherwise.

\subsubsection{R13: Lightless Labyrinth}

This network of 5-foot-wide, 10-foot-high tunnels zigzags in incomprehensible directions.

\paragraph{Entrances}

The tunnel leading north from this labyrinth connects with area R12. The south entrance is a 3-foot-tall crawlway that ends 45 feet above the floor of area R14.

\paragraph{Navigating the Labyrinth}

When the characters enter the labyrinth, they're faced with a choice of three paths: one lit by violet \textit{continual flame} spells, one by red \textit{continual flame} spells, and one shrouded in impenetrable magical darkness. Following any of the paths leads them to another three-way fork with the same choices available, and then another. The way to get through the labyrinth most quickly is always to take the fork shrouded in darkness.

After choosing three paths, the characters emerge at one of the entrances. If they chose the darkened path all three times, they emerge at the opposite side from where they entered. If they chose any of the illuminated paths, they emerge at the same place where they entered, and each character must succeed on a DC 15 Constitution saving throw or gain 1 level of \textit{exhaustion} as the endlessly zigzagging pathways drain their resolve.

\paragraph{Hints}

A character who makes a successful DC 14 Intelligence (Religion) check recalls one of Torog's commandments: "Seek and exalt places where no light touches." If they aren't sure what that means, a character can guess that the darkened passages of the labyrinth are related to this commandment by making a successful DC 10 Wisdom (Insight) check.

\subsubsection{R14: Blood Font of the Ruiner}

\begin{DndReadAloud}
This chamber has a flat, fifty-foot-high ceiling, and its black stone walls are covered with deep gouges and lined with alcoves to the east and west. In the middle of the area is a circular pool filled to the brim with blood. Standing in the pool are two seven-foot-tall orcs, one pouring blood from a golden bowl over the other's head. The creatures snarl and turn to you.
\end{DndReadAloud}

The \textbf{creatures} are two \textbf{orc war chiefs} that were once valiant soldiers of the Aurora Watch. Both have been corrupted by demonic magic and have the following changes to their statistics:

\begin{itemize}
\item Their creature type is Fiend.
\item They speak Abyssal as well as Common and Orc.
\item They have immunity to fire and poison damage.
\end{itemize}

On their first turn in combat, the orcs bellow a prayer to Gruumsh in Abyssal, activating the power of the blood font.

\paragraph{Blood Font}

This pool is 2 feet deep and filled with blood kept fresh by demonic magic. When the orcs pray to Gruumsh, they activate the blood font, causing the following effects to occur:

\begin{itemize}
\item Stone slabs rise from the floor to block the north and south exits (unless both slabs were already closed). A \textit{knock} spell or similar magic can force a slab to sink back into the floor.
\item When both exits are sealed after the orcs pray to Gruumsh, the pool of blood begins to boil and overflow. The boiling blood starts to flood the room, rising at a rate of 5 feet per minute to a maximum depth of 45 feet.
\item A creature that ends its turn in the boiling blood takes 7 (2d6) fire damage.
\end{itemize}

Killing both orcs or forcing open either exit causes the boiling blood to drain out of the room through holes that open in the floor, lowering the blood's depth by 5 feet per round.

\paragraph{Crawlway}

A character who searches for a way out of the sealed-off room can use an action to make a DC 14 Wisdom (Perception) check. On a successful check, the character spots a 3-foot-high opening in the back wall of the southeast alcove, 45 feet above the ground. The opening is the entrance to a crawlway that leads to area R13. Climbing the wall to reach this opening requires a successful DC 12 Strength (Athletics) check.

\subsubsection{R15: Misty Chasm}

A short hallway connects area R14 with the chasm south of it. The archway at the south end of the corridor is carved to resemble the open-mouthed, fanged visage of Gruumsh the Ruiner when seen from inside the chasm. Fire fills one of the visage's eye sockets, courtesy of a \textit{continual flame} spell.

Describe the stone-carved face as follows when one or more characters can see it:

\begin{DndReadAloud}
A snarling face of stone is carved into the north wall of the foggy chasm, its gaping mouth forming the entrance to a hallway. Fire blazes in one of the face's eye sockets, and its fangs are stained with blood.
\end{DndReadAloud}

The 1,000-foot-deep chasm swirls with fog that makes the entire area Vision and Light. The carving of Gruumsh's face is 100 feet below the opening in the ceiling to area R1 and 900 feet above the chasm's floor.

When a creature passes through the gaping mouth of the stone-carved face, a planar rift opens in the floor on the east side of the chasm (see "Abyssal Rift" below), and a hostile \textbf{vrock} flies out of it. The vrock begins searching the chasm for food, attacking the characters on sight.

\paragraph{Abyssal Rift}

The planar rift is a gateway to the Abyss and looks like a 10-foot-wide, 20-foot-long orifice filled with impenetrable darkness. Horrible shrieks emanate from the rift and can be heard throughout the chasm for as long as the rift remains open. The rift closes after 1 minute, but a new one appears whenever a creature passes through the carving of Gruumsh's face.

A creature that enters the rift from this side is magically transported, along with anything it is wearing or carrying, to the Abyss. The creature emerges from a similar rift located inside a colossal cyclone of poisonous gas. Trapped in the cyclone are the shrieking spirits of countless evil souls condemned to the Abyss. Demons and other creatures are sometimes swept up in the cyclone as well, where they remain trapped until a planar rift appears and they can escape through it. Even if the rift closes, the cyclone remains.

A creature caught in the cyclone is trapped there and takes 14 (4d6) poison damage at the start of each of its turns. A creature that has a flying speed can escape the cyclone by flying into the rift; a creature without a flying speed can use an action to try to harness the winds of the cyclone, propelling itself through the rift with a successful DC 19 Dexterity (Acrobatics) check.

\paragraph{Climbing the Chasm Walls}

The chasm's walls are honeycombed with niches, each no more than 5 feet wide and 10 feet deep, set between stucco statues of people bearing expressions of horror and pain. A character can climb a section of wall without magic or climbing gear by succeeding on a DC 10 Strength (Athletics) check. A failed check indicates that no progress is made; if a check fails by 5 or more, the character slips and plunges to the cavern floor, taking damage from the fall as normal.

\paragraph{Bodies of the Fallen}

Lying in a heap on the chasm floor directly below the shaft that leads to area R1 are a dozen corpses. These are the remains of people who fell to their doom. Most are clad in the chitinous armor of the Aurora Watch. Two of the bodies look more recently deceased and wear the blue garb of the Cobalt Soul; these are the unlucky allies of Question the tiefling.

\paragraph{Treasure}

A character who inspects the remains of the two Cobalt Soul explorers finds a \textit{cloak of protection} and a \textit{cracked driftglobe}. In its damaged state, the \textit{driftglobe} can be used only to cast the \textit{light} spell.

\subsubsection{R16: Prayer Site of Avandra}

The only way to access this chamber is by going through the doors in the statue in area R12. When the characters follow the passageway inside the statue, read:

\begin{DndReadAloud}
The passage defies reality, distorting any sense of direction as it twists and turns, making its way through the earth to another place deep inside the Betrayers' Rise. The passage ends on the east side of a circular chamber. Its walls are covered with softly glowing amber-colored crystals that fill a crater at the room's center. At the bottom of the crater sits an altar made of the same crystal.
\end{DndReadAloud}

% Image placeholder: The key to awakening the {@item Jewel of Three Prayers|CRCotN} lies at the bottom of this crystal crater

This location is the prayer site where Avandra saved Alyxian from death and granted him the ability to stand against the Betrayer Gods. The crater is roughly 30 feet deep. At its base is a cleared area about 10 feet in diameter. A creature can climb down the crater's crystal slope with a successful DC 13 Strength (Athletics) check. On a failed check, the creature slips and tumbles to the bottom, taking 10 (3d6) slashing damage from the sharp crystals.

\paragraph{Curse of the Crawling King}

Creatures that were affected by the curse of Torog in area R12 can rid themselves of the curse by entering this area. If any of the characters were cursed when they arrived here, describe the end of the effect as follows:

\begin{DndReadAloud}
Your incessant urge to kill is gone, replaced by a torrent of noise that coalesces into a threatening scream.
"Change Bringer! Your luck shall not hold!"
The voice echoes in your mind and then becomes silent. The sheltering embrace of Avandra's lingering presence has dispelled the curse of the Crawling King.
\end{DndReadAloud}

\paragraph{The Jewel Awakens}

If a character who has the \textit{Jewel of Three Prayers} reaches the bottom of the crater, the jewel tugs the character in the direction of the altar, as if it wants to be placed there. If the jewel is placed on the altar, all the characters in this chamber experience the following vision:

\begin{DndReadAloud}
A spectral figure blossoms forth from the amulet. This warrior, who called himself Alyxian in your previous vision, looks up at you with a faint smile and says, "You came... you followed... you can find me." The crystals around you glow brightly, and your vision goes white. Then you see a company of armored soldiers, Alyxian among them, marching into the Betrayers' Rise. Their expressions are grim. Though they do not speak, you can tell that none of them expects to come home.
The vision shifts again, and everyone in the company of soldiers is dead except for Alyxian, who continues to fight while surrounded by the corpses of mortals and demons. He is hurled across the battlefield by the claws of a gorilla-like demon. The scene blurs, and he falls to his knees, pleading, before a simple wooden altar that bears the holy symbol of Avandra. At the end of his prayer, he's lifted to his feet by a tall young woman with light brown skin and flowing black hair.
A foreboding voice cuts through this scene of divine intervention, and Alyxian freezes. "The Change Bringer and a mortal," the voice intones. "What brings you into my sacred, devouring darkness? Even if you save him, Change Bringer, he will suffer. He will die. He will be forgotten. All that he has worked to save will crumble and be devoured by worms. Why invite further suffering? Let go. Give in. All is futile in the end, so why..."
As the voice trails off, Alyxian turns to face you and falls to his knees in tears. The scenery shifts, and the forsaken warrior is now underwater, amid the ruins of a sunken city. "The Crawling King spoke true," he mutters. "All was futile. All has been forgotten... and I am lost in darkness. Please, help me set it right."
\end{DndReadAloud}

The final scene of the vision takes place in the Drowned City of Cael Morrow; recalling the appearance of this location will help the characters as they seek more information about the Apotheon in chapter 4 (p. 4).

The vision persists long enough for the characters to speak to Alyxian for a few moments. Once the characters have had three questions answered, the vision ends as new arrivals come on the scene (see "Confrontation for the Jewel" below). Alyxian addresses the characters' questions with the following information:

\begin{itemize}
\item Alyxian was born in the lands of Wildemount under full Ruidus, which cursed him to endure a life of suffering. He decided to spend his life shielding others from pain, since he was already doomed to suffer.
\item He fought in the Calamity and was called the Apotheon because he beseeched the gods for power three times and three times received it---first on the shores of Wildemount, again in the depths of the Betrayers' Rise, and once more in a temple of Corellon the Arch Heart in a city of elves and orcs in the jungles of Marquet. These divine gifts transformed him from a mortal being into something more. Alyxian says that the city of elves and orcs was destroyed centuries ago. (This is partly true; the city Alyxian speaks of is now the underwater ruin called Cael Morrow, described in chapter 5 (p. 5).)
\item Alyxian is imprisoned in an underwater place lit by a crimson glow. He grows angry as he explains that mortals are harvesting the source of the glow and leaving him in darkness, and that this act is torture for him. (This is both true and an indication of the Apotheon's emotional state; his fury is simmering beneath his sorrowful demeanor.)
\end{itemize}

Alyxian answers other questions as well as he can, drawing from information in the "Story Overview" section of the introduction (p. 0). He portrays himself as sympathetic and helpless.

The next time the characters look at the \textit{Jewel of Three Prayers}, they find it has transformed into its \textit{Awakened State} (as described in appendix B (p. 9)). The transformed jewel can be removed from the altar safely.

\subsection{Confrontation for the Jewel}

The following boxed text assumes that \textbf{Aloysia Telfan} is still alive; if that's not the case, replace Aloysia with another \textbf{occult initiate} (see appendix A (p. 8)) from the Consortium of the Vermilion Dream:

\begin{DndReadAloud}
Just as the vision of Alyxian fades, you become aware of the presence of intruders. Your rivals appear at the edge of the crater, along with a red-robed figure. "Well done, heroes!" says \textbf{Aloysia Telfan}. "Now, if you would simply hand over the jewel, we can all get out of here without spilling any more blood."
\end{DndReadAloud}

Aloysia and the rivals did not use the same route as the characters to reach area R16. They were transported to the crater's edge by a magical force they encountered in their version of the Betrayers' Rise.

Aloysia is determined to claim the \textit{Jewel of Three Prayers}. The rival adventurers stand arrayed behind her with their hands on their weapons.

If characters refuse to surrender the jewel, Aloysia orders the rivals to kill them and take the jewel by force.

Before running this encounter, consider who among the rivals is friendly, indifferent, or hostile toward the characters. The rivals' overall attitude determines how they respond to Aloysia's order, and if there is a tie, Ayo Jabe's opinion breaks it:

\begin{description}
\item[Friendly Rivals.] Ayo Jabe congratulates the characters for reaching the prayer site and awakening the \textit{Jewel of Three Prayers}. The rivals not only allow the characters to keep the jewel but also offer to help them return to Bazzoxan. Aloysia is shocked by the rivals' betrayal.
\item[Indifferent Rivals.] The rivals try to knock the characters \textit{unconscious}, take the jewel, and give it to Aloysia.
\item[Hostile Rivals.] Ayo strides to the edge of the crater and snarls, "I guess this is the end of your hunt, huh?" The rivals attack with intent to kill.
\end{description}

\subsubsection{Aloysia's Next Move}

Aloysia carries a satchel that holds three \textit{teleportation tablets} (see appendix B (p. 9)), all connected to the same teleportation circle in Ank'Harel (see chapter 4). Her next actions are determined by who winds up with the \textit{Jewel of Three Prayers}.

\paragraph{Aloysia Gets the Jewel}

If Aloysia gets the jewel, she uses a \textit{teleportation tablet} to transport herself to Ank'Harel. She invites the rivals to follow her, assuming they didn't betray her; any creature that enters the circle before the start of Aloysia's next turn is also teleported.

\paragraph{Aloysia Doesn't Get the Jewel}

If the characters retain possession of the jewel, Aloysia loses her temper and begins ranting on her next turn. Midway through her rant, she pulls a scroll from her belt. The scroll is a \textit{spell scroll} of \textit{earthquake}, which she casts (choosing the Structures option) as an action on her turn. Aloysia then flees up the passageway, accidentally dropping two \textit{teleportation tablets} while she rummages through her satchel. A huge chunk of stone falls from the ceiling behind her, separating her from the characters and blocking their only exit. On her next turn, she uses her third and final \textit{teleportation tablet} to escape to Ank'Harel.

Each creature in the prayer site when Aloysia casts \textit{earthquake} must make a DC 13 Dexterity saving throw, taking 17 (5d6) bludgeoning damage from falling rubble on a failed save, or half as much damage on a successful one.

The boulder that seals off the prayer site can be cleared if creatures with a combined Strength of 70 or higher spend a total of 10 hours working to move it, or if creatures with a combined Strength score of 50 or higher spend a total of 20 hours working to move it.
\section{Next Steps}

The characters' choices at the end of this chapter determine which allies and enemies await their arrival in Ank'Harel.

If the rivals took the \textit{Jewel of Three Prayers}, or if Aloysia fled to Ank'Harel on her own, the characters can use Aloysia's \textit{teleportation tablets} (see appendix B (p. 9)) to travel to Ank'Harel directly, without backtracking or spending more time in Bazzoxan. Otherwise, they can find their way out of the Betrayers' Rise, return to Bazzoxan, and plan their next move with or without the help of the following nonplayer characters:

\begin{description}
\item[Prolix Yusaf.] Prolix has his own \textit{teleportation tablet}, which he can use to teleport himself, the characters, and Question the tiefling (see below) to Ank'Harel. Prolix offers to introduce the characters to his peers in the Allegiance of Allsight, who can help the characters learn more about the Apotheon and the \textit{Jewel of Three Prayers}.
\item[Question.] If the characters tell Question what happened in the prayer site of Avandra, she is happy to learn what they discovered but sad to hear that her Cobalt Soul colleagues were lost. That news makes her determined to ensure that her colleagues' deaths were not in vain. She tells the characters that her organization could use people of their skills and invites them to Ank'Harel to report their findings to the monks of the Cobalt Soul. She says she has made plans with \textbf{Prolix Yusaf} to return to Ank'Harel together if the Library of the Cobalt Soul's expedition was lost.
\item[Taskhand Verin Thelyss.] If the characters report to Verin, he is eager to hear about their experiences in the Betrayers' Rise. If they show him one of Aloysia's \textit{teleportation tablets}, Verin has a mage of the Aurora Watch cast \textit{identify} on the item. The mage says that the tablet is linked to a teleportation circle on the continent of Marquet. Verin encourages the characters to travel there, telling them that he believes their fates are somehow intertwined with "divine forces that can't be ignored." He bids them farewell and looks forward to seeing them again, after their adventure has concluded.
\end{description}

\chapter{The Jewel of Hope}\label{ch:the-jewel-of-hope-5-5}

The gloom of Bazzoxan gives way to the sunlight and warm winds of Marquet as the characters arrive in the metropolis of Ank'Harel. While they get their bearings in a new city on a new continent, the characters discover potential allies among the factions of the city and plumb the mystery of the Apotheon, an investigation that culminates in an expedition into the sunken city beneath Ank'Harel's foundations.
% Content: Unknown (dict)
\section{Running This Chapter}

This chapter has three main parts:

\paragraph{Arrival in Ank'Harel}

This section describes the ways in which the characters enter the city.

\paragraph{Ank'Harel Gazetteer}

The gazetteer is a district-by-district description of Ank'Harel that you can use to bring the city to life. Use it to improvise situations as the characters move from place to place or engage in self-directed side quests.

\begin{description}
\item[Faction Story Tracks.] Each of the three major factions that the characters can join has an episodic story track that details how the characters can advance the cause of that faction. These sections consist of "What Lies Beneath" (for the Allegiance of Allsight), "Vermilion Gambits" (for the Consortium of the Vermilion Dream), and "Knowledge Is Power" (for the Library of the Cobalt Soul).
\end{description}

\subsection{Faction Missions}

By completing missions for their chosen faction, the characters learn about the corrupting element ruidium, the existence of the Drowned City of Cael Morrow, and the existence of the Netherdeep. The missions prompt characters to explore these regions in exchange for rewards from their faction.

Each faction's story track consists of six missions, the first three of which take place in Ank'Harel, referencing locations described in the "Ank'Harel Gazetteer" section. The fourth and fifth missions take the characters to Cael Morrow, which is described in chapter 5 (p. 5). The sixth mission takes place in the Netherdeep, which is detailed in chapter 6 (p. 6).

\subsubsection{Joining a Faction}

Once the characters have successfully completed the first two missions for a faction, that faction invites them to become full members. The faction won't offer them any more missions unless the characters join, because the information in the last four missions of each story track is too privileged to be revealed to common sellswords.

\subsubsection{Joining Multiple Factions}

Typically, a character is associated with only one faction at a time. But if the characters want to change their allegiances or engage in espionage, they can try to join a second faction.
\section{Rivals in Ank'Harel}

The following descriptions of the rivals' actions during this chapter assume that the rivals came to Ank'Harel by means of a \textit{teleportation tablet} (see appendix B (p. 9)):

\begin{description}
\item[Friendly Rivals.] If the rivals are predominantly friendly toward the characters, they cordially part ways after the "Arrival in Ank'Harel" scenario. The rivals find an inn that they make their semi-permanent residence, and a few days later Galsariad shares the address with the characters using a \textit{spell scroll} of \textit{sending}.
\item[Indifferent Rivals.] If the rivals are predominantly indifferent toward the characters, they say farewell after the "Arrival in Ank'Harel" scenario and make no attempt to contact the characters thereafter. Rivals who are friendly with one character might still send an invitation to have lunch with that character from time to time.
\item[Hostile Rivals.] If the rivals are predominantly hostile toward the characters, they part company with the characters as soon as they are able. Rivals who are still friendly with one character might sheepishly, secretively try to meet that character in the dark of night to swap stories.
\end{description}

\subsection{Rival Developments}

The rivals are shaken by their experience in the Betrayers' Rise and have evolved in the following ways:

\paragraph{Ayo Jabe (Tier 2)}

If the rivals allied with Aloysia, Ayo worries that the group has lost respect for Ayo's leadership and questions her ability to make wise decisions.

\paragraph{Dermot Wurder (Tier 2)}

Dermot fears that Ayo is faltering under the responsibilities of leadership and is being overly kind in an attempt to keep her happy. He is also worried by the way she has withdrawn from their longstanding friendship.

\paragraph{Galsariad Ardyth (Tier 2)}

Galsariad shuts himself off from everyone. He feels underqualified for everything his group has encountered, and he stubbornly refuses help as he continues trying to prove his worth as an adventurer.

\paragraph{Irvan Wastewalker (Tier 2)}

The possibility of traveling to another continent---far removed from the Luxon beacons---terrifies Irvan. But his commitment to his companions outweighs his fear, and he chooses to stay with them and face his mortality head-on.

\paragraph{Maggie Keeneyes (Tier 2)}

Always pensive, Maggie becomes even quieter after the events in the Betrayers' Rise. Her mind whirls as she tries to understand the hidden forces behind the jewel, the temple, and the factions they've encountered thus far.
\section{Arrival in Ank'Harel}

Located thousands of miles from Wildemount, Ank'Harel is vibrant and alive, a sharp contrast to the war-scarred land the characters have left behind. Instead of the grim towers and besieged houses of Bazzoxan, the characters now find themselves in a realm of sprawling roads and towering structures humming with life, all surrounded by the golden dunes of the Marquesian desert. Ank'Harel is a bustling metropolis, and the characters will find no shortage of opportunities for adventure within its walls.

The characters arrive in Ank'Harel in one of two ways. Continue with either "Following Aloysia" or "Arriving with Prolix" below.

\subsection{Following Aloysia}

Characters who use one of \textbf{Aloysia Telfan}'s \textit{teleportation tablets} to travel to Ank'Harel arrive on a rooftop overlooking the Suncut Bazaar (see area T8 on the Streets of the Suncut Bazaar map (p. 4)). Read:

\begin{DndReadAloud}
The chilly air and gloomy clouds of Bazzoxan are replaced by dry heat and a brilliant blue sky. You are standing on a rooftop overlooking a bazaar. Beneath your feet is a teleportation circle, from which the glow of magic is slowly subsiding.
\end{DndReadAloud}

If the characters did not follow closely behind Aloysia, there is no one here to greet them. They must wander the city and find their own way around. They can purchase a map of Ank'Harel (use the poster map (p. 13)) for 10 gp in any shop that caters to tourists.

If the characters arrive just moments behind Aloysia, she tries to flee at the first sight of them in Ank'Harel, hoping to lose her pursuers in the Suncut Bazaar. If the characters give chase, use the "Chases, p. 8" section and the urban chase complications in the Dungeon Master's Guide to determine whether Aloysia escapes. If caught, she frantically offers the characters employment with the Consortium of the Vermilion Dream. The Consortium, she says, can provide them with money, lodging, and opportunity. If the characters express interest, she offers to lead them through the Suncut Bazaar to a tavern named First Eclipse.

\subsubsection{Meeting the Consortium}

Though it appears to be an ordinary, well-appointed tavern, First Eclipse is the Consortium's hidden base of operations in Ank'Harel (see the "Suncut Bazaar" section). If the characters allow Aloysia to guide them here, read:

\begin{DndReadAloud}
Aloysia leads you through the crowded tavern to a wall decorated with old cask lids. She raps her knuckles three times on a lid painted with a red crescent moon, causing a secret door in the wall behind it to open. Through the opening, you see a private storeroom filled with cabinets and corked bottles of wine in wooden racks.
\end{DndReadAloud}

Aloysia leads the characters inside the storeroom, then asks them to wait while she goes to get someone. After a few minutes, she returns with one of the Consortium's leaders: Vrill the Moth, a lawful evil, elf \textbf{occult silvertongue} (see appendix A (p. 8)).

Hailing from Wildemount's Menagerie Coast, Vrill is a white-haired elf with eyes that seem to penetrate every creature and object she looks upon. She shields herself from the sun with a floppy red hat. Though Vrill regards the characters with haughty disdain, Aloysia's recounting of events in Bazzoxan has piqued her interest enough to grant the characters a brief audience.

Vrill immediately asks the characters to show her the \textit{Jewel of Three Prayers}, and she returns the jewel to the characters after examining it. She correctly identifies the jewel as a Vestige of Divergence in its Awakened State and asks how the characters acquired it.

If the characters mention the vision they received from the Apotheon, Vrill's haughty demeanor changes to one of curiosity. She explains that her organization's research has turned up fragments of the ancient history of the Apotheon---a legendary figure said to have been blessed by Ruidus, the Moon of Ill Omen.

\subsubsection{Developments}

Vrill offers the characters a place to stay at First Eclipse until they find lodging of their own. She also gives them a map of Ank'Harel (use the poster map (p. 13)), then excuses herself, bidding Aloysia to follow. Before leaving the characters to their own devices, Vrill says to them, "The Consortium will be in touch again after you find your own lodgings. No need to tell us where you're staying. We will find you."

This branch of the adventure continues in "Vermilion Gambits," the story track tied to the Consortium of the Vermilion Dream.

\subsection{Arriving with Prolix}

If Prolix used his \textit{teleportation tablet} to transport himself and the characters to Ank'Harel, the group arrives inside the Teleportation Atrium of the Crystal Chateau (area S2 on the Streets of the Sigil District map (p. 4)). If Question the tiefling is with them, she arrives here as well. Read:

\begin{DndReadAloud}
The chilly air and gloomy clouds of Bazzoxan are replaced by dry heat and a brilliant blue sky visible through the arched window of a magnificent building with walls of white marble. Beneath your feet is a teleportation circle, from which the glow of magic is slowly subsiding.
\end{DndReadAloud}

Prolix asks the characters if they'd be comfortable meeting with one of his superiors. If they say yes, continue with the "Meeting the Allegiance" section below. Otherwise, they are free to explore Ank'Harel and secure lodging for themselves.

\subsubsection{Friends of the Cobalt Soul}

If Question is present, she thanks Prolix for helping her return to Ank'Harel and tells the characters that she must inform her superiors at the Library of the Cobalt Soul about the loss of her expedition---and about the characters' involvement in Bazzoxan. "Only good things," she reassures them cheerily.

Before leaving their company, Question tells the characters that they'll be called upon in a few days' time and gives them one stone from a pair of \textit{sending stones} that they can use when that time comes. She says that her superior, Iwo Zalarre, will probably be the one to contact them through the stone. She also entrusts them with a map of Ank'Harel (use the poster map (p. 13)), so they can become familiar with the city during their stay. She then catches a carriage to the Guided District. This branch of the adventure continues in "Knowledge Is Power," the story track for the Library of the Cobalt Soul.

\subsubsection{Meeting the Allegiance}

Prolix leads the characters across the Sigil District, which features well-watered green grass and beautiful fig trees, as well as dozens of buildings. They arrive at Teres Schoolhouse, one of two universities run by the Allegiance of Allsight in Ank'Harel (see the "Sigil District" section later in the chapter).

Prolix guides the characters across the school's campus and into one of the administrative buildings:

\begin{DndReadAloud}
Prolix leads you to a structure the size of a small house. The bottom half is the same brick that makes up many of the older buildings in the neighborhood, but halfway up the outside, the walls turn to delicate stonework, as if someone had built a new floor on top of an old foundation. Glittering spires of iron and crystal adorn its sloping roof, and silver lettering across the door reads, "Headmaster Alakritos."
\end{DndReadAloud}

The more charitable of the Allegiance of Allsight's two leaders is Alakritos, a neutral good, goblin \textbf{scholarly mastermind} (see appendix A (p. 8)). The headmaster is pleased to see that Prolix has returned from Bazzoxan safely and offers tea as he listens to Prolix and the characters talk.

Prolix asks the characters to recount their experiences with Aloysia in Bazzoxan. Alakritos listens to them quietly, pondering what the \textit{Jewel of Three Prayers} must mean to the Consortium if it was willing to go to such lengths to obtain it.

If the characters have the jewel, Alakritos asks to examine it. He successfully identifies it as a Vestige of Divergence, noting that though it is not directly tied to any deity, it nonetheless bears the hallmarks of divine creation. If the characters mention their vision, Alakritos claims to know nothing about it, but a character who succeeds on a DC 16 Wisdom (Insight) check discerns that the mention of the sunken city depicted in the vision sparked recognition in his eyes. If confronted, Alakritos admits that there are sunken ruins beneath Ank'Harel's oasis, though access to these ruins is limited to members of the Allegiance of Allsight. If pressed further, he holds up his hands and says, "All in good time. I've already said more than I ought to a group of untested hopefuls."

\subsubsection{Developments}

Alakritos fields questions about Ank'Harel in general (recounting information from the "Ank'Harel Gazetteer" below). When the session with the headmaster is over, Prolix offers to let the characters stay in his apartment for the night before they start to find their way around this new city. Prolix also hands them a map of Ank'Harel (use the poster map) (p. 13), if they haven't already received one.

A few days later, Prolix visits them at whatever new lodgings they've found and extends an invitation to undertake a mission for the Allegiance of Allsight. This branch of the adventure continues in "What Lies Beneath," the Allegiance Story Track.
\section{Ank'Harel Gazetteer}

Ank'Harel translates to "Jewel of Hope" in the northern dialect of the Marquesian language. The city was established as an oasis trading post. A rare source of underground fresh water made this one of the few habitable locations in the vast desert.

As more travelers arrived to take part in building this metropolis, unpaved roads were upgraded to stone streets, and walls were bolstered with towers and parapets. Ank'Harel flourished, becoming a cultural melting pot of people from across the continent and some from farther away.

\subsection{Ank'Harel at a Glance}

\begin{description}
\item[Population.] 293,500 (predominantly humans, orcs, and elves)
\item[Leader.] J'mon Sa Ord, a lawful neutral \textbf{ancient brass dragon}, helped found the city over 400 years ago and has ruled it ever since. Depending on which supposition one subscribes to, J'mon is either an archfey of the Feywild, a demigod in disguise, or multiple individuals pretending to be a single entity. J'mon is rarely seen in public, and it is extremely difficult to gain an audience with them. Because of the dragon's reclusive nature, an elderly female human named Gemeshega is Ank'Harel's Grandmaven, or acting chancellor.
\item[Laws.] The laws of Ank'Harel are based on the Code of Ord. Theft, exploitation, and murder are the worst crimes, punishments for which range from life in prison to permanent exile from the city. Laws are enforced by the Hands of Ord, the city guard. Offenses for which culpability is in question or that involve particularly heinous crimes are ruled on by the judges in the Cerulean Palace and, in some cases, by J'mon Sa Ord.
\item[Religion.] Worship of all Prime Deities is welcomed in Ank'Harel. Worship of the Betrayer Gods, when such activity is discovered, is punished on a case-by-case basis depending on the nature of the reverence. Someone in possession of forbidden iconography is typically sentenced to a few weeks of imprisonment, but someone who conducts rites dedicated to a Betrayer God is treated more severely.
\item[Neighborhoods.] The city is divided into eight districts: the Alluvium District, the Circlet Walk, the Guided District, the Ridge, the River District, the Sand-Herald District, the Sigil District, and the Suncut Bazaar.
\item[Geography.] Ank'Harel lies in the center of a vast desert, nestled among clay-rich mountains. Three of these peaks have been leveled off at different elevations to form the Ridge, the Guided District, and the Cerulean Palace. Canals filled with fresh water circulate through the River District, branching outward from the nexus of the oasis.
\end{description}



\subsection{Lodging}

The Ank'Harel Lodgings table gives examples of lodgings that are available in the city.

% Table: Ank'Harel Lodgings
\begin{DndTable}[header={Ank'Harel Lodgings}]{Xcc}
Lodging & Type & Lifestyle \\
Back alley & Side Street & Wretched \\
Ajir's Whetstone (in the Alluvium District) & Hostel & Squalid \\
Golden Chip (in the Suncut Bazaar) & Casino & Poor \\
Harried Mongoose (in the River District) & Tavern & Modest \\
Step Aside (in the Circlet Walk) & Tavern/Inn & Comfortable \\
Lap of the Gods (in the Sand-Herald District) & Resort & Wealthy \\
\end{DndTable}

\subsection{Major Factions}

Numerous political, academic, and criminal factions are active in Ank'Harel. Driven by ambitions both apparent and unseen, these organizations and their rivalries shape the city's social and political landscape, and they offer opportunities for adventurers to rub shoulders with powerful individuals and thereby enhance their reputations.

Three of the groups have a strong interest in the Apotheon and what he represents: the Allegiance of Allsight, the Consortium of the Vermilion Dream, and the Library of the Cobalt Soul. Each of these factions has a set of missions the adventurers can undertake, which are detailed later in the chapter.

\subsubsection{Allegiance of Allsight}

\textit{"To peer into the past is to illuminate the future."}

\begin{description}
\item[Headquarters.] The Crystal Chateau, a university in the Sigil District, is the Allegiance's headquarters.
\item[Leader.] The Allegiance is led by two headmasters: the dour \textbf{James Cryon}, a lawful neutral elf, who is also the head of the Crystal Chateau's program for the arcane arts; and the kindly \textbf{Gryz Alakritos}, a neutral good \textbf{goblin}, whose infectious enthusiasm makes him the more popular of the pair. Both use the \textbf{scholarly mastermind} stat block (see appendix A (p. 8)).
\item[Allies.] The Allegiance and the Library of the Cobalt Soul are staunch allies.
\end{description}

% Image placeholder: Headmasters {@creature James Cryon|CRCotN} and {@creature Gryz Alakritos|CRCotN} of the Allegiance of Allsight

\begin{description}
\item[Opponents.] The Allegiance opposes the Consortium of the Vermilion Dream and the Sentinels of Memory (see "Minor Factions" later in the chapter).
\end{description}

The Allegiance of Allsight is an academic collective known across Marquet. Its original members came to Ank'Harel from across the continent to rebuild the robust academic tradition that defined the elves and orcs of ancient Cael Morrow.

The Allegiance of Allsight is embodied in two of Ank'Harel's academic institutions: Teres Schoolhouse, the city's largest center of education, and the Crystal Chateau, an elite school of arcane principles and the center of the Allegiance's operations. Students from anywhere in Ank'Harel are welcome at Teres Schoolhouse, but the Crystal Chateau's programs are extremely competitive, and all pupils there are required to join the Allegiance of Allsight.

In recent years, the faction's leadership has been focused on uncovering the secrets of Cael Morrow, the sunken ruins beneath the city. For decades previously, the headmasters of the Allegiance petitioned J'mon Sa Ord for permission to unseal the long-lost ruins, and finally, the city's mysterious leader relented. The opening of an entrance to the ruins infuriated the Sentinels of Memory, a minor faction whose members fanatically believe the sunken city is best left forgotten, lest some unknown terror left over from the Calamity be disturbed.

\paragraph{How to Join}

Joining the Allegiance of Allsight requires one to be referred by an active member, then be reviewed and audited by the Allegiance's council of leadership prior to the issuance of a judgment on the candidate.

\paragraph{Allegiance Story Track}

The series of missions offered by the Allegiance of Allsight is described in "What Lies Beneath," later in the chapter.

\subsubsection{Consortium of the Vermilion Dream}

\textit{"Seek the stories others fear to know."}

\begin{description}
\item[Headquarters.] Consortium members meet in First Eclipse, a tavern in the Suncut Bazaar, and hold private conversations in a secret storeroom there.
\item[Leader.] The Consortium is led by a council of five masters who all use the \textbf{occult silvertongue} stat block (see appendix A (p. 8)). They are \textbf{Aradrine the Owl}, a lawful neutral goliath (see the "Goliaths of Exandria" sidebar later in the chapter); Dendarron the Sun Bear, a chaotic neutral halfling; Larthul the Wolf, a chaotic evil human; Khelkur the Gull, a neutral evil dwarf; and Vrill the Moth, a lawful evil elf.
\item[Allies.] The Consortium maintains a tenuous alliance with the Sentinels of Memory (see "Minor Factions" later in the chapter).
\item[Opponents.] The Consortium opposes the Allegiance of Allsight and the Library of the Cobalt Soul.
\end{description}

This group of occult mystics has assembled a membership of like-minded souls from across Exandria. The Consortium of the Vermilion Dream focuses on studying the magic described in folklore that has been dismissed as fanciful by more "proper" institutions. The group takes its name from its foremost object of fascination: Ruidus, the ruddy moon of Exandria that is fabled to bestow ill fortune and suffering upon those who are born, enter into contracts, or begin new ventures under its baleful light.

The Consortium was founded by five influential individuals, all renowned in occult circles from across Exandria, who sought to gain wealth and prestige from their studies. The Consortium is still young, and thus all five of its founders still live. Older academic institutions such as the Allegiance of Allsight and the Library of the Cobalt Soul express concern and sometimes disdain toward the upstart Consortium and its aggressive, profit-driven attitude toward scholarship.

Due to the expensive nature of its research, the Consortium is in constant need of funding. It leans heavily on its members for such support, selling the services of its magically skilled devotees to government officials, wealthy diplomats, and other persons of means. A large portion of its income goes to support the Sentinels of Memory, the Consortium's tentative ally against the Allegiance of Allsight.

\paragraph{How to Join}

The Consortium gladly welcomes any enthusiast of the strange and occult to its ranks---as long as the faction's leaders can be reasonably certain that the individual has no ulterior motives that would undermine their efforts.

\paragraph{Consortium Story Track}

The series of missions offered by the Consortium of the Vermilion Dream is described in "Vermilion Gambits," later in the chapter.

\subsubsection{Library of the Cobalt Soul}

\textit{"Reason. Knowledge. Truth."}

\begin{description}
\item[Headquarters.] The Marquesian branch of the Library of the Cobalt Soul is based at the Temple of the Mentor---a temple of Ioun the Knowing Mentor---in the Guided District.
\end{description}

% Image placeholder: High Curator {@creature Jamil A'alithiya|CRCotN} of the Library of the Cobalt Soul

\begin{description}
\item[Leader.] At 29 years old, \textbf{Jamil A'alithiya}, a chaotic good human, is the youngest \textbf{monastic high curator} (see appendix A (p. 8)) in the history of the Cobalt Soul. He is often underestimated by his contemporaries and uses that fact to his advantage.
\item[Allies.] The Cobalt Soul maintains alliances with the Allegiance of Allsight and the Hands of Ord (see "Minor Factions" below).
\item[Opponents.] The Cobalt Soul opposes the Consortium of the Vermilion Dream.
\end{description}

The Library of the Cobalt Soul was founded in the heart of Wildemount centuries ago, before any of the nations that currently wage war over that land were born. Its archivists have stood strong against conflict, propaganda, and upheaval during all this time in their dogged hunt for the truth. Their mission has spread across Exandria, with advocates in various locales defying local governments in the pursuit of truth.

The Cobalt Soul founded the Temple of the Mentor in Ank'Harel nearly 150 years ago. The archivists here focus on rooting out evil in the city's civilian factions---principally the Consortium of the Vermilion Dream, whose members reportedly wield magic weapons made of some strange, red mineral the likes of which no Cobalt Soul archivist has ever seen.

The faction is wary of the Allegiance of Allsight's fascination with Cael Morrow, but as fellow academic institutions, the Allegiance of Allsight and the Cobalt Soul work together most of the time, sharing resources from their libraries.

\paragraph{How to Join}

The Library of the Cobalt Soul accepts for consideration any volunteers who express interest in gathering and preserving knowledge. These volunteers are put to work as librarians, performing menial tasks such as organizing records and transcribing notes from expeditions---drudgery designed to weed out those who want to join the organization for the sake of fame and adventure. Typically, this duty lasts about six months, after which a librarian is formally inducted.

\paragraph{Cobalt Soul Story Track}

The series of missions offered by the Library of the Cobalt Soul is described in "Knowledge Is Power," later in the chapter.

\subsection{Minor Factions}

Other, smaller factions are important to life in Ank'Harel but take a backseat in the plot of this adventure.

\subsubsection{Hands of Ord}

\textit{"By the code of Ord, we stand vigilant."}

\begin{description}
\item[Headquarters.] The city guard is based in Ord Bastion in the Sand-Herald District.
\item[Leader.] Ironhand Sem, a lawful good \textbf{minotaur}, commands the Hands of Ord and oversees all military operations in Ank'Harel.
\item[Allies.] The Hands of Ord and the Library of the Cobalt Soul are staunch allies.
\item[Opponents.] The Hands of Ord aim to root out and destroy the Veil (see below).
\end{description}

Established four hundred years ago by J'mon Sa Ord, the Hands of Ord are a peacekeeping order of desert warriors who have long watched over Ank'Harel and kept the city safe and stable.

The order is governed by the code of Ord, which is the backbone of the city's legal system. Members of the Hands dedicate their lives to the city and the code, and many come from a long, respected bloodline of previous Hands. Well-appointed barracks in the Sand-Herald District provide members of the Hands with a comfortable home; places of worship; and sources of food, armor, and weapons.

Despite their best efforts, the Hands of Ord have failed to thwart the Veil, the crime syndicate that pervades Ank'Harel society.

\subsubsection{Scarbearers}

\textit{"We do not break."}

\begin{description}
\item[Headquarters.] The Bowl of Judgment, an arena in the Sand-Herald District used for combat tournaments, is the Scarbearers' headquarters.
\item[Leader.] Once a mercenary in his own right, Quartermaster Croog Lynn, a chaotic neutral, goliath \textbf{gladiator}, has now taken to running the guild in his later years (see the accompanying "Goliaths of Exandria" sidebar).
\item[Allies.] None.
\item[Opponents.] None.
\end{description}

The Scarbearers are the most famous mercenary company in Ank'Harel. Named for the scars their founders displayed to J'mon Sa Ord as proof of their service to the realm, the Scarbearers occupy an honorable position in Marquesian society. They often take high-profile jobs, such as serving as bodyguards for diplomats and protecting caravans and expeditions into the desert. The members promote themselves as a guild that respects honor and guidelines, but they often come across as brash, thuggish fighters who are out for a paycheck.

In addition to their mercenary exploits, the Scarbearers also supervise the tournaments that take place in the Bowl of Judgment.

\subsubsection{Sentinels of Memory}

\textit{"Let the past die in peace."}

\begin{description}
\item[Headquarters.] The Tower of Memory in the Guided District is the meeting place for the Sentinels of Memory.
\item[Leader.] Watcher Trast, a neutral, elf \textbf{priest} and a former cleric of Ioun, founded the Sentinels of Memory and maintains command over the faction to this day.
\item[Allies.] The Sentinels are allied with the Consortium of the Vermilion Dream.
\item[Opponents.] The Sentinels oppose the Allegiance of Allsight.
\end{description}

The Sentinels of Memory are a group of fanatics who believe the sunken city beneath Ank'Harel isn't some grand ancient wonder---it's a prison that must be kept sealed at all costs. Citing the lack of recorded history about Cael Morrow as proof that powerful forces wanted to expunge the city from memory, the Sentinels of Memory are willing to do whatever it takes to ensure that no one speaks of that ill-fated city ever again.

Since they oppose the excavation of Cael Morrow, the Sentinels of Memory are constantly at odds with the Allegiance of Allsight. Though their efforts to date have mostly been political, some of their members have been connected to violent acts intended to thwart the excavation efforts.

Recently, the Consortium of the Vermilion Dream has been supplying the Sentinels of Memory with funds and agents to support the Sentinels' conflict against the Allegiance of Allsight. The Sentinels are grateful for the aid but treat the Consortium with caution, suspecting the faction has ulterior motives.

\subsubsection{The Veil}

\textit{"Welcome to the other side."}

\begin{description}
\item[Headquarters.] None.
\item[Leader.] Twenty-year-old Ilena Hapayhari, a chaotic evil, human \textbf{assassin} and ruthless pirate from the western coast of Marquet, recently disposed of the Veil's old leadership. She now rules the Veil from the shadows, though only high-ranking members are privy to this knowledge.
\item[Allies.] None.
\item[Opponents.] The Veil opposes the Hands of Ord.
\end{description}

A secret network of spies, thieves, and assassins, the Veil is the largest crime syndicate in Ank'Harel. It is the chaotic counterpoint to the order of the Hands of Ord.

Not much is publicly known about the Veil; the identities and personal details of its members are kept secret even among the group, and those who let such secrets slip are often found murdered before they can be apprehended. The Veil has no permanent headquarters in the city, constantly changing bases to cover its trail.

Although the Veil doesn't play a role in this adventure, you can use this faction as an adversary in your own adventures set in Ank'Harel.

\subsection{Alluvium District}

The Alluvium District is the northernmost neighborhood of Ank'Harel. Named for the rich clay mined in the cliffs that border the district, this area is a mixture of dense residential sections, sites of clay strip mining, and spacious plazas used for oral performances and artisan markets.

% Image placeholder: Alluvium District

\subsubsection{Alluvium Gardens}

\textit{Arboretum}

The arboretum known as the Alluvium Gardens is famous for the meticulous, geometric layout of its hedges and planters, mimicking the tile art of the district's ceramic masters. After a long day of hauling clay, miners often retreat to the shade of the gardens to unwind and chat with friends. At the center of the gardens is a grove of fig, apricot, and pomegranate trees, and visitors to the gardens are welcome to help themselves to any ripe fruit they can reach. The gardens also contain kurrak, a hard-to-cultivate fruit native to the jungles of Marquet that has a tough, spiny shell and soft, sweet flesh like that of a mango.

\subsubsection{Opalite Forum}

\textit{Entertainment Square}

Eight pillars of iridescent glass, created from discarded sand and debris from the clay mines, ring the outside of this wide stone courtyard in the heart of the Alluvium District. The Opalite Forum is a central gathering place for district residents, often used to host festivals and holy celebrations.

The forum is also a place where storytellers from across Exandria tell their tales. During the day, children gather around the elderly chroniclers, who relate the legends of Exandria's founding and stories of the Calamity in the oral tradition of their forebears. At night, the forum is overtaken by crowds of burgeoning writers, reading their latest works aloud and, on occasion, engaging in friendly duels of poeticism and wit.

% Image placeholder

\begin{DndSidebar}{Goliaths of Exandria}
The goliaths of Exandria are a mighty people with stone giant blood running in their veins. Most stand well over seven feet tall and are blessed with a naturally powerful physique. Their pale gray skin is mottled with splotches or streaks of darker gray.
In some parts of Exandria, goliaths love to compete and keep score, counting their deeds and tallying their accomplishments to compare to others. Above all else, they are driven to outdo their past achievements.
A goliath nonplayer character in Exandria typically has proficiency in the Athletics skill, as well as one or both of the following traits:
\paragraph{Mountain Born}

The goliath has resistance to cold damage and is naturally acclimated to high altitudes, including elevations above 20,000 feet.
\paragraph{Powerful Build}

The goliath counts as one size larger when determining its carrying capacity and the weight it can push, drag, or lift.
\end{DndSidebar}

\subsubsection{Sa'irah Mines}

\textit{Clay Strip Mines}

The angular ridges of the Sa'irah strip mines loom over the rest of the Alluvium District, and the cacophony of hammering and drilling fills the nearby streets almost constantly during the workday. Most of the clay harvested here is used for the construction and repair of buildings. Some of it is worked by artisans who have perfected the art of blending and firing the clay, creating vibrantly colored mosaic tiles and other ceramic works.

The quarry is run by a gruff, middle-aged miner named Zala Keencutter, a neutral, tiefling \textbf{thug}. Though he puts up an abrasive facade, Zala has a shy and empathetic soul, and he wants dearly to work up the courage to read his prose writings at the Opalite Forum some evening.

\subsection{Cerulean Palace}

The purple and blue spires of the Cerulean Palace reflect the sun's light off their edges like a sun-dappled waterfall. Most citizens are familiar only with the palace's ornate exterior, covered with glittering designs of lapis lazuli and turquoise with brass accents. The inside of the palace, where only those with official business are typically allowed, contains chambers used for deliberation, debate, and judgment regarding legal matters. Beneath the palace lies a labyrinthine dungeon called the Oblag, also known as the Scarlet Oubliette.

At the center of the palace complex is the Chamber of Judgment. Intricate tile-work covers the outer surface of the chamber. The inside is dimly illuminated by sunlight that shines through the translucent panes of bluish crystal that make up the roof. On a raised stone dais sits an intricately carved bronze throne studded with sapphires---the Seat of J'mon Sa Ord, said to be a duplicate of the one the leader uses in their personal quarters at the top of the palace's highest tower. In the Chamber of Judgment, perpetrators of the most terrible crimes are brought to justice. J'mon Sa Ord descends to occupy the seat while presiding over these hearings and then personally delivers the verdict.

\subsection{The Circlet Walk}

The Circlet Walk is primarily a residential district, though taverns and other places of commerce can be found here as well. The district is so packed with buildings that it's customary for people to travel through the district by hopping from rooftop to rooftop rather than navigating the narrow roads.

\subsubsection{Lyrean Linen}

\textit{Front for the Black Market}

At first glance, Lyrean Linen looks like any other cloth-dyeing establishment, with long sheets of fabric hanging off every rafter and vats of colored liquid arrayed across the front of the shop. Asking for cloth dyed "whitestone green" gives a customer access to the store's other wares: a dangerous narcotic called suude and other forbidden concoctions sold on the black market.

The shop is run by Bas and Kovra Lyrean, twin sisters who are neutral evil, half-elf \textbf{bandits}. Originally alchemists by trade, they use their knowledge to brew magic potions deemed by the Hands of Ord to be too dangerous for public consumption.

\subsubsection{Step Aside}

\textit{Inn and Tavern}

This cozy establishment is a local favorite, known for its delicious food and a cheery atmosphere that invites folks to "step aside" from their daily troubles and rest a while. Desert flowers flourish in planters that decorate the tavern's windowsills and bar, and the upstairs rooms are redolent with the scents of garlic rice and fried meats wafting up from the kitchen. On the most sweltering days, the tavern offers its signature dessert: haluh, a dish from the southwestern Marquesian coast consisting of shaved ice and sweet milk topped with jackfruit or kurrak, coconut strips, and sweet purple yam.

The place is run by two former mercenaries: Irawan, a lawful good, elf \textbf{veteran}, and his husband Calinao, a neutral good, elf \textbf{druid}. Unbeknownst to many, Irawan and Calinao were among the original Scarbearers who came to Ank'Harel during the city's nascent years and defiantly showed to J'mon Sa Ord the scars they earned during their defense of the realm. The couple has long since retired from mercenary life, but they serve the people of the city in other ways. If the phrase "Pardon my steps as I pass through" is whispered to either Irawan or Calinao, they'll lead the individual to a safe house beneath the tavern, where the beneficiary is welcome to stay until other arrangements can be made.

\subsection{Guided District}

The Guided District contains several houses of worship, with each of the Prime Deities represented. The district's architecture is more ornamental than in other areas of the city; many temples are adorned with intricate mosaics and crowned by minarets with gilded spires.

% Image placeholder: Guided District

The few vagabonds to be found in Ank'Harel frequent the alleys in this district, where many temples send representatives to feed and clothe the downtrodden. The Hands of Ord also maintain a notable presence here.

\subsubsection{Blessing Well}

\textit{Holy Site}

At the heart of the Guided District is a concave stone courtyard known as the Blessing Well. A line of alabaster tiles spirals down to the center of the plaza, ending at the well itself, whose deep blue bricks are speckled with silver in a representation of the night sky.

The Blessing Well is not dedicated to a deity and instead is a public space for anyone in Ank'Harel who seeks spiritual guidance. On many occasions, an individual has been lowered into the well, away from the bright lights and bustle of the city, to calm their thoughts and commune with the divine; they emerge hours later, eyes gleaming with renewed purpose as they savor the divine vision they experienced while in the well's depths.

The water of the well is separate from that in the city canals and is used only in holy ceremonies. Drinking the well's water bestows a terrible divine curse upon the drinker that even the most potent mortal magic can't undo.

\subsubsection{Crossings of Eventide}

\textit{Necropolis}

Over two hundred feet high, the Crossings of Eventide are a series of tombs carved into the cliffs that border the Guided District. Reliefs depicting the most important deeds of the tombs' occupants adorn the spaces between the stone slabs that seal the graves. At sunset, crystal braziers that hang throughout the necropolis catch the fading light, painting the nearby cliffside in vibrant hues of pink, orange, and purple.

At the base of the cliff rests a small temple dedicated to the Matron of Death. In Ank'Harel (and in much of Marquet at large), the god of death and fate is perceived differently from how she is regarded in the lands of Wildemount, Tal'Dorei, and Issylra. Bereft of her moniker "the Raven Queen," here the god is associated with twilight, inevitability, and the passage of time, and she presides over the transition between life and death. Some Marquesian sects refer to her as the "Duskmaven," portraying her as a keen-eyed vulture with a golden funerary mask and feathers whose colors range from sunset hues to midnight black.

\subsubsection{Temple of the Mentor}

\textit{Temple to Ioun and Headquarters of the Cobalt Soul}

A ceramic mosaic of three open eyes framed in cobalt blue crowns the entrance to the Temple of the Mentor. Dedicated to Ioun, the temple is more a library than a hall of worship, and it doubles as the base of operations for the Marquesian branch of the Library of the Cobalt Soul. Shelves line the walls of the temple's sanctuary, all filled near to capacity with meticulously catalogued tomes, scrolls, and periodicals that are available for public inspection.

Staircases lead up from the sanctuary to the residents' apartments and the scriptorium, and up farther to the quarters of the Cobalt Soul's higher-ranking members. Stairs lead down from the sanctuary to a locked door opened with a key held only by select individuals and the Cobalt Soul's high curator. Behind this door is the Cobalt Soul's vast subterranean archive, containing all sorts of esoteric knowledge that is purposely kept away from the public. The entire expanse is lit by \textit{continual flame} spells.

At the high curator's behest, a sardonic monk named Iwo Zalarre, a chaotic good, half-orc \textbf{monastic operative} (see appendix A (p. 8)), oversees all visitors to the public archives and registers all requests for access to the private archive. Though just as dedicated to Ioun's service as any other member of the Cobalt Soul, Iwo is eager to get out of the temple and take part in a challenging mission.

\subsubsection{Tower of Memory}

\textit{Headquarters of the Sentinels of Memory}

Situated on the far western side of the Guided District, the octagonal Tower of Memory juts out of the ground at a slightly crooked angle. The building was slated for demolition when the founder of the Sentinels of Memory, Watcher Trast, purchased it to use as a home for his small but zealous group of followers. The Sentinels have since spent little time repairing the structure, so cracks still show between the ivory bricks, and the wooden floorboards creak with every step.

Watcher Byron, a lawful evil, gnome \textbf{veteran}, oversees most of the sentinels' operations for Watcher Trast and rarely leaves the Tower of Memory. Gentle of voice but volatile of temper, Byron has grown increasingly impatient with his faction's political strategy and has begun to plot more aggressive ways of hampering the Allegiance of Allsight's excavation of Cael Morrow. Though he would never admit it, Byron secretly longs to see the wonders of Cael Morrow for himself.

\subsection{The Ridge}

Sitting atop one of the city's three plateaus, the Ridge is Ank'Harel's industrial district. Its streets are crowded with smithies, foundries, shipyards, and warehouses.

\subsubsection{Alsfarin Union Shipyard}

\textit{Skyship Conglomerate}

The foremost producer of skyships in Exandria, the Alsfarin Union, has its base of operations in Ank'Harel. Its shipyard, thundering with the peal of hammers and glowing with arcane energy when work is under way, is off limits to anyone except employees. The Alsfarin Union protects its supply of brumestone---the enchanted crystal that magically holds skyships aloft---with an iron fist.

Since all skyships in operation across Exandria must come to the shipyard once a year for maintenance and repair, the area near the shipyard has become a hodgepodge of cultures from around the world. Curious folks can purchase iconic headwear from Whitestone in Tal'Dorei at the Lord Percy Haberdashery (a name not endorsed by Lord Percival de Rolo III of Whitestone), then patronize a cart selling pastries from Port Damali in Wildemount and indulge in a mug of ale brewed in Vasselheim in distant Issylra, all in the same city block.

\subsubsection{Sunfire Forge}

\textit{Smithy}

Built partially into the side of the mountain that borders the north side of the district, the Sunfire Forge is the premier blacksmith and weapons manufacturer in Ank'Harel. Its name comes from the forge's unique power source: liquid sunlight, harvested from the iridescent crystals lining its roof. The consistent heat of this concentrated form of energy allows for the creation of fine metalwork that incorporates fragile components such as arcane minerals and gem dust. Two glowing streams of Sunfire cascade down the sides of the forge's pitched roof before coming to rest in stone reservoirs, where the liquid is stored before being funneled into the forge.

The Sunfire Forge is under the control of Rohaya Tak, a neutral good, half-elf \textbf{commoner}, who inherited the business from her mother two decades ago. An innovative metalworker, Rohaya combines her smithcraft with shrewd business intuition. In addition to crafting blades and tools, Rohaya uses the forge to produce cannons for use on sailing ships and skyships, modeled after the bronze cannons of her family's homeland in southern Marquet.

\subsection{River District}

Its atmosphere permeated by the smell of wet earth, the River District is where most of the city's farming is done. Strips of irrigated cropland are intermingled with newer hydroponic farms, and in place of roads, canals connect the subdivisions of the district. Most folks traverse the canals of the River District in skiffs and lightweight canoes.

\subsubsection{Life Dome}

\textit{Nexus of the Canals}

This glass dome crowns the source of water in Ank'Harel: the oasis around which the city was built. All the canals in the River District originate from the Life Dome, and the structure is a center of celebration and a monument to the city's history. At night, the inside of the dome is magically lit by drifting orbs of blue and green luminescence, painting the nearby area in a display of colored light.

The Life Dome also serves an important agricultural purpose. Because the water that fills Ank'Harel's oasis flows not from a natural source but from underground caverns created by the acts of the gods, the water level frequently fluctuates. The water can surge forth or slow to a trickle, threatening to bring about floods or dry spells, respectively. To counteract this problem, the Life Dome instantly transmutes any water that exceeds the accepted level into vapor and stores it in the seams of the dome. When the water level drops, the stored vapor is transmuted back into liquid and fed into the irrigation canals.

The Allegiance of Allsight scholar Carliale Kroogan, a neutral good, gnome \textbf{scholarly excavator} (see appendix A (p. 8)), is currently the chief supervisor for the Life Dome. Lonely and a bit of a chatterbox, Carliale is always ready to chat about arcane theory with any individual who comes his way.

\subsubsection{Old Man Kruuk's}

\textit{Bakery and Fence for Stolen Goods}

Old Man Kruuk, a neutral, half-orc \textbf{bandit captain}, is a mediocre baker; his pitas are tough and chewy, and the caramel of his baklava is bitter and burned. Kruuk's talent lies in his true vocation as a fence for stolen goods. After an individual gains his trust, he drops his bumbling baker facade and reveals himself to be an excellent businessperson, able to sweet-talk even the most uncompromising thieves into a favorable deal.

The scope of Old Man Kruuk's operation far exceeds what anyone might suspect---or even what he should be capable of as a single individual. Some rumormongers suggest that Kruuk is a high-ranking member of the Veil, but no one has verified this supposition.

\subsubsection{Steam Gardens}

\textit{Arboretum and Bathhouse}

The most luxurious of Ank'Harel's arboretums, the Steam Gardens are popular because they include a bathhouse and spa. Water from the canals is fed into the gardens' magically fueled hot springs, which bubble as they release massaging jets of water.

An hour's visit at the most exclusive spa costs 5 gp. Characters who spend 8 hours eating, luxuriating, and relaxing in the spa awaken the next morning with 1d10 temporary hit points, p. 9 that last for the next 24 hours, in addition to the usual benefits of finishing a long rest.

\subsection{Sand-Herald District}

The center of military activity and law enforcement in Ank'Harel, the Sand-Herald District houses the headquarters of the Hands of Ord. The organization's stronghold is a major feature of the district, dwarfed only by the Cerulean Palace.

The residences in the Sand-Herald District are occupied mostly by wealthy folk: powerful merchants, decorated mercenaries who have earned riches by winning tournaments in the Bowl of Judgment, and local celebrities of song and stage. The district is also the site of Ank'Harel's skyport.

\subsubsection{Bowl of Judgment}

\textit{Combat Tournament Arena}

On the north side of the Sand-Herald District, an immense circular depression has been carved into the ground. The Bowl of Judgment is where warriors are made into legends. Sporting tournaments are held here on a regular basis, highlighted by a yearly event known as the Grand Tournament. At each competition, spectators wager on who will be the winner of the upcoming match. Anyone who can pay the entry fee of 20 gp can sign up to compete. Although most bouts end with one participant yielding or falling \textit{unconscious}, death is always a possible outcome in the Bowl of Judgment.

Sheed Caltor, a chaotic neutral, orc noble, is the suave ringmaster of the Bowl. Sheed is mostly concerned with his image, his income, and his status among the upper crust of Ank'Harel.

\subsubsection{Indala Skyport}

\textit{Skyship Port}

Skyship travel is undertaken primarily by those who want to avoid the time and danger of an ocean voyage or a long trek through the desert---and who can afford the cost of such transportation. The Indala Skyport dates to the earliest days of skyship travel. Its central terminal is world-famous for its elaborate stained-glass ceiling, which depicts a mosaic of the surface of Exandria with glowing dots moving across it, representing each skyship the Alsfarin Union currently has in flight.

Outbound flights from the Indala Skyport connect to other major skyports in Exandria, including Vasselheim in Issylra, Port Damali in Wildemount, and Emon and Whitestone in Tal'Dorei.

\subsubsection{Ord Bastion}

\textit{Headquarters of the Hands of Ord}

In contrast to the heavily embellished architecture of the Cerulean Palace, the towering walls of Ord Bastion are built from sturdy mud bricks, undecorated except for simple, angular carvings that adorn the tops of the building's watchtowers.

Ord Bastion serves principally as the barracks and training center for the members of the Hands of Ord, but since membership in the Hands is a life-long commitment, many members have brought their families to live in the bastion, creating a small community bound together by love and loyalty.

\subsection{Sigil District}

The Sigil District is one of the smaller districts in Ank'Harel but also one of the busiest. People from across the city commute here every day to study at the district's various centers of learning. From basic skills and trades to the most esoteric arcane ideals, various intellectual pursuits are represented in the Sigil District.

% Image placeholder: Sigil District

The Allegiance of Allsight is a pervasive presence in this district; many of the faction's members and affiliates are on the faculties and staffs of these schools. Institutions of learning constructed in Ank'Harel's infancy abut structures of more modern architecture, and buildings representing cultures from across Exandria are a counterpoint to the grand, Marquesian-style domes of the Crystal Chateau. At the heart of the district is the excavation that leads to Cael Morrow, the mysterious sunken city (see "Maw of Cael Morrow" below).

The Streets of the Sigil District map shows a small segment of the district associated with the faction missions later in the chapter. It includes the following locations:

\begin{description}
\item[S1: Crystal Chateau Entrance.] This elite university is described below.
\item[S2: Teleportation Atrium.] A permanent teleportation circle stands in the center of this austere, domed building. Most \textit{teleportation tablets} (see appendix B (p. 9)) made by the Allegiance of Allsight are keyed to this circle.
\item[S3: The Frog and Kebab.] No actual frogs are served at this popular tavern, which is frequented by students of the Crystal Chateau.
\item[S4: Emerald Ibis Relic Hall.] Watched over by statues made of green glass and shaped like giant ibises, this hall is where the Allegiance of Allsight examines newly discovered items. See "Consortium Mission 3: Elephant in the Room" in the Consortium Story Track.
\item[S5: Scroll's Alcove.] This apartment building is used mostly by professors at the Crystal Chateau and visiting guest lecturers. See "Cobalt Soul Mission 1: The Cultist of Zehir" in the Cobalt Soul Story Track.
\item[S6: Lawbearer's Librams.] One of dozens of libraries in the Sigil District, this is the most extensive legal library in the city.
\item[S7: Boughs of the Wild Mother.] Students can be found studying in the shade of this courtyard's giant sycamore tree during the day. Some say a ghost lurks here at night (see "Consortium Mission 1: A Ghost in Our Midst" in the Consortium Story Track).
\end{description}



\subsubsection{Crystal Chateau}

\textit{Elite University and Headquarters of the Allegiance of Allsight}

An immense structure of smoky quartz and white marble, the Crystal Chateau is a major landmark in the Sigil District. A small stream winds through the gardens around the building before flowing under an opening into the main hall, where the bubbling sound of its water is said to help inspire the highest of thoughts. A white iron fence closes off the premises to all except members of the Allegiance of Allsight.

The Crystal Chateau serves double duty as a university for the teaching of spellcraft and arcane theory and as the center for the Allegiance's operations. Though not all members of the Allegiance are students or professors at the university, every would-be pupil must first become a member of the faction. Tuition is free, since the Chateau's expenses are subsidized by the city in the expectation that the university's graduates will use their gifts to protect Ank'Harel and enhance its prestige across the world.

Jor Raashid, a lawful neutral, halfling \textbf{scholarly agent} (see appendix A (p. 8)), is the gatekeeper for the Crystal Chateau. Suspicious and condescending, Jor sternly prohibits nonmembers from entering the Chateau.

\subsubsection{Maw of Cael Morrow}

\textit{Excavation Site Entrance}

What was once a public square has been converted into an excavation site, from where the Allegiance of Allsight conducts its exploration into the ruins of Cael Morrow. The "maw" is a 150-foot-diameter, roped-off hole, admission into which is prohibited to all except for holders of Allegiance of Allsight badges. A 10-foot-wide path spirals down along the edge of the maw, descending hundreds of feet into Cael Morrow (see chapter 5 (p. 5)).

\paragraph{Protection}

A contingent of five lawful neutral Scarbearer \textbf{veterans} patrols the rim of the maw, on the lookout for unauthorized visitors. They are led by a loyal, lawful good, orc \textbf{scholarly excavator} (see appendix A (p. 8)) named Hakzorne. He permits only people who display Allegiance badges to enter the maw; Hakzorne gives only one warning before ordering the mercenaries to attack.

\subsubsection{Omnival Library}

\textit{Public Library}

Tan stone columns and pointed archways form the circumference of this enormous tower on the northeast side of the Sigil District. Inside, texts from across Exandria fill shelves that line the wall space between the floor and the library's vaulted gold ceiling. Staircases and magical lifts provide access to the highest shelves, and on the floor sits a mixture of weathered wooden desks and colorful cushions.

Any individual is welcome to borrow from the library, though access to the most esoteric and delicate materials requires the approval of an Allegiance of Allsight headmaster. Recent additions to the library include copies of notes from the latest Cael Morrow expeditions.

The Omnival Library is managed by Bookkeeper Khime, a neutral good, orc \textbf{mage}. He's convinced that shadowy forces are constantly trying to undermine Ank'Harel, and this penchant for conspiracy theories makes him incredibly observant. He's an excellent source of information about happenings in the city, despite his wild conclusions.

\subsubsection{Teres Schoolhouse}

\textit{University}

The larger and less stringent of the city's two universities, Teres Schoolhouse offers a curriculum that features practical applications of magic as well as a wide range of trade skills. Its physical presence, though not as imposing as that of the Crystal Chateau, occupies a large portion of the Sigil District, made up of both older buildings of weathered brick and newer structures built from stone quarried from northern Marquet.

Academic schedules at Teres Schoolhouse are typically divided into two semesters, each five months long. The institution also puts on single-session educational seminars that anyone in the city is welcome to attend. As with the Crystal Chateau, tuition at Teres Schoolhouse is free for all residents of Ank'Harel.

\subsection{Suncut Bazaar}

Crimson banners, hanging flags of all colors, and cloth drapery that provides shade over the crowded streets herald the entrance to the Suncut Bazaar, the center of commerce in Ank'Harel. Awnings stretch above open-air food stalls that offer treats ranging from fresh fruit to sweet semolina cakes to warm meat carved directly off a roasting spit. Most of the densely packed stone buildings are stores that sell artisan goods and shops that buy and sell magic items and nonmagical trinkets. In the center of the district lies the palatial Luck's Run casino.

% Image placeholder: Suncut Bazaar

The Streets of the Suncut Bazaar map shows a small segment of the district that comes into play in the faction missions later in the chapter. It includes the following locations:

\begin{description}
\item[T1: Welcome Arch.] This vaulted stone archway draped with crimson banners welcomes everyone to the bazaar.
\item[T2: Battlements.] The walls of the Suncut Bazaar are made of sandstone and patrolled day and night by Hands of Ord \textbf{guards} and \textbf{veterans}.
\item[T3: Central Street.] This dusty red road that runs throughout the district is so crowded with shoppers, pickpockets, tourists, and locals that it can be hard to see the paving stones.
\item[T4: Market Stalls.] Market stalls shaded by awnings line the streets and alleys of the bazaar.
\item[T5: Luck's Run.] This casino is described below.
\item[T6: Bone Garden.] The bazaar's most infamous pawn shop is described below.
\item[T7: Mystic Pursuits.] This arcanist's abode is described below.
\item[T8: First Eclipse.] Both a tavern and the secret headquarters of the Consortium of the Vermilion Dream, this venue is described below. Its roof is where the characters appear if they used one of Aloysia's \textit{teleportation tablets} at the end of chapter 3.
\item[T9: Courtyard.] During business hours, hungry folk line up in front of the food stalls on the west side of this courtyard while others sit around the edge of the courtyard's fountain and share gossip.
\end{description}



\subsubsection{Bone Garden}

\textit{Purveyor of Oddities and Exceptionalities}

The clatter of fish-bone wind chimes announces one's entrance into the Bone Garden: a combination pawn shop, antique store, and repository of strange and misplaced items. The shop, which stands near the main entrance of the bazaar, has become a running joke among the residents of Ank'Harel. Fruitless searches for a missing key or a lost earring often end with the lighthearted comment "It'll probably end up among the Bones within the week."

The Bone Garden is run by Rerosha, a chaotic good, human \textbf{acolyte} who's a nonverbal young woman with a brazen attitude toward life. She uses facial expressions, gestures, and a sassy gray parrot named Akil to communicate.

\subsubsection{First Eclipse}

\textit{Tavern and Headquarters of the Consortium of the Vermilion Dream}

In sharp contrast to the glimmering Luck's Run casino across the street, First Eclipse is notable for its maroon stone walls and ash-black door. Inside, old ale barrels have been turned into tables, and the back wall is covered with painted lids from ales, wine, and liquor casks from across Exandria.

One of these cask lids bears a symbol not associated with any known brewer: a red crescent moon. Servers, if asked about it, are told to explain that this lid represents the "house ale." In truth, knocking three times on the cask lid reveals a secret door that leads to a hidden storeroom. Members of the Consortium of the Vermilion Dream use this room as a base of operations. (The Consortium changes its headquarters frequently to avoid detection.)

The secret room is little more than a cellar with a long table in it, with a few locked cabinets containing fist-sized chunks of ruidium. It is used only for private mission briefings and rare gatherings of all five of the faction's masters.

First Eclipse is owned and operated by Satzrak Runestrider, a neutral evil, blue dragonborn \textbf{occult extollant} (see appendix A (p. 8)). He is a loyal member of the faction, and he longs to keep the Consortium headquartered at his place permanently; the group has already used the tavern for three months longer than originally planned, so his hopes are up.

\subsubsection{Luck's Run}

\textit{World-Renowned Casino}

The Luck's Run casino stands three stories tall and is made of speckled red porphyry and topped with metallic domes in the palatial Marquesian style. After dark, faint purplish light escapes from the slit-like windows of the casino, and the sounds of laughter and music carry into the nearby streets.

Inside the casino, tables draped in crimson cloth fills the floor space, each location crowded with patrons trying their hand at one of the games of chance. Employees dressed in uniforms of dark gray and gold push through the throng while carrying trays of food, drinks, and coin between tables.

The casino is owned and run by Adima Shemsilver, a chaotic neutral, halfling \textbf{noble} who walks with the confidence of someone twice his height. Adima secretly uses half of the casino's earnings to fund the Veil, as his family has done for generations.

Three of the most popular games at Luck's Run are described below.

\paragraph{Avandra's Favor}

This dice game has a 25 gp minimum bet. Each participant rolls 2d6 and wins back twice their bet on a roll of 7 or 12. A participant who doesn't win can roll another d6 and add it to their total by wagering an additional 25 gp. A participant can do this as many times as they want, until they "bust" by accumulating a total higher than 12.

\paragraph{Gambit of Ord}

This card game requires a 50 gp initial bet, which goes into a pot in the center of the table. Each participant draws a card, then a second, and then a third. After each draw, each participant has a chance to raise the bet, stand pat, or fold. The participants reveal their hands after the third card is drawn. Whoever has the highest-ranking hand takes the pot.

To simulate the play of this game, have each participant roll dice in secret. At the first draw, all participants roll a d8. The second draw is a d6 roll, and the third is a d4 roll. Then, all reveal their dice, and whoever has the highest total wins.

\paragraph{Quon a Drensal}

Also known as "Run of Luck," this game is a contest of lizard racing. Each player can bet 10 gp or more on a single racer in a field of three, and anyone who backs the winner gets back double the amount wagered. Those who bet on the second-place lizard get back half their wager.

To simulate the play of this game, roll 3d4 for each lizard. The winning lizard is the one with the highest total (ties are possible).

\subsubsection{Mystic Pursuits}

\textit{Arcane Shop}

Colorful tapestries frame the entrance to this establishment. Beyond the threshold, silk curtains cover the walls, and magical baubles cover low wooden tables illuminated by scented candles. A soothsayer sits in the back of the room, her deep blue hood pulled down over her eyes and a deck of cards spread out before her---offering an invitation to peer into one's future, if one so dares.

The "soothsayer" is a charlatan named Amkezne, a chaotic neutral, tiefling \textbf{mage}. She is a Crystal Chateau dropout who makes a living defrauding wide-eyed tourists with her mystical displays and fake enchanted trinkets. Customers who see through her act but treat her with respect can purchase any spell component worth 1,000 gp or less from Amkezne. She also has a collection of fine jewelry for sale, including five gold bracelets (25 gp each), three gem-studded gold rings (250 gp each), and two jewel-encrusted necklaces and tiaras (750 gp each). An \textbf{invisible stalker} watches over this collection and attacks anyone who tries to steal from it.

% Image placeholder: {@creature Ruidium Elephant|CRCotN} in Figurine Form
\section{Faction Story Tracks}

% Image placeholder: A {@item Figurine of Wondrous Power, Marble Elephant|DMG|figurine of wondrous power} corrupted by ruidium transforms into a rampaging crystalline elephant

\subsection{What Lies Beneath}

\subsubsection{Allegiance Story Track}

This section details the missions in the Allegiance Story Track, which are summarized in the accompanying flowchart.



\subsection{Allegiance Mission 1: Deliver the Figurine}

As rumors about Cael Morrow continue to spread, the Allegiance of Allsight's expeditions have been targeted by other factions and would-be thieves. The Allegiance has recently located an item that was stolen during an encounter with one such thief: a \textit{figurine of wondrous power (marble elephant)}, now lying among the oddities and trinkets for sale at the Bone Garden in the Suncut Bazaar. The proprietor of the shop, Rerosha, has agreed to hand over the item to the Allegiance of Allsight. The faction tasks the characters with retrieving the figurine from Rerosha and delivering it safely to the Crystal Chateau. See the "Suncut Bazaar" section of the "Ank'Harel Gazetteer" for information about the Bone Garden.

\subsubsection{Getting the Mission}

This mission is communicated to the characters in a letter dropped on the doorstep of wherever they are staying. The message reads:

\begin{DndReadAloud}
Allegiance hopefuls,
We need you to pick up and deliver a marble figurine of an elephant that was stolen from us. We have located it at a curio shop called the Bone Garden in the Suncut Bazaar and arranged for the proprietor, Rerosha, to hand over the merchandise without a fuss. Simply bring the figurine to Gatekeeper Raashid at the Crystal Chateau, and your task is complete.
Welcome in advance to the team!---Professor Lymmle Wist, Allegiance of Allsight
\end{DndReadAloud}

\subsubsection{Volatile Figurine}

Unknown to anyone who has come into contact with it, the elephant figurine was infected with ruidium during its time in Cael Morrow, and that substance has warped its magical properties.

When a character tries to retrieve the figurine, it reacts as if it had been activated, transforming into a crystalline version of the creature it was meant to become (use the \textbf{ruidium elephant} stat block earlier in the chapter). This creature can't be controlled.

The figurine transforms even if the space where the elephant appears is occupied by other creatures. A creature in the space where the elephant appears must make a successful DC 13 Dexterity saving throw or take 11 (2d10) force damage and be pushed into the nearest unoccupied space.

The elephant exits the shop and begins rampaging through the streets of the bazaar. As the characters pursue, have everyone roll initiative; this encounter can be played out using the Streets of the Suncut Bazaar map (p. 4).

Reducing the elephant to 0 hit points causes it to transform back into a figurine, after which it can be handled and transported safely. The figurine can't transform again for seven days.

\paragraph{Success}

When the characters bring the figurine to Gatekeeper Raashid, outside the Crystal Chateau (area S1 on the Streets of the Sigil District map (p. 4)), their task is complete.

\paragraph{Failure}

If the characters fail to defeat the elephant or don't deliver the figurine, the Allegiance refuses their applications for membership. They'll need to find another faction to help them investigate the mysteries of the Apotheon.

\subsubsection{Keeping the Figurine}

If the characters keep the elephant figurine for themselves, the elephant it creates uses the \textbf{ruidium elephant} stat block (see earlier in the chapter) and can't be controlled.

If the characters slay or redeem the Apotheon (see chapter 7), the figurine is rid of its ruidium corruption and functions like a normal \textit{figurine of wondrous power (marble elephant)}.

\subsection{Allegiance Mission 2: Search the Life Dome}

Chasing down a rumor for the Allegiance of Allsight, the characters are sent to investigate a possible second point of access into Cael Morrow, said to be in the River District. Their time spent learning about the sunken city is cut short when mysterious forces attack the group with intent to kill.

\subsubsection{Getting the Mission}

Early one morning, \textbf{Prolix Yusaf} meets with the characters:

\begin{DndReadAloud}
The bright-eyed tiefling says, "As you might be aware, the Allegiance of Allsight is conducting a massive excavation project in the Sigil District. It's an effort to explore the ruins of Cael Morrow, a much older city below Ank'Harel.
"Well, the Allegiance has heard rumors of another entrance into Cael Morrow that runs through the city's oasis. Would you please do us the favor of investigating these claims? I'm authorized to pay your group three hundred gold pieces for your time and trouble."
\end{DndReadAloud}

If the characters agree to help, Prolix informs them that Carliale Kroogan, the gnome supervisor at the Life Dome, is expecting them in the early afternoon (see "Life Dome" earlier in the chapter).

\paragraph{True Motives}

Unknown to Prolix, the rumors of a second entrance to Cael Morrow were planted by Lymmle Wist, a member of the Sentinels of Memory who has infiltrated the Allegiance of Allsight by posing as a highly accredited professor of antiquities. Though Lymmle doesn't appear during this mission, her machinations drive the plot. The characters are walking into an ambush---an effort by the Sentinels of Memory to wipe out those who would bring Cael Morrow back into living memory.

\subsubsection{Into the Life Dome}

When they arrive at the Life Dome, the characters are met by Carliale Kroogan, who's standing outside:

\begin{DndReadAloud}
"I rarely get visitors," the gnome explains before turning to face the Life Dome. He holds aloft a thin shard of iridescent, glowing quartz, and, with a quiet hum, one of the dome's glass panels suddenly comes alight with blue and purple arcane runes---which then flicker and fade, leaving behind an opening. "Come on in! I'll give you the tour."
\end{DndReadAloud}

The inside of the dome is humid and sweltering, the smell of damp earth hitting one's senses along with a blast of hot air. A series of crystal bridges span the oasis's 60-foot-diameter lake, and the edges of the dome are lined with vegetation. Characters who have a passive Wisdom (Perception) score of 17 or higher notice a brief, cool current of air as they begin to move about inside the dome (the side effect of an \textbf{invisible stalker} stealing Carliale's magic crystal).

\paragraph{Inspecting the Oasis}

The lake's bottom is a thick layer of silt. A character who surveys the area and makes a successful DC 15 Wisdom (Perception) check finds no disturbance in the sediment that would suggest an underground passage to lower caverns.

\paragraph{Talking with Carliale}

Carliale tries to answer any questions the characters have about the oasis. He knows the following information:

\begin{itemize}
\item The oasis is the primary source of fresh water in Ank'Harel.
\item The water originates from the flooded caverns below Ank'Harel. These caverns are also where Cael Morrow is located, which lends credence to the rumors of an ancient passageway that connects Cael Morrow with the oasis.
\item Relics retrieved from the Allegiance of Allsight's expeditions indicate that Cael Morrow was the site of a pre-Calamity society originally built above the water. Carliale theorizes that natural tectonic shifting caused the city to slip beneath the surface, but the fluctuation of the water level in the caverns seems to defy explanation.
\item The Life Dome regulates the water flow to and through the canals. Using a combination of enchantment and engineering, the Life Dome dispenses water from the lake into the major canals while also drawing extra moisture from the air to supplement the water flow when necessary.
\item The Life Dome normally functions automatically. As chief supervisor, Carliale can manually override the dome's controls and adjust the water level as needed.
\end{itemize}

At some point during the conversation, Carliale notices that his arcane override---the magic crystal he uses to open the entrance to the Life Dome and to adjust the dome's controls---has disappeared from his pocket. He assures the party that he must have dropped it back at the entrance.

\subsubsection{Sudden Surge}

As Carliale begins searching for his missing arcane override crystal, read:

\begin{DndReadAloud}
The surface of the lake begins to churn white with foam, and seconds later an enormous geyser of water bursts upward and outward, sending you flying.
\end{DndReadAloud}

Everyone in the Life Dome is swept outside the dome and propelled along a 5-foot-deep canal by the sudden surge of water.

When they see that the characters survived the surge, two \textbf{invisible stalkers} move in for the kill. The \textbf{invisible stalkers} were summoned by Professor Lymmle Wist and infiltrated the Life Dome a short time ago without Carliale's knowledge, slipping through the entrance as the gnome arrived for work. The \textbf{invisible stalkers} focus their attacks on the characters, ignoring Carliale.

After the battle, Carliale suspects that the \textit{invisible} stalkers' summoner is associated with either the Consortium of the Vermilion Dream or the Sentinels of Memory. He describes these factions to the characters if they ask about them (see "City Factions" earlier in the chapter).

Characters who encountered the \textbf{invisible stalker} at Mystic Pursuits (described earlier in the chapter) might assume that Amkezne, the owner of that shop, is behind the Life Dome incident. Amkezne denies any involvement and is telling the truth. If the characters use magical coercion or threaten to report her to the Hands of Ord, Amkezne tells them that she learned how to summon \textbf{invisible stalkers} in her final year at the Crystal Chateau and that the professor who taught her the arcane ritual was Lymmle Wist (see mission 3 of this story track).

\subsubsection{Returning to the Chateau}

After foiling the ambush, the characters can take a moment to regroup before returning to the Crystal Chateau. After he hears their report, Prolix pays the characters the agreed amount of 300 gp and deeply apologizes for the events of the day. He expresses gratitude for the characters' skill, saying that he wishes the Allegiance of Allsight had more like them among their ranks.

\subsubsection{Joining the Allegiance}

If this mission is successful, Prolix recommends the characters to Headmaster \textbf{Gryz Alakritos} as potential new members. Within a week, Alakritos presents each character with a formal invitation to join the faction's ranks. Membership comes with the following benefits:

\begin{itemize}
\item Each new member receives an Allegiance of Allsight badge, which can be presented as proof of membership. If a badge is lost, a new one can be issued for 50 gp.
\item Each new member receives a stylish Allegiance of Allsight outfit. If the outfit is lost or damaged, a replacement can be purchased for 10 gp.
\item Each new member receives free lodging in Ank'Harel for as long as their membership lasts. The lodging is of modest quality (see "Lodging" earlier in the chapter).
\end{itemize}

The characters can't receive any more missions from the Allegiance of Allsight unless they become members of the faction.

\subsection{Allegiance Mission 3: The Proxies of Prolix}

Months ago, Watcher Lymmle Wist of the Sentinels of Memory infiltrated the Allegiance of Allsight. She has since advanced through the ranks and become one of the individuals responsible for assigning missions to Allegiance agents. During this time, she has also been feeding information about Allegiance operations to the Sentinels of Memory and the Consortium of the Vermilion Dream.

Lymmle previously owned a \textit{ring of red fury} (see appendix B (p. 9)), which was given to her by the Consortium as a reward for her services. She has decided to use the ring as part of a grander scheme. Recently, with the Allegiance of Allsight on high alert after the attack at the Life Dome, Lymmle planted the ring on \textbf{Prolix Yusaf} and plans to falsely accuse the tiefling of being a traitor. Her contact with the ruidium in the ring has left its mark on her skin.

\subsubsection{Getting the Mission}

The characters notice a dark figure clumsily following them as they stroll through Ank'Harel one night. Confronting the figure reveals it to be a harried and distressed \textbf{Prolix Yusaf}, who immediately asks if they can talk in private.

Prolix wears nondescript dark clothes instead of his usual Allegiance of Allsight attire. When he and the characters are away from prying ears, he explains his plight:

\begin{DndReadAloud}
"I think someone is trying to frame me for the attack at the Life Dome," he whispers.
"Earlier today, a thief snatched my satchel as I was walking down the street. I reported the theft, then went about my business. A few hours ago, someone used a \textit{sending} spell to notify me that my satchel had found its way to the Crystal Chateau and that I should claim it at once. I was on my way to retrieve it when I found this."
Rummaging in his pockets, he pulls out a silver ring with a stripe of red metal. "I think someone slipped it into my pocket, but I swear it's not mine. It looks like one of the magic items salvaged from Cael Morrow, but I can't be sure. You believe me, right?"
\end{DndReadAloud}

Prolix is earnest and desperate in his claims of innocence. He fears that whoever planted the ring on him also planted something in his satchel to further incriminate him. He has no intention of returning to the Crystal Chateau or checking in with his superiors until he learns who is trying to set him up.

If the characters agree to help, Prolix points them toward his friend, Bookkeeper Khime, who manages the Omnival Library in the Sigil District and has records of every item of antiquity salvaged from Cael Morrow. Prolix believes that if the characters can determine where the ring came from, they can find the mastermind behind this plot.

\subsubsection{Hunt for the Truth}

When the characters go to the Omnival Library, Khime greets them from behind the front desk. He isn't a member of the Allegiance of Allsight, but he is willing to assist the characters by producing the following information from the library's archives:

% Image placeholder: The Maw of Cael Morrow is the only known entrance to the Drowned City

\begin{description}
\item[Expeditions to Cael Morrow.] The records contain only one mention of a ring: in the inventory of an expedition from two months ago, which was staffed by agents Anbara Flintbreaker and Idris Lornen and led by Professor Lymmle Wist. The log indicates that the group was accosted by a thief on its way back to the surface, but mention of what was stolen is missing from the record.
\item[Sentinels of Memory.] Allegiance documents confirm that the Sentinels of Memory are a new faction, established at around the time the project to explore the sunken ruins began. The documents assert that the Sentinels of Memory believe the so-called "Under-Temple" is a prison, meant to be kept sealed and forgotten.
\item[Ruidium.] The name coined by the Consortium of the Vermilion Dream to identify the reddish mineral found in the ruins of Cael Morrow has entered the academic lexicon. The extent of its properties has yet to be known, but early reports say that the mineral produces red crystalline veins in other substances---inorganic and organic alike---that remain in prolonged contact with it.
\end{description}

A character who examines the expedition records and makes a successful DC 10 Intelligence (Investigation) check notices a pattern: some of the expedition logs and inventories of recovered items have been redacted or left unfinished. A character who succeeds on this check by 5 or more notices that all these records were from expeditions led by Lymmle Wist.

If asked for his opinion, Khime confides in the characters that he's inclined to believe the rumors that the Sentinels of Memory will do anything to achieve their goal: keeping Ank'Harel safe from whatever might be lurking below it. Furthermore, he believes the Consortium of the Vermilion Dream and the Sentinels of Memory are working together against their common enemy, the Allegiance of Allsight.

\paragraph{Interrogating the Excavators}

To learn more about what they discovered in the library records, the characters can question the expedition members at the Crystal Chateau.

Anbara and Idris (lawful neutral, human \textbf{acolytes}), the two agents from the expedition that recovered the ring, vaguely recall being attacked during their work, but they can't provide any details. A character who succeeds on a DC 15 Intelligence (Arcana) or Wisdom (Insight) check discerns that this claim of ignorance is not a lie---Anbara and Idris have had their memories of the incident tampered with. Restoring their memories with a \textit{remove curse} spell or similar magic causes them to recall that Professor Wist led the group through a wrong turn, to an area where they were accosted by an armed ruffian. They were knocked out in the ensuing scuffle, but they can't identify who cast the spell or wielded the weapon that \textit{incapacitated} them.

If Lymmle Wist (a neutral evil, human \textbf{mage}) is questioned, she deftly addresses the characters' inquiries, and her responses conform to the expedition records. A character who succeeds on a DC 16 Wisdom (Insight) check notices a guarded expression in her eyes and a terse undertone in her voice, indicating her discomfort. Characters who have a passive Wisdom (Perception) score of 15 or higher notice thin, elevated red veins encircling the base of her right ring finger---the result of her exposure to the \textit{ring of red fury} before she planted it on Prolix.

\subsubsection{Infiltrator Exposed}

The characters can present their findings to Headmaster \textbf{Gryz Alakritos} (see "City Factions" earlier in the chapter) and make a DC 18 Charisma (Persuasion) check. The DC drops to 13 if Anbara and Idris are with the characters and at least one of them has had their memory restored. On a successful check, the headmaster immediately dispatches agents to apprehend Lymmle.

The characters can confront Lymmle directly by going to her office in the Crystal Chateau. Lymmle initially denies the accusation, then tries to turn the tables on the characters by accusing them of having ulterior motives. When it becomes clear to her that the characters are insistent, she casts \textit{fly} on herself and tries to escape by smashing through the window behind her desk. If she's unable to escape, Lymmle fights to the death.

\subsubsection{Epilogue}

After Lymmle is exposed as a traitor, Headmaster Alakritos gives the characters a pouch containing a reward of 50 pp. He also hands them a satchel and asks them to return it to Prolix---minus the Life Dome schematics and other incriminating paperwork planted in it by Lymmle.

In gratitude, Prolix lets the characters keep the \textit{ring of red fury} (see appendix B (p. 9)) as long as they promise to share anything they learn about the ring's properties.

\subsection{Allegiance Mission 4: The Double Agent}

Seven days after their previous mission for the Allegiance of Allsight, the characters are summoned to meet with Headmasters \textbf{James Cryon} and \textbf{Gryz Alakritos} in Cryon's office at the Crystal Chateau. The headmasters want the characters to find a double agent who is undermining the Allegiance's activities in the ruins of Cael Morrow.

\subsubsection{Getting the Mission}

When the characters arrive for their meeting with the headmasters, read or paraphrase the following:

\begin{DndReadAloud}
As you step into the office of Headmaster Cryon, you see that both he and Headmaster Alakritos are present. The former welcomes you curtly, while the latter gestures for you to take seats in armchairs.
"It seems you can add counterespionage to your list of accomplishments," Headmaster Cryon says coolly.
"And a good thing, too!" Headmaster Alakritos follows up hastily. "There is trouble in Cael Morrow."
\end{DndReadAloud}

\textbf{James Cryon} is a condescending elitist, while \textbf{Gryz Alakritos} is less abrasive and more down-to-earth. Each, in his own way, has the best interests of the Allegiance of Allsight at heart. They brief the characters on the following details that Allegiance spies have uncovered:

\begin{itemize}
\item The Consortium of the Vermilion Dream has infiltrated the Allegiance's excavation site in Cael Morrow.
\item There is a double agent in the archaeological team.
\item This spy is presumably helping the Consortium find and smuggle ruidium.
\end{itemize}

The headmasters request that the characters accomplish the following tasks and report back within seven days:

\begin{itemize}
\item Enter Cael Morrow, travel to the Allegiance of Allsight's base camp, and meet with \textbf{Insight Acuere}, a professor of archaeology and the leader of the current expedition.
\item Explore Cael Morrow and capture or kill the Consortium spy.
\end{itemize}

Alakritos gives each character an Allegiance of Allsight badge if they don't already have one, as well as a \textit{potion of water breathing} (in case the character wants to travel outside the air-filled areas of Cael Morrow). The badges are proof of the characters' Allegiance affiliation and allow them to enter Cael Morrow and move around freely within the excavation site. Cryon warns them that the spy must have already acquired a badge.

The reward for eliminating the double agent is 1,000 gp and continued access to Cael Morrow for as long as the characters remain members of the Allegiance of Allsight.

\subsubsection{Finding the Double Agent}

The number of workers in the Cael Morrow archaeological expedition has dwindled recently. Currently, only Professor \textbf{Insight Acuere}, a kenku named \textbf{Scribble}, and a goliath named \textbf{Xot} are there. None of them is the double agent.

If the characters ask Professor Acuere about other members of the expedition, she mentions that many excavators have been lost. She seems most interested in learning the fate of her friend \textbf{Galeokaerda}, an adjunct professor of ancient literature at the Crystal Chateau. Unknown to Professor Acuere, \textbf{Galeokaerda} is the Consortium spy.

\subsubsection{Key Locations}

This mission takes place primarily in Cael Morrow. Key locations the characters must visit are as follows:

\begin{description}
\item[Maw of Cael Morrow.] This location is the entrance to the Drowned City (see the "Maw of Cael Morrow" section earlier in the chapter).
\end{description}

\paragraph{Allegiance Base Camp (Chapter 5, Area M3)}

This site is the center of Allegiance operations in Cael Morrow. Professor \textbf{Insight Acuere} runs the show from here.

\paragraph{Hall of the Royal Library (Chapter 5, Area M7a)}

Here is where the spy, \textbf{Galeokaerda}, can be found. She is trying to find the so-called Key to the Netherdeep. If seven days pass before the characters reach this location, she is gone.

\paragraph{Cliffside Villa (Chapter 5, Area M13)}

This site is where the characters can find the Key to the Netherdeep that \textbf{Galeokaerda} is searching for. If seven days pass and the characters haven't defeated \textbf{Galeokaerda}, the key is gone.

\subsubsection{Returning to the Chateau}

If the characters defeat \textbf{Galeokaerda}, the headmasters give them the 1,000 gp reward as promised, and Alakritos tells them to expect their next mission in a week's time. If the characters bring back the \textit{ruidium greataxe} (the Key to the Netherdeep), they are awarded an additional 1,000 gp.

\subsection{Allegiance Mission 5: Scout the Rift}

Seven days after the characters return from their first expedition into Cael Morrow, Headmaster Cryon invites them back to his office to brief them on a new mission that will lead them through a planar rift and into the Netherdeep.

\subsubsection{Getting the Mission}

Headmaster Cryon begins the meeting by sternly proclaiming, "The survival of the Allegiance of Allsight likely depends on the success of this mission." He then provides characters with the following details:

\begin{itemize}
\item The Allegiance of Allsight has found a planar rift in Cael Morrow that leads to an unknown destination.
\item The Consortium of the Vermilion Dream has sent agents to the rift to figure out how to pass through it, with unknown results.
\item If the Allegiance of Allsight is to prevent the Consortium from taking over Cael Morrow, the characters must enter the rift and learn what is on the other side.
\end{itemize}

A character who succeeds on a DC 19 Wisdom (Insight) check realizes that Headmaster Cryon is on edge because he arranged this meeting without Headmaster Alakritos's knowledge. Prior to the characters' arrival, Alakritos and Cryon had a disagreement over the final point in the briefing. Alakritos believes that the rift must not be disturbed, but Cryon thinks that it contains a source of power that would help destroy the Consortium of the Vermilion Dream and the Sentinels of Memory.

Headmaster Cryon requests that the characters accomplish the following tasks and report back to him before seven days have elapsed:

\begin{itemize}
\item Find a way to open the rift.
\item Enter the rift and confirm that ruidium can be found inside it.
\end{itemize}

If the characters acquired the Key to the Netherdeep in the previous mission, they have already accomplished their first goal. In that case, Cryon gives the greataxe to them so they can use it to open the rift. He also offers each character a \textit{potion of water breathing}, in case they have no other way to navigate the water-filled areas of Cael Morrow.

The reward for entering the rift (and thus, the Netherdeep) and reporting on what lies beyond is 1,000 gp.

\subsubsection{Key Locations}

This mission takes place primarily in Cael Morrow. Key locations the characters must visit are as follows:

\begin{description}
\item[Maw of Cael Morrow.] This location is the main entrance to the Drowned City (see the "Maw of Cael Morrow" section earlier in the chapter).
\end{description}

\paragraph{Cliffside Villa (Chapter 5, Area M13)}

This site is where the characters can find the Key to the Netherdeep.

\paragraph{Temple of the Arch Heart (Chapter 5, Area M9)}

This structure is Alyxian's final prayer site. If the characters haven't acquired the Key to the Netherdeep, advancing the \textit{Jewel of Three Prayers} to its \textit{Exalted State} at this prayer site will enable them to enter the rift.

\paragraph{Rift to the Netherdeep (Chapter 5, Area M17)}

This planar rift connects the Netherdeep with the Material Plane.

\subsubsection{Returning to the Chateau}

When they enter the Netherdeep, the characters must contend with the water pressure (see "Netherdeep Features" in chapter 6 (p. 6)). If they don't have sufficient protection, they'll need to return to the Crystal Chateau and acquire it.

After the mission concludes, the characters come back to the Chateau to find Cryon and Alakritos arguing furiously. Cryon turns to them with an expression of barely concealed glee and thanks them for their service. Alakritos glumly thanks them as well, then encourages them to rest up while they await their next mission. There is no turning back now that the rift has been breached.

\subsection{Allegiance Mission 6: Secure the Netherdeep}

The final mission in the Allegiance Story Track requires the characters to delve into the Netherdeep and secure a large enough amount of ruidium to permanently thwart the Allegiance's enemies.

\subsubsection{Getting the Mission}

Seven days after the characters return from their previous mission, Headmasters Cryon and Alakritos brief the characters on their new job:

\begin{itemize}
\item In recent days, the Allegiance of Allsight has perfected the crafting of ruidium weapons and armor. Initial experimentation has established that such equipment provides its user with protection from the hazards of an underwater environment.
\item Several of these weapons were issued to Allegiance scouts, who were given instructions to submerge themselves and explore the rift. All but one of them have failed to return and are presumed dead.
\item The lone survivor briefly spoke to an expedition member before succumbing to her wounds. She said her ruidium weapon enabled her to survive the water pressure beyond the rift, and she described hearing the voice of a "sorrowful intelligence" as she navigated the Netherdeep, before running into a creature that laid her low.
\end{itemize}

The headmasters request that the characters accomplish the following tasks and inform them of the results before fourteen days have elapsed:

\begin{itemize}
\item Enter the Netherdeep.
\item Contact the "sorrowful intelligence" that resides within, and convince it to allow the Allegiance of Allsight to extract ruidium without being threatened by the creatures of the Netherdeep.
\end{itemize}

If the party agrees to undertake the mission, the headmasters equip each character with a \textit{ruidium dagger}. The headmasters' reward for completion of this mission is a promotion to the upper ranks of the Allegiance of Allsight and 5,000 gp each.

\subsubsection{Failure}

This mission, unbeknownst to the characters or anyone else, is doomed to fail. If the characters talk to Alyxian about harvesting ruidium while they're in the Netherdeep, Alyxian promises them whatever they want, as long as they free him from the Heart of Despair. The characters have no chance of claiming the reward, regardless of the adventure's outcome (see chapter 7 (p. 7)).

\begin{DndSidebar}{Which Faction Should the Rivals Join?}
Use the characters' choice of faction to determine which faction the rival party joins, as follows:
\begin{itemize}
\item If the characters join the Allegiance of Allsight or the Library of the Cobalt Soul, the rivals join the Consortium of the Vermilion Dream.
\item If the characters join the Consortium of the Vermilion Dream, the rivals join the Allegiance of Allsight.
\end{itemize}
\end{DndSidebar}

\subsection{Vermilion Gambits}

This section details the missions in the Consortium Story Track, which are summarized in the accompanying flowchart.

\subsubsection{Consortium Story Track}



\subsection{Consortium Mission 1: A Ghost in Our Midst}

Characters who weren't introduced to the Consortium of the Vermilion Dream by Aloysia might have a hard time joining the faction, given its secretive nature. Those who want to pursue membership in the faction without Aloysia's help can spend a day gathering information from Ank'Harel's citizens. At the end of that time, any character who succeeds on a DC 15 Charisma (Investigation) check has learned that an organization of red-robed occultists meets in a tavern called First Eclipse.

If the characters enter First Eclipse (see the "Suncut Bazaar" section earlier in the chapter), they are greeted by the bartender, Satzrak Runestrider, a neutral evil, blue dragonborn \textbf{occult extollant} (see appendix A (p. 8)). If they ask him about the Consortium, he gives each of them a drink on the house and tells them to keep quiet. When they finish their drinks, Satzrak hisses, "Go home and wait for a letter."

\subsubsection{Getting the Mission}

Characters who have expressed interest in joining the Consortium of the Vermilion Dream receive an unsigned letter at their lodgings. It reads:

\begin{DndReadAloud}
Aspirants,
We have a mission that will test your abilities. Do not try to contact us again until you have fulfilled it.
Rumors say a ghost haunts the shrine to the Wild Mother in the Sigil District in the dark hours. Folk who enter or even pass too close the shrine at night hear dreadful wailing and feel the touch of blood-chilling, ghostly hands.
Learn the truth about this spirit and, if possible, turn its wrath to our benefit. Any advantage we can have in the Sigil District is a useful one.
\end{DndReadAloud}

\subsubsection{Search for the Ghost}

Finding the shrine is simple, since there is only one shrine to the Wild Mother in the Sigil District (area S7 on the Streets of the Sigil District map (p. 4)). If the characters investigate this area at night, one of them is targeted by a \textit{chill touch} spell as a lonely wail fills the air. If they enter the shrine, they see movement inside.

The "ghost" is \textbf{Shira}, a shy, anxious 15-year-old human with innate spellcasting ability. Fearing for the safety of those around her as she learns to control her powers, \textbf{Shira} has taken to using the shrine as a nighttime hiding place. She mimics the behavior of a spirit to frighten away others.

\textbf{Shira} is a chaotic good, human \textbf{commoner}, with the following additional action option:

\begin{DndSidebar}{}
\paragraph{Spellcasting}

\textbf{Shira} casts one of the following spells, requiring no material components and using Charisma as the spellcasting ability (spell save DC 10, +2 to hit with spell attacks):

At will: \textit{chill touch}

1/day each: \textit{blindness/deafness}, \textit{ray of sickness}
\end{DndSidebar}

\textbf{Shira} uses her spells to drive away intruders. A character can use an action to try to force her to surrender, doing so with a successful DC 14 Charisma (Intimidation or Persuasion) check, whereupon \textbf{Shira} breaks down into tears and ceases her attacks. Only then can she be convinced to try to find another place in the city to sleep. She will grudgingly accept being taken in by a foster parent, if the characters offer to help her with this solution.

\subsubsection{A Home for Shira}

If the characters want to find a safe home for \textbf{Shira} and bring her to First Eclipse for that purpose, Satzrak the bartender shows a soft side of his persona and offers her a small, private room in the tavern, along with free room and board if she's willing to perform simple chores and "not make trouble." Satzrak is inwardly happy to be a kind, loving father figure to \textbf{Shira}.

\subsection{Consortium Mission 2: When Luck Runs Out}

As a small organization with expensive needs, the Consortium of the Vermilion Dream is in constant need of funds, prompting the faction's leaders to loan out the services of their agents and associates to customers who want protection.

Adima Shemsilver, the proprietor of the Luck's Run casino, has heard rumors that a group of thieves plans to rob the casino's coin vault. To avoid involving the Hands of Ord, Adima has turned to the Consortium for protection, and the Consortium turns to the characters.

\subsubsection{Getting the Mission}

The characters receive an urgently worded missive summoning them to First Eclipse, where Satzrak orders them a round of ale on the house and slides a note beneath one of the drinks that reads, "She's waiting in the back for you." If pressed for more information, Satzrak doesn't elaborate.

Inside the Consortium's hidden storeroom in the back of the tavern, the characters find \textbf{Aloysia Telfan}. If Aloysia died earlier in the adventure, replace her with another \textbf{occult initiate} (see appendix A (p. 8)), and modify the following text accordingly:

\begin{DndReadAloud}
"Well, you made it," Aloysia says. She hops off her stool and holds out a leather bag, jingling with the sound of coins. "I hope you're not disappointed by mercenary work. It's not as exciting as other jobs, but it pays well."
\end{DndReadAloud}

Aloysia explains that the Consortium has been hired by Adima Shemsilver, the owner of Luck's Run, to guard the casino against a robbery rumored to be happening that night.

If the characters accept the task, Aloysia pays them 250 gp now and says that Luck's Run will pay them another 750 gp for the successful capture of the thieves, 500 gp of which is to be handed over to the Consortium. She tells them to arrive at Luck's Run an hour before sundown, when the captain of the casino's security staff will brief them further.

\subsubsection{At the Casino}

When the characters get to Luck's Run, Security Captain Nedosi Anay, a neutral, half-orc \textbf{veteran}, brusquely briefs them. She relays the following information:

\begin{itemize}
\item Nedosi's intelligence reports say that at least three thieves will be working together to pull off this operation, which begins at sundown.
\item Nedosi assumes the thieves will target the casino's coin vault on the third floor, where most of the establishment's earnings are stored each day.
\item The coin vault is sealed with a magic lock. Narrow chutes, each 6 inches in diameter, extend from the inside of the vault to dispensers on the lower floors. Only casino employees can access these dispensers, which they must do before doling out winnings to the casino's patrons.
\end{itemize}

Nedosi wants two characters to station themselves right outside the coin vault while the rest patrol the casino's ground floor with her. If the characters don't have the ability to communicate with each other from a distance, she gives each of them a fully charged \textit{earring of message} (see appendix B (p. 9)) to use for the night before ordering them to their posts. She demands that the characters use force only if necessary, and only of the nonlethal variety.

\paragraph{Casino Floor}

At night the casino is crowded, as people filter in after work has finished for the day to try their hand at one of the casino's games. Each floor of the casino is patrolled by four \textbf{guards}; in addition, the ground floor is staffed by ten \textbf{commoners}. A pair of finely upholstered spiral staircases in the back of the building connects the three floors. Each staircase is magical and functions like an escalator---one has stairs that carry patrons upward, while the other has stairs that move downward.

\paragraph{Vault Doors}

Forged of gleaming steel, the casino's cylindrical coin vault sits in the center of the third floor---in full view from any direction, so no one can approach the vault without being seen.

The vault is 10 feet in diameter and rises 20 feet to the apex of the casino's domed ceiling. It appears to have no door, but a character who examines the vault and succeeds on a successful DC 20 Intelligence (Investigation) check finds a thin keyhole near its base. The vault's lock can be opened by using a master key, which only Nedosi and Adima possess, by a character who makes a successful DC 21 Dexterity check using \textit{thieves' tools}, or by magic. Nedosi has her copy of the master key on her person.

\paragraph{Vault Trap}

If the vault is opened by any means other than using the master key, the vault's magical trap is triggered. Each creature within 5 feet of the vault must make a DC 16 Wisdom saving throw. On a failed saving throw, a creature takes 36 (8d8) psychic damage and is \textit{stunned} for 1 minute. On a successful save, a creature takes half as much damage and isn't \textit{stunned}.

The trap can't be disarmed, but successfully casting \textit{dispel magic} (DC 15) on the lock before an attempt is made to open it suppresses the trap for 1 minute, allowing the lock to be manipulated without negative consequences.

\subsubsection{Casino Heist}

As the sun sinks below the horizon, the thieves put their plan in motion. Three individuals are involved in the operation:

\begin{itemize}
\item Ena (neutral, human \textbf{berserker}) masquerades as a casino patron.
\item Duskwood (chaotic neutral, half-elf \textbf{spy}), like Ena, masquerades as a casino patron and wears a \textit{hat of disguise}.
\item \textbf{Koris} (lawful neutral, tiefling \textbf{assassin}), the crew's leader, works at one of the casino's card tables---a job she has held for about a month. She carries a \textit{potion of gaseous form}, a tiny packet containing \textit{dust of disappearance}, and a \textit{bag of holding}.
\end{itemize}

If any of the thieves are confronted before they can pull off the heist, they try to flee the casino and foil any pursuers by running through the streets. Resolve such a chase using the chase rules, p. 8 and the Urban Chase Complications table in the Dungeon Master's Guide.

The planned heist has four stages:

\begin{description}
\item[Stage 1:] Lead-Up. Ena, Duskwood, and \textbf{Koris} begin on the ground floor of the casino. A character who scans the crowd and succeeds on a DC 11 Wisdom (Insight) check notices Ena's disinterest in the dice game she's standing at. Shortly thereafter, Duskwood begins chatting with one of the guards near Nedosi, and a character who succeeds on a DC 15 Wisdom (Insight) check notices the half-elf watching someone over the shoulder of the guard.
\item[Stage 2:] Infiltration. Characters on floor duty who know thieves' cant or have a passive Wisdom (Perception) score of 15 or higher notice the three thieves flashing quick hand signals to each other. \textbf{Koris} abruptly excuses herself from the game she's running and heads toward the employee area; a character who succeeds on a DC 13 Dexterity (Stealth) check can follow her without being noticed. If she is detained, \textbf{Koris} explains that she's on the way to collect winnings for one of her patrons, but a character who succeeds on a DC 13 Wisdom (Insight) check sees through the lie. If she is not prevented from doing so, \textbf{Koris} imbibes a \textit{potion of gaseous form} and enters one of the coin chutes that connect with the vault.
\item[Stage 3:] Coin with Two Heads. When \textbf{Koris} has entered the coin chute, Ena stands up from her seat and starts berating the guard whom Duskwood is chatting with. As Nedosi tries to break up the fight, Duskwood lifts the casino master key from her pocket, an act that a character on floor duty can spot with a successful DC 14 Wisdom (Perception) check. Duskwood then ascends the spiral staircase and uses their \textit{hat of disguise} to assume Nedosi's visage before arriving on the third floor. A character who succeeds on a DC 13 Intelligence (Investigation) check sees through Duskwood's trick as the thief approaches the vault.
\end{description}

\paragraph{Stage 4}

\textbf{Getaway}. Inside the vault, \textbf{Koris} dismisses her gaseous form, dumps a total of 10,000 gp into her \textit{bag of holding}, and then uses \textit{dust of disappearance} to become \textit{invisible}. Duskwood, disguised as Nedosi, unlocks the vault from the outside, pretends to inspect the interior as the \textit{invisible} \textbf{Koris} escapes, and shuts it again. Characters standing guard outside the vault hear \textbf{Koris} slip past them with a successful DC 24 Wisdom (Perception) check. Duskwood and \textbf{Koris} both leave the third floor while Ena faints in the arms of a guard downstairs to create a minor distraction. Ena then heads outside to "get some air." If the other two thieves are not stopped, the pair reunites with Ena outside the casino, and all three escape into the night with their spoils.

% Image placeholder: {@creature Koris|CRCotN} flees the scene with adventurers in pursuit

\paragraph{Aiding and Abetting}

If any of the thieves are approached privately about their intent, they offer the characters a cut of the stolen money, amounting to 800 gp, if they let the thieves perpetrate the heist. A character who makes a successful DC 11 Wisdom (Insight) check involving any of the three thieves discerns they are earnest about this offer.

If the characters aid the thieves without getting caught, \textbf{Koris} gives them their cut and thanks them, promising to return the favor in the future. After the heist, finding \textbf{Koris} in Ank'Harel requires 8 hours of searching and a successful DC 18 Intelligence (Investigation) check, since she's keeping a low profile. A failed check can be repeated, but only after another search.

\subsubsection{Returning to First Eclipse}

The characters' stint as mercenaries ends at sunrise, after which they are expected to return to First Eclipse. Their mission has one of two outcomes, each described below.

\paragraph{Thieves Are Caught}

If the characters apprehended the thieves, Nedosi thanks them and pays them the agreed-upon 750 gp before they depart the casino. She explains that, even though the casino is loath to interact with the Hands of Ord, the thieves will be turned over to the Hands for justice to take its course. The characters are welcome to keep any magic items they confiscated from the thieves.

Aloysia is waiting at First Eclipse to collect the 500 gp that is due to the Consortium. She says she hopes to see the characters advance in the ranks of the Consortium.

\paragraph{Thieves Escaped}

If the characters were unable to stop the thieves or let them go, Nedosi is frustrated by their failure, but she still pays them 375 gp (half the agreed-upon amount) for their efforts.

When the characters return to First Eclipse and tell Aloysia what happened, she expresses disappointment but says the Consortium will overlook their failure if they're willing to provide the expected 500 gp.

\subsubsection{Joining the Consortium}

If the characters complete the mission and hand over the expected 500 gp, they are offered membership in the Consortium. Membership comes with the following benefits:

\begin{itemize}
\item Each new member receives a red robe like those worn by other members of the Consortium. If the robe is lost or damaged, a replacement can be purchased at First Eclipse for 1 gp.
\item Each new member receives free lodging in Ank'Harel for as long as their membership lasts. The lodging is of squalid quality (see "Lodging" earlier in the chapter).
\item Each new member receives a 10 percent discount on purchases at the arcane shop Mystic Pursuits (see "Mystic Pursuits" earlier in the chapter). The discount expires in thirty days.
\end{itemize}

The characters can't receive any more missions from the Consortium unless they become members of the faction.

\subsection{Consortium Mission 3: Elephant in the Room}

The Allegiance of Allsight has recently come into possession of a ruidium magic item found in Cael Morrow. The leaders of the Consortium want the characters to infiltrate the Crystal Chateau and obtain the item for them.

\subsubsection{Getting the Mission}

The characters are summoned to First Eclipse, where Satzrak quietly informs them that one of the masters of the Consortium wants to speak with them, though he doesn't know why. Following Satzrak's directions, the characters enter the tavern's secret storeroom to find Master \textbf{Aradrine the Owl}, a lawful neutral, goliath \textbf{occult silvertongue} (see appendix A (p. 8)), waiting for them. For more information on goliaths, see the "Goliaths of Exandria" sidebar earlier in the chapter.

Aradrine expresses appreciation for the characters' decision to join their ranks before asking if they are now ready to learn the truth about the Consortium's studies in Ank'Harel:

\begin{DndReadAloud}
"Behold," Aradrine says, rolling up the sleeve of her robe to expose the web of pulsing red veins that spreads across her forearm. She brings out a dagger from the folds of her robe. Its silver blade is veined with organic growths the same color as the ones on her arm. "This," she says, pointing to the strip of metal, "is ruidium. A gift from Ruidus, the Moon of Ill Omen, salvaged from the ruins beneath the city. Some recoil at its power, but we embrace it."
\end{DndReadAloud}

Aradrine explains that the Consortium first became aware of this strange mineral through the efforts of a spy who had access to Cael Morrow. The Consortium has sent other agents to infiltrate the excavation site and steal samples of the mineral, but the Allegiance of Allsight's control over Cael Morrow makes this task exceedingly difficult, prompting further acts of espionage in Ank'Harel.

Aradrine wants the characters to retrieve a \textit{figurine of wondrous power} in the shape of an elephant that has been affected by ruidium. She believes the figurine is being kept inside the Crystal Chateau. She asks the characters to press the Crystal Chateau's gatekeeper, Jor Raashid, for information.

Locations in this mission are keyed to the Streets of the Sigil District (p. 4) map earlier in the chapter.

\subsubsection{Grilling the Gatekeeper}

The characters find Gatekeeper Jor Raashid (see "Crystal Chateau" earlier in the chapter) standing next to the double door that leads into the Crystal Chateau's entry hall (area S1). To coax information out of the halfling, a character must either deceive, intimidate, or otherwise persuade him by succeeding on a DC 16 Charisma (Deception, Intimidation, or Persuasion) check. If the party has successfully completed a mission for the Allegiance of Allsight and thus has been seen by Jor before, this check is made with advantage.

Jor knows that the elephant figurine has been moved from the Crystal Chateau to a vault inside the Emerald Ibis Relic Hall (area S4 on the Streets of the Sigil District map presented earlier in the chapter). If the characters fail to obtain this information by other means, a character can pry it from his mind with a \textit{detect thoughts} spell or similar magic.

\subsubsection{Assassins Above}

As the characters travel to the storehouse where the elephant figurine is being kept, they are watched from the rooftops by two lawful evil, human \textbf{assassins} who have been hired by the Sentinels of Memory to grab the figurine from the storehouse and return it to the faction's headquarters. After watching the characters interact with Jor Raashid, the assassins follow them to the storehouse. When the characters arrive, the assassins attack. Characters who have a passive Wisdom (Perception) score of 19 or higher spot the assassins before they attack and are not surprised during the first round of combat.

\paragraph{Long Hands of the Law}

On initiative count 20 of the third round of combat, five Hands of Ord \textbf{knights} arrive to apprehend everyone involved in the brawl. The assassins try to flee rather than face arrest and incarceration. Two of the knights pursue the assassins while the remaining knights try to take the characters into custody. If the characters don't resist, they are held overnight in a Sigil District jail, then released the next morning and fined 50 gp each for disturbing the peace. This delay enables the Allegiance of Allsight to keep the \textit{figurine of wondrous power} safe, and the mission ends in failure (see "Mission Failed" below).

\subsubsection{Emerald Ibis Relic Hall}

During the day, this storehouse is open to the public. Students of the Crystal Chateau mingle and study the items that are displayed in glass bells throughout this warmly furnished research hall.

\paragraph{Daytime Entry}

The hall is managed by Relickeeper Kareema, a neutral good, orc \textbf{scholarly agent} (see appendix A (p. 8)) with messy dark curls and round glasses. If the characters convince her that they are members of the Allegiance of Allsight with a successful DC 18 Charisma (Deception) check, she reveals that the figurine was placed in the downstairs vault (see "Basement Vault" below). On a failed check, Kareema becomes suspicious of the characters' intentions; she gives them the information, but then scuttles away to alert her superiors.

\paragraph{Nighttime Entry}

The hall's front door is locked at night. A character who makes a successful DC 21 Dexterity check using \textit{thieves' tools} can open the door. It can also be forced open by a character who succeeds on a DC 15 Strength (Athletics) check. Anyone who breaks into the hall in this way is attacked by six animated statues that resemble ibises. The statues, which are made of green glass, patrol the public spaces during off hours. A single character can avoid the statues' notice with a successful DC 15 Dexterity (Stealth) check; if two or more characters are present, have them make a group check instead.

Each \textbf{animated glass statue} uses the \textbf{giant vulture} stat block, with these changes:

\begin{itemize}
\item Each statue is a Construct that has darkvision out to a range of 60 feet.
\item Each statue has immunity to the \textit{charmed}, \textit{exhaustion}, \textit{frightened}, \textit{paralyzed}, \textit{petrified}, and \textit{poisoned} conditions.
\end{itemize}

A search of the room reveals that the figurine isn't in the public display area. But the characters discover a staircase in the southeast corner of the building that leads down to the basement (see "Basement Vault" below), which is not patrolled by animated statues.

\subsubsection{Basement Vault}

The basement of the hall contains a hidden vault. A character who searches the basement and succeeds on a DC 15 Wisdom (Perception) check notices the faint outline of a circular stone door in the wall next to a bookcase that is bolted to the floor.

A character who examines the bookcase and succeeds on a DC 13 Intelligence (Investigation) check spots four tiny symbols engraved in its wood. A character who then succeeds on a DC 10 Intelligence (Arcana) check recognizes them as symbols of abjuration, evocation, divination, and transmutation. When a character casts a spell from one of those schools of magic while within 10 feet of the bookcase, the bookcase absorbs the spell (which is expended with no effect), and the corresponding rune begins to glow. If a spell from each of the four schools is absorbed by the bookcase, the vault door swings open. Magic cannot otherwise open the door.

A character can use an action to try to forcibly pry open the vault door, doing so with a successful DC 21 Strength (Athletics) check. Forcing the vault open, however, produces a loud, concussive blast of energy that fills a 15-foot cube immediately outside the vault. This trap can't be disarmed. Any creature in the cube must make a DC 18 Constitution saving throw, taking 33 (6d10) thunder damage on a failed save, or half as much damage on a successful one.

\paragraph{Treasure}

A short flight of stairs descends into the main chamber of the vault. Square alcoves line the walls, and inside one of them is the \textit{figurine of wondrous power}.

Most of the other alcoves are empty, but a thorough search of the vault also yields a set of \textit{Nolzur's marvelous pigments} and a \textit{wand of wonder}.

\subsubsection{Hasty Getaway}

If the characters aroused Kareema's suspicions or activated the trap in the basement vault, they are intercepted as they exit the hall by Headmaster \textbf{James Cryon}, a lawful neutral, elf \textbf{scholarly mastermind}, and two lawful neutral \textbf{scholarly excavators} serving as bodyguards (see appendix A (p. 8) for their stat blocks). Headmaster Cryon is willing to talk as long as the characters seem willing to relinquish the figurine, but if any character tries to escape with it, Headmaster Cryon and his bodyguards attack.

The three Allegiance members try to incapacitate the characters and reclaim any items that were removed from the vault. If Headmaster Cryon is reduced to fewer than 25 hit points, he curses and calls for a retreat, allowing the characters to get away.

% Image placeholder: Master {@creature Aradrine the Owl|CRCotN}

\subsubsection{Returning with the Elephant}

Master Aradrine is delighted if the characters deliver the elephant figurine to her, and she allows them to keep any plunder they gained from their mission. In addition, she makes arrangements to have one \textit{ruidium weapon} or a suit of \textit{ruidium armor} of their choice (see appendix B (p. 9)) delivered to them within 1d6 days.

\subsubsection{Keeping the Figurine}

The characters can keep the elephant figurine, but they would be wise not to use it, for the elephant it creates uses the \textbf{ruidium elephant} stat block (see earlier in the chapter) and can't be controlled.

If the characters slay or redeem the Apotheon (see chapter 7), the figurine is rid of its ruidium corruption and functions like a normal \textit{figurine of wondrous power (marble elephant)}.

\subsubsection{Mission Failed}

If the characters return to Aradrine without the figurine, the goliath beholds them coolly and snarls, "I have precious little sympathy for failures. Get out of my sight." Characters who beg for Aradrine's forgiveness only anger her further. If they refuse to leave, she calls in six \textbf{occult initiates} (see appendix A (p. 8)) to remove them.

After a week of silence, the characters receive a letter from \textbf{Aloysia Telfan}. If Aloysia died earlier in the adventure, the letter is sent by another Consortium agent (modify the following text accordingly):

\begin{DndReadAloud}
Fellow initiates,
Master Aradrine has decided to overlook your failure. Your connection to the Apotheon is too strong to allow a single misstep to disqualify you from further service. She requests your presence for a significant assignment---one that will require utmost stealth. Another failure will not be tolerated.
---\textbf{Aloysia Telfan}
\end{DndReadAloud}

\subsection{Consortium Mission 4: Key to the Netherdeep}

Seven days after the end of their last mission, the characters are invited to meet with Aradrine at her home---a modest apartment with shuttered windows in Ank'Harel's Guided District. Read or paraphrase the following when they arrive:

\begin{DndReadAloud}
A panel on the door to Aradrine's apartment slides open. Her eyes peer at you for a moment; then the panel closes and the door opens. The interior is dimly lit, and the walls are covered with astronomical charts and crimson drapery.
Aradrine is dressed not in her usual red robes, but in a simple, sleeveless tunic. The crimson growths and crystalline spines on her arms are in full view as she walks back to her chair.
\end{DndReadAloud}

Master Aradrine, hoping to gain the characters' trust by appearing vulnerable, provides the characters with the following information:

\begin{itemize}
\item A Consortium agent named \textbf{Galeokaerda} has infiltrated the archaeological site in Cael Morrow by posing as a member of the Allegiance of Allsight and befriending some Allegiance excavators.
\item \textbf{Galeokaerda} reports that a rift has been opened in Cael Morrow, but no one can enter it without the proper key. Diviners hired by the Consortium claim that an object known as the Key to the Netherdeep, a greataxe encrusted with ruidium, lies somewhere in Cael Morrow.
\item Aradrine fears that the Allegiance might be suspicious of \textbf{Galeokaerda} and doesn't want to risk losing the key.
\end{itemize}

Aradrine requests that the characters accomplish the following tasks and report back within three days:

\begin{itemize}
\item Enter Cael Morrow and avoid contact with Allegiance of Allsight personnel as much as possible.
\item Rendezvous with \textbf{Galeokaerda}, who says she will be waiting inside Cael Morrow's Royal Library (area M7a in chapter 5 (p. 5)).
\item Secure the Key to the Netherdeep. (Unknown to Aradrine at this time, any \textit{ruidium weapon} can be used as the key.)
\end{itemize}

Aradrine gives each character a stolen Allegiance of Allsight badge. These badges allow the characters to pose as Allegiance agents, enter Cael Morrow, and move around freely within the excavation site.

She promises the characters a reward of 1,000 gp for bringing back the Key to the Netherdeep.

\subsubsection{Key Locations}

This mission takes place primarily in Cael Morrow. Key locations the characters must visit are as follows:

\begin{description}
\item[Maw of Cael Morrow.] This location is the main entrance to the Drowned City (see the "Maw of Cael Morrow" section earlier in the chapter).
\end{description}

\paragraph{Royal Library Hall (Chapter 5, Area M7a)}

\textbf{Galeokaerda} is here, searching for the Key to the Netherdeep. If three days pass before the characters reach this area, \textbf{Galeokaerda} has been found and killed by the characters' rivals, who are in league with the Allegiance of Allsight.

\paragraph{Cliffside Villa (Chapter 5, Area M13)}

This site holds the Key to the Netherdeep: a \textit{ruidium greataxe}. If three days pass before the characters reach this location, the key has already been removed by the characters' rivals.

If the rivals obtain the Key of the Netherdeep, they bring it to the Allegiance of Allsight's base camp in Cael Morrow (area M3 in chapter 5 (p. 5)). The rivals are tasked with guarding the key but would sooner surrender than be killed for it.

\subsubsection{Success}

If the characters return to Aradrine with the Key to the Netherdeep, she takes it off their hands for further study and gives them their reward.

\subsection{Consortium Mission 5: Slay the Aboleth}

Seven days after the characters successfully complete mission 4, Aradrine invites them to her home to brief them on a new mission. First, she conveys the following information:

\begin{itemize}
\item Cael Morrow contains a rift to the Netherdeep, the source of all ruidium.
\item The rift is guarded by an aboleth that believes it is the Apotheon, a hero of legend.
\end{itemize}

Aradrine asks the characters to enter Cael Morrow and slay the aboleth, thus making it easier for the Consortium to salvage ruidium from the Netherdeep. Their reward for completing the mission is 1,000 gp.

\subsubsection{Key Locations}

This mission takes place primarily in Cael Morrow. Key locations in this mission include the following:

\begin{description}
\item[Maw of Cael Morrow.] This location is the main entrance to the Drowned City (see the "Maw of Cael Morrow" section earlier in the chapter).
\end{description}

\paragraph{Kelp Forest (Chapter 5, Area M11)}

The \textbf{Alyxian Aboleth} lairs here, though the characters might encounter it elsewhere.

\subsection{Consortium Mission 6: Ruidium Monopoly}

The final mission in this story track requires the characters to delve into the Netherdeep and ensure that the Consortium is the only faction in Ank'Harel that has access to ruidium.

Seven days after they successfully complete mission 5, Aradrine briefs the characters on their next assignment:

\begin{itemize}
\item To open Cael Morrow's rift to the Netherdeep, one needs a \textit{ruidium weapon} (see appendix B (p. 9)). The Consortium has determined that any such weapon functions as a Key to the Netherdeep.
\item The Allegiance of Allsight has sent agents into the rift. Their motives are unknown.
\end{itemize}

Aradrine asks the characters to accomplish the following tasks and report back before fourteen days have elapsed:

\begin{itemize}
\item Enter the Netherdeep. If the characters don't have at least one \textit{ruidium weapon} in their possession to serve as a Key to the Netherdeep, Aradrine gives them the \textit{ruidium greataxe} they retrieved in mission 4, assuming that mission was successful. (In its Exalted State, the \textit{Jewel of Three Prayers} also enables travel through the rift.)
\item Find the source of ruidium. If this source is a creature, make it an ally or capture it.
\end{itemize}

The reward for completion of this mission is a promotion to the rank of master within the Consortium of the Vermilion Dream, which comes with a salary of 1,000 gp per month.

\subsubsection{Failure}

Ultimately, this mission is doomed to fail. The characters can't give Aradrine what she wants. If the Apotheon is killed or redeemed, all ruidium ceases to exist; moreover, the characters can't hope to ally with the Apotheon until he is released, and releasing him when he is in his corrupted state makes an alliance with him impossible.

\subsection{Knowledge Is Power}

This section details the missions in the Cobalt Soul Story Track, which are summarized in the accompanying flowchart.

\subsubsection{Cobalt Soul Story Track}



\subsection{Cobalt Soul Mission 1: The Cultist of Zehir}

This introductory mission is used by the archivists of the Cobalt Soul to assess the characters' ability and potential value to the organization.

\subsubsection{Getting the Mission}

If the characters arrived in Ank'Harel with Question, the tiefling scholar they met in Bazzoxan in chapter 3, she greets them at their lodgings the morning after their arrival. Question knocks on the characters' door and cheerfully states that her superior, Iwo Zalarre (a chaotic good, half-orc \textbf{monastic operative}; see appendix A (p. 8)), wants to speak with them. If the characters accept the meeting, Question takes them to the Temple of the Mentor, the headquarters of the Cobalt Soul in Ank'Harel (see "Temple of the Mentor" earlier in the chapter).

If the characters aren't invited to join the Cobalt Soul by Question, they can apply for membership by speaking to Iwo Zalarre at the Temple of the Mentor.

Read or paraphrase the following when the characters meet with Iwo Zalarre:

\begin{DndReadAloud}
Iwo Zalarre stands at the edge of a balcony and points down at a pale wizard wearing academic robes. The wizard, who has numerous body piercings, seems to be minding his own business while reading a book. Next to him leans a staff topped with a carving of a snake's head.
Iwo whispers, "Our investigators report that this man is likely a cultist of Zehir the Cloaked Serpent, one of the Betrayer Gods. Your orders are to follow him back to his home and, without causing a scene, find out why he's here. If it's to cause trouble, apprehend him and turn him over to the Hands of Ord. Report back to me when you're done."
\end{DndReadAloud}

The person of interest is a chaotic evil half-elf named \textbf{Laurin Ophidas}. He uses the \textbf{cult fanatic} stat block, with these changes:

\begin{itemize}
\item He has immunity to poison damage and the \textit{poisoned} condition.
\item He wields a \textit{staff of the adder} and is attuned to it as though he were a cleric.
\end{itemize}

\subsubsection{Ophidas's Motives}

\textbf{Laurin Ophidas} is a disgraced member of the Cerberus Assembly, an organization of spellcasters in the Dwendalian Empire in Wildemount. He fled from that organization after stealing a \textit{staff of the adder} from a superior. Convinced that the assembly's assassins are pursuing him, Ophidas is trying to locate other Zehir cultists in the city who will protect him. His search is not going well.

\subsubsection{Following Ophidas}

Shortly after the characters receive their orders, Ophidas leaves the Temple of the Mentor in a foul mood and hails a carriage to take him back to his lodgings in Scroll's Alcove, a stately apartment building in the Sigil District (see area S5 on the Streets of the Sigil District map (p. 4)). The characters can follow him in a carriage of their own, at a cost of 2 sp per rider.

\subsubsection{Showdown}

Characters who confront Ophidas can force him to divulge the information they seek in one of the following ways:

\begin{itemize}
\item A character can use an action to try to bully Ophidas into revealing his motives, doing so with a successful DC 16 Charisma (Intimidation) check.
\item If Ophidas is reduced to half his hit points or fewer, he surrenders and answers any questions the characters put to him.
\end{itemize}

If the characters accost Ophidas in public, he hollers for the Hands of Ord. Five Hands of Ord \textbf{knights} arrive 3 rounds later. If the characters are still nearby when the Hands arrive, the guards try to arrest them and hold them overnight. If they don't resist, the characters are detained in a Sigil District jail, then released the next morning and assessed a fine of 50 gp each for disturbing the peace. While the characters are cooling their heels in jail, Ophidas (who is questioned by the Hands of Ord but not arrested) collects his belongings and disappears. The characters have no chance of finding him again.

\paragraph{Success}

If the characters report back to Iwo with the truth about Ophidas's motives, Iwo assures the characters that they'll be called upon again in a few days' time and gives them one stone from a pair of \textit{sending stones} for future communication.

\paragraph{Failure}

If the characters fail to learn Ophidas's motives before he disappears, Iwo says they must work their way up as librarians before he will give them another mission. Although this path might eventually lead to full membership, the characters know they can't wait that long to uncover the mysteries of the Apotheon, so they need to find another faction to join.

\subsection{Cobalt Soul Mission 2: Half-Baked Scheme}

The characters' second mission for the Cobalt Soul requires them to stake out Old Man Kruuk, a half-orc baker suspected of being a purveyor of stolen goods. Their quiet surveillance explodes into a furious chase across the canals as the characters try to keep powerful magic out of the wrong hands.

\subsubsection{Getting the Mission}

Seven days after the characters successfully finish their first mission, Iwo Zalarre sends one of them a message via \textit{sending stone}. Whichever character has the matching stone receives the message, which is as follows:

\begin{DndReadAloud}
"Still interested in joining the Cobalt Soul? Meet me at the Temple of the Mentor. I have another mission for you."
\end{DndReadAloud}

When the characters reconnect with Iwo at the Temple of the Mentor, read:

\begin{DndReadAloud}
Iwo shows you a map of the city, with a circle noting a location in the River District. "We suspect that a local baker named Kruuk is involved in the illegal sale of dangerous magic items, but we need proof before involving the Hands of Ord. The circle on the map shows where the bakery is located. We assume that Kruuk does all his black-market business after sundown, when the bakery closes. I suggest you wait until then to approach the place."
\end{DndReadAloud}

Iwo asks the characters to accomplish the following tasks:

\begin{itemize}
\item Obtain proof that Old Man Kruuk is selling magic items on the black market.
\item Confiscate any magic items found at Kruuk's bakery and bring them to Iwo.
\end{itemize}

Iwo pays the characters 250 gp up front, with the promise of an additional 250 gp if they obtain proof that Kruuk is selling magic items and another 250 gp for each magic item retrieved from his bakery.

\subsubsection{Kruuk Makes a Deal}

While outwardly appearing as a kindly, elderly baker, Old Man Kruuk runs an efficient operation as a fence for stolen goods (see "Old Man Kruuk's" earlier in the chapter). His exploits have earned him connections with not only petty thieves but also professional mercenaries and black-market spellcasters.

\subsubsection{River District Canal Map}

The River District Canal map shows the following locations:

\begin{description}
\item[Q1: Old Man Kruuk's.] This establishment is described in the "Bakery by the Canal" section below.
\item[Q2: Trees.] Several giant cedar trees grow in this district.
\item[Q3: Canal.] This 5-foot-deep, 20-foot-wide canal is one of many such watercourses in the River District. The canals have ropes or bridges over them to make pedestrian travel through the district easier.
\end{description}



\subsubsection{Bakery by the Canal}

When the characters arrive at Old Man Kruuk's bakery shortly after sunset, read or paraphrase the following:

\begin{DndReadAloud}
The soft sound of water lapping against canal banks is audible as you approach the building. Scents of honey and rose mingle with the smell of freshly cut grass. "Old Man Kruuk's" is painted in large letters above the door, and a sign hanging from the door's handle reads, "Sorry, we're closed! See you in the morning!"
\end{DndReadAloud}

Inspecting the building's front and making a successful DC 16 Wisdom (Perception) check enables a character to notice that the sign hanging from the door's handle is surrounded by a field of colored dots. The dots seem erratically placed to the untrained eye, but a character who knows thieves' cant recognizes the pattern as part of a code, identifying this place as a repository of stolen goods. The sign alone, however, is not enough to prove that Kruuk is trafficking in illicit magic items.

\paragraph{Q1a: Bakery Shop}

The front door is unlocked, and a small silver bell on the door jingles as it's opened. Unsold items are still sitting in their display cases. Old Man Kruuk, a neutral, half-orc \textbf{bandit captain}, sits behind the counter, counting the day's proceeds.

The entry bell spoils any chance of the characters entering undetected. While some characters keep Kruuk occupied, another character can sneak away by making a successful DC 13 Dexterity (Stealth) check.

If the characters try to engage Kruuk in conversation, he replies that the bakery is closed but offers to sell them some leftover tarts or rolls before they go.

A character who speaks to Kruuk in thieves' cant or succeeds on a DC 15 Charisma (Persuasion) check earns his trust, whereupon he divulges that a "magic shortsword" has recently come into his possession and that he plans to sell it later tonight (see "A Deal Goes Down" below). If the characters express interest in buying the weapon, Kruuk asks them to wait until his buyer arrives, after which he'll sell the blade to the highest bidder. If a character fails the check, Kruuk says nothing about the magic shortsword or his illegal business.

\paragraph{Q1b: Bakery}

Kruuk's bakery is filthy, with waste bins filled with moldy dough and stale flatbread. Behind a table on the north wall is a secret door that leads to his back room (area Q1e). A character who examines the wall and succeeds on a DC 10 Intelligence (Investigation) check finds the hidden door, which can be opened by pushing on it.

\paragraph{Q1c: Bedroom}

Kruuk's bedroom contains a writing desk and chair, a mattress on the floor, shelves covered with personal belongings, and a pair of 25-pound dumbbells. On the north wall hangs a full-body painting of a scantily clad tiefling. The painting conceals a door to his back room (area Q1e).

\paragraph{Q1d: Courtyard}

In the center of Kruuk's property is a grass-covered courtyard. It has wooden doors leading to all four rooms surrounding it. These doors are locked, but any of them can be opened by using a key Kruuk keeps on his person. A lock can be picked by a character who makes a successful DC 16 Dexterity check using \textit{thieves' tools}. A door can be broken down by a character who makes a successful DC 11 Strength (Athletics) check.

\paragraph{Q1e: Back Room}

The exterior door on the west wall is locked. The lock can be picked by a character who makes a successful DC 15 Dexterity check using \textit{thieves' tools}. The door can be broken down by a character who succeeds on a DC 11 Strength (Athletics) check, but the noise alerts anyone in the building.

This room smells of decay, and iron manacles are bolted to the wall. There are stains of various sorts on the floor. The room contains jugs of water, loose pieces of jewelry, and crates that hold low-value stolen goods.

A character who makes a successful DC 15 Intelligence (Investigation) check locates a metal lockbox stowed underneath an upside-down crate in a corner of the room. The lockbox can be opened by a character who makes a successful DC 21 Dexterity check using \textit{thieves' tools}. It can also be pried open with a successful DC 16 Strength (Athletics) check. If the box is forced open, poison gas sprays outward, and the character who opened it must succeed on a DC 15 Constitution saving throw or take 22 (4d10) poison damage and be \textit{poisoned} for 1 hour.

The lockbox holds the \textit{ruidium shortsword} that Kruuk plans to sell (see "A Deal Goes Down" below). If the characters have gained Kruuk's trust, he opens the lockbox to show them the sword, then explains its magical properties (see appendix B (p. 9)).

\subsubsection{A Deal Goes Down}

Kruuk plans to meet one hour after sunset with \textbf{Ashann}, a chaotic neutral, gnome \textbf{occult extollant} (see appendix A (p. 8)) working for the Consortium of the Vermilion Dream. \textbf{Ashann} carries a pouch containing five peridots (500 gp each).

% Image placeholder: {@creature Ashann|CRCotN}

At the appointed time, \textbf{Ashann} arrives at Kruuk's bakery. If the characters haven't neutralized him, Kruuk leads \textbf{Ashann} to the back room (area Q1e) so she can examine the \textit{ruidium shortsword}. If he has more than one potential buyer, Kruuk holds an impromptu auction in the back room and starts the bidding at 1,000 gp. \textbf{Ashann} can't afford to pay more than 2,500 gp for the shortsword, and she attacks anyone who bids over that amount. \textbf{Ashann} surrenders if she is reduced to 10 hit points or fewer.

If \textbf{Ashann} has been taken out of the equation, a character can convince Kruuk to sell the shortsword for 500 gp by succeeding on a DC 14 Charisma (Intimidation or Persuasion) check, either by threatening his life or by convincing him that the characters are interested in being repeat customers.

\paragraph{Urban Chase}

The characters can wait to snatch the \textit{ruidium shortsword} from \textbf{Ashann} after her deal with Kruuk is done, in which case \textbf{Ashann} tries to flee with the sword rather than fight, resulting in a chase (use the chase rules, p. 8 and the Urban Chase Complications table in the Dungeon Master's Guide). If \textbf{Ashann} is caught, she fights to keep the \textit{ruidium shortsword} but surrenders it if she is reduced to 10 hit points or fewer.

\subsubsection{Returning to the Temple}

Iwo is happy to receive the \textit{ruidium shortsword} and hands over the agreed-upon reward in exchange for it. Iwo is careful not to touch the weapon bare-handed. (After examining the sword carefully for a few days, the monks of the Cobalt Soul destroy it to ensure that it doesn't corrupt anyone.) If the characters seem reluctant to part with the sword, Iwo warns them that the weapon might be dangerous to its wielder. If necessary, Iwo increases the reward to 1,000 gp, hoping the extra coin will convince the characters to part with the weapon.

If the characters have Kruuk, \textbf{Ashann}, or both in their custody, Iwo has the criminals detained until the Hands of Ord can be summoned to take them away.

\subsubsection{Joining the Cobalt Soul}

Seven days after the characters successfully complete this mission, the archivists of the Cobalt Soul deem them worthy and motivated enough to join their organization as full members. The characters receive a brief, congratulatory message from Iwo via \textit{sending stone}. Membership comes with the following benefits:

\begin{itemize}
\item Each new member receives a monastic robe like those worn by other members of the Cobalt Soul. If the robe is lost or damaged, a replacement can be purchased at the Temple of the Mentor for 1 gp.
\item Each new member receives free lodging in Ank'Harel for as long as their membership lasts. The lodging is of comfortable quality (see "Lodging" earlier in the chapter).
\item New members can buy the following magic items at the Temple of the Mentor: a \textit{spell scroll} of \textit{detect magic} (10 gp), a \textit{spell scroll} of \textit{divination} (125 gp), a \textit{spell scroll} of \textit{commune} (250 gp), and a pair of \textit{sending stones} (250 gp).
\end{itemize}

The characters can't receive any more missions from the Cobalt Soul unless they become members of the faction.

\subsection{Cobalt Soul Mission 3: Elephant Uproar}

After the Cobalt Soul receives reports of strange happenings in the Suncut Bazaar, High Curator \textbf{Jamil A'alithiya} surmises that the Consortium of the Vermilion Dream must be making progress researching the crimson substance known as ruidium. Concerned about the potentially destructive repercussions if the Consortium learns how to use it, Jamil sends the characters to infiltrate the Consortium's headquarters.

\subsubsection{Getting the Mission}

The next time the characters visit the Temple of the Mentor, Iwo informs them that the high curator wants to speak with them. The high curator's office is on the third floor. When the characters enter it, read:

\begin{DndReadAloud}
Seated at the wide table is a young man no older than thirty, with brown skin and hazel eyes. His white robes are adorned with blue designs.
He smiles warmly as you enter. "Please, shut the door behind you," he says, closing the book in front of him. "I am High Curator \textbf{Jamil A'alithiya}, and I've heard much about you.
"We know that the Consortium of the Vermilion Dream is experimenting with a red substance its leaders call ruidium. Rumor has it the substance comes from the ruins of Cael Morrow, the underwater city beneath Ank'Harel.
"I'm worried that the Consortium is tampering with powers beyond its control. That is why I would like you to infiltrate the Consortium's headquarters---a tavern called First Eclipse---and find out anything you can about its research on ruidium. If you happen to come across any magic items in the Consortium's possession, please confiscate them and bring them to our temple for examination and safekeeping."
\end{DndReadAloud}

\textbf{Jamil A'alithiya}, a neutral good, human \textbf{monastic high curator} (see appendix A (p. 8)), offers a reward of 1,000 gp for completing the mission.

Although he suspects that the Consortium of the Vermilion Dream has begun crafting ruidium weapons and armor, Jamil keeps this information to himself for the time being. He is hoping the characters will either confirm or dispel his suspicions.

\subsubsection{A Bazaar Encounter}

When the characters arrive at the Suncut Bazaar, read:

\begin{DndReadAloud}
The sound of wood splintering and cries of panic echo through the streets, then are suddenly drowned out by a trumpeting roar that heralds the arrival of two red, crystalline elephants. One smashes through a stall as the other strides a few paces behind its companion.
\end{DndReadAloud}

This combat encounter takes place on the Streets of the Suncut Bazaar map (p. 4). The crystalline creatures are two \textbf{ruidium elephants} (see their stat block earlier in the adventure). These elephants are running amok and must be destroyed before they cause any further damage.

The first elephant that drops to 0 hit points turns into a tiny marble elephant \textit{figurine of wondrous power} with thin veins of ruidium running through it.

The second elephant crumbles into ruidium dust when reduced to 0 hit points. A character who makes a successful DC 17 Intelligence (Arcana) check determines that this elephant was created using magic, with a small amount of ruidium as a material component; in other words, it was never a magic item to begin with.

Once the elephants have been dispatched, the characters can follow the elephants' trail of destruction back to First Eclipse. The characters are familiar with this tavern if they've had dealings with the Consortium of the Vermilion Dream.

\subsubsection{Keeping the Figurine}

The characters can keep the elephant figurine, but they would be wise not to use it, for the elephant it creates uses the \textbf{ruidium elephant} stat block (see earlier in the chapter) and can't be controlled.

If the characters slay or redeem the Apotheon (see chapter 7), the figurine is rid of its ruidium corruption and functions like a normal \textit{figurine of wondrous power (marble elephant)}.

\subsubsection{First Eclipse}

The elephants were released by Khelkur the Gull, a neutral evil, dwarf \textbf{occult silvertongue} who's one of the masters of the Consortium of the Vermilion Dream. He and the bartender, Satzrak Runestrider, a neutral evil, blue dragonborn \textbf{occult extollant} (see appendix A (p. 8) for both stat blocks), are in First Eclipse's taproom, celebrating the success of their experiment. The tavern is otherwise empty.

When the characters enter, Satzrak barks, "We're closed!" and expects the characters to leave at once. If they don't, Khelkur says, "Leave now, or meet your doom!"

Unless the characters leave immediately, Khelkur and Satzrak attack.

After the battle, any character who searches Khelkur finds a crumpled note stuffed in one of his pockets. This note, written by one of his colleagues, bears the following message in Common:

\begin{DndReadAloud}
Master Khelkur,
Our attempts to gain access to the rift have been disastrous. An aboleth haunts the Drowned City. It slaughtered all but one member of our last expedition. The creature calls itself Alyxian---certainly not an aboleth name, but not one I am familiar with either. But know this: Cael Morrow is where we must focus our efforts, lest the Allegiance of Allsight claim the only known source of ruidium for itself.
Our key to the rift is lost, but we have a spy in the Drowned City searching for it. Meanwhile, we have decided to secure a new headquarters. Please see that First Eclipse is cleared out---and quickly---before the Hands of Ord search the place.
Humbly, Master Dendarron
\end{DndReadAloud}

Characters who search First Eclipse find no magic items or other treasure. Consortium agents under Khelkur's command have already emptied the tavern of valuables and moved the faction's headquarters to a new location.

\subsubsection{Returning to the Temple}

Jamil welcomes the characters back to the Temple of the Mentor at the end of the mission. If they turn over Dendarron's note, the elephant figurine, or both, he gives them the promised reward of 1,000 gp.

The handwritten note from Master Dendarron (whom Jamil recognizes as one of the leaders of the Consortium of the Vermilion Dream) makes it clear that the Consortium is keen to secure more ruidium, which is accessible only through a rift in Cael Morrow. Jamil informs the characters that he will contact allies in the Allegiance of Allsight and arrange safe passage for the characters into the Drowned City. Until they hear from him, the characters can do as they please.

\subsection{Cobalt Soul Mission 4: Allegiance Alliance}

Seven days after they complete the previous mission, the characters are invited to the Temple of the Mentor to meet with High Curator \textbf{Jamil A'alithiya}, who plans to send them on a mission in the Drowned City.

\subsubsection{Getting the Mission}

Read or paraphrase the following when the characters arrive at Jamil's study:

\begin{DndReadAloud}
As you enter his study, the high curator spreads his arms wide and softly says, "Welcome. I am glad to see you again.
"The Allegiance of Allsight has granted you access to Cael Morrow. The Allegiance's control of the excavation site is threatened by the Consortium of the Vermilion Dream, so it's important that we act quickly.
"Ruidium is a magical substance that corrupts everything it touches. It comes from a place called the Netherdeep, which can be reached by passing through a rift in the heart of Cael Morrow. I've been told that passage through the rift requires some sort of key, but I don't know what form the key takes.
"I fear further unrest in Ank'Harel so long as the rift exists. The Netherdeep must be sealed off or destroyed, lest ruidium continue to find its way into Ank'Harel. But first thing's first. We need that key."
\end{DndReadAloud}

Jamil believes that the Allegiance of Allsight and the Consortium of the Vermilion Dream are acting selfishly by taking advantage of their access to the Netherdeep. He asks the characters to accomplish the following tasks and report back before seven days have elapsed:

\begin{itemize}
\item Enter Cael Morrow and rendezvous with Professor \textbf{Insight Acuere} at the Allegiance of Allsight base camp.
\item Earn \textbf{Insight}'s trust and learn how to access the Netherdeep.
\item Acquire the "key" to the Netherdeep.
\end{itemize}

Jamil promises the characters a reward of 1,000 gp if they find a key to the Netherdeep.

At the end of the meeting, Jamil gives each character an Allegiance of Allsight badge on loan from that organization. These badges will allow them to enter Cael Morrow and move around freely within the excavation site.

\subsubsection{Insight's Quest}

The characters can meet \textbf{Insight} in the Allegiance base camp (area M3 in chapter 5 (p. 5)). She tells them that her companions are searching for the writings of an ancient hero who tried to save Cael Morrow from destruction during the Calamity. Earlier expeditions found dozens of stone tablets from the period in question at the south end of the excavation site. Her team, however, no longer has the force of arms to fight the dangerous creatures that lurk in this region. If the characters can find more of these tablets and return them to her, she'll be grateful.

\paragraph{Insight's Reward}

If the characters bring back the tablet from area M10 in chapter 5 (p. 5), \textbf{Insight} shares the following information:

\begin{itemize}
\item \textbf{Insight} reveals the location of the rift leading to the Netherdeep (area M17 in chapter 5 (p. 5)) and informs the characters that weapons infused with ruidium can open this rift.
\item \textbf{Insight} believes there might be keys to the rift in Cael Morrow---left behind by agents of the Consortium of the Vermilion Dream.
\end{itemize}

\subsubsection{Key Locations}

This mission takes place primarily in Cael Morrow. Key locations the characters must visit are as follows:

\begin{description}
\item[Maw of Cael Morrow.] This location is the main entrance to the Drowned City (see the "Maw of Cael Morrow" section earlier in the chapter).
\end{description}

\paragraph{Allegiance Base Camp (Chapter 5, Area M3)}

This location is the heart of Allegiance operations in Cael Morrow.

\paragraph{Unstable Building (Chapter 5, Area M10)}

This location contains the tablet that \textbf{Insight} seeks.

\paragraph{Cliffside Villa (Chapter 5, Area M13)}

The "key" to enter the Netherdeep can be any \textit{ruidium weapon}. Such an item can be found in this sunken villa.

\subsubsection{Returning to the Temple}

If the characters complete the mission and return to Jamil, he gives them the 1,000 gp reward and requests that they await another summons in seven days' time.

\subsection{Cobalt Soul Mission 5: Enemy of Our Enemies}

Seven days after the characters complete the previous mission, Jamil requests another meeting with them, during which he imparts the following information:

\begin{itemize}
\item Tensions between the Allegiance of Allsight and the Consortium of the Vermilion Dream have grown in the past seven days.
\item The rift to the Netherdeep is guarded by an aboleth that believes it is the Apotheon of legend, defending its realm against invaders.
\item The Apotheon is said to have died in Cael Morrow at the end of the Calamity. Perhaps the aboleth is, indeed, some bizarre manifestation of this fallen hero.
\end{itemize}

Jamil needs the characters to accomplish one of the following tasks and report back within three days:

\begin{itemize}
\item If the aboleth is truly Alyxian, it might listen to reason. The characters must inform the aboleth that they are not enemies of the Apotheon and convince it to allow them to enter the rift.
\item If the first task can't be accomplished, the characters must slay the aboleth so they have easier access to the Netherdeep when the time comes.
\end{itemize}

Jamil's reward for this task is 1,000 gp. He promises he will continue researching the Netherdeep in hopes that he can help them make their eventual journey into it.

\subsubsection{Key Locations}

This mission takes place primarily in Cael Morrow. Key locations the characters must visit are as follows:

\begin{description}
\item[Maw of Cael Morrow.] This location is the main entrance to the Drowned City (see the "Maw of Cael Morrow" section earlier in the chapter).
\end{description}

\paragraph{Temple of the Arch Heart (Chapter 5, Area M9)}

This location is Alyxian's final prayer site. Accessing it is necessary to unlock the full power of the \textit{Jewel of Three Prayers}.

\paragraph{Kelp Forest (Chapter 5, Area M11)}

The Alyxian Aboleth lairs here, though the characters might encounter the creature elsewhere.

\paragraph{Rift to the Netherdeep (Chapter 5, Area M17)}

This planar rift connects the Netherdeep with the Material Plane.

\subsubsection{Negotiating with the Aboleth}

The \textbf{Alyxian Aboleth} attacks the characters on sight. As an action, a character claiming to be Alyxian's friend or ally can try to persuade the aboleth to leave the party alone, doing so with a successful DC 25 Charisma (Persuasion) check, or DC 15 if that character presents the \textit{Jewel of Three Prayers}.

If the check fails, the aboleth rails against the characters, bellowing, "I am Alyxian! I am the Apotheon! I am divine, I am invincible! I am beyond mortal ken!" The delusional monster attacks its foes ruthlessly.

If the check succeeds, the aboleth gurgles contentedly and says, "Then, my friends, I will join your fight and defeat our shared enemies, just as I did so long ago." The aboleth no longer attacks the characters and, if they request it, harries their rivals and any other adversaries they identify.

\subsection{Cobalt Soul Mission 6: Destroy the Netherdeep}

The final mission in the Cobalt Soul Story Track requires the characters to enter the Netherdeep and try to destroy it from the inside.

Seven days after they return from the previous mission, the characters receive their next mission from Jamil, who imparts the following information:

\begin{itemize}
\item Both the Consortium of the Vermilion Dream and the Allegiance of Allsight are trying to stockpile ruidium. If the flow of ruidium out of the Netherdeep isn't cut off immediately, the consequences could be disastrous.
\item The Consortium has sent a group of skilled adventurers into the Netherdeep. (This group is the rival party.)
\item To survive the underwater pressure of the Netherdeep, one needs an item at least partially made of ruidium. (The \textit{Jewel of Three Prayers} also allows its wearer to withstand the pressure.)
\end{itemize}

Jamil requests that the characters find a way to destroy the Netherdeep and report back to him before fourteen days have elapsed. Jamil says he can think of no reward appropriate for accomplishing a task of this magnitude, but he offers the characters 2,000 gp each and hopes that will suffice.

\subsubsection{Success}

This mission can succeed only if the characters slay or redeem Alyxian, thus destroying the Netherdeep and all ruidium in the process (see chapter 7 for details).

\chapter{The Drowned City}\label{ch:the-drowned-city-6-6}

Cael Morrow lies in a flooded cavern deep beneath the foundations of Ank'Harel. Only ruins of the place remain, but even so there is no doubt that Cael Morrow was one of the most magnificent cities in the world before the Calamity. (The city's original name is unknown; "Cael Morrow" is the name given to the place by the archaeologists of the Allegiance of Allsight.) In those days, it was home to elves of all kinds---including those who were transformed into Exandria's first orcs by the blood of Gruumsh the Ruiner after Corellon pierced his eye on the field of battle. The city was a bastion of hope, its magical wards blessed by the Arch Heart.
Infuriated by his loss to Corellon, Gruumsh swore to annihilate Cael Morrow. The Betrayer stood more than a hundred feet tall, looming over a temple to Corellon in the center of the city. There, he raised his terrible spear, intending to strike the earth with such force that all of Marquet would be reduced to ash---only to have his blow parried by Alyxian the Apotheon at the last second.
Cael Morrow, the Drowned City, is the grave of thousands of Gruumsh's victims. Even Alyxian's sacrifice, which protected much of the land of Marquet at the cost of his freedom, could not save the city. Now, the center of Cael Morrow is a planar rift born from the place where Gruumsh's spear struck the earth. This rift is the gateway to the Netherdeep, the fountainhead of the corrupting element ruidium, and an object of fascination to Ank'Harel's factions. The rift will draw the characters and their rivals into conflict once more---with the chance to become heroes.
% Image placeholder: Two goblin children run through the streets of Jigow with a delicious meat pie
\section{Running This Chapter}

This chapter contains a description of the ruins of Cael Morrow, which the characters explore over the course of the various faction missions detailed in the previous chapter. These missions, which differ depending on which faction the characters have allied themselves with, form the story of this chapter. The missions aren't necessary for advancing the adventure plot; if the characters want to save Alyxian and enter the Netherdeep, they can do so without the prompting of their faction. But if they need a push to enter the Netherdeep, the missions in chapter 4 (p. 4) can help motivate them.

\subsection{Rival Developments}

Like the characters, the rivals entered Cael Morrow on behalf of their faction. But the treacherous ruins and horrific creatures have taken their toll on the group.

When the characters encounter their rivals in Cael Morrow, it might be the first time they've met since Bazzoxan. In the meantime, the rivals have been working with an opposing faction, exploring Cael Morrow and the Netherdeep in their own way---and facing the eerie beings within.

By the time the characters meet them again in Cael Morrow, the rivals have changed dramatically. They now use their tier 3 stat blocks (see appendix A (p. 8)). The changes the rivals have undergone are as follows:

\textbf{Ayo Jabe}. Ayo's drive to prove herself a hero, and to prove herself superior to the characters, has brought her to this place. She is aware of the trauma that her quest has put her friends through. This feeling has led her to doubt her leadership abilities. Ayo is filled with cold fury toward anything that stands in her way, including her own insecurities.

\textbf{Dermot Wurder}. Dermot's desire to protect his friends is still at the core of his being. But dozens of battles and near-death experiences have turned his uncertainty into steely confidence.

\textbf{Galsariad Ardyth}. Galsariad's haughtiness has drained from his heart---as have all his other emotions. He has witnessed unspeakable horrors in the name of gaining greater magical power and has protected himself from being overcome by fear and self-loathing by suppressing all his feelings.

\textbf{Irvan Wastewalker}. Irvan's cavalier demeanor disappeared when he lost his arm in a battle with a slithering bloodfin in Cael Morrow. His faction allies fitted him with a stylish and functional arcane prosthetic, but this brush with death has made him short-tempered, cautious, and mistrustful. Irvan fears that any further mistake could cost him his life---permanently.

\textbf{Maggie Keeneyes}. A deep melancholy has taken root in Maggie's heart. She continues the mission out of love for Ayo, but she fears that her leader's determination might not be enough to achieve victory. Maggie doesn't trust the Allegiance of Allsight agents in Cael Morrow (even if the rivals have allied with that faction), believing them to be influenced by a malevolent force. She has shared her misgivings about this organization with Ayo on several occasions.

\subsubsection{Encountering the Rivals}

The rivals have become affiliated with a faction of your choice, most likely the Consortium of the Vermilion Dream. Their reason for joining their faction and coming into conflict with the characters depends on how the characters have interacted with them so far:

\begin{description}
\item[Friendly Rivals.] Rivals who are friendly toward the characters are happy for the groups to work together.
\item[Indifferent Rivals.] If the rivals are indifferent to the characters, then this competition is nothing personal---it's just business, and they happen to be on different sides. If they are working for the same faction, the rivals are willing to cooperate, but they make it clear they intend to come out on top and be considered the real heroes.
\item[Hostile Rivals.] Rivals who are hostile toward the characters hold a grudge against them and aim to prove their superiority.
\end{description}

A good time for the characters to encounter the rivals is when the characters try to leave Cael Morrow with something valuable in their possession, such as the Key to the Netherdeep (see the faction story tracks in chapter 4 (p. 4)) or the \textit{Jewel of Three Prayers} in its \textit{Exalted State} (see area M9).

If the characters get the better of the rivals in this chapter, you can use the rivals' defeat to change the pace of the exploration of the Netherdeep in the next chapter. The rivals could take a few days to lick their wounds and prepare for the next encounter with the characters. Or the rivals' defeat could reenergize them, prompting them to try to get a head start on exploring the Netherdeep.

\begin{DndSidebar}{Allegiance of Allsight Badges}
Cael Morrow is currently under the jurisdiction of the Allegiance of Allsight, whose members wear badges inside their robes. (For more information on these badges and how the characters can obtain them, see chapter 4 (p. 4).) Interlopers who don't show their official badge when asked are attacked on sight. Allegiance agents try to apprehend intruders by knocking them \textit{unconscious} and then take them to Professor \textbf{Insight Acuere} (see area M3) for questioning.
\end{DndSidebar}
\section{Cael Morrow Overview}

Rare are the things that remain beautiful after their deaths---Cael Morrow is one of them. Through the efforts of the Allegiance of Allsight, portions of the sunken city have been reclaimed and magically warded against the pervasive deluge. In the last few months, the faction has done its best to reinforce the city's eroding architecture, securing the place for the sake of the faction's archaeological pursuits.

Despite these efforts, though, Cael Morrow remains a largely inhospitable place. Otherworldly aquatic creatures thrive in the ruins, some adapted to and warped by the impenetrable darkness and others aberrant growths spawned in the Netherdeep. To make matters worse, the spirits of the dead and the damned from the Calamity long ago still haunt the wreckage of the city.

\subsection{Features of Cael Morrow}

The areas of the Drowned City, depicted on the map of Cael Morrow, have the features described below.

\subsubsection{Architecture}

Unless otherwise noted, the areas in Cael Morrow have 30-foot-high ceilings, and the ceilings in the connecting hallways are 20 feet high. Doorways are 8 feet high and 3 feet wide.

\subsubsection{Flooded Cavern}

The cavern that contains the sunken city is 500 feet below ground and filled with fresh water that flows out of the Netherdeep rift (area M17). The water bubbles up to Ank'Harel through a narrow, natural channel in the roof of the cavern (not shown on the map).

The characters first enter the flooded cavern by descending the hollow interior of a natural stone column (area M1) that stretches from the top of the cavern to the bottom.

\subsubsection{Light}

Unless otherwise noted, all keyed locations on the map are lit by \textit{continual flame} spells. The only areas not illuminated in this way are places where the Allegiance of Allsight hasn't yet erected protective wards (described below).

\subsubsection{Magical Barriers and Keystones}

The Allegiance of Allsight has constructed 1-inch-thick magical barriers of shimmering, light blue force around certain areas of Cael Morrow, as indicated on the map. These wards have forced the water out of those areas, leaving behind dry, air-filled chambers and hallways. The rest of the Drowned City is underwater.

The barriers are soundproof, and their edges overlap with walls, doors, ceilings, and floors without harming those features. In places where a barrier overlaps with a stone wall, door, ceiling, or floor, it is \textit{invisible} and untouchable.

Any ward that encloses an area also creates a barrier across the doorways leading to and from that area. These barriers function as airlocks.

A barrier is immune to all damage, is impermeable to liquids, and can't be dispelled. A creature can pass through a barrier, in either direction, by spending 5 feet of movement, but no creature can be pushed or pulled through by another one. As a creature passes through a barrier, it experiences a harmless tingling sensation.

Each warded area has a keystone located somewhere in it. For example, a keystone might be inserted into the top of a door frame or set into the floor. A keystone is a cube of solid gray stone measuring 6 inches on a side, each side inscribed with arcane runes that emit the same blue light as the barriers. A keystone has AC 17, 50 hit points, and immunity to poison and psychic damage. Destroying an area's keystone, removing a keystone from the area it affects, or targeting a keystone with a \textit{dispel magic} spell causes the barrier around that area to dissipate. If that happens, each creature in the area must make a DC 20 Dexterity saving throw, taking 21 (6d6) bludgeoning damage on a failed saving throw, or half as much damage on a successful one, as the area fills with water.

\subsection{The Alyxian Aboleth}

Lurking in the waters of Cael Morrow is an aboleth that believes it is the physical embodiment of the Apotheon. It uses the \textbf{Alyxian Aboleth} stat block in appendix A (p. 8).

\subsubsection{Where Is the Aboleth?}

The aboleth's lair is in Cael Morrow's kelp forest (area M11). The creature routinely swims about in the ruins, striking fear into those who are delving into the remains of the city. If the Alyxian Aboleth is reduced to half its hit points or fewer, it retreats to its lair and tries to take an uninterrupted short rest, regaining 9d10 + 36 hit points at the end of it.

To determine the aboleth's whereabouts while the characters explore Cael Morrow, roll a d20 and consult the Aboleth's Location table, or choose one of the suggested locations. The aboleth has free access to all the submerged map locations. It's recommended that you determine a new location for the aboleth whenever the characters move from one area to another, though the creature can move less frequently if you see fit.

% Table: Aboleth's Location
\begin{DndTable}[header={Aboleth's Location}]{cX}
d20 & Location \\
1--10 & Brooding in its lair in area M11 \\
11--15 & Searching area M8 for archaeologists to devour \\
16--20 & Patrolling the rift in area M17 \\
\end{DndTable}

\subsubsection{Poisoned Emotions}

The aboleth's psychic powers have been amplified by the ruidium that pervades its flesh.

The first time the characters enter each of several keyed locations (areas M2, M4, M5a, M7c, M8, M10, and M15), the aboleth transmits a telepathic message to one randomly determined character, who hears the aboleth's voice---a distorted version of the true voice of Alyxian that they've heard in their visions---echo in their mind. Roll on the Poisoned Emotions table to determine the aboleth's message. After receiving the message, the character must succeed on a DC 10 Wisdom saving throw or suffer the message's corresponding effect, as noted in the table. A character can repeat the saving throw after finishing a short or long rest, ending the effect on themself on a success. The effect can also be removed from a character by any magic that ends a curse.

The aboleth loses this ability when the consciousness of the Apotheon manifests in the Temple of Corellon (see "Second Vision" in area M9), and any creatures currently affected by it feel its effects end immediately.

% Table: Poisoned Emotions
\begin{DndTable}[header={Poisoned Emotions}]{cll}
d10 & Message & Effect \\
1--2 & "Ungrateful wretches---they have forgotten all I have done!" & The character has disadvantage on Intelligence checks and Intelligence saving throws. \\
3--4 & "My power was not enough. It never will be." & The character has disadvantage on attack rolls. \\
5--6 & "Why have the gods forsaken me?" & The character has disadvantage on Wisdom checks and Wisdom saving throws. \\
7--8 & "They were right. I bring nothing but despair." & The character has disadvantage on Charisma checks and Charisma saving throws. \\
9--10 & "If I can't live, I will burn this world with me." & Whenever the character hits a creature with a weapon attack, the target takes an extra 1d6 psychic damage, and the character also takes 1d6 psychic damage. \\
\end{DndTable}

\subsection{Uncharted Waters}

Many areas of Cael Morrow have yet to be secured by the Allegiance of Allsight. Exploring areas unprotected by magical wards is a dangerous proposition, with hungry sea creatures liable to be lurking around any corner. When the characters enter an area of Cael Morrow not protected by a ward, roll on the Cael Morrow Encounters table to determine what they face. Don't roll on this table if hostile creatures are already in the area.

Cael Morrow includes many other areas beyond what this book describes. You can use the encounters from this table to populate other parts of the sunken city for the party to explore.

% Table: Cael Morrow Encounters
\begin{DndTable}[header={Cael Morrow Encounters}]{cX}
d12 & Encounter \\
1 & A \textbf{swarm of sorrowfish} (see appendix A (p. 8)) flocks around the detritus of a shark's corpse, ravenous for more prey. \\
2 & Three \textbf{scuttling serpentmaws} (see appendix A (p. 8)) search for food. \\
3 & A \textbf{giant octopus} wrestles with a locked chest. The chest contains 1,000 ep in coins minted long ago in Cael Morrow. \\
4 & Tentacles burst from the darkness as a \textbf{death embrace} (see appendix A (p. 8)) notices the characters. \\
5 & Two \textbf{slithering bloodfins} (see appendix A (p. 8)) streak through the water, eager for a meal. \\
6 & Two \textbf{aboleth spawn} (see their stat block later in the chapter) spot the characters and flee to alert the Alyxian Aboleth, which appears 1 minute later. \\
7--12 & No encounter occurs. \\
\end{DndTable}

\subsection{Cael Morrow Locations (M1-M8)}

The locations described below are keyed to the Cael Morrow map. This map doesn't show the sunken city in its entirety, just the areas important to the adventure. You can expand Cael Morrow by adding locations and encounters of your own design.



\subsubsection{M1: Stone Column}

A 10-foot-wide spiral staircase descends along the interior of a hollow stone column that extends down from the Maw of Cael Morrow in Ank'Harel's Sigil District (described in chapter 4).

Read or paraphrase the following when the characters reach the bottom of the column for the first time:

\begin{DndReadAloud}
You descend a spiral staircase for hundreds of feet. Windows in the nearby wall offer glimpses of a sunken city below Ank'Harel. Magical lights illuminate several of its ruins, but most of the city is hidden from view.
At the bottom of the staircase is a plain, windowless room containing boxes of archaeological tools and crates marked with the emblem of the Allegiance of Allsight. Not far away is a passage that leads to a stone door also marked with the Allegiance's symbol. Leaning against the wall next to the door is an orc wearing the badge of an Allegiance of Allsight member. He doesn't seem to have noticed you.
\end{DndReadAloud}

The person near the door is a sullen, lawful neutral orc named Gortag Inkdrinker (use the \textbf{scholarly agent} stat block in appendix A (p. 8)). The characters can surprise him if they act quickly. If the characters are members of the Allegiance of Allsight or the Library of the Cobalt Soul, Gortag treats them as allies and invites them to meet with his superior, Professor \textbf{Insight Acuere}, in the Allegiance base camp (area M3). If the characters accept the invitation, he escorts them there.

If one or more characters aren't wearing Allegiance badges or claim to be in league with the Consortium of the Vermilion Dream, Gortag tells them to halt and return up the stairs and does his best to hinder their progress without provoking conflict. If combat breaks out, he opens the door he's guarding, pushes himself through the magical barrier that extends across the doorway (see "Features of Cael Morrow" earlier in the chapter), and alerts the guards in area M2. If he's successful, he runs to area M3 to warn \textbf{Insight} that enemies have entered the Drowned City. If Gortag is unable to flee, he throws himself on the characters' mercy and can be persuaded to lead the party past the guards in area M2, if only to avoid harm.

\paragraph{Crates}

The crates contain recovered relics and specimens in glass jars, all packed for transport.

\subsubsection{M2: Warded Entrance Hall}

\begin{DndReadAloud}
A corridor framed by shimmering blue light extends ahead of you. This airtight tunnel ends before a door set into the wall of a submerged building. Above the doorway is a small stone cube inscribed with glowing blue symbols. Standing in front of the door are two guards, a human and a halfling, wearing Allegiance of Allsight apparel and badges. Verdant patches of kelp grow just outside the tunnel's shimmering walls, and colorful fish flit between the swaying strands.
\end{DndReadAloud}

The first time the characters enter this area, one among them (determined randomly) receives a telepathic message from the Alyxian Aboleth (see "Poisoned Emotions" earlier in the chapter).

This hall is composed entirely of magical barriers created by the keystone placed above the doorway to area M3. See the "Features of Cael Morrow" section for more information.

Two \textbf{scholarly agents} (see appendix A (p. 8)) stand outside the door to area M3. One is a lawful neutral human named Kal, and the other is a lawful good halfling named Nadiya.

These guards have dealt with raiders and infiltrators before. They are hostile to anyone who tries to get past them without a badge. If the characters are members of the Allegiance of Allsight or are accompanied by Gortag, the two guards let the group through with minimal questioning. If they believe the characters to be lying, the guards turn hostile and attack.

\paragraph{Keystone}

The stone cube above the door to area M3 is a keystone (see "Features of Cael Morrow" earlier in the chapter). Destroying the keystone or removing it from this area causes the magical barriers that enclose area M2 to disappear. When that happens, all creatures in the corridor are engulfed by water. The barriers around area M3 remain intact.

% Image placeholder: The Allegiance of Allsight maintains an underwater base camp, from where it coordinates archaeological excavations

\subsubsection{M3: Allegiance Base Camp}

\begin{DndReadAloud}
This storage building has three floors connected by a spiral staircase. The archaeologists are using the shelf space to store journals, recovered relics, and excavation equipment. Windows warded by barriers of shimmering blue light offer views of the underwater cavern outside this structure.
Three individuals wearing Allegiance of Allsight garb and badges---a tiefling, a goliath, and a kenku---bustle about the space, engrossed in their work.
\end{DndReadAloud}

The Allegiance's base of operations in the ruins is maintained by an adjunct professor of archaeology named \textbf{Insight Acuere}, a lawful good tiefling. Most of her team has been recalled to the Crystal Chateau, but she still has two research technicians helping her: the laconic \textbf{Scribble}, a chaotic neutral kenku, and the rambunctious \textbf{Xot}, a neutral good goliath (see the "Goliaths of Exandria" sidebar in chapter 4 (p. 4)). See below for their statistics and roleplaying information.

Nearly inseparable, \textbf{Scribble} and \textbf{Xot} have been working together at the camp for months. They complement each other as friends and scholars. \textbf{Xot} tends to talk ceaselessly about her latest discoveries, and \textbf{Scribble} patiently listens, absorbing many of \textbf{Xot}'s excited ramblings into his own speech.

If \textbf{Insight}, \textbf{Scribble}, and \textbf{Xot} become hostile toward intruders, they fight to kill. On the first turn of combat, \textbf{Insight} uses a \textit{sending stone} to contact Headmaster Cryon in the Crystal Chateau and request aid. Twenty minutes later, ten \textbf{scholarly agents} and five \textbf{scholarly excavators} (see appendix A (p. 8) for both stat blocks) arrive by way of area M1 to secure the base camp.

% Image placeholder: Professor {@creature Insight Acuere|CRCotN}

\paragraph{Insight Acuere}

The \textbf{tiefling professor} uses the \textbf{scholarly mastermind} stat block (see appendix A (p. 8)), with these changes:

\begin{itemize}
\item The professor speaks Aquan, Common, Infernal, and Orc.
\item She has resistance to fire damage and darkvision out to a range of 60 feet.
\end{itemize}

\textbf{Insight} won't cooperate with the Consortium of the Vermilion Dream unless she is tricked or magically compelled into doing so. If the characters are members of the Allegiance of Allsight, \textbf{Insight} informs them that they have arrived earlier than she expected, but she's nonetheless happy to see them. If they're allied with the Cobalt Soul, she requires them to prove their usefulness before she reveals any information (see "Insight's Quest" in mission 4 of the Cobalt Soul Story Track in chapter 4 (p. 4)).

\textbf{Insight} is the primary archivist for the Cael Morrow archaeological project and would put her life on the line to ensure its success. She is willing to share the results of her research with fellow Allegiance members. \textbf{Insight} can offer the following information in response to the characters' questions:

\begin{itemize}
\item The wards keeping the water at bay are magically generated by keystones in each warded location. These keystones look like small stone cubes with glowing runes carved into their sides. Removing a keystone from the area it protects causes the barriers around that location to vanish, allowing the water in.
\item The wards that have been erected end at an ancient temple to Corellon the Arch Heart, which seems to have its own magical way of holding back the water. Allegiance members have yet to breach the temple's ward.
\item A fourth member of the expedition, \textbf{Insight}'s friend \textbf{Galeokaerda}, a high elf professor of ancient literature, hasn't been seen in three days. \textbf{Insight} is concerned but hasn't yet considered raising a search party because \textbf{Galeokaerda} has a shield guardian to protect her.
\end{itemize}

\paragraph{Scribble}

\textbf{Scribble} uses the \textbf{scholarly agent} stat block (see appendix A (p. 8)), with this change:

\begin{itemize}
\item \textbf{Scribble} understands Auran and Common but speaks only through mimicry. He can imitate any sounds he has heard, including voices.
\end{itemize}

\textbf{Scribble} seems pleasant enough, but a character who spends more than 10 minutes with him can tell that the kenku is haunted. While scouring Cael Morrow for relics, he received telepathic threats from the Alyxian Aboleth but lacks the vocabulary to convey the horror of that experience; however, he can repeat any of the phrases listed on the Poisoned Emotions table.

\paragraph{Xot}

This goliath uses the \textbf{scholarly excavator} stat block (see appendix A (p. 8)), with these changes:

\begin{itemize}
\item \textbf{Xot} speaks Common and Giant.
\item She has resistance to cold damage and counts as one size larger when determining her carrying capacity and the weight she can push, drag, or lift.
\end{itemize}

If the characters are members of the Allegiance of Allsight or if they succeed on a DC 14 Charisma (Deception or Persuasion) check, \textbf{Xot} shows them the items she and \textbf{Scribble} have stored on the second floor of the building. She explains that these objects were deemed too damaged or insignificant to warrant their transport topside, but she found them too charming to discard entirely.

\textbf{Xot}'s collection consists of ancient baubles, rusted spear heads, and worthless bits of jewelry---but a character who makes a successful DC 12 Intelligence (Investigation) check finds a \textit{ring of the ram} and a sludgy but still potent \textit{potion of healing (superior)}. \textbf{Xot} can be convinced to give up these two items from her collection by a character who makes a successful DC 18 Charisma (Persuasion) check.

\paragraph{Double Agent}

Characters who are following the Allegiance Story Track might come to Cael Morrow hoping to expose a double agent in the expedition. A character who speaks with \textbf{Insight Acuere}, \textbf{Scribble}, or \textbf{Xot} can make a DC 15 Wisdom (Insight) check to get a sense of that person's loyalty. On a successful check, it's apparent to the character that the individual is fully dedicated to the expedition and doesn't have any intention of betraying their faction.

\paragraph{Exploring the Building}

The map of Cael Morrow shows only the first floor of area M3, but the upper two floors have a similar floor plan. The floors are described below.

\begin{description}
\item[First and Second Floors.] These floors have shelves holding organic and mineral specimens taken from Cael Morrow. Characters interested in Exandrian history might find several clay tablets, sound-storing magic crystals, and water-damaged vellum manuscripts containing accounts and images of daily life in the city before the events of the Calamity that brought it to ruin. Tucked among the specimens is a selection of baubles that \textbf{Xot} has hoarded.
\item[Third Floor.] Allegiance members come up here to rest. Large holes in the roof are covered by magical barriers of blue force, thanks to a keystone hidden on this floor (see "Keystone" below). This former attic holds six small crates of preserved food, a dozen casks of fresh water, six cots with soft pillows and blankets, and an unlocked wooden case containing two \textit{breathing bubbles} (see appendix B (p. 9)) with full air supplies and three \textit{potions of water breathing}. \textbf{Insight} and her companions won't permit the theft of these items, but \textbf{Insight} is willing to share them with the characters if they request her aid and their goals align with hers.
\end{description}

\paragraph{Keystone}

The keystone that keeps the water out of this building and the L-shaped corridor linking it to area M4 (see "Features of Cael Morrow" earlier in the chapter) is hidden in a wall on the building's third floor behind a loose stone. \textbf{Insight}, \textbf{Scribble}, and \textbf{Xot} know where the keystone is, but they won't divulge its location to others if doing so might endanger them or any other Allegiance members.

It takes 10 minutes for a character to thoroughly search the third floor. At the end of that time, the character can make a DC 15 Wisdom (Perception) check, finding the keystone on a successful check. If the keystone is destroyed or removed from this area, water comes gushing in through the rooftop. In addition, the barriers that form the corridor to M4 disappear, engulfing the corridor's occupants in water.

\subsubsection{M4: Royal Guest House}

\begin{DndReadAloud}
Before you are the ruins of a once-magnificent house of white stone. The ceiling has fallen through, leaving the marble floors littered with masonry, broken bed frames, and other furniture from at least three collapsed stories. A staircase leads from the entrance hall to the upper level of the main room. Walls of blue light hold back the water on all sides, leaving the interior dry.
\end{DndReadAloud}

The first time the characters enter this area, one among them (determined randomly) receives a telepathic message from the Alyxian Aboleth (see "Poisoned Emotions" earlier in the chapter).

Long ago this building was a guest house for royal visitors to Cael Morrow. Now it is an empty husk, mostly picked clean of valuables by the researchers of the Allegiance of Allsight. The keystone for the wards in this location (see "Features of Cael Morrow" earlier in the chapter) is above the north door.

\subsubsection{M4a: Parlor}

This roofless chamber is filled with rubble from the collapsed upper floors and the staircase that once led up to them.

\subsubsection{M4b: Private Study}

This ruined study is filled with enormous chunks of fallen masonry.

\paragraph{Treasure}

Any character who spends 10 minutes searching through the rubble can make a DC 20 Intelligence (Investigation) check. On a successful check, the character finds a \textit{spell scroll} of \textit{teleport}. Only one such scroll can be found here.

\subsubsection{M5: Access Corridor}

This corridor of shimmering blue barriers intersects with the remains of a guard tower (area M5a), which is enclosed by barriers as well. The keystone that maintains the barriers is set into the floor of the tower. See the "Features of Cael Morrow" section for more information on the magical barriers and keystones.

As they traverse this hall, the characters can see a deep reddish glow to the east (emanating from the ruidium spines in area M17).

\subsubsection{M5a: Crumbling Guard Tower}

\begin{DndReadAloud}
Cold water drips from above as the wards surrounding this ruined tower flicker. Resting in a shallow indentation in the middle of the floor is a stone cube six inches on a side that hisses and sparks.
\end{DndReadAloud}

The first time the characters enter this area, one among them (determined randomly) receives a telepathic message from the Alyxian Aboleth (see "Poisoned Emotions" earlier in the chapter).

Characters who have a passive Wisdom (Perception) score of 15 or higher notice the silhouette of a giant shark circling the tower. As long as the barriers remain in place, the shark doesn't pose a threat.

\paragraph{Malfunctioning Keystone}

Any character who examines the keystone closely and succeeds on a DC 15 Intelligence (Arcana) check determines that its magic is weakening, but there's no way to tell when it will cease to function. The Allegiance of Allsight explorers in Cael Morrow are unaware of the problem, since the keystone began to malfunction only recently.

It's only a matter of time before the faulty keystone fails, causing the barriers around the tower and the barriers that protect the adjoining hallways to disappear. Removing the keystone from the floor also causes this to happen. In its defective state, this cube has 25 hit points instead of the normal 50.

If the characters leave the keystone undisturbed, they can pass through the tower safely and continue onward. If any of them return to the tower a second time, the keystone makes a loud popping noise in their presence and cracks in half, bringing down the magical barriers. The shark, which is a \textbf{corrupted giant shark} (see appendix A (p. 8)), attacks characters who now find themselves underwater.

\subsubsection{M6: Sunken Tavern}

This stone-walled tavern is underwater. Sections of its exterior walls have collapsed, leaving openings through which the interior can be seen:

\begin{DndReadAloud}
Cracked stone tiles and fields of kelp surround a submerged building with light emanating from its interior. Through a hole in one wall, you see a blue-skinned elf with long, white hair. It's hard to see exactly what she's doing from this viewpoint, but she appears to be tending bar.
\end{DndReadAloud}

\textit{Continual flame} spells illuminate the interior of this underwater tavern, which is strewn with detritus. A neutral good sea elf named \textbf{Olara} floats behind a granite countertop. If approached, the barkeep cheerfully introduces herself and welcomes the characters to what she calls the Last Willow Tavern. \textbf{Olara} uses the \textbf{scout} stat block, with these changes:

\begin{itemize}
\item She has a swimming speed equal to her walking speed, and she can breathe air and water.
\item She has darkvision out to a range of 60 feet.
\item She has resistance to cold damage and advantage on saving throws to end the \textit{charmed} condition on herself. Magic can't put her to sleep.
\end{itemize}

\textbf{Olara} speaks an ancient dialect that formed the basis for both Elvish and Orc, and any character who speaks and understands either of those languages can converse with her. This is a trait she shares with some of the ghosts of her kinfolk that still haunt the ruins of Cael Morrow, but \textbf{Olara} is no ghost; she is one of the few mortals who survived Gruumsh's attack centuries ago. She subsists on the kelp growing throughout the Drowned City.

Decades spent near the Netherdeep and the Alyxian Aboleth have damaged \textbf{Olara}'s memories. A few of the Allegiance of Allsight archaeologists have interacted with \textbf{Olara}, to no avail, and since she shows no sign of hostility toward their efforts, they have decided to leave her alone.

\paragraph{Buying Goods from Olara}

\textbf{Olara} treats the characters as she would patrons of her tavern in any normal setting, commenting on the fact that, though the tavern is out of food and spirits, she can still offer some special brews. If the characters humor her, \textbf{Olara} shows off her collection of mundane wines and magic potions---their containers all tightly sealed. Characters can purchase any of these items:

\begin{description}
\item[Magically Preserved Wine.] \textbf{Olara} has three bottles of this delicious, pre-Divergence vintage in stock and asks 50 gp per bottle. An Ank'Hareli sommelier will pay up to 1,000 gp for a bottle of this rare vintage.
\item[Potions of Healing.] \textbf{Olara} has five of \textit{potions of healing} in stock and sells them for 50 gp each.
\item[Potions of Heroism.] \textbf{Olara} has two \textit{potions of heroism} in stock and sells them for 300 gp each.
\end{description}

\paragraph{Clearing Olara's Mind}

The haze that clouds \textbf{Olara}'s mind is magically induced, a result of many years spent near the Alyxian Aboleth and the Netherdeep. Any magic that ends a curse cures \textbf{Olara}'s affliction. Killing the aboleth also ends the effect.

If the characters ask \textbf{Olara} about Alyxian after clearing her mind, she fondly recalls serving wine to him on one occasion. She remembers Alyxian as a haunted and tragic figure, but beautiful in a way.

\subsubsection{M7: Hall of the Royal Library}

A dry, air-filled hallway enclosed by magical barriers and illuminated by \textit{continual flame} spells runs along the south side of a crumbling stone building. The keystone that maintains the hallway's barriers is set above the western door. See the "Features of Cael Morrow" section for more information on the magical barriers and keystones.

A door in the north wall of the corridor leads to a dry, air-filled library (area M7a). A door in the south wall has a flooded chamber (area M8) beyond it, with a protective barrier across the doorway that holds back the water. A door at the east end of the hallway has a similar barrier, through which characters can see the Temple of the Arch Heart (area M9).

\subsubsection{M7a: Royal Reading Room}

Magical barriers enclose this room, keeping the water out. The room's keystone is set above the south door. If the keystone is destroyed or taken from this area, the barriers around the room disappear, causing water to burst in through the north doors and seep through cracks in the walls and roof. It takes 1 minute for the room to become entirely flooded. See the "Features of Cael Morrow" section for more information on the magical barriers and keystones.

The first time the characters enter this area, read or paraphrase the following:

\begin{DndReadAloud}
Even though this magically illuminated room has been drained of water, the smell of rotting, waterlogged wood fills the air. Bookshelves lining the walls are filled with tomes and scrolls that have been turned to mush by centuries of submersion.
A female elf is sifting through the room's detritus, seeming exasperated. Standing near her but looking in your direction is a ten-foot-tall construct, its bipedal form made of metal, stone, and wood.
\end{DndReadAloud}

A neutral evil, high elf \textbf{occult extollant} (see appendix A (p. 8)) named \textbf{Galeokaerda} has set up camp in this room along with Crescent, her \textbf{shield guardian}. \textbf{Galeokaerda} wears the \textit{shield guardian's amulet} around her neck but hides it under her shirt.

Modify \textbf{Galeokaerda's} stat block as follows:

\begin{itemize}
\item \textbf{Galeokaerda} speaks Celestial, Common, and Elvish.
\item She has darkvision out to a range of 60 feet.
\item She has advantage on saving throws against being \textit{charmed}, and magic can't put her to sleep.
\end{itemize}

As described in the faction story tracks in chapter 4 (p. 4), \textbf{Galeokaerda} has long posed as a professor of the Allegiance of Allsight and has become \textbf{Insight}'s friend. Now she regrets betraying her friend and secretly hopes that she can convince \textbf{Insight} to turn against the Allegiance of Allsight when \textbf{Galeokaerda} and the Consortium of the Vermilion Dream take over Cael Morrow.

% Image placeholder: {@creature Galeokaerda|CRCotN}

If the characters are on a mission that doesn't involve \textbf{Galeokaerda}, she tells them to get lost while claiming that she's doing research for the Allegiance of Allsight. If the characters refuse to leave, she threatens to unleash her shield guardian. But if the characters call her bluff, she doesn't follow through on her threat, afraid that doing so will lead to her defeat. If the characters attack \textbf{Galeokaerda} or Crescent, however, the two of them react in kind.

\textbf{Galeokaerda} is suspicious of anyone she doesn't recognize, even someone affiliated with the Consortium. If the characters are on a mission that involves \textbf{Galeokaerda} and portray themselves as fellow explorers, she demands that they prove their worth by retrieving a clay tablet, which she describes as having a "sigil sequence" inscribed on it, from area M7b. If they do so, she tells them that she has identified the location of the Key to the Netherdeep: a ruined villa on the southern cliffs (area M13).

\paragraph{Double Agent}

At the end of any conversation with \textbf{Galeokaerda}, a character can discern her true intentions and affiliation by making a successful DC 15 Wisdom (Insight) check. Any character who observes \textbf{Galeokaerda} closely can, with a successful DC 15 Wisdom (Perception) check, spot the telltale markings of ruidium corruption on her arms. If a character calls attention to the marks, \textbf{Galeokaerda} hastily tries to explain how she got them in the line of duty; then she realizes that any explanation is fruitless and attacks.

\subsubsection{M7b: Royal Study Chamber}

This damaged, water-filled chamber contains a hungry \textbf{swarm of sorrowfish} (see appendix A (p. 8)).

\paragraph{Treasure}

Lying on the floor in the middle of the room is a \textit{teleportation tablet} (see appendix B (p. 9)) that was left behind by a member of the Consortium of the Vermilion Dream. This individual was searching the area for relics when a giant shark entered through a hole in the roof. The agent panicked and fled, dropping the tablet on their way back to area M7a. \textbf{Galeokaerda} wants to acquire the tablet, which connects to a teleportation circle on the roof of First Eclipse in Ank'Harel's Suncut Bazaar (see chapter 4).

\subsubsection{M7c: Royal Ritual Chamber}

This flooded chamber has been stripped bare by water and time.

The first time the characters enter this area, one among them (determined randomly) receives a telepathic message from the Alyxian Aboleth (see "Poisoned Emotions" earlier in the chapter).

A character who spends at least 1 minute searching the area can make a DC 14 Intelligence (Investigation) check, uncovering a permanent teleportation circle on a success. A character who knows this circle's sigil sequence (which can be learned by studying the circle for 1 minute) can teleport to it using the \textit{teleportation circle} spell.

\subsubsection{M8: Submerged Library}

\begin{DndReadAloud}
Broken clay tablets inscribed with writing lie among the detritus of this former library. What were once polished marble shelves are now coral-encrusted ruins overgrown with kelp. The shattered remains of a circular glass window give off flashes of reddish light.
\end{DndReadAloud}

The first time the characters enter this area, one among them (determined randomly) receives a telepathic message from the Alyxian Aboleth (see "Poisoned Emotions" earlier in the chapter).

The writing on the tablets is in a script that has elements familiar to anyone who understands Elvish or Orc. Any character who spends at least 10 minutes searching through the detritus can make a DC 13 Intelligence (Investigation) check, finding three unbroken tablets amid the rubble on a success.

When placed side by side in the proper order, the three tablets tell the story of the gathering of elves across Exandria; the blessing of Corellon upon their settlement, Cael Morrow; and the birth of the first orcs. Before the tablets can be manipulated in this way, however, the characters must deal with the ancient ghosts that haunt them.

\paragraph{Haunted Tablets}

Each of the three unbroken tablets is inhabited by a lawful neutral \textbf{ghost}, desperate for a peaceful end to its suffering. When a Humanoid first touches a tablet, the ghost inside tries to possess the creature using its Possession action. If the creature resists that attempt, the ghost appears in an unoccupied space within 5 feet of the tablet. After one ghost has emerged from its tablet, whether by possessing a creature or otherwise, the other two ghosts appear alongside it.

Two of the ghosts look like male elves, and the third looks like a female orc. All three are dressed in elegant robes trimmed with fine embroidery. The ghosts are not hostile; they try to possess creatures only so they can share their stories and thus find peace after death, but they defend themselves if attacked. While possessing a creature, a ghost moves the creature's body as far from combat as possible and doesn't force the creature to attack.

Any character who is possessed by one of the ghosts experiences a vision of the ghost's past life: their families and friends, the shining alabaster spires of their city in its prime, and the frantic attempts at evacuation as Gruumsh began his assault. The vision ends with a scene showing the two elves and the orc in a splendid library, about to sacrifice their lives in a futile attempt to preserve their history so that those they loved would not be forgotten. They look out from a library window and behold a nearby temple. A lone warrior with a spear and shield---Alyxian---stands atop the building. He is cast in shadow while the hundred-foot-tall figure of Gruumsh looms over him and laughs. As Gruumsh's spear descends on the temple, the vision goes dark. The vision takes 1 minute to play out in full. When it is over, the ghost ends the possession.

\paragraph{Speaking with the Ghosts}

Characters who speak Elvish or Orc can converse with the ghosts, who speak an ancient dialect that forms the basis for both modern languages. The ghosts respect Alyxian the Apotheon, who they believe died in a vain attempt to defy Gruumsh. They know little about Alyxian beyond his divine connection to Sehanine, Avandra, and Corellon, and they are unaware of the Netherdeep and the existence of ruidium.

\paragraph{Laying the Dead to Rest}

The ghosts find peace if they are reassured that they and their city will be remembered. A character can try to bring peace to the ghosts by offering reassuring words and then making a DC 12 Charisma (Persuasion) check. A character who was possessed by one of the ghosts has advantage on the check. On a successful check, the ghosts thank the character before disappearing in a shimmer of light. As a token of the ghosts' gratitude, the character gains inspiration, p. 4 (see the Player's Handbook). On a failed check, the ghosts are not laid to rest, and further attempts by that character to reassure the ghosts fail automatically.

\subsection{Cael Morrow Locations (M9-M17)}

% Image placeholder: Divine magic keeps water from flooding the temple of Corellon in the sunken heart of Cael Morrow

\subsubsection{M9: Temple of the Arch Heart}

\begin{DndReadAloud}
A once-majestic stone temple stands in the middle of a sunken courtyard. Alabaster pillars, glowing softly, surround the open area. Pulsing red light issues from an archway that leads inside the temple.
\end{DndReadAloud}

Characters who have a passive Wisdom (Perception) score of 15 or higher notice a distortion in the light filtering through the water between the alabaster pillars and the center of the temple. The pillars project an \textit{invisible} protective dome of divine magic around the temple. Only characters who are accompanied by a devout worshiper of Corellon can pass through this dome; otherwise, the dome can't be penetrated, is immune to all damage, and can't be dispelled in any way. If no characters in the party meet this requirement, the ghost of one of Corellon's priests can be found in area M14.

\paragraph{Temple Interior}

If the characters move to a position where they can see through the archway in the temple's south wall, read:

\begin{DndReadAloud}
The interior of the temple is dry. Lavishly sculpted pillars line the interior walls, and the floor is adorned with a mosaic of silver and green glass that depicts two verdant crescent moons atop a four-pointed star. At the center of the mosaic is a red crystal shard, pulsing with light in a mockery of a heartbeat. Scarlet tendrils radiate from it and crawl across the stonework. Against the north wall stands a statue of Corellon.
\end{DndReadAloud}

The air in the temple is fresh and smells faintly of incense, although the source of the odor is not apparent. Corellon's divine magic hedges out the water, and the barrier stays in place as long as the statue of Corellon remains intact. The statue has AC 20; 80 hit points; and immunity to poison, psychic, and radiant damage. If the statue is destroyed, the divine ward around the temple ends, allowing water to pour in through the entrance and enabling creatures to enter the temple without being accompanied by a priest or cleric of Corellon.

\paragraph{First Vision}

When the characters enter the temple for the first time, a sense of peace descends upon them, and a vision fills their minds:

\begin{DndReadAloud}
An image comes into focus. Alyxian is on his knees in this place, his head bent and his expression haggard. He looks old beyond his years. He raises his gaze to the statue of Corellon, his eyes clear despite the \textit{exhaustion} showing on his face.
"Whatever it takes," he whispers, his voice filling the room. "Even if it costs my life and my soul. I will do what it takes to protect them."
\end{DndReadAloud}

Any creature that comes within 10 feet of the pulsating red crystal at the center of the temple must succeed on a DC 15 Wisdom saving throw or become \textit{frightened} of it. This effect lasts until the creature leaves the temple. A creature that succeeds on the saving throw becomes immune to the crystal's fear-inducing effect.

\paragraph{Second Vision}

The ruidium shard protruding from the mosaic is clearly out of place in the temple. The character who first makes physical contact with the crystal experiences a second vision:

\begin{DndReadAloud}
A wash of light engulfs you, as does the sensation of traveling at great speed. You are going down, faster than you thought possible, as a sharp pain blooms above your breastbone. You keep plummeting. The light diminishes and then goes out.
Suddenly, you are looking down upon Alyxian as he floats in the blackness. His expression is one of slowly growing horror. He stretches an arm up to you, as though begging you to take his hand and save him.
"When you see me, permit me to leave," he says. "I can't do it myself. You must. You will!"
\end{DndReadAloud}

The vision ends, and the red crystal lets out a loud hiss just before it cracks and explodes. Creatures in the temple who did not experience the second vision see the red crystal flare with light as it shatters. For a moment, a translucent image of Alyxian, his face stern and impatient, appears in the center of the temple. It then disappears, and all is quiet once more.

When the crystal shatters, the character who touched it must make a DC 16 Charisma saving throw. On a failed save, the character takes 33 (6d10) psychic damage and gains 1 level of \textit{exhaustion}. If the character isn't already suffering from ruidium corruption, the character becomes corrupted when they fail the save. On a successful save, the character takes half as much damage and suffers no other effects.

When a character experiences the second vision, the Alyxian Aboleth loses its ability to poison the emotions of nearby creatures, and the \textit{Jewel of Three Prayers} (see appendix B (p. 9)) transforms into its \textit{Exalted State}---both consequences of gaining the blessing of Corellon the Arch Heart.

\subsubsection{M10: Crumbling Building}

This structure appears to be on the verge of collapse. Small, harmless fish dart in and out of the many cracks in the walls.

The first time the characters enter this area, one among them (determined randomly) receives a telepathic message from the Alyxian Aboleth (see "Poisoned Emotions" earlier in the chapter).

If the characters explore the interior of the building, read:

\begin{DndReadAloud}
The walls are covered in faded murals that depict orcs and drow working together to build a city in the heart of a verdant jungle. Dozens of clay tablets are scattered across the floor.
\end{DndReadAloud}

Four \textbf{scuttling serpentmaws} (see appendix A (p. 8)) lurk amid the rubble and scuttle across the floor to attack characters who investigate the room. Characters who have a passive Wisdom (Perception) score of 18 or higher notice the serpentmaws and aren't surprised when they attack.

\paragraph{Clay Tablets}

Characters who spend 10 minutes searching the floor of the building find, among other clay tablets that describe ancient prayers and catechisms of the monks of the temple of Corellon, the particular tablet that \textbf{Insight} has been seeking (see "Insight's Quest" in mission 4 of the Cobalt Soul Story Track in chapter 4 (p. 4)). This tablet is written in Celestial script. It reads:

\begin{DndReadAloud}
I know not who will read this, for if I survive, reading it will not be necessary, and if I die, none will be alive to do so. Yet I feel compelled to record my thoughts. The Ruiner and his legions close upon us. The walls are breached, reduced to rubble by the one-eyed god. Thousands have already died, and thousands more will as well, unless I can muster the courage to stop him.
Already the people call me "Apotheon." I do not deserve the title---but I shall claim it. If calling me a demigod gives them the courage to fight alongside me, I shall bear this mantle. And even if I am to fall in the hours to come, they will honor me by letting this sacrifice fuel their fight against the Betrayers. To that end, may the memory of this day never die.
\end{DndReadAloud}

\subsubsection{M11: Kelp Forest}

% Image placeholder: Two aboleth spawn guard the lair of the Alyxian Aboleth

\begin{DndReadAloud}
Once a lush garden, this area is now filled with strands of kelp that rise to a height of about thirty feet. Small shrines built of pale stone stand among the growth. Their surfaces are carved with images of the face of Corellon, serene and benevolent.
\end{DndReadAloud}

Two \textbf{aboleth spawn} (see the accompanying stat block) guard the kelp forest. These tentacled creatures are bound to the Alyxian Aboleth and do its bidding. Their task is to guard the aboleth's treasure (see "Treasure" below).

The kelp forest is the lair of the \textbf{Alyxian Aboleth} (see appendix A (p. 8)). If the aboleth retreated here to take a short rest, it tries to hide from interlopers until its rest is finished before attacking them. If the aboleth is not present but returns to find any of its treasure missing (see below), it scours Cael Morrow in search of the thieves, determined to destroy them.

\paragraph{Treasure}

Any character who spends at least 5 minutes searching the kelp forest can make a DC 15 Intelligence (Investigation) check, finding the aboleth's accumulated treasure on a success. This hoard consists of a pouch that holds 156 gp, a \textit{ruidium shield} (see appendix B (p. 9)), a \textit{ruidium shortsword} (see appendix B (p. 9)), and a \textit{brooch of shielding} still pinned to the garments of a water-bloated human corpse (the remains of an Allegiance of Allsight explorer).

% Image placeholder: {@creature Aboleth Spawn|CRCotN}

\subsubsection{M12: Cliffside Path}

\begin{DndReadAloud}
A narrow, winding path ascends a steep, rocky slope strewn with oddly shaped chunks of stone. The path splits into two branches that each wind up a different eighty-foot-high peak. What's left of a once-elegant villa rests on one peak, and a crumbling watchtower adorns the other.
\end{DndReadAloud}

Closer examination reveals that the chunks of stone are shaped like the dismembered hands, arms, legs, heads, and torsos of Humanoids. The head-shaped chunks all have screaming faces. These are the \textit{petrified} remains of the victims of the creature that has taken up residence in the villa (see area M13). A character who succeeds on a DC 12 Intelligence (Arcana) check recognizes the signs of petrification and knows that the bodies are too broken to be restored to life.

\subsubsection{M13: Villa}

\begin{DndReadAloud}
The roof of this building has collapsed, but its twenty-foot-high walls remain mostly intact. Coral grows around a closed stone door at the base of the structure.
\end{DndReadAloud}

The characters might run into their rivals in this location. The rivals, if present, are gathered in front of the villa and debating whether to enter through the door or by way of the open roof. Use the following text to describe what's inside:

\begin{DndReadAloud}
Floating in the middle of the ruined villa is a huge, jellyfish-like creature with barbed tentacles. Its bell is covered with a crystalline lattice and has a greataxe with veins of ruidium arcing up the blade wedged in it. Beneath the bell, a tendrilous crimson mass crackles with energy and provides shelter for a school of tiny, sharp-toothed fish.
\end{DndReadAloud}

A \textbf{death embrace} (see appendix A (p. 8)) has made its lair in the ruins of this once magnificent location and attacks anyone who comes within reach of its tentacles. A \textbf{swarm of sorrowfish} (see appendix A (p. 8)) lurks among the tendrils of the \textbf{death embrace}, gaining cover. When the \textbf{death embrace} grapple a creature, the swarm loses its cover and the sorrowfish emerge to feed on the \textit{grappled} creature.

\paragraph{Treasure}

Embedded in the \textbf{death embrace}'s bell is a \textit{ruidium greataxe} (see appendix B (p. 9)), the so-called Key to the Netherdeep that the Allegiance of Allsight and the Consortium of the Vermilion Dream are after (see chapter 4 (p. 4)). The axe can be removed from the \textbf{death embrace} only after the creature is dead.

A character who spends 10 minutes searching the villa and succeeds on a DC 15 Intelligence (Investigation) check finds a locked jewelry box. The jewelry box can be opened by a character who makes either a successful DC 15 Dexterity check using \textit{thieves' tools} or a successful DC 15 Strength (Athletics) check. Inside are two emerald rings (1,500 gp each), a string of pearls (750 gp), and a \textit{ring of shooting stars}.

\subsubsection{M14: Watchtower}

\begin{DndReadAloud}
This watchtower is perched on a high hilltop overlooking the rest of the sunken city. The structure is roughly thirty feet high but used to be taller. Part of it has collapsed, leaving the remaining structure with a jagged crown of loose-fitting stone. A single stone door leads into the tower at ground level.
\end{DndReadAloud}

Characters who peer into the tower, either through the open door or by swimming up to the roof and looking down, view the following scene:

\begin{DndReadAloud}
A crumbling staircase follows the inside wall as it climbs to a fifteen-foot-high balcony. The stairs continue upward until they disappear a few feet short of the open top of the tower. Armed, wraith-like forms move about inside the tower, like a battle-ready garrison that doesn't know the war is long over. Floating a few feet above the balcony is a pale, spectral orc.
\end{DndReadAloud}

The tower is haunted by eight \textbf{sword wraith warriors} under the command of Beltreath, a lawful neutral \textbf{sword wraith commander}. See appendix A (p. 8) for these creatures' stat blocks.

The sword wraiths are former defenders of Cael Morrow who fell during the battle against Gruumsh. They are protecting the \textbf{ghost} of Hadarai, a neutral good, orc priest of Corellon. Hadarai is soft-spoken and kindhearted. He swore a vow of pacifism centuries ago that he has never broken; even after being stripped of his mortal form, he won't fight or use Possession.

\paragraph{Roleplaying the Spirits}

If the characters enter the tower or otherwise make their presence known, the discordant sound of a bugle pierces the water, and the warriors gather immediately to defend Hadarai and the tower.

Before raising his weapon, Beltreath explains that his warriors are the last wardens of the Drowned City and informs the characters that any invasion is unwelcome. If the characters explain their reason for being in Cael Morrow, Hadarai and Beltreath listen closely. If the characters' stated purpose involves helping the Apotheon, which Hadarai and Beltreath both regard as a worthy goal, the spirits become friendly toward the characters.

Hadarai met Alyxian once long ago, while the warrior was praying to Corellon for aid in the hours before Cael Morrow was destroyed. Hadarai is slow to accept that the Apotheon yet lives and can be saved, but a character can convince him of the truth of this claim with a successful DC 15 Charisma (Persuasion) check. If a character shows Hadarai the \textit{Jewel of Three Prayers}, the check is made with advantage, since the ghost recognizes the jewel and knows its properties.

If combat breaks out and Beltreath is reduced to 50 hit points or fewer, he surrenders and begs the characters not to destroy him, for Hadarai's sake. If the characters show mercy, Hadarai is grateful and offers to assist them in any way he can. For example, the ghost, as a priest of Corellon, can help the characters bypass the magic ward around the temple (area M9) or can warn the characters about the \textbf{aboleth} that haunts the city.

Wherever Hadarai goes, Beltreath and his sword wraiths follow. None of these spirits can leave Cael Morrow, nor would they want to.

\subsubsection{M15: Collapsed Guard Tower}

\begin{DndReadAloud}
This tower has both toppled over and collapsed inward. Around its foundation are heaps of rubble.
\end{DndReadAloud}

The first time the characters enter this area, one among them (determined randomly) receives a telepathic message from the Alyxian Aboleth (see "Poisoned Emotions" earlier in the chapter).

\paragraph{Hidden Symbol and Trapdoor}

A character who searches the tower ruins and succeeds on a DC 15 Intelligence (Investigation) check discovers a large symbol of Corellon inscribed on the floor, mostly hidden under sand and debris. Closer examination discloses a 5-foot-diameter, circular stone trapdoor disguised as part of the symbol. The trapdoor is closed and stuck. A character can use an action to try to pry it up by using a crowbar or similar tool, doing so with a successful DC 17 Strength (Athletics) check. A \textit{knock} spell or similar magic can also open it. If all else fails, the trapdoor (which has AC 17, 60 hit points, and immunity to poison and psychic damage) can be destroyed.

\paragraph{Secret Armory}

Below the trapdoor is a 10-foot-high, 20-foot-diameter circular armory. A corroded iron ladder descends into the chamber.

A character who searches the opening for traps and succeeds on a DC 17 Intelligence (Investigation) check spots a tiny glyph inscribed along the edge of the opening, in a location that's not visible while the trapdoor is closed. If this glyph is dispelled with a successful casting of \textit{dispel magic} (DC 17) or defaced using \textit{mason's tools} or similar equipment, the following trap ceases to function.

\paragraph{Trap}

If a creature enters the armory without first saying a prayer to Corellon, a burst of radiance flares from the chamber. Each creature in the armory and within 15 feet of the trapdoor must make a DC 18 Constitution saving throw, taking 33 (6d10) radiant damage on a failed save, or half as much damage on a successful one. This trap can be triggered only once.

The burst of light from this trap attracts the \textbf{slithering bloodfins} in area M16 and possibly the \textbf{Alyxian Aboleth} (see appendix A (p. 8) for their stat blocks). If the aboleth is in the nearby kelp forest (area M11) when the trap is sprung, it sees the light and immediately moves to investigate it.

\paragraph{Treasure}

The walls of the armory are festooned with longswords, shortswords, spears, longbows, shortbows, arrows in leather quivers, and steel shields---all corroded or rotted beyond usefulness. Among these objects are two magic items, both of which can be located quickly by using a \textit{detect magic} spell: a \textit{sentinel shield} and a \textit{frost brand longsword}. A character must otherwise spend at least 1 minute searching the armory to find these magic items.

\subsubsection{M16: Collapsed Ruins}

\begin{DndReadAloud}
The toppled ruins of a palace are strewn across the cavern floor. Yellow-green kelp sprouts from the cracks between heaps of crumbled stone, and corroded silver and bronze spires lie bent and broken as schools of iridescent fish swim between them.
\end{DndReadAloud}

Unless the characters killed them in area M15, three \textbf{slithering bloodfins} (see appendix A (p. 8)) prowl this area in search of prey. The characters spot them---three streaks of crimson darting through the ruins. Unless the characters move away from the ruins immediately, the bloodfins swim toward the intruders and attack.

\subsubsection{M17: Rift to the Netherdeep}

\begin{DndReadAloud}
Amid the ruins of a sunken palace is a gaping rift that pulsates with sickly red light. Black energy swirling with oily colors shimmers and hisses inside the opening of the rift. Glowing spines of ruidium sprout from its edges, forming a jagged, partially intact dome around the space.
\end{DndReadAloud}

If the characters have yet to defeat the \textbf{Alyxian Aboleth} (see appendix A (p. 8)), it senses their intrusion into this area and arrives 1 minute later, attacking the characters in a rage.

The rift to the Netherdeep is covered with, and protected by, a film of black energy streaked with all the colors of the spectrum, appearing much like an oil slick. Water, liquids, and creatures native to the Netherdeep can pass through this barrier in either direction, but no other creatures or objects can do so unless the barrier is pierced by a \textit{ruidium weapon} (see below).

\paragraph{Opening the Rift}

To enter the rift, the characters must force it open using one of the following items:

\begin{itemize}
\item A \textit{ruidium weapon}, like the one in area M13. Touching the weapon's edge to the barrier opens the rift.
\item The \textit{Jewel of Three Prayers} in its Exalted State. When a creature brings the amulet within 10 feet of the rift, it glows with holy light that pierces the barrier protecting the rift.
\end{itemize}

Each creature within 60 feet of the rift when it comes open must make a DC 18 Charisma saving throw, taking 33 (6d10) psychic damage on a failed saving throw, or half as much damage on a successful one.

When the rift is opened, the characters can use this route to enter the Netherdeep. The rift closes after 1 hour but can be forced open again.

\paragraph{Into the Netherdeep}

The rift to the Netherdeep opens directly into area N1 in chapter 6 (p. 6). The first time they reach this area, the characters might not carry out a full exploration. It's likely that they'll return to Cael Morrow and then to the surface, especially if anyone in the party is susceptible to the effects of water pressure, as described in chapter 6. If they are slow to reenter the Netherdeep, a directive from their faction might be necessary to encourage them to follow their mission to its completion.

\chapter{The Netherdeep}\label{ch:the-netherdeep-7-7}

Leaving behind the bustling streets of Ank'Harel and the gloom of Cael Morrow, the party now navigates the dark waters of the Netherdeep. This otherworldly tomb of Alyxian's own making is similar to a demiplane or an extradimensional space connected to the Material Plane but not part of it. In other words, the Netherdeep is its own plane of existence. It is where the Apotheon's negative emotions have festered, creating labyrinthine trenches of regret, fury, yearning, and despair.
% Image placeholder: Five adventurers journey across the wastes of Xhorhas, watched by the moons Catha and Ruidus
\section{Running This Chapter}

Re-familiarize yourself with the Apotheon's back story in the introduction (p. 0) before running this chapter. Because this chapter leads up to the final confrontation with the Apotheon, understanding his past and his frame of mind is crucial. The ending of the adventure hinges on whether the characters can help him overcome his pain and despair.
\section{Netherdeep Overview}

The Netherdeep is an underwater dungeon at the bottom of a lightless abyss. Spells or magic items that enable characters to breathe underwater are a necessity for exploring the Netherdeep.

The following sections describe the Netherdeep's recurring features, the Fragments of Suffering that the characters must acquire before they can face the Apotheon, and the main threats the characters must overcome as they make their way toward the Apotheon's prison.



\subsection{Netherdeep Features}

The following sections summarize important or recurring features of the Netherdeep.

\subsubsection{Ceilings and Floors}

Tunnels have 10-foot-high ceilings, while a chamber or grotto has no ceiling unless its description states otherwise. If an area has no ceiling, assume that its walls are about 100 feet tall, and beyond them is a dark, underwater abyss. Creatures that swim upward never reach the surface and can only return downward to the area they left.

Unless otherwise noted, all tunnels have irregular floors that are at the same elevation as the floors of the chambers they adjoin.

\subsubsection{Fragments of Suffering}

To enter the Heart of Despair, where the Apotheon is imprisoned, a creature must have absorbed at least one Fragment of Suffering---a tangible representation of an emotion felt by the Apotheon. Nine fragments exist in the Netherdeep. A Fragment of Suffering looks like an immobile mote of light about the size of a fig. The Fragments of Suffering table shows the names and locations of the fragments.

% Table: Fragments of Suffering
\begin{DndTable}[header={Fragments of Suffering}]{ll}
Fragment Name & Area \\
Fragment of Despondence & N4a \\
Fragment of Attachment & N5a \\
Fragment of Pity & N7a \\
Fragment of Rancor & N14a \\
Fragment of Abhorrence & N16a \\
Fragment of Melancholy & N18 \\
Fragment of Intransigence & N22a \\
Fragment of Deception & N24a \\
Fragment of Loathing & N25 \\
\end{DndTable}

\paragraph{Absorbing a Fragment}

As an action, a creature can touch a fragment it can see and thereby absorb the fragment into its body. If a character absorbs a fragment, give that character's player Fragments of Suffering (see appendix D (p. 11)). The fragment disappears and becomes part of the creature's soul until the creature dies, transfers the fragment to another willing creature (see "Releasing a Fragment" below), or enters the Heart of Despair.

A creature can absorb up to three Fragments of Suffering. If a creature that carries three fragments tries to absorb a fourth, the attempt fails automatically.

\paragraph{Releasing a Fragment}

A creature can use an action to transfer one fragment it is carrying to another creature by touching the fragment's intended recipient. When a fragment leaves a creature, that creature loses the benefit and drawback conferred by the fragment. If a character initiates a transfer, have that character's player give the corresponding card to the new recipient.

If a creature that is carrying a fragment dies or enters the Heart of Despair, the fragment immediately leaves the creature and returns to its original location (noted in the Fragments of Suffering table).

% Image placeholder: Alyxian is seen in visions as a young man struggling to come to grips with his destiny

\subsubsection{Light}

Locations in the Netherdeep are either engulfed in darkness or dimly lit by ruidium deposits (see "Ruidium" below). All non-ruidium light sources in the Netherdeep shed light half as far as usual. For example, a \textit{light} spell gives off bright light in a 10-foot radius and dim light for an additional 10 feet.

\subsubsection{Ruidium}

Ruidium is plentiful throughout the Netherdeep. Brief contact with the vermilion mineral isn't harmful, but contact lasting more than a few seconds can lead to ruidium corruption (see "Ruidium Corruption" in the introduction (p. 0)). The save DC to resist the corruption is 20.

\subsubsection{Secret Doors}

Secret doors in the Netherdeep are made of the same ruidium-veined stone as the walls surrounding them. In any location that has a secret door, the text explains how the door can be found and opened.

A character who spends at least 10 minutes searching an area for secret doors can make a DC 20 Intelligence (Investigation) check at the end of that search. If the check succeeds and there's a secret door to be found, the character finds it. Secret doors in the Netherdeep are unlocked unless otherwise noted, and they open easily despite their great weight.

When the characters pass through a secret door, it closes behind them unless it is held or wedged open. A character who has passed through a secret door once can always find and open it again from either side of the door.

A secret door has AC 17, 50 hit points, and immunity to poison and psychic damage. Any creature within 10 feet of a secret door that deals damage to it must make a DC 15 Constitution saving throw, taking 22 (4d10) psychic damage on a failed save, or half as much damage on a successful one.

\subsubsection{Water Pressure}

The Netherdeep is a supernatural abyss, and the water pressure it exerts on creatures within it is beyond anything most mortals can survive. Creatures not native to the Netherdeep take 35 (10d6) force damage at the end of every minute they spend within the Netherdeep, unless they are wearing or wielding an item infused with ruidium or have ruidium embedded in their bodies. See appendix B for more information on ruidium items.

\subsection{Apotheon Lore}

In this chapter, the characters experience fleeting visions that shed light on Alyxian's past. The characters experience these visions as a group regardless of how close they are to one another. The visions come to them in the order given in the Apotheon Lore table.

A vision occurs whenever one of the following events takes place:

\begin{itemize}
\item The characters, as a group, finish a short or long rest in the Netherdeep.
\item A character absorbs a Fragment of Suffering.
\item A character falls \textit{unconscious} in the Netherdeep.
\item Some other event in the Netherdeep triggers a vision, as noted in the text.
\end{itemize}

After the characters receive all twenty visions, they can piece together the Apotheon's whole story.

% Table: Apotheon Lore
\begin{DndTable}[header={Apotheon Lore}]{cX}
Order & Vision \\
1st & A child is born, eerily silent, and the midwife refuses to hold him. "Look. Even he knows," she rasps to his mother as she clutches the baby, trembling. "His little soul prepares for the dread omens that will haunt him the rest of his life." \\
2nd & Two parents place a baby in front of a priest of Pelor and beg a blessing. The priest makes a gesture over the infant's head, then recoils. "There will be no help for him here. Don't be foolish. There is no cure for the mark of the red moon." \\
3rd & A baker hides a mussed, dirty boy behind the counter in his shop, slipping him a tart when no one is looking. Outside, children crow and call for the omen-touched boy, laughing, as a patron furtively mutters to the baker, "He's full of Betrayer's blood, I heard." \\
4th & A teenager sleeps fitfully under a vermilion moon, its glow washing him in faint red tones. He doesn't rest easily until a cloud drifts in front of the moon, masking its sight from his slumber. \\
5th & A young man stands atop a mountain and beholds a green Xhorhasian vista. He holds a handful of lush flowers. The vision shifts, and now he looks out over the same landscape blighted by the Calamity---the greenery is burned to the ground, corpses litter the hillside, and he grips a sword dripping blood. \\
6th & A soldier crouches over her dead commander in a plaza filled with the bodies of people and monsters. The weight of responsibility settles uncomfortably on her. "Harbinger!" she spits at Alyxian. "War-bringer! Omen! We sheltered you!" \\
7th & A guttural roar leaves the throat of the young warrior as he cleaves through the foes in front of him. His rapid breathing makes allied soldiers around him skittish. "Omen of malice," one hisses to another. "Don't get on his bad side." \\
8th & The warrior, now an adult, kneels before an altar in a ruined temple of Avandra the Change Bringer. "Fight for the freedom of others," he whispers in reverence. His hand, scarred and shaking, grips the hilt of his sword. "Luck favors the daring," he says with determination. "The Change Bringer doesn't bless cowards." \\
9th & The warrior holds his hands up to the sky beneath the sheltering, silvery glow of the moon Catha, and a drop of moonlight falls into his cupped hands. It turns into a swirl of colors, then takes the form of a small golden amulet. The warrior, awestruck, whispers a prayer as he presses the \textit{Jewel of Three Prayers} to his chest. \\
10th & Bleeding heavily, the warrior limps through a burning battlefield toward a wailing child. She screams when she sees him; he flinches, but he picks her up, then carries her toward the remnants of civilization in the distance. \\
11th & Close to sleep, huddled beneath a tree, and cloaked in a ratty blanket, the warrior runs his finger along the edge of the \textit{Jewel of Three Prayers}, finding some comfort in its familiar shape. \\
12th & During a thunderstorm, the warrior bellows, "Why? Avandra, this is not the change I seek! Tell me why I walk these steps!" The roar of the storm is all he hears in response. He rakes his hands through his hair. "Tell me!" \\
13th & The warrior kneels in front of two moonlit graves bearing the names of his parents. He places incense between them, resting it on a simple box decorated with wedding bells. The box bears the inscription "For those whose love never wavered." Around his neck, he wears the \textit{Jewel of Three Prayers}, which carries the blessings of Avandra the Change Bringer and Sehanine the Moon Weaver. \\
14th & The warrior's sword cleaves a beast in two. Its corpse falls onto the blood-soaked ground, joining many other victims. The warrior, wounded, stands above the carnage. Two people lean out of nearby houses---then four. Ten. Dozens. They run to him and cry out, some of them sobbing. "The Apotheon!" "They said you were coming!" "We knew it!" \\
15th & The warrior kneels before an elf monarch. "I've heard the most curious thing about you, Alyxian," the elf drawls. "You keep refusing the titles I offer you. The land, the riches, the servants." The warrior whispers in reply, "The power to change fate. That's all I have ever asked." \\
16th & The light is warm and dim in a tavern, where the weary warrior listens to a bard's song: "Don't kiss underneath the red moon, nor promise yourself to sweet eyes! They'll leave for another one's home too soon, or string up your heart in their lies!" He pays for his drink and leaves, despite the early hour. \\
17th & The warrior stands at the door of a ruined house and peers inside at a huddled, \textit{frightened} family. They draw back when they see him, but he pulls off his cloak and drapes it over a shivering youth. "Go to the city," he rasps. "It is a bastion of hope." \\
18th & The night air is so thick with the metallic smell of blood that the warrior can sense the massacre before he sees it. He crests a hill and stares out over a city in southern Marquet, trembling with a mixture of anticipation and dread as the light of the moon Ruidus bathes the bodies in red. \\
19th & Standing in front of a mirror, the warrior considers the toll the Calamity has taken on him---the scars, the \textit{exhaustion} etched into his features, all the result of the endless clash between faith and fear. He clasps the \textit{Jewel of Three Prayers}, now complete with the addition of the blessing of Corellon the Arch Heart. "Three divine mantles," he whispers. "Let us all be free of this duty." \\
20th & A warrior stands atop a temple, beneath the towering form of a one-eyed titan, as if intending to test his human-sized spear against the Betrayer God's massive weapon. "I am cloaked in moonlight," he whispers, alight with a silvery aura. "My heart is steeled by deep magic." Brilliant green suffuses the silver of the aura. The demigod faces down Gruumsh's apocalyptic spear. "And armed with the power to change fate!" Golden light flares, just before the god's spear thrusts downward and all goes dark. \\
\end{DndTable}

\subsection{Rivals in the Netherdeep}

In this chapter, the characters' rivals use their tier 3 stat blocks (see appendix A (p. 8)).

Unless they were routed or killed earlier, the rivals are also trying to reach the Apotheon. Depending on how earlier events played out, the rivals are either ahead of the characters or behind them. Regardless, you must choose when and where the characters first meet the rivals in the Netherdeep. If you're not sure where this encounter should occur, choose one of the following options:

\begin{itemize}
\item The rivals lie in wait for the characters in area N1. Their goal is to force the characters to flee while warning them never to come back.
\item The two groups meet in area N26. If the rivals are hostile toward the characters, this is their final stand---a fight to the death. If the rivals are friendly or indifferent toward the characters, they fight as hard as they can but are willing to retreat if the battle turns against them.
\end{itemize}

If the characters are trailing the rivals, you can drive home the point that "the race is on" by having the rivals leave some trace of their recent passage through an area. Determine what they left behind by rolling on the Evidence of Rivals table. Don't use this table more than once in any particular area.

% Table: Evidence of Rivals
\begin{DndTable}[header={Evidence of Rivals}]{cX}
d4 & Evidence \\
1 & The corpses of one or more Netherdeep creatures killed by the rivals \\
2 & A broken piece of armor, twisted and bent out of shape \\
3 & A taunting message carved into a wall, telling the characters that they have fallen behind \\
4 & A lost or discarded weapon, piece of ammunition, potion bottle, or trinket \\
\end{DndTable}

\subsubsection{Race to the Bottom}

If the characters have a friendly relationship with the rival party, the Apotheon tries to ruin their friendship and put the two groups at odds. Here are some ways in which the Apotheon's interference can manifest:

\begin{itemize}
\item The characters discover the rivals tending their wounds after a skirmish with one or more Netherdeep creatures. Ayo warns the characters of a possible threat or a treasure in the next area, but the Apotheon twists locations around, making it seem as if Ayo has lied to them. (Essentially, the characters are moved to another part of the dungeon without their knowing it. If they head back the way they came, they end up in the area they previously visited and are not displaced.)
\item The Apotheon whispers to Galsariad that the characters know secrets of the Netherdeep that they refuse to share. Galsariad bitterly accuses the characters of knowing more than they've let on, or he convinces his companions that one of the characters should be captured and interrogated.
\item When the characters are close to discovering a Fragment of Suffering, the Apotheon recreates the screams of the rival party to lure the characters to another location where danger awaits.
\end{itemize}

If your goal is to turn the rivals into hated adversaries, consider using one or more of the following tactics:

\begin{itemize}
\item After the characters overcome a challenge to acquire a Fragment of Suffering, they find that the rivals have already claimed the fragment and left behind some kind of calling card.
\item When the characters arrive at a new location, they find the rivals exiting it, being chased by several creatures snapping hungrily at them. Ayo spots the characters, winks maliciously at them, and casts \textit{pass without trace} on her party, causing the monsters to lose sight of their quarry and turn their attention to the newcomers.
\item Galsariad uses his \textit{telekinesis} spell to trigger the partial collapse of a tunnel, blocking the characters' path until they spend 10 minutes clearing the rubble.
\end{itemize}
\section{Netherdeep Regions}

The Netherdeep has three interconnected regions:

Grottoes of Regret (Areas N1--N9). Characters who pass through the rift in Cael Morrow arrive in this region, which is shaped by the Apotheon's sorrows of a life unfulfilled. The Apotheon's memories and regrets manifest here in various ways.

Vents of Fury (Areas N10--N18). This region is dimly lit by fiery vents that belch the Apotheon's wrath. It is filled with violent creatures and shards of the Apotheon's persona.

Chasm of Yearning (Areas N19--N25). These twisting corridors of illusion represent the Apotheon's desire for freedom. The region's creatures and architecture try to disorient intruders.

The three regions are linked by a central hub (area N26) that contains the Heart of Despair, a cocoon of pain and misery that the Apotheon has constructed for himself. Chapter 7 (p. 7) describes the interior of his self-made prison in more detail. Mortal creatures can enter the Heart of Despair only by taking his pain into their souls (see "Fragments of Suffering (p. 6)" earlier in the chapter), thus becoming more like the Apotheon.
\section{Grottoes of Regret}

\begin{quotation}
\em
I lived in service of others, selflessly, for I knew no other way. Now, forgotten, hanging between life and death, I can think of naught but regret for a life haunted and unlived.

\hfill --- Alyxian the Apotheon
\end{quotation}

Rough walls of ruidium-veined stone are interrupted by grand friezes of warriors in their phalanxes, standing proudly against monstrous hordes and towering titans. Wispy figments of memory linger before the carved walls, lost in endless, wistful dreams. Crevasses that turn into winding tunnels connect the grottoes, which the Apotheon uses as vessels for his sorrowful memories.

Walls and floors in the Grottoes of Regret are festooned with foliage as varied in color as coral---often ghostly white, but eerie colors also pierce the softer, lush landscape, interrupted by glowing crimson veins of ruidium.

\subsection{Living Memories}

When they enter certain chambers (areas N2, N3, N4, N5, and N7), the characters are transported to new sites that are not underwater---places where they can interact with characters and locations from Alyxian's past without actually changing the past. The Netherdeep is never fully isolated from any of these quasi-real locations. Ruidium formations, sea grass, and white coral appear on the outskirts of these places, and puddles of water lie on the ground---indications that the Netherdeep is not far away.

Like living dioramas, these scenarios consist of more than just the characters witnessing events of Alyxian's life that he regrets; the characters are experiencing these moments from Alyxian's perspective and are given a chance to change what he did or might have done---not to change the past, but to help the Apotheon confront the choices he made. Some players will figure out this situation quickly, but others might need additional hints. For example, figures dressed in old styles of clothing and using obscure idioms might suggest to the characters that they've been transported not only to another place, but another time.

\begin{DndSidebar}{Emotional Torment}
Various effects in the Netherdeep can twist and amplify the emotions of visitors, causing them to act irrationally. They might be tempted to give in to their vices, face up to their flaws, or suffer in the same ways that Alyxian has. This effect doesn't use the rules for madness; rather, the visitors are being tormented by unresolved suffering, and this distinction will make for a more rewarding outcome if the characters help Alyxian reach catharsis in the final confrontation.
\end{DndSidebar}

\subsection{Roleplaying the Apotheon}

The Apotheon's voice accompanies the characters while they explore the Grottoes of Regret. Within the grottoes, the Apotheon's tone is sorrowful, even reluctant, but helpful. He has spent many years wallowing in these torments and clings to them---they are familiar comforts at the same time that they wound him.

The Apotheon is never more than a disembodied voice in the Grottoes of Regret, but he's more willing to talk to the characters in this region than anywhere else in the Netherdeep. When the characters enter certain areas for the first time, the Apotheon pulls them out of the Netherdeep and drops them into scenes that hearken back to past events that continue to haunt him. His reason for involving the characters in these regretful episodes comes from one or more of these motivations:

\begin{itemize}
\item Alyxian is trying to see some of his feared memories through the eyes of others to gain perspective on his current imprisonment.
\item Instinctively and without malice, Alyxian is trying to elicit empathy from the party.
\item Alyxian is fixated on his own tragedy and can't help but pull others into it.
\end{itemize}

You can choose one motivation for all the areas in the Grottoes of Regret, or you can choose a different option for each area, but do your best to portray the Apotheon sympathetically in this region.

\subsection{Grottoes of Regret Locations}

The Grottoes of Regret encompass areas N1 through N9 on the Netherdeep map.

\subsubsection{N1: Antechamber}

Read or paraphrase the following when the characters first pass through the rift in Cael Morrow and enter the Netherdeep:

\begin{DndReadAloud}
You burst from the planar rift and are plunged into deep, cold water.
\end{DndReadAloud}

Characters who emerge from the rift are immediately subjected to the pressure of the deep (see "Water Pressure" earlier in the chapter). Describe their surroundings as follows:

\begin{DndReadAloud}
You are floating ten feet above the floor of a cavernous pit, the rift at your back. Two enormous statues have settled into the seafloor at awkward angles. Each statue is missing its head, and one of the statues has an outstretched hand that looks like it might grab anything that gets too close to it. Covering the statues and clinging to the walls are crystalline tendrils that shed dim, eerie red light. A ten-foot-high tunnel at floor level leads out of this sunken chamber, which has patches of sea grass growing everywhere.
\end{DndReadAloud}

Swimming in the dark water 20 feet above the characters (30 feet above the floor) are two \textbf{devourers} and a \textbf{slithering bloodfin} (see appendix A (p. 8) for their stat blocks) that act as sentries. Characters who find themselves outmatched by the monsters or unable to cope with the water pressure can retreat through the open rift and return to Cael Morrow.

\paragraph{Statues}

The ruidium-covered statues are depictions of Alyxian the Apotheon as a noble warrior in his prime, but the missing heads make identification difficult. The heads are nowhere to be found, symbolizing Alyxian's loss of self.

\subsubsection{N2: Child's Lament}

This water-filled grotto has tunnels to the south, northeast, and northwest. These short passages lead to areas N1, N3, and N4, respectively.

The first time the characters enter this area, read:

\begin{DndReadAloud}
The water shimmers as you are transported out of the undersea gloom and into a child's room. Simple comforts are scattered around the room's perimeter: a teddy bear overgrown with ruidium, a cradle lined with ruidium spikes, and a rocking horse with tendrils of ruidium crawling up one side of it. Simple toys are scattered on the floor, with more of them on small wooden shelves. All of them are laced with ruidium.
Sunlight shines into the room through two small, high windows. You hear muffled voices coming from outside and an occasional dull thud.
\end{DndReadAloud}

The characters are transported to Alyxian's childhood home, a modest wooden dwelling with three rooms: a 20-foot-square room containing a kitchen and a living area, with doors leading to two 10-foot-square bedrooms. The characters appear in Alyxian's bedroom. A door off the kitchen leads outside.

% Image placeholder: Alyxian's Childhood Bedroom

\paragraph{Alyxian's Toys}

A character who examines the rocking horse notices the name "Alyxian" carved on its flank in Common. Characters who want to keep one or more of Alyxian's toys can do so, and those toys remain with the characters after this vision ends. The first time a character picks up one of these toys, the Apotheon whispers, "Such a thing no longer holds a special place in my heart, but it might be worth something to you." These toys prove useful in area N6. Any character who handles a toy without gloves or similar protective gear might be subject to ruidium corruption, at your discretion (see "Ruidium" in the introduction (p. 0)).

\paragraph{Bedroom Windows}

Alyxian's bedroom has a 10-foot-high ceiling and two windows. Each window is a 2-foot-square opening whose bottom is 7 feet above the floor. A Medium character can peer out a window easily enough by standing on the back of the rocking horse.

\paragraph{Outdoor Scene}

Any character who looks out a window or exits the cottage sees two humans being attacked by a band of six people armed with farm implements. The two folks being accosted are representations of Alyxian's parents. His father has a kindly face and a bushy brown beard. His mother has a worry-lined, scarred face. Both have olive skin and wear simple, dirt-stained garments made of a single piece of cloth. Their attackers are shadowy, faceless facsimiles of people draped in worn-out clothes. Red eyes burn from beneath their wide-brimmed hats. These shadowy assailants use the \textbf{berserker} stat block, but their weapon attacks deal psychic damage. Their vague forms are due to the fact that Alyxian's memory of them has faded.

As soon as one or more characters see what's going on outside, Alyxian's voice whispers to them:

\begin{DndReadAloud}
"You have found me, my friends. Witness my earliest memory. The pain of a child born under the ill-fated sign of Ruidus. We were turned on by friends and kinfolk as war raged around us. Many people suffered from my presence, my parents most of all. My story begins with them. Their story dies with me."
With a loud crack, one of the shadowy figures strikes Alyxian's father, sending him to the ground.
\end{DndReadAloud}

Characters who exit the cottage not only witness the unfolding scene but also attract the attention of the assailants. Four of them break off to attack the characters while the remaining two continue to strike Alyxian's parents on the ground. These two assailants turn to fight the characters only when the four who attacked the characters are dead.

If the characters confront the faceless assailants, have the players roll initiative for their characters, Make one initiative roll for all the assailants.

\paragraph{Placating the Apotheon's Regret}

To placate this regret of Alyxian's, the characters must try to rescue his parents. If the assailants haven't all been defeated by the start of their fourth turn, Alyxian's father dies from his wounds, and his mother is left for dead. Administering magical healing to the father delays his death at the hands of the assailants by 1 round.

As they attack, the assailants hiss justifications for why young Alyxian and his parents should die. Examples of what they might say:

\begin{itemize}
\item "You have brought this ruin upon us!"
\item "Monsters birth monsters!"
\item "No good ever comes from sheltering a child of Ruidus!"
\end{itemize}

If Alyxian's parents are saved, they speak to the characters as if they were speaking to Alyxian:

\begin{DndReadAloud}
"Get back inside, Alyx. Everything's all right. They just don't understand."
\end{DndReadAloud}

If one or more characters offer words of comfort to Alyxian's parents after rescuing them from the assailants, this interaction has a positive effect on the final battle against Alyxian (see "Emotional Healing" in chapter 7 (p. 7)).

\paragraph{Ending the Vision}

Once this episode has played out, the characters are transported back to the underwater grotto along with anything they are wearing or carrying, including any toys they gathered from Alyxian's bedroom. Once they are back in the water-filled chamber, Alyxian whispers the following to them in their minds:

\begin{DndReadAloud}
"What I understood from a young age is this: because of me, they would always suffer."
\end{DndReadAloud}

\paragraph{Subsequent Visits}

The vision involving Alyxian's childhood home is not repeated if the characters leave this area and return later. On future visits, this grotto is dark and empty except for a few patches of sea grass.

\subsubsection{N3: Unforgotten Fallacy}

This water-filled grotto is connected to area N2 by a short tunnel. A longer tunnel in the north wall bends west and passes under area N4a on its way to area N5.

The first time the characters enter this area, read:

\begin{DndReadAloud}
The walls of this chamber shimmer as you enter another memory. Fragmentary recollections fill your mind: you recall Alyxian as a daring teenage boy who wanted to take part in a village hunt. The vermilion moon was shining full that night, and you remember Alyxian telling himself that he was strong enough to overcome it. You remember him donning a cloak and joining the hunt in secret.
Your awareness shifts to a clearing in a moonlit forest. Its beauty is belied by the thick scent of blood and the sound of frantic shouts.
\end{DndReadAloud}

The characters are whisked away to a forest where a hunt has just gone terribly wrong. The misfortune that has followed Alyxian his entire life has now doomed his hunting party. One of the hunt captains, a neutral \textbf{drow elite warrior} named Saqiri Yestrana, was mauled by two \textbf{owlbears} in the red moonlight, and another hunter caught a glimpse of the young Alyxian before he parted company with Saqiri and darted into the woods.

When the characters enter the scene, the owlbears are escaping, scattering a herd of deer in the process, and Saqiri is bleeding out. As Saqiri coughs up blood, the characters hear the following shouts of panic from the hunters:

\begin{itemize}
\item "Those beasts were going to see us through the winter! We'll be short now!"
\item "Alyxian has run off!"
\item "Saqiri's not going to make it! She needs tending!"
\end{itemize}

Alyxian whispers the following in the characters' minds after they take in the scene:

\begin{DndReadAloud}
"I learned the price of my selfishness early. I just wanted to---I just wanted to hunt among them, to be with them." He sighs and growls ruefully. "It doesn't matter. Saqiri died that night, and I will carry that regret until I join her."
\end{DndReadAloud}

\paragraph{Placating the Apotheon's Regret}

To placate Alyxian's regret, the characters must hunt for meat, heal Saqiri, or both. Accomplishing one or both tasks has a positive effect on the final battle against Alyxian (see "Emotional Healing" in chapter 7 (p. 7)).

\paragraph{Hunting for Meat}

One or more characters can hunt the owlbears and the deer to bring back meat to the hunters' camp. At least one character must succeed on a DC 16 Wisdom (Survival) check to track the owlbears through the forest. As they explore the forest, the characters see beds of long, flowing kelp emerging from among the trees---a hint that the Netherdeep isn't far away. If one or more characters succeed on the check, they encounter the two \textbf{owlbears}. When an owlbear dies, its body splits open and a \textbf{swarm of sorrowfish} (see appendix A (p. 8)) bursts out and attacks. These sorrowfish swim through the air as if it were water. Once the swarms are defeated, the characters can take the owlbear meat---enough to sustain the village---back to the hunters' camp. If no characters succeed on the check, they don't encounter the owlbears and find only the carcasses of sickly deer that couldn't keep up with the rest of the herd. If these carcasses are returned to the camp, they provide some edible meat---enough to placate the Apotheon's regret.

\paragraph{Healing Saqiri}

Four hunters (\textbf{scouts}) guard Saqiri. These hunters have shadowy forms and hazy faces, because Alyxian's memory of them has faded somewhat.

The hunters regard the characters with the same contempt they hold for Alyxian, shouting to them, "Be off with you! You're a curse on all of us!" A character can get to Saqiri's side by succeeding on a DC 14 Dexterity (Stealth) check to slip past the guards. A character can also approach the guards and convince them not to attack with a successful DC 14 Charisma (Persuasion) check. The guards attack any character who fails either check. Characters who try to force their way past the guards are also attacked.

Saqiri is \textit{unconscious} and dying from her wounds. If healing restores her to 1 hit point or more, she struggles to sit up and speaks the following words to the character closest to her, as if that character were Alyxian: "Go on, boy. Take one of the carcasses to your ma. Tell her you did a good job tonight. I don't know if a curse truly follows you---but people believe it does, and that's enough for them." The characters then hear Alyxian whisper the following words: "I have watched Saqiri's death a thousand times. It never stops tearing my soul apart---but now I know what she would have said had she lived." More quietly, he adds, "It is not what I expected after so long."

\paragraph{Ending the Vision}

Once this scene has played out, the forest fades away and the characters are returned to the underwater grotto. If they accomplished one or both tasks in the vision, a small, unlocked chest appears on the floor nearby. It contains the following rewards based on what tasks the characters completed:

\begin{itemize}
\item If the characters healed Saqiri, the chest contains fifteen tiny rubies (100 gp each).
\item If the characters brought owlbear meat back to the camp, the chest contains two 6-inch-tall, solid gold owlbear figurines (750 gp each), one on all fours and the other reared up on its hind legs. If they returned with deer meat for the camp instead, the chest contains a 3-inch-tall figurine of a deer carved out of translucent, grayish-blue chalcedony (150 gp).
\end{itemize}

\paragraph{Subsequent Visits}

The vision of Alyxian's first hunt is not repeated if the characters leave this area and later return. On future visits, this grotto is dark and empty except for a few patches of sea grass.

\subsubsection{N4: Haunted Ashes}

This water-filled grotto has short tunnels leading to areas N2 and N5. A third tunnel leading to area N4a is hidden behind a secret door.

The first time the characters enter this area, read:

\begin{DndReadAloud}
The water in this chamber feels as slick as oil. A shiver moves down your spine as halos of pale red light surround you and your companions. You all sink to the bottom of the chamber and land softly as your feet touch the floor.
You are no longer underwater but standing on the edge of a small village surrounded by hills. A melee has broken out about a hundred yards away in the middle of the village. Shadowy soldiers are pillaging the settlement while villagers mount a futile effort to rebuff them. Before you can act, one of these soldiers raises his sword above his head and howls, "You ought to be kissing the ground we walk upon, you spineless cowards!" In a flash, he stabs a villager, whose death wail rings in your ears.
\end{DndReadAloud}

The characters have been transported to a sacked village in the distant past, during the time of the Calamity.

\paragraph{Red Halos}

The halo surrounding each of the characters sheds bright light in a 10-foot radius and dim light for an additional 10 feet. These halos vanish when the characters leave this memory and are returned to the Netherdeep.

A character encircled by a red halo leaves blotches of ruidium wherever their feet touch the ground. These ruidium deposits slowly send out tendrils across the ground in the character's wake. In this vision, any successful attack a character makes with a melee or ranged weapon deals an extra 11 (2d10) psychic damage to the target.

\paragraph{Raiders}

A group of soldiers---allies of Alyxian in the battle against Torog---have decided that since they saved this village, it is theirs to pillage. Four \textbf{sword wraith warriors} led by a \textbf{sword wraith commander} (see appendix A (p. 8) for their stat blocks) take the shadowy shape of Alyxian's crass allies. They are bullying a group of ten \textbf{commoners} in the middle of the village. As the characters approach, the commander, a neutral evil human in his mid-twenties named Kalagothe, looks to them as if they were Alyxian and roars, "Commander Alyxian, these people refuse to pay us tribute!"

Have the players roll initiative for their characters, then make one initiative roll for Kalagothe; the remaining sword wraiths act immediately after him in the initiative order. Roll initiative once for all the commoners, but don't waste time devising tactics for them; they cower or retreat from the soldiers on their turn.

A character can use an action to try to convince the soldiers to stand down, doing so with a successful DC 27 Charisma (Intimidation or Persuasion) check. The DC drops to 17 if Kalagothe is dead and only the \textbf{sword wraith warriors} need to be dealt with. Only three such attempts can be made, after which the sword wraiths are immune to such coercion.

Unless the soldiers are slain or forced to stand down, they capture and murder the commoners.

\paragraph{Placating the Apotheon's Regret}

If the characters rescue one or more of the villagers, the Apotheon's voice tentatively whispers, "There are bright spots amid the ruin, sometimes. Beauty. Kindness." This interaction has a positive effect on the final battle against Alyxian (see "Emotional Healing" in chapter 7 (p. 7)).

\paragraph{Ending the Vision}

If the characters rescue at least five of the ten commoners, the Apotheon whispers to them, "You handled that better than I did. I wish I could have seen their faces happy with relief that they were saved, rather than fraught with sorrow at their losses." After Alyxian speaks these words, the vision fades and the characters are returned to the underwater grotto. As the characters reappear in the grotto, a secret door in the north wall opens (see "Secret Doors" earlier in the chapter). Beyond this door lies a short tunnel leading to a small chamber (area N4a).

\paragraph{Subsequent Visits}

The vision of the village is not repeated if the characters leave this area and later return. On future visits, this grotto is dark and empty except for a few patches of sea grass growing near the walls.

\subsubsection{N4a: Fragment of Despondence}

\begin{DndReadAloud}
In the middle of this chamber stands a marble pedestal, above which floats a mote of opalescent light about the size of a fig. A bolt of crimson energy occasionally crackles across its glowing surface. Resting on the pedestal below this light is a tiny gray doll in the shape of some kind of four-legged animal.
\end{DndReadAloud}

When they enter this area, the characters hear the Apotheon's bitter words in their minds:

\begin{DndReadAloud}
"Yes. You let go of hope because you know nothing will ever change."
\end{DndReadAloud}

\paragraph{Doll}

The doll on the pedestal looks like a moorbounder. It has sentimental value to Alyxian. If a character picks it up, everyone in the room hears the Apotheon whisper, "Maeska, my old friend." The moorbounder doll might prove useful to the characters in area N6.

\paragraph{Fragment of Despondence}

The mote of light above the pedestal is the Fragment of Despondence (see "Fragments of Suffering (p. 6)" earlier in the chapter). When a character touches the mote, it disappears and becomes part of the character's soul. When this happens, all the characters receive a fleeting vision (see the Apotheon Lore table earlier in the chapter).

The character who is carrying the Fragment of Despondence gains the following benefit and drawback, and is aware of both:

\begin{DndSidebar}{}
\begin{description}
\item[Benefit.] You are immune to the \textit{charmed} condition.
\item[Drawback.] You can't take the Help action.
\end{description}
\end{DndSidebar}

Give the Fragment of Despondence card (see appendix D (p. 11)) to the player whose character is carrying the fragment.

\paragraph{Tunnel Exits}

Two short tunnels lead away from this chamber, one toward area N4 and the other toward area N6. Each tunnel ends at a secret door (see "Secret Doors" earlier in the chapter). A creature leaving this location automatically finds either secret door.

\subsubsection{N5: Alyxian's Refuge}

This water-filled grotto has short tunnels leading to areas N4 and N6. Between them, a longer tunnel travels east, passing under area N4a on its way to area N3. A fourth tunnel leading to area N5a is hidden behind a secret door.

The first time the characters enter this area, read:

\begin{DndReadAloud}
As your consciousness melds with the Apotheon's, you remember a tale from the Calamity when Alyxian took shelter in the home of three women. For one night, the women fed Alyxian and gave him a bed to sleep in. Now, you stand outside their home, near which are planted three unsightly gravestones made of jagged red crystal.
In your mind, you hear Alyxian say, "They welcomed me into their home. Why can't I remember their names or faces? Please, help me remember."
\end{DndReadAloud}

The characters have been transported to another place Alyxian remembers from his troubled past. When he was a young man fated to be caught up in the Calamity, Alyxian stayed in the home of a family for one evening, and ruin followed him to their doorstep. The two women and their teenage daughter were killed when the village was sacked the day after Alyxian's departure. He learned about their deaths only several months later when another soldier told him about the village's destruction. Alyxian is desperate to remember the names of the three victims, which is why the characters have been brought here.

\paragraph{Gravestones}

The gravestones near the ruined house are made of ruidium and bear no inscriptions. No graves lie beneath them, so no corpses are available to examine or question by using magic.

\paragraph{Ruined House}

If the characters explore the ruins of the house, read:

\begin{DndReadAloud}
Corpses and smashed furniture are strewn throughout the ground floor of the house, which contains a living room and a kitchen plus a small dining room. A damaged staircase leads up to a second-floor bedroom.
\end{DndReadAloud}

A thorough search of the ground floor reveals ten corpses. They belong to the three women, five grimlocks, and two human warriors clad in bloodstained chain mail. The warriors were apparently slain trying to protect the family from the grimlocks. All these creatures look as if they died several days a day ago.

\paragraph{Forgotten Names}

To satisfy Alyxian, the characters must learn the names of the three women: \textbf{Marisa}, \textbf{Celeste}, and \textbf{Meri}. \textbf{Marisa} was the eldest, \textbf{Celeste} was her wife, and \textbf{Meri} was their teenage daughter. Their family name is Zenthas. \textbf{Marisa}'s name can be discovered in the kitchen, \textbf{Celeste}'s in the living room, and Mari's in the bedroom, as described below:

\begin{description}
\item[Kitchen.] \textbf{Marisa} was making a \textit{loaf of bread} when the attack came. A character who spends at least 1 minute searching the kitchen can make a DC 12 Intelligence (Investigation) check. On a successful check, the character finds a folded note next to a torn bag of cinnamon spattered with blood and tied shut with a green ribbon. The note reads: "\textbf{Marisa}---for that little cake you like so much."
\end{description}

\paragraph{Living Room}

\textbf{Celeste} used the living room as a space to indulge her hobby: incense-making. Herbs, bottles, and incense sticks lie scattered across a worktable and on the floor in one corner of the room. Any character who spends at least 1 minute searching the living room can make a DC 12 Intelligence (Investigation) check. On a successful check, the character finds a book titled Scents for the Seasons lying under a chair. Written on the book's inside cover is the name "\textbf{Celeste} Zenthas."

\begin{description}
\item[Bedroom.] \textbf{Meri}, on the cusp of adulthood, was training to be a soldier. Any character who spends at least 1 minute searching the bedroom can make a DC 12 Intelligence (Investigation) check. On a successful check, the character finds the name "\textbf{Meri}" carved into the empty sheath of a shortsword.
\end{description}

\paragraph{Ghosts of the Remembered}

When the characters learn all three names, three \textbf{ghosts} rise from the women's corpses and attack the intruders until at least one of the ghosts' names is spoken aloud. The ghosts resemble the three women, but they are not restless spirits. Rather, they are manifestations of Alyxian's rage and regret. The characters can destroy these ghosts by reducing each of them to 0 hit points.

If a character addresses a ghost by the name of the woman it resembles, that ghost's attitude toward the party changes from hostile to indifferent, and the ghost addresses each of the characters as if they were Alyxian. If a character apologizes to a ghost on Alyxian's behalf for not being around in its time of need, the ghost's attitude changes to friendly.

The following sections describe the three women and their behavior as ghosts:

\begin{description}
\item[Marisa.] \textbf{Marisa}'s ghost smells faintly of sugar and cinnamon. On her first turn in combat, she wails, "Our home, defiled!" If a character addresses \textbf{Marisa}'s ghost by name and apologizes to her on Alyxian's behalf, the ghost says, "Alyxian! When they came for us, I feared they would kill you, too. It is good to hear your voice. Be well, and do not mourn. No one can harm us anymore." After speaking these words, the ghost fades away.
\end{description}

\paragraph{Celeste}

\textbf{Celeste}'s ghost smells faintly of incense. On her first turn in combat, she snarls, "We sheltered him! We died for him!" If a character addresses \textbf{Celeste}'s ghost by name and apologizes to her on Alyxian's behalf, the ghost's visage softens, and she says, "I would die for you, moon-cursed and moon-blessed. A thousand times, if need be." After speaking these words, the ghost fades away.

\begin{description}
\item[Meri.] \textbf{Meri} has wavy, shoulder-length hair, strong arms, and sharp eyes. The resemblance to \textbf{Marisa} is unmistakable. She doesn't speak unless she is addressed by name. If a character does so and then apologizes to \textbf{Meri} on behalf of Alyxian, she says, "I died a soldier, like my mother. What happened was neither our fault nor yours. Be well, Alyxian." After speaking these words, the ghost fades away.
\end{description}

% Image placeholder: Three ghosts of women from Alyxian's past haunt a ruined house

\paragraph{Placating the Apotheon's Regret}

Alyxian is placated if one or more ghosts are laid to rest. After speaking the ghosts' names aloud, he adds, "The Crawling King's horde destroyed them, and I was not there to help. Yet they forgave me. Was that truly all it took?"

Each ghost that vanishes on its own instead of being defeated in combat has a positive effect on the final battle against Alyxian (see "Emotional Healing" in chapter 7 (p. 7)).

\paragraph{Ending the Vision}

When all three ghosts are destroyed or vanish on their own, the characters are returned to the underwater grotto. A secret door opens in the northwest wall, revealing a narrow tunnel that leads to area N5a (see "Secret Doors" earlier in the chapter).

\paragraph{Subsequent Visits}

The vision of the ruined house is not repeated if the characters leave this area and later return. On future visits, this grotto is empty except for a few patches of sea grass.

\subsubsection{N5a: Fragment of Attachment}

\begin{DndReadAloud}
Rising from the floor in the center of this grotto is a marble pedestal above which floats a wispy mote of light.
\end{DndReadAloud}

The first time one or more characters enter this area, all the characters hear Alyxian whisper the following words in their minds:

\begin{DndReadAloud}
"These regrets are all I have. The only thing that remains is to take the path I walked to get here."
\end{DndReadAloud}

\paragraph{Fragment of Attachment}

The mote of light above the pedestal is the Fragment of Attachment (see "Fragments of Suffering (p. 6)" earlier in the chapter). When a character touches the mote, it disappears and becomes part of the character's soul. When this happens, all the characters receive a fleeting vision (see the Apotheon Lore table earlier in the chapter).

The character who is carrying the Fragment of Attachment gains the following benefit and drawback, and is aware of both:

\begin{DndSidebar}{}
\begin{description}
\item[Benefit.] You can't be \textit{frightened} while within 10 feet of an ally. If you're already \textit{frightened} and move within 10 feet of an ally, the \textit{frightened} condition ends on you.
\item[Drawback.] You have disadvantage on Wisdom and Charisma saving throws while you aren't within 10 feet of an ally.
\end{description}
\end{DndSidebar}

Give the Fragment of Attachment card (see appendix D (p. 11)) to the player whose character is carrying the fragment.

\paragraph{Tunnel Exits}

Two tunnels lead from this chamber: one travels east to area N5, and the other spirals downward, passing under the eastern tunnel before opening in the wall of a wider tunnel that connects areas N10 and N11. Each tunnel ends at a secret door (see "Secret Doors" earlier in the chapter). A creature leaving this location automatically finds either secret door.

\subsubsection{N6: Loss of Innocence}

\begin{DndReadAloud}
This chamber is filled with hauntingly beautiful seaweed and ghostly white coral. A stone altar in the middle of the room is decorated with carvings of cyclopean skulls and grasping hands.
A barrier made of red energy covers a doorway in the north wall, beyond which you can make out another underwater chamber. Carved above the doorway in Common is the following inscription:
\textit{Apotheon, Apotheon}.
\textit{Leave your childhood yearnings behind.}
\textit{It is time to don your armor and go to war.}
\end{DndReadAloud}

\paragraph{Altar}

A character who examines the altar and makes a successful DC 13 Intelligence (Religion) check recognizes its imagery as symbolic of the gods Gruumsh and Torog. Placing certain objects on this altar can help the characters bypass the door of red energy (see below).

\paragraph{Door of Red Energy}

This magical barrier is impervious to damage but can be negated with a successful \textit{dispel magic} (DC 19). Any creature that touches the door must make a DC 15 Constitution saving throw. On a failed saving throw, the creature takes 35 (10d6) force damage and is pushed 10 feet away from the door. On a successful save, the creature takes half as much damage and isn't pushed.

If the characters can't dispel the door, they can place a sentimental item from Alyxian's childhood on the altar ("Leave your childhood yearnings behind"), which causes the item to disappear and dispels the door for 8 hours. Items that qualify include a toy taken from Alyxian's childhood bedroom (area N2) and the moorbounder doll found in area N4a. A character can substitute a trinket or toy from their own childhood, with the same effect. Any item that disappears from the altar can't be retrieved by any means short of a \textit{wish} spell.

In addition, a character who dons a suit of armor in this chamber or in area N7 can pass through the doorway ("It is time to don your armor and go to war"). A character who entered either chamber already wearing armor must first take off the armor and then put it on again before they can pass safely through the doorway.

% Image placeholder: Sword wraiths haunt a vision of the ancient past, with the Netherdeep never far away

\subsubsection{N7: Battle at the Betrayers' Rise}

When the characters enter this area for the first time, describe it as follows:

\begin{DndReadAloud}
The roof of this underwater sepulcher has three naturally formed domes, each one reaching its apex about thirty feet above the floor. Veins of ruidium cling to the walls, forming a lattice that casts red light on the clumps of sea grass growing along the room's perimeter.
\end{DndReadAloud}

The characters have only a few seconds to observe their surroundings before they are whisked away to another time and place:

\begin{DndReadAloud}
The underwater gloom gives way to a scene showing a parched battlefield under a bleak, ashen sky. All around you, shadowy figures clash swords, and the air is filled with battle cries and the drums of war.
Familiar-looking black towers in the distance provide clues to where and when you are: at the gate of the Betrayers' Rise during one of the fiercest battles of the Calamity. In this conflict, Alyxian fought the Crawling King's horde and scored a hard-won victory. In this version of the scene, it seems, the battle is yours to win or lose.
\end{DndReadAloud}

The characters are experiencing the Apotheon's memories of a terrifying battle. Most of the battlefield is occupied by clashing armies---Alyxian's allies versus the monstrous hordes of Torog the Crawling King. These creatures have shadowy forms and hazy faces, reflecting the fact that Alyxian's memory of them has faded somewhat, and most of them ignore the characters.

\paragraph{Battle Rules}

As the battle rages around them, the characters are about to be faced with three waves of foes (described below) and have little choice but to defend themselves. Have the players designate one character to act as the party's leader for the duration of the battle. (It might be the character with the highest Charisma score or a character who customarily assumes a leadership role in the group.) This character remains the designated leader for the entire battle, even if they are killed or unable to act. Once the leader is determined, have the players roll initiative for their characters while you roll initiative for the hostile creatures in each wave as they enter the fray. The characters don't reroll initiative when the second and third waves appear. At no point during this battle can the characters take a short or long rest.

The characters can move anywhere they want on the battlefield, but a character who moves farther than 30 feet from the party leader or starts their turn in such a location takes 16 (3d10) psychic damage from the chaos around them, even if the leader is dead or \textit{incapacitated}. A character can take this damage no more than once per turn.

\paragraph{First Wave}

This wave consists of one \textbf{sword wraith warrior} (see appendix A (p. 8)) per character, up to a maximum of five. These grim knights represent the evil soldiers of the Crawling King. As the battle is joined, Alyxian chimes in:

\begin{DndReadAloud}
"After so long, their faces have faded into the haze of the past. I try to remember them---not my enemies, but the soldiers who fought alongside me. The least I could do was wade into the heart of the battle, where the suffering was already the worst, and take it all into me."
\end{DndReadAloud}

Alyxian's words should make it clear that the only way to put this battle to rest is to conquer wave after wave of foes.

\paragraph{Second Wave}

After the first wave is defeated, the characters have 2 rounds to heal and prepare for the second wave: two \textbf{moorbounders} (see appendix A (p. 8)) accompanied by five \textbf{sword wraith warriors}. Before the characters can do anything to prevent it, the sword wraiths merge into one creature---a \textbf{slithering bloodfin} (see appendix A (p. 8)) that swims through the air as easily as it normally swims through water. The howling faces of the warriors are visible on the bloodfin's flesh. The bloodfin is not what the Apotheon sought to conjure in this moment; rather, it represents Alyxian's inability to keep the Netherdeep from intruding upon his memories.

When the second wave is defeated, Alyxian's allies roar in unison, and Alyxian chimes in:

\begin{DndReadAloud}
"You cut them down, endlessly, a scythe in the hands of a reaper. It doesn't matter how many you kill. You water the lands with blood, and your allies die anyway."
\end{DndReadAloud}

\paragraph{Third Wave}

After the second wave is defeated, the characters have 1 round to heal and prepare for the final wave: a \textbf{sword wraith commander} and four \textbf{sword wraith warriors} on the back of a \textbf{horizonback tortoise} (see appendix A (p. 8) for these creatures' stat blocks). Alyxian shares his thoughts as the tortoise lumbers toward the party:

\begin{DndReadAloud}
"To break the spine of the Crawling King's horde, I knew I had to slay both the commander and the tortoise. I don't remember which one I killed first."
\end{DndReadAloud}

When the commander drops to 0 hit points, it lets out a terrible scream, and any remaining sword wraith warriors flee on their next turn. When the tortoise drops to 0 hit points, it falls \textit{prone} and lets out one last shudder before expiring. The characters are victorious when both the commander and the tortoise are slain.

\paragraph{Placating the Apotheon's Regret}

When all three waves of foes are defeated, Alyxian's allies cheer as the enemy is routed. A \textbf{veteran} covered in blood limps toward the characters and says, "You won the day, brave Alyxian. My wounds burn, but if you command it, I will follow you to the end." Alyxian whispers an answer that only the characters can hear: "It never ends." The wounded soldier, named Talmyth, has 1 hit point. If a character heals him, that act has a positive effect on the final battle against Alyxian (see "Emotional Healing" in chapter 7 (p. 7)).

\paragraph{Secret Doors}

Once the battle has ended and the scene with Talmyth has played out, the characters are returned to the underwater grotto. At the same time, two secret doors open in the north wall (see "Secret Doors" earlier in the chapter). Behind one secret door is a narrow, dimly lit tunnel that connects to area N7a; behind the other is area N8.

\paragraph{Subsequent Visits}

The vision of the battle at the Betrayers' Rise is not repeated if the characters leave this area and later return. On future visits, this grotto is dark and empty except for the patches of sea grass growing near the walls.

\subsubsection{N7a: Fragment of Pity}

\begin{DndReadAloud}
Rising from the floor in the center of this grotto is a marble pedestal, above which floats a wispy mote of light. At the foot of the pedestal, half buried in the sandy floor, is a small wooden chest.
\end{DndReadAloud}

\paragraph{Fragment of Pity}

The mote of light above the pedestal is the Fragment of Pity (see "Fragments of Suffering (p. 6)" earlier in the chapter). When a character touches the mote, it disappears and becomes part of the character's soul. When this happens, all the characters receive a fleeting vision (see the Apotheon Lore table earlier in the chapter).

The character who is carrying the Fragment of Pity gains the following benefit and drawback, and is aware of both:

\begin{DndSidebar}{}
\begin{description}
\item[Benefit.] Each time you spend a Hit Die to regain hit points, you regain additional hit points equal to your proficiency bonus.
\item[Drawback.] You have disadvantage on death saving throws.
\end{description}
\end{DndSidebar}

Give the Fragment of Pity card (see appendix D (p. 11)) to the player whose character is carrying the fragment.

\paragraph{Treasure}

The wooden chest is closed and unlocked. It contains four stoppered \textit{potions of healing (greater)} in red crystal vials and a fully charged \textit{wand of secrets}. If the rivals reach this area before the characters do, Dermot takes the magic items and doesn't bother closing the chest after emptying it.

\paragraph{Tunnel Exits}

Two tunnels, one to the north and one to the south, lead away from this chamber. The south tunnel heads to area N7. The north tunnel is longer, descends almost vertically to area N16, and is dimly lit by red light emanating from that location. Each tunnel ends at a secret door (see "Secret Doors" earlier in the chapter). A creature leaving this location automatically finds either secret door.

\subsubsection{N8: Ruiner's Spear}

When the characters peer into the grotto, it appears to be devoid of inhabitants or noteworthy features. When they enter the chamber, the secret door to area N7 closes behind them unless it is being held or propped open. At the same time, a dark form rises from the floor:

\begin{DndReadAloud}
A shadow rises from the sandy floor and assumes a giant form. Its single, glowing red eye remains fixed on you as it rises to its full height of one hundred feet. An enormous spectral spear is gripped in its hands.
\end{DndReadAloud}

This chamber and its challenge represent the last great sacrifice of the Apotheon. The spectral colossus is an image of Gruumsh the Ruiner. This image can't be fought or harmed. Read the following as the Betrayer God brings his spear down upon the characters:

\begin{DndReadAloud}
The water swirls around the spear as it plunges down toward you.
\end{DndReadAloud}

Every character in the grotto and any character who enters the grotto while the manifestation of the Betrayer God is present falls \textit{prone} on the floor and has their speed reduced to 0 by the force of the Betrayer God's spear, as though the character were being impaled by it. No force, magical or otherwise, can move a character from the space in which they fall \textit{prone}.

One you know which characters are \textit{prone}, have the players roll initiative. Any character who starts their turn in the grotto must succeed on a saving throw. Have each player describe how their character tries to resist the pressure of the Betrayer God's spear on their turn, then use the ability that most closely matches that description for the saving throw:

\begin{description}
\item[Strength.] The character tries to use their physical prowess to push back against the god's attack.
\item[Dexterity.] The character contorts their body to avoid the full force of the god's attack.
\item[Constitution.] The character braces themself against the pain of the attack, trying to endure it.
\item[Intelligence.] The character tries to refute the attack as unreal.
\item[Wisdom.] The character tries to combat the psychic power of the Netherdeep by concentrating on their own joyful memories.
\item[Charisma.] The character tries to use sheer force of will to counteract the pressure.
\end{description}

The saving throw DC is 15 to start and increases by 2 each time the initiative count reaches 0, making it harder for characters to resist the spear as time goes on. On a failed saving throw, a character takes 14 (4d6) psychic damage, and their mind is filled with the voices of lost loved ones telling them to stop resisting and surrender to the attack instead. On a successful saving throw, the character takes half as much damage and hears Alyxian's voice in their mind, pondering why he gave his life to save others when they wouldn't have done the same thing for him. The effect ends on any character who either is reduced to 0 hit points by the damage or succeeds on the saving throw three times. Such a character no longer feels the spear or sees the Betrayer God pinning them down with his mighty spear (though a character reduced to 0 hit points is dying and either needs to stabilize or be healed to avoid death).

Alyxian whispers in the mind of the first character to succeed on three saves, saying, "Perhaps you will be remembered. Now, what about your companions?" That character can take the Help action to speak words of encouragement to other characters still trapped by the spear, granting them advantage on the next saving throw they make to resist it.

When all the characters who were pinned by the Betrayer God's spear end the effect on themselves, the challenge is over, and the vision ends. Read:

\begin{DndReadAloud}
You have felt the Betrayer God's fury, as Alyxian did. The towering image of the one-eyed god is gone, and the chamber stands empty once more. Part of the north wall crumbles away, creating a jagged, open doorway between this grotto and another one.
\end{DndReadAloud}

The doorway leads to area N9.

\subsubsection{N9: Hall of the Portal}

The characters can enter this chamber by stepping through the open doorway in the south wall after surviving the attack in area N8. Hidden in the north wall is a secret door that leads to area N26 (see "Secret Doors" earlier in the chapter).

When the characters enter, read:

\begin{DndReadAloud}
An oval portal with glowing edges appears in the middle of the room, floating between the hands of a giant statue depicting Alyxian in armor. Suspended vertically in the water, the portal is ten feet tall and five feet wide. Through it, you can see the giant statues that occupy the first chamber of the Netherdeep that you entered.
"You have endured," whispers Alyxian's voice. "Take this opportunity to recover and prepare for what yet lies in wait."
\end{DndReadAloud}

The portal in this chamber appears only if the characters enter from area N8. The portal links to an identical one that appears simultaneously in area N1, permanently connecting these two areas. Looking through the portal is like looking through an open doorway into the other chamber. Any creature or object that passes through a portal emerges in the other location. Neither portal can be dispelled, and the portals last until Alyxian leaves the Netherdeep.
\section{Vents of Fury}

\begin{quotation}
\em
I sacrificed everything for them! And now they tear at my flesh to slake their greed, with no thought for their savior. There is only one way to repay their ill gratitude.

\hfill --- Alyxian the Apotheon
\end{quotation}

The lowest chambers of the Netherdeep are called the Vents of Fury because of the steam and thermal fissures in the floor. These vents are manifestations of Alyxian's wrath, which also coalesces into monsters that inhabit this region. These creatures are marred with ugly ridges and streaks of ruidium that resemble festering scabs.

\subsection{Roleplaying the Apotheon}

In the Vents of Fury, the Apotheon's presence is a cruel and malicious hindrance. He doesn't seek understanding; he wants the characters to suffer as he has. Bitterness flavors every word he speaks.

The first time a character speaks out against the Apotheon while in the Vents of Fury, the Apotheon's voice audibly booms out, "You will not speak of me this way! Not after all we have shared!" This voice can be heard out to a range of 100 feet underwater.

The second time a character speaks out against the Apotheon in the Vents of Fury, the water around the character turns boiling hot, dealing 21 (6d6) fire damage to the character, and the Apotheon's voice roars, "Fool! You reveal the truth of your greedy, mortal heart."

If a character speaks out against the Apotheon a third time in the Vents of Fury, the Apotheon bellows, "So be it. You too shall know the suffering of betrayal." Sludgy ichor oozes out from the character's skin and coalesces, taking the form of a \textbf{cloaker} whose face resembles that of someone the character loves or thinks of fondly. This cloaker has the following changes to its stat block:

\begin{itemize}
\item The cloaker can breathe air and water.
\item It has a swimming speed equal to its flying speed.
\end{itemize}

The cloaker attacks the character, ignoring all other creatures. When either the cloaker or its target drops to 0 hit points, the cloaker disappears in a vermilion haze that fills its space and disperses after 1 minute. The area within the haze is Vision and Light.

\subsection{Vents of Fury Locations}

The Vents of Fury encompass areas N10 through N18 on the Netherdeep map.

\subsubsection{N10: Belching Gallows}

\begin{DndReadAloud}
This cavernous space slightly expands and contracts like the lung of a living, breathing creature. Natural fissures in the floor belch out columns of steam and bubbles, and the water here is warm. Red light also emanates from these vents, illuminating the chamber and the mouths of the tunnels that lead away from it.
\end{DndReadAloud}

Characters who have a passive Wisdom (Perception) score of 17 or higher notice something more:

\begin{DndReadAloud}
At the far end of the thirty-foot-high cavern, a monstrous fish with a red-streaked body chases a school of much smaller fish.
\end{DndReadAloud}

The expansion and contraction of the cavern is a natural phenomenon and can't be dispelled.

In the distance, a \textbf{slithering bloodfin} chases a \textbf{swarm of sorrowfish} (see appendix A (p. 8) for these creatures' stat blocks). A character who isn't carrying a light source can move around in the flooded cavern without attracting the attention of these creatures by succeeding on a DC 14 Dexterity (Stealth) check. If the check fails or the character makes no attempt to hide, the bloodfin notices and attacks the intruder. The sorrowfish are attracted to blood and attack any creature in the cavern that is wounded, including the bloodfin.

\paragraph{Bubbling Water}

The bubbling columns of water rising from the vents are hot, but not hot enough to damage creatures that enter them. If the bloodfin is killed, the vents in the floor flare with bright red light and spew forth a flood of boiling-hot water. Any creature that starts its turn in the boiling-hot water must make a DC 15 Constitution saving throw, taking 21 (6d6) fire damage on a failed saving throw, or half as much damage on a successful one. The scalding heat lasts for 1 hour, after which the vents return to their normal illumination and the water temperature in the cavern again becomes bearably warm.

\paragraph{Secret Door}

Hidden in the north wall is a secret door (see "Secret Doors" earlier in the chapter). On the other side of the secret door is a narrow passage that climbs steeply and ends before another secret door that opens into area N19. Either secret door is easily found from inside the connecting passage.

\subsubsection{N11: Frothing Hollow}

\begin{DndReadAloud}
A current threatens to pull you into this chamber, the walls of which are lined with ruidium spines. The Apotheon's ragged breathing is audible in every heave of water. Thermal vents in the floor give off swirling bubbles, dim light, and warmth.
\end{DndReadAloud}

A \textbf{light devourer} (see appendix A (p. 8)) swims in the dark water 50 feet above the cavern floor. It attacks any character who brings a light source into the chamber.

\paragraph{Current}

If the characters kill the light devourer, the current intensifies for 1 hour. Until the current returns to normal, any creature that starts its turn in this chamber must succeed on a DC 14 Strength saving throw or be thrust against a wall by the current and skewered by ruidium spines. The creature takes 5 (1d10) piercing damage from the spines and must succeed on a DC 20 Charisma saving throw or gain 1 level of \textit{exhaustion}. If the creature is not already suffering from ruidium corruption, it becomes corrupted when it fails the saving throw.

\paragraph{Secret Doors}

Hidden in the south wall are two secret doors (see "Secret Doors" earlier in the chapter). One opens to reveal a narrow, spiraling tunnel that climbs gradually to area N5a. The other leads to a small side chamber (area N11a).

\subsubsection{N11a: Hidden Treasure}

\begin{DndReadAloud}
This small cave has a fifteen-foot-high ceiling and a sandy floor covered with tarnished gold coins. Lying among the coins are two curious items: a ceremonial dagger made of jade and a stoppered vial containing a bubbling purple liquid.
\end{DndReadAloud}

The characters can rest here without being disturbed, assuming they don't mind being underwater.

\paragraph{Treasure}

Characters can recover the following treasure from the sandy floor:

\begin{itemize}
\item 750 gp (minted before the Calamity, these coins are worth double their monetary value to scholars or historians)
\item A jade dagger, its hilt and crossguard carved to resemble two green snakes coiled around one another (250 gp)
\item A potion that combines the beneficial effects of a \textit{potion of healing (superior)} and a \textit{potion of heroism}
\end{itemize}

\subsubsection{N12: Scalding Pit}

\begin{DndReadAloud}
Natural vents spew red light and jets of bubbling water into this warm, oblong chamber. Two enormous jellyfish-like creatures, faintly illuminated by the light coming from the vents, drift in the water at opposite ends of the chamber, their tentacles swaying. Each is within easy reach of one of the tunnels leading to this chamber.
\end{DndReadAloud}

% Image placeholder: Slithering Bloodfin

The creatures floating in the chamber are two \textbf{death embraces} (see appendix A (p. 8)). They are oblivious to what happens around them and attack only if damaged or otherwise disturbed (see "Apotheon's Fury" below). Moving around a death embrace without touching it doesn't disturb it.

\paragraph{Treasure}

In the middle of the chamber, lodged in two vents 10 feet apart, are a \textit{ruidium greatsword} and a \textit{ruidium shield} (both described in appendix B (p. 9)). To extract either item from its location, a character must reach into the boiling water, taking 5 (1d10) fire damage. A \textit{telekinesis} spell or similar magic can be used to dislodge these items safely.

\paragraph{Apotheon's Fury}

If either the \textit{ruidium greatsword} or the \textit{ruidium shield} is removed from its resting place, the vents in the floor flare with bright red light and spew forth a flood of boiling-hot water, drastically raising the temperature of the water in the chamber, but only to a height of 30 feet. (The walls climb to a height of 100 feet; see "Features of the Netherdeep" earlier in the chapter.)

Any creature that starts its turn in the boiling water must make a DC 15 Constitution saving throw, taking 21 (6d6) fire damage on a failed save, or half as much damage on a successful one. The scalding heat lasts for 1 hour, after which the vents return to their normal illumination and the water temperature changes from boiling hot to bearably warm.

The death embraces, if they are still present, also take damage from the boiling water at the start of their turns. Roused by this, they attack any other creatures in the grotto, using their tentacles to grapple as many foes as they can before they float upward, away from the boiling-hot water.

\paragraph{Secret Door}

A secret door in the north wall has a narrow, winding tunnel behind it (see "Secret Doors" earlier in the chapter). The tunnel ascends steeply and ends before another secret door, beyond which is area N21. Either secret door is easily found from inside the connecting tunnel.

\subsubsection{N13: Warriors of Wrath}

% Image placeholder: Ruidium-encrusted statues of warriors jut from the walls of this ominous, sunken cave

\begin{DndReadAloud}
Ruidium covers almost every surface of this thirty-foot-high cavern, in some places forming crystalline spikes that range from one foot to eight feet long. Life-sized, ruidium-encrusted stone statues of warriors jut from the walls, floor, and ceiling at odd angles, like jagged teeth in a diseased maw.
\end{DndReadAloud}

The statues in the cavern depict warriors who served with Alyxian. All are carved to look like they've been gravely wounded, and a few are missing limbs or their heads.

Three \textbf{scuttling serpentmaws} (see appendix A (p. 8)) hide behind the statues and ruidium spikes in the middle of the area. A character who tries to cross the cavern must succeed on a DC 19 Wisdom (Perception) check to avoid being surprised when the serpentmaws attack.

\paragraph{Apotheon's Fury}

When all three serpentmaws are dead, dozens of tendrils of ruidium emerge from the statues in the cavern. The tendrils persist for 1 minute, after which they recede back into the statues.

As long as the tendrils are present, any creature that enters the cavern or starts its turn there must succeed on a DC 15 Dexterity saving throw or be \textit{grappled} by a tendril (escape DC 15). While \textit{grappled} in this way, a creature is also \textit{restrained}. No more than one tendril can grapple a single creature. Each tendril is an Ooze that has AC 15; 10 hit points; immunity to all damage except slashing; Strength, Dexterity, and Constitution scores of 20; and Intelligence, Wisdom, and Charisma scores of 1. A tendril reduced to 0 hit points disintegrates in a crimson haze, like blood diluted in water.

On initiative count 0, each creature \textit{grappled} by a tendril takes 3 (1d6) piercing damage and 3 (1d6) psychic damage as it is flung against one of the cavern's ruidium spikes. The creature remains \textit{grappled} after taking this damage; in addition, it must succeed on a DC 20 Charisma saving throw or gain 1 level of \textit{exhaustion}. If the creature is not already suffering from ruidium corruption, it becomes corrupted when it fails the saving throw.

\subsubsection{N14: Carmine Sepulcher}

\begin{DndReadAloud}
Thermal vents in the floor bring light and heat into this eerie sepulcher, which has a ceiling thirty feet high. Strewn around the vents are the bones of large fish, though it's not clear what killed them.
The ruidium growing on the walls seems to have engulfed dozens of people, trapping them in red crystal sarcophagi as if they were bugs in amber. Every figure bears an expression of anguish.
\end{DndReadAloud}

The figures attached to the walls and entombed in ruidium sarcophagi are illusions. Most of them are images of unfamiliar people from Alyxian's past, but characters who examine the figures closely can see individuals they met while exploring the Grottoes of Regret, such as Alyxian's parents (area N2), Saqiri the drow hunter (area N3), the soldiers Kalagothe (area N4) and Talmyth (area N9), and young \textbf{Meri} (area N5). Any character who tries to chip away at the ruidium sarcophagus to free an illusory figure must succeed on a DC 15 Charisma saving throw or take 11 (2d10) psychic damage. As the attempt is made, the illusory figure trapped inside pleads for release, although its voice is indistinct through its coating of ruidium.

Each ruidium sarcophagus is a Large object that has AC 13, 27 hit points, and immunity to poison and psychic damage. Reducing a sarcophagus to 0 hit points destroys it and causes the illusory figure inside the sarcophagus to vanish.

One of the sarcophagi is empty---a hint that it is actually a secret door (see "Secret Door" below).

\paragraph{Bones}

The floor contains the bones of three slithering bloodfins. If the secret door to area N14a is opened, the bones quietly knit themselves back together and attack any other creatures in the cavern. These \textbf{skeletal creatures} have glowing red eyes and use the \textbf{slithering bloodfin} stat block (see appendix A (p. 8)), with these changes:

\begin{itemize}
\item The skeletal bloodfins are Undead.
\item They have immunity to poison damage; resistance to piercing and slashing damage; and immunity to the \textit{charmed}, \textit{deafened}, \textit{frightened}, \textit{paralyzed}, and \textit{poisoned} conditions.
\item They require no food, drink, air, or sleep.
\end{itemize}

\paragraph{Secret Door}

In the southwest wall is a secret door made to look like one of the chamber's ruidium sarcophagi (see "Secret Doors" earlier in the chapter). Beyond the secret door is area N14a.

\subsubsection{N14a: Fragment of Rancor}

\begin{DndReadAloud}
A pale glow suffuses this small, dead-end chamber. The ceiling is fifteen feet high, and the floor is covered with fine, white sand. Tucked in an alcove is an altar of glittering crystal carved with small feet. A shimmering mote of light floats above it.
\end{DndReadAloud}

The crystal altar bears inscriptions in Celestial that say this shrine is dedicated to Avandra the Change Bringer. A character who doesn't speak Celestial can discern that the altar is tied to Avandra by succeeding on a DC 15 Intelligence (Religion) check. A follower of Avandra has advantage on this check.

\paragraph{Fragment of Rancor}

The mote of light above the pedestal is the Fragment of Rancor (see "Fragments of Suffering (p. 6)" earlier in the chapter). When a character touches the mote, it disappears and becomes part of the character's soul. When this happens, all the characters receive a fleeting vision (see the Apotheon Lore table earlier in the chapter).

The character who is carrying the Fragment of Rancor gains the following benefit and drawback, and is aware of both:

\begin{DndSidebar}{}
\begin{description}
\item[Benefit.] When you hit a creature with an attack, you can deal either an extra 2d6 psychic damage to the creature or 4d6 psychic damage to each creature within 5 feet of it. After you use this benefit, you must finish a short or long rest before you can use it again.
\item[Drawback.] Whenever you are not \textit{unconscious} and fail a saving throw, you take 2d6 psychic damage.
\end{description}
\end{DndSidebar}

Give the Fragment of Rancor card (see appendix D (p. 11)) to the player whose character is carrying the fragment.

% Image placeholder: In the depths of the Netherdeep, the Apotheon's spear waits to be claimed

\subsubsection{N15: Spear of the Apotheon}

\begin{DndReadAloud}
This pit is devoid of thermal vents. The floor is strewn with humanoid skeletons clad in rusty armor as well as the bones of other creatures. Tendrils of ruidium weave along and through the battlefield like ivy. A shining spear stands upright in the center of the room, its shaft emerging from a pile of bones and rusting weapons.
\end{DndReadAloud}

The skeletons of warriors who died during the Calamity are mixed with the bones of giants, mastodons, and moorbounders. These bones are magical fabrications that remind the Apotheon of the carnage he witnessed as a warrior on the battlefield.

\paragraph{Treasure}

Characters who search the area notice a dark recess in the northeast wall, at the back of which is a human skeleton in rusted chain mail. Tendrils of ruidium coil around the remains. The skeleton has a \textit{ruidium shortsword} (see appendix B (p. 9)) clenched in its bony hand.

\paragraph{Taking the Spear}

The \textit{spear} sticking out of the pile of bones is a replica of Alyxian's spear. It shows no signs of rust, decay, or ruidium corruption, unlike the other weapons found throughout the chamber. It is imbued with the properties of a \textit{weapon of warning}. If the characters have already confronted Alyxian in the Heart of Despair (see chapter 7 (p. 7)), the spear is just another piece of treasure waiting to be claimed, and there are no repercussions for someone who takes it.

If a character pulls the spear from the pile of bones before the characters confront Alyxian in the Heart of Despair, all the characters receive a fleeting vision (see the Apotheon Lore table earlier in the chapter). After the vision concludes, two beings suddenly appear in the chamber:

\begin{DndReadAloud}
A loud growl reverberates throughout the cavern. As the sound fades, a soft, melodic voice says, "You shouldn't have done that." The voice comes from a translucent, pale blue figure of a child who appears to have materialized out of nowhere. Floating next to him is a slender sculpture of a man carved out of ruidium and bearing a solemn expression.
"I am Theo Nathope," says the ghostly boy. He points to the ruidium golem. "This is Alyxian the Hunter. It has come to destroy you, but it won't do anything as long as you're with me. We must go now. Do you want to save the Apotheon and end this nightmare?"
\end{DndReadAloud}

\paragraph{Theo}

Theo Nathope bears a striking resemblance to Alyxian as a young boy. (One or more players might realize that "Theo Nathope" is an anagram of "The Apotheon.") This apparition can't be harmed, banished, or dispelled. If the characters accept his offer to guide them, Theo leads them through area N16, shows them a secret door to area N17, and follows them into that location. There he hopes the characters will learn secrets that will help them redeem Alyxian. See the "Roleplaying Theo" sidebar for more information on how to portray the spectral child.

Theo glides through the water as an apparition should, moving at a speed of 30 feet. He can pass through solid objects and end his turn inside one, and he has truesight out to a range of 60 feet.

\begin{DndSidebar}{Roleplaying Theo}
Everything about Theo hints that there is more to him than what appears. His words are carefully chosen and intelligent. He often pauses before he speaks, and he tends to give vague or cryptic responses to questions. Although he might seem unhelpful, his coyness serves a purpose: he wants the characters to \textit{want} to save Alyxian, not simply to follow instructions.
Theo's intent is to assess the moral fiber of the characters. If he thinks they can save Alyxian by healing his centuries-old trauma, rather than letting him go free without a second thought, Theo tries to lead them to the tools that will help them do so. He tries to bond with characters who are empathetic and gentle, but he also takes a keen interest in discerning and clever characters, and he will ask them questions both personal and philosophical. Only after he bonds with a character does he show his more childlike side, sometimes revealing an innocent sense of wonder, a desire to play games and laugh, or a need for fondness and affection.
\end{DndSidebar}

\paragraph{Alyxian the Hunter}

This ruidium golem resembles Alyxian the Apotheon. It remains as rigid as a statue until the characters attack it or until Theo vanishes from the Netherdeep (which happens in area N18). Either event causes the golem to animate and attack the characters.

Alyxian the Hunter can track the characters unerringly and knows the locations of all the secret doors in the Netherdeep. The golem can't leave the Netherdeep, however, and it is instantly destroyed if the spear from this area is taken into the Heart of Despair.

\textbf{Alyxian the Hunter} uses the \textbf{stone golem} stat block, with these changes:

\begin{itemize}
\item The golem is Medium. (Its Hit Dice and hit points remain the same.)
\item The golem has a swimming speed of 30 feet.
\item As an action, the golem can conjure a magic dart made of ruidium and hurl it at another creature it can see within 120 feet of itself. The dart strikes the target unerringly and vanishes on contact. The target must make a DC 17 Charisma saving throw. On a failed saving throw, the target takes 24 (7d6) psychic damage and gains 1 level of \textit{exhaustion}. In addition, if the target is not already suffering from ruidium corruption, it becomes corrupted when it fails the saving throw. On a successful saving throw, the target takes half as much damage and suffers no other effects.
\end{itemize}

When Alyxian the Hunter drops to 0 hit points, the golem turns into an immobile vermilion haze that can't be harmed or dispelled. If the spear from this area has been taken into the Heart of Despair, the haze disperses as the golem is destroyed. Otherwise, the haze coalesces and solidifies into a new ruidium body 1 minute later, whereupon \textbf{Alyxian the Hunter} regains all its hit points and resumes the hunt.

\subsubsection{N16: Curse of Ruidus}

\begin{DndReadAloud}
Floating twenty feet above the floor in the middle of this chamber, rotating slowly on a slightly tilted vertical axis, is a ruidium sphere pockmarked with crater-like indentations. The moon-like sphere is not quite fifty feet in diameter. Two giant sharks, their mouths and skin encrusted with ruidium, swim around the moon's equator in opposite directions. Thermal vents in the floor heat the grotto and bathe the lower hemisphere of the moon in red light.
\end{DndReadAloud}

Two \textbf{corrupted giant sharks} (see appendix A (p. 8)) orbit the red moon. The sharks don't attack until certain conditions are met (described below) or they are attacked.

If Theo is with the characters, he crinkles his nose and gives the moon a disapproving look, then says, "Ruidus was the one enemy Alyxian could never face, yet the one he hated the most." Theo warns the characters against attacking the sharks and tells them not to "wake the red moon" by touching it, moving too quickly through the grotto, or casting a spell.

\paragraph{Secret Doors}

Hidden in the southwest wall are two secret doors (see "Secret Doors" earlier in the chapter). Opening either door doesn't awaken the red moon.

\paragraph{Sinister Red Moon}

The moon is immune to all damage and can't be affected by magic, nor can it be moved. If a character in the chamber makes an attack, casts a spell, touches the moon, or moves 15 feet or more during a single turn, dozens of large bloodshot eyes open on the face of the moon and gaze in all directions. Three things happen at once when this occurs:

\begin{itemize}
\item The secret door to area N16a opens.
\item If the sharks in this area are still alive, they attack the character who caused the moon's eyes to open.
\item Each character inside the chamber when the eyes open must succeed on a DC 15 Wisdom saving throw or be cursed by the moon's corrupted gaze. Determine the effect of each character's curse by rolling on the Curses of Ruidus table. The effect on a character can be ended by any magic that removes a curse.
\end{itemize}

Once the moon's eyes come open, they remain open, watching creatures that enter and leave the area without harming them.

If the moon disappears because of events that transpire in area N25 and the sharks are still alive, they attack any creatures that enter this chamber.

% Table: Curses of Ruidus
\begin{DndTable}[header={Curses of Ruidus}]{cX}
d6 & Curse \\
1--2 & Agent of Doom. In your mind, a voice that sounds like Alyxian's says, "Those you love must die for your greatness, and believe me, I feel your pain." While within 30 feet of you, your allies have disadvantage on their saving throws until this curse on you ends. \\
3--4 & Harbinger of Misadventure. In your mind, a voice that sounds like Alyxian's says, "Your victories will bring ruin to those you love, as mine have done." While within 30 feet of you, your allies have disadvantage on their attack rolls until this curse on you ends. \\
5--6 & Vortex of Malice. In your mind, a voice that sounds like Alyxian's says, "Embrace the malice that swirls deep within your heart, as it does within mine." You can't take the Help action, nor can you use a spell, magic item, or class feature to restore hit points or grant temporary hit points to another creature, until this curse on you ends. \\
\end{DndTable}

\subsubsection{N16a: Fragment of Abhorrence}

This small chamber is hidden behind a secret door. If the characters explore this area while Theo is with them, he waits for them outside.

\begin{DndReadAloud}
In the middle of this small chamber is a half-buried stone statue of Alyxian tilted back at a slight angle. Bits of the statue have broken off, and tendrils of ruidium rise from the damaged areas. A dull red key with a crescent moon-shaped handle is sticking out from a hole in the statue's chest, and a bright mote of light is cupped in the statue's outstretched hands.
\end{DndReadAloud}

The key is made of rusty iron and is safe to touch. It opens the door in area N23. The characters can remove the key by carefully jiggling it free. In area N23, the key glows brightly enough to shed dim light in a 5-foot radius.

\paragraph{Fragment of Abhorrence}

Cupped in the statue's hands is the Fragment of Abhorrence (see "Fragments of Suffering (p. 6)" earlier in the chapter). When a character touches the mote of light, it disappears and becomes part of the character's soul. When this happens, all the characters receive a fleeting vision (see the Apotheon Lore table earlier in the chapter).

The character who is carrying the Fragment of Abhorrence gains the following benefit and drawback, and is aware of both:

\begin{DndSidebar}{}
\begin{description}
\item[Benefit.] Once on each of your turns when you hit a creature with a weapon attack roll, you can force the target to move up to 10 feet away from you in a direction of your choice. A creature that can't be \textit{frightened} is immune to this effect.
\item[Drawback.] If you start your turn \textit{frightened}, you take 2d6 psychic damage.
\end{description}
\end{DndSidebar}

Give the Fragment of Abhorrence card (see appendix D (p. 11)) to the player whose character is carrying the fragment.

\subsubsection{N17: Heart of Truth}

This cavern is hidden behind two secret doors (see "Secret Doors" earlier in the chapter). If Theo has guided the characters to this location, he draws their attention to the northernmost secret door, then passes through it and waits for them on the other side.

When the characters enter this area, read the following boxed text:

\begin{DndReadAloud}
Thermal vents in the floor heat and illuminate this thirty-foot-high, oblong cavern. The ruidium-veined walls expand and contract in regular pulses like the beating of a heart. A ruidium archway in the far wall leads to a darker chamber.
\end{DndReadAloud}

The expansion and contraction of the cavern is a natural phenomenon and can't be dispelled.

\paragraph{Ruidium Archway}

The ruidium archway that stands between this chamber and the adjoining grotto (area N18) is 10 feet tall and 8 feet wide. The first time a character passes through the archway, all the characters receive a fleeting vision (see the Apotheon Lore table earlier in the chapter).

\paragraph{Theo}

If Theo is with the characters, he shares some thoughts on how they can help Alyxian:

\begin{DndReadAloud}
"Alyxian's suffering has made him callous and selfish," Theo says. "I know he can be reminded of his goodness."
A human-sounding howl echoes throughout the grotto. Darkness engulfs the cavern, swallowing you in an inky embrace, flooding your mind with visions. Of Ank'Harel. Of Wildemount. Of a godlike wrath that ravages the world until only corpses and smoking ruins remain. The visions abate, and you are back where you started. "That is what awaits you in the Heart of Despair," says the ghostly child. "Well, kind of."
\end{DndReadAloud}

If asked what he means by "kind of," Theo solemnly beckons for the characters to follow him as he passes through the ruidium archway into area N18.

% Image placeholder: Two aspects of Alyxian—one a ghostly child and the other a merciless golem—shed light on the Apotheon's pain

\subsubsection{N18: Fragment of Melancholy}

\begin{DndReadAloud}
This pit is devoid of thermal vents, and the water here is cold and dark. A few tendrils of ruidium cling to the walls around the archway. Within the darkness, you see a small, pale white light hovering six feet above the ground in a crevice to the north.
\end{DndReadAloud}

When a character touches the mote of light (see "Fragment of Melancholy" below), three \textbf{chuuls} hidden under the sandy floor in the middle of the grotto rise and attack. Theo shrieks and cowers from the chuuls, which do not attack him. The chuuls pursue prey into area N17 but won't go any farther.

\paragraph{Fragment of Melancholy}

The mote of light in the northern crevice is the Fragment of Melancholy (see "Fragments of Suffering (p. 6)" earlier in the chapter). When a character touches the mote, it disappears and becomes part of the character's soul. When this happens, all the characters receive a fleeting vision (see the Apotheon Lore table earlier in the chapter).

The character who is carrying the Fragment of Melancholy gains the following benefit and drawback, and is aware of both:

\begin{DndSidebar}{}
\begin{description}
\item[Benefit.] When you fail a Wisdom or Charisma saving throw, you can reroll it with advantage, potentially turning a failure into a success. After you use this benefit, you must finish a long rest before you can use it again.
\item[Drawback.] If you use this fragment's benefit and it doesn't turn a failed saving throw into a success, you are \textit{incapacitated} until the end of your next turn, overcome with despair.
\end{description}
\end{DndSidebar}

Give the Fragment of Melancholy card (see appendix D (p. 11)) to the player whose character is carrying the fragment.

\paragraph{Theo}

If Theo is with the characters, he shares the following information with them before they leave the grotto:

\begin{DndReadAloud}
"You have seen and heard the creatures in here. Their rage is Alyxian's rage. Their sorrow is his own. Their yearning to escape is his yearning to escape. And when he is angry, Alyxian curses and scorns the gods. But he always returns to prayer, as if hoping for an answer to why the gods abandoned him.
"When Alyxian's rage is spent, temporarily, the Netherdeep grows dark and quiet. In those moments, I hear gentle laughter echoing through the grottoes and I feel a lightness... dare I say a budding joy. I think in such moments he finds peace.
"A piece of the Apotheon remains uncontaminated by regret, fury, yearning, and despair, submerged beneath all his anger. When you see him in his present form, appeal to this shard of his psyche. Please, you must not simply free him. Do whatever it takes to truly save him."
\end{DndReadAloud}

When the characters are prepared to move on, Theo adds:

\begin{DndReadAloud}
"Please don't judge him too harshly. There is goodness in him still, buried deep under so many centuries of anguish."
\end{DndReadAloud}

The first chance he gets after sharing this information, Theo disappears---at a moment when no one is looking at him---and never returns.

\paragraph{Alyxian the Hunter}

If \textbf{Alyxian the Hunter} is stalking the characters, they find the golem waiting for them at the northern end of the tunnel outside area N17. See area N15 for more information about the golem.
\section{Chasm of Yearning}

\begin{quotation}
\em
Those I saved live on. Their joy, their struggles, their ambitions, filter into my home in the depths like sunlight through the waves. How I long to feel that again---the fluttering of another living soul.

\hfill --- Alyxian the Apotheon
\end{quotation}

This region of the Netherdeep reflects the furtive tunneling of an imprisoned mind yearning for freedom. Yet not all is as it seems. The Chasm of Yearning is a region filled with falsehoods and mirages of things too good to be true. If lit by magic, the walls of this region are a deep, soothing blue, except where interrupted by illusions or ruidium growths.

The Apotheon's longing for freedom and friendship is felt most strongly here, and the creatures that reside in this region of the Netherdeep appear withered, starved, and sometimes translucent.

\subsection{Roleplaying the Apotheon}

Within the Chasm of Yearning, the Apotheon's presence manifests as raw desire: the desire to be seen and understood by others, the desire for rest, and the desire to be among heroes who know the meaning of sacrifice. In this region, Alyxian can manifest forms that represent his loneliness and yearning, appearing as a human made of sand, a will-o'-wisp, or an air bubble (see "Weird Events" below).

\subsection{Weird Events}

Whenever the characters leave one location and move to another one in the Chasm of Yearning, roll on the Weird Events table to determine what occurs. An event can occur more than once.

% Table: Weird Events
\begin{DndTable}[header={Weird Events}]{cX}
d10 & Event \\
1 & The characters suddenly find themselves in another area in the Chasm of Yearning, along with all their equipment, but with no memory of how they got there. You choose the new location, excluding areas N24 and N25. \\
2 & A 1-foot-diameter air bubble drifts toward the party, moving gently while avoiding creatures in its path. If a creature touches it, the bubble pops and releases a crash of thunder that can be heard out to a range of 60 feet. Any creature within 10 feet of the popping bubble takes 5 (1d10) thunder damage. The bubble follows one randomly determined character and disappears when that character leaves the Chasm of Yearning or when you roll on this table again. \\
3 & A \textbf{will-o'-wisp} becomes visible and tries to lure the characters to area N25, then turns \textit{invisible} once it arrives there. If attacked, the will-o'-wisp retaliates. Otherwise, it is indifferent toward the characters and harmless. \\
4--5 & Sand on the floor assumes a vaguely humanlike form in an unoccupied space as close to the characters as possible. This sandy shape follows one randomly determined character. It is harmless and can't be damaged or dispelled, but it loses its form and reverts to ordinary sand when the character it is following leaves the Chasm of Yearning or when you roll on this table again. \\
6--10 & The characters experience a vision from Alyxian's past (see the Apotheon Lore table earlier in the chapter). \\
\end{DndTable}

\subsection{Chasm of Yearning Locations}

The Chasm of Yearning encompasses areas N19 through N25 on the Netherdeep map.

% Image placeholder: The Cavern of Many Hands embodies the Apotheon's yearning to feel the gentle touch of other creatures

\subsubsection{N19: Cavern of Many Hands}

\begin{DndReadAloud}
This narrow cavern is fifty feet long, with a ten-foot-high ceiling and an exit at the far end. The floor and walls are covered in a five-foot-thick, rippling carpet of thin, pale arms ending in hands that have long, delicate fingers. At first glance, these limbs might be taken for seaweed. Whispering voices begin to echo in your mind, saying things like "Join us," "Be with us," and "Never be alone again."
\end{DndReadAloud}

The arms and hands that grow out of the floor and walls reach for passersby, trying to touch them. This effect is a manifestation of the Apotheon's longing to feel the touch of another living creature. These limbs are not aggressive---in fact, they are chillingly gentle as they try to hold the characters' hands, caress their arms, and cradle their heads.

Any character who crosses the cavern notices, with a successful DC 13 Wisdom (Perception) check, that several of the hands are holding items: a box, a key, a flask of red liquid, and a rolled-up scroll. These items are as follows:

\begin{itemize}
\item The locked box is decorated with wedding bells and inscribed with the words "For those whose love never wavered." The box contains a black pearl (500 gp), a violet garnet (100 gp), and a pair of delicate platinum rings (250 gp each).
\item The key is made of tarnished silver and unlocks the box. (Without this key, a character can use \textit{thieves' tools} to try to pick the box's lock, doing so with a successful DC 14 Dexterity check.)
\item The flask contains a \textit{potion of healing (superior)}.
\item The rolled-up scroll is a \textit{spell scroll} of \textit{Otiluke's resilient sphere} tied with a green ribbon.
\end{itemize}

\paragraph{Secret Door}

Hidden in the south wall is a secret door that requires a successful DC 25 Wisdom (Perception) check to locate (see "Secret Doors" earlier in the chapter). This door is hard to spot because of the spindly arms and grasping hands that grow on and around it. Behind the secret door is a narrow, winding tunnel that steeply ascends to area N10.

\subsubsection{N20: Cavern of Many Eyes}

\begin{DndReadAloud}
Weblike strands of bright ruidium cling to the walls of this grotto, spreading out from dozens of large, bloodshot eyes that follow your every move. These eyes are embedded in the walls between ten and twenty feet off the floor.
Five patches of seaweed sprout from the sandy floor. Inside each patch is an oval mirror four feet tall and two feet wide. The seaweed not only holds the mirrors in place but also frames them. You see displayed in each mirror what appears to be an important moment in the life of the Apotheon.
\end{DndReadAloud}

\paragraph{Eyes}

The bloodshot eyes embedded in the walls track creatures as they move through the room. Their stony eyelids snap shut if necessary to protect the eyes against incoming attacks, blinding light, or harmful magic. Each eye is 1 foot in diameter and resembles a bloodshot human eyeball in appearance and texture. The eyes do nothing but watch and stare.

\paragraph{Mirrors}

A \textit{detect magic} spell reveals an aura of divination magic around each mirror. The five mirrors show particular moments in Alyxian's past, each one playing a short, silent scene that repeats every 30 seconds. If the characters haven't yet seen all the visions in the Apotheon Lore table presented earlier in the chapter, you can replace any or all of the scenes in the mirrors with entries from that table. Otherwise, the scenes are as follows:

\begin{description}
\item[Mirror 1:] Born in the Light of Ruidus. A baby boy lifts his arms into the air as he cries. A red light washes over him. Instead of toys lying around him, a spear and a dagger rest near where he lays his head.
\item[Mirror 2:] A Selfless Soul. A young Alyxian, ten years old at most, holds out a piece of bread to a sick, elderly figure, who reaches for it with a warm smile.
\item[Mirror 3:] Beginnings of a Hero. This solemn scene depicts Alyxian as a young man, holding a sword and looking at it with a pained expression. The ground around his feet is strewn with rubble and bodies.
\item[Mirror 4:] Defending the Weak. Alyxian, as an adult, faces a gloomstalker that bears down on him while two children huddle behind him.
\item[Mirror 5:] Beacon of Hope. Alyxian, his demeanor exuding determination, stands atop a raised mound, thrusting his spear in the air as he delivers a rallying cry. The soldiers below respond by raising their weapons in salute.
\end{description}

A character who gazes into a mirror while standing directly in front of it sees themself instead of Alyxian, and that mirror's image now depicts an important or defining moment in the character's past. Ask the player to describe the scene that their character sees reflected in the mirror's surface. If the player does a particularly good job describing a scene from their character's past, award inspiration, p. 4 to the character; no character can receive inspiration in this way more than once. If the player doesn't come up with a scene before deciding that the character looks away, the character takes 22 (4d10) psychic damage and is \textit{incapacitated} for 1 minute.

Each mirror has AC 13, 8 hit points, and immunity to poison and psychic damage. When a mirror is destroyed or cut loose from the seaweed that holds it in place, it ceases to display images and becomes nonmagical.

\paragraph{Leaving the Grotto}

Any character who tries to leave the grotto without first seeing themself in at least one mirror hears a garbled voice in their mind say, "Don't leave yet! Let us see! Show us a moment from your past---one that shaped your heroic journey." If met with belligerence or silence, the voice replies, "Oh, have you no worth, then? You, too, shall be forgotten." If met with confusion or gentleness, the voice replies, "Please, friend, let us adore you. Let us catch a glimpse of the path you had to walk to become the hero we see before us." A character can freely ignore the voice, and once the character leaves this area, the voice no longer speaks to them (even if the character returns to this area later).

\subsubsection{N21: Cavern of Many Peaceful Ends}

When the characters first peer into this area, but before they enter it, describe it as follows:

\begin{DndReadAloud}
Weblike strands of bright ruidium cling to the walls of this otherwise empty underwater grotto.
\end{DndReadAloud}

When one or more characters enter this chamber for the first time, add:

\begin{DndReadAloud}
A dim red light appears above your head and takes the form of a crown with jagged spikes.
\end{DndReadAloud}

A crown that sheds dim light in a 5-foot radius appears above the head of each character who has just entered the chamber for the first time. When a crown appears, the character sporting it must make a DC 19 Wisdom saving throw. On a failed save, the character is \textit{incapacitated}. While \textit{incapacitated} in this way, the character is also \textit{charmed} and does nothing except make their way to area N22 by the most direct route---most likely the north tunnel (unless that path is blocked). A character who can't be \textit{charmed} can still be \textit{incapacitated} by the crown, but isn't compelled to go to area N22.

While \textit{charmed} by the crown, a character experiences a vision of a wonderful future in which the character believes they have achieved their goals, won happiness, and found paradise. In effect, the crown shows the character the sort of happy ending that Alyxian wishes he had experienced. Ask the player to describe their character's idea of perfect happiness. Perhaps the character falls in love with Ank'Harel and becomes a person of influence in the city, or perhaps the character imagines returning to Jigow and raising a family. If a player is having trouble with this request, use one or more of the following questions as prompts:

\begin{itemize}
\item What would make your character happy?
\item Where does your character imagine spending the best years of their life?
\item Whom does your character imagine spending the rest of their life with?
\end{itemize}

When a character bearing a crown enters area N22 or is targeted by any magic that ends a curse, the crown disappears and its effect on that character ends.

\paragraph{Secret Door}

Hidden in the west wall is a secret door (see "Secret Doors" earlier in the chapter), beyond which is a narrow, winding tunnel that steeply ascends to area N12. The secret door outside area N12 is found automatically from inside the tunnel.

\subsubsection{N22: Cavern of Many Graves}

Each character who peers into this area sees the same thing:

\begin{DndReadAloud}
Jutting out of the sandy floor of this grotto are a dozen seaweed-choked gravestones. Most of the stones are engraved with the names of people you know or knew, but one of them has your name on it. Below your name on the stone is a shallow indentation with an unusual yet familiar shape.
\end{DndReadAloud}

A \textit{detect magic} spell reveals an aura of illusion magic around each gravestone, and this magic is the reason why each character sees different inscriptions on the stones. The gravestone that bears a character's name is the same one for each character. It doesn't take an ability check to recognize that the shallow indentation beneath the name perfectly matches the size and shape of the \textit{Jewel of Three Prayers} in its Exalted State. If the jewel is pressed into the indentation, the secret door to area N22a swings open.

\paragraph{Gravedigging}

If the characters start digging up graves, they find a number of faceless corpses buried under the sand in shallow graves equal to the number of characters in the party. These corpses have nothing in their possession. Removing a corpse from its grave causes all the corpses in the area to animate and attack. Each corpse uses the \textbf{revenant} stat block and transforms into a 1-ounce pebble of ruidium when reduced to 0 hit points.

\paragraph{Secret Door}

A secret door is hidden in the southwest wall (see "Secret Doors" earlier in the chapter). Unlike most of the secret doors in the Netherdeep, this one is locked. It can be opened only by pressing the \textit{Jewel of Three Prayers} into the indentation (as described above) or by the use of a \textit{knock} spell or similar magic.

\subsubsection{N22a: Fragment of Intransigence}

\begin{DndReadAloud}
Floating about this grotto are deteriorated bedroom furnishings: a bed frame with no mattress, a chair, a table, a dresser missing its drawers, and a wooden chest, all within easy reach. The chest's lid is emblazoned with the symbol of Corellon: a four-pointed star flanked by a pair of crescents. A soft white glow spills out of the chest's many cracks.
\end{DndReadAloud}

The chamber's furnishings fall apart if handled roughly. The chest is unlocked and contains a tiny mote of light---the Fragment of Intransigence (see "Fragments of Suffering (p. 6)" earlier in the chapter).

\paragraph{Fragment of Intransigence}

When a character touches the mote of light, it disappears and becomes part of the character's soul. When this happens, all the characters receive a fleeting vision (see the Apotheon Lore table earlier in the chapter).

The character who is carrying the Fragment of Intransigence gains the following benefit and drawback, and is aware of both:

\begin{DndSidebar}{}
\begin{description}
\item[Benefit.] Another creature can't force you to move anywhere you don't want to go.
\item[Drawback.] You can't take the Disengage or Dodge action.
\end{description}
\end{DndSidebar}

Give the Fragment of Intransigence card (see appendix D (p. 11)) to the player whose character is carrying the fragment.

\subsubsection{N23: Grotto of Many Futures}

The following description assumes that the characters arrive from area N20 or N22:

\begin{DndReadAloud}
The walls of this grotto have been meticulously carved with images of shops and town houses. The features of the buildings change from time to time, as though an \textit{invisible} force were swapping out one facade for another or replacing one architectural flourish with another. The one feature that seems never to change is a wooden door set into the far wall at floor level.
In the middle of the grotto floats a giant jellyfish-like creature with sixty-foot-long tentacles. Swarms of tiny, ugly fish flit between its translucent tentacles.
\end{DndReadAloud}

% Image placeholder

This fake city street is a manifestation of Alyxian's longing for a home he can love, a fact that a character can intuit with a successful DC 15 Wisdom (Insight) check.

In the center of the cavern is a \textbf{death embrace} that has a \textbf{swarm of sorrowfish} darting around its tentacles (see appendix A (p. 8) for these creatures' stat blocks). These creatures ignore the characters, attacking only in self-defense or when something becomes entangled in the death embrace's tentacles, such as what can happen when the east door is touched (see below).

\paragraph{East Door}

The door in the east wall resembles a sturdy cellar door with a built-in lock. The first time the door is touched, it gives off a flash of crimson light. Any creature within 10 feet of the door when this happens must succeed on a DC 18 Strength saving throw or be pushed 20 feet away from the door---and into the death embrace's tentacles, if that creature is still alive and present.

The key to the door can be found in area N16a. A character can use an action to try to pick the lock with \textit{thieves' tools}, doing so with a successful DC 22 Dexterity check. A \textit{knock} spell or similar magic also opens the door.

After the door is opened, the characters can see that the water doesn't pass beyond its threshold. On the other side of the door is a flight of dry steps leading down to a red-lit cavern (area N24).

\subsubsection{N24: Cavern of Many Waterfalls}

The following boxed text assumes that the characters are coming from area N23:

\begin{DndReadAloud}
This thirty-foot-high cavern is not submerged. The air smells stale, coral-like growths made of ruidium festoon the walls, and the floor is covered with a soft layer of dry sand. Sparkling waterfalls trickle down the walls in a few places, but the water disappears in a glittery froth as it strikes the floor.
At the east end of the cavern, opposite you, is a floor-to-ceiling opening with a waterfall pouring down in front of it. Standing in front of this waterfall is a blue-skinned figure with glowing red eyes and veins of ruidium on her face and arms. Folded behind her back is a pair of feathery white wings.
\end{DndReadAloud}

The blue-skinned angel is a \textbf{deva} named \textbf{Perigee}. She is a servant of the Moon Weaver who fought with Alyxian in his final battle. \textbf{Perigee} was destroyed but later re-formed at the Moon Weaver's side. For centuries thereafter, she searched for Alyxian and eventually found her way to the Netherdeep, where she became mired in despair and corrupted by ruidium. In the Celestial tongue, she accuses the characters of disturbing the Apotheon's dreams and attacks them.

% Image placeholder: {@creature Perigee|CRCotN} the Deva

A character can use an action to try to convince \textbf{Perigee} that the party wants to save Alyxian. Allow the player to roleplay this interaction, then have the character make a DC 16 Charisma (Persuasion) check, with advantage if you think the player roleplayed the interaction well. On a failed check, \textbf{Perigee} isn't swayed. On a successful check, \textbf{Perigee} does nothing on her next turn as she considers the character's words (as long as she's not attacked or threatened in the meantime). At the end of that turn, she coolly remarks that anyone who can't defeat her has little hope of besting Alyxian in the Heart of Despair. On her next turn, she resumes her attack. Each character can attempt the check only once.

\paragraph{Perigee's Defeat}

If \textbf{Perigee} is defeated, she disappears with a sigh. No longer bound to the Netherdeep, the deva returns to the Moon Weaver's side with all traces of ruidium gone. When \textbf{Perigee} disappears, a secret door in the north wall swings open (see "Secret Door" below).

\paragraph{Placating the Apotheon}

If one or more characters tried to convince \textbf{Perigee} of the party's good intentions, their attempt to sway \textbf{Perigee} has a positive effect on the final battle against Alyxian (see "Emotional Healing" in chapter 7 (p. 7)), regardless of the success of the attempt.

\paragraph{Secret Door}

The secret door in the north wall (see "Secret Doors" earlier in the chapter) opens to reveal a short, dry tunnel leading to area N24a. The tunnel is dimly lit by silvery moonlight emanating from that chamber, and characters can hear the sound of the waterfalls coming from that direction.

\paragraph{Waterfalls}

The waterfalls are illusory and can't be dispelled. Nothing that touches a waterfall or passes through it gets any wetter than it already is. The waterfall in the passageway to N25 doesn't block line of sight and is harmless; characters can move through it as easily as they would walk through a curtain.

\subsubsection{N24a: Fragment of Deception}

This area, like area N24, is not submerged.

\begin{DndReadAloud}
The air smells fresh here. Waterfalls tumble down the walls of this moonlit grotto, the water disappearing when it touches the soft soil that covers the floor. The area is filled with jungle foliage, and water droplets roll off the leaves onto fragrant tropical flowers.
A beam of silvery moonlight shines down on a curved alabaster bench facing the west wall. Carved into the west wall is a frieze depicting a young girl with wavy strands of light for limbs, accompanied by a pack of wolves. The girl's sculpted hands protrude from the wall and are held out in a cupped position.
\end{DndReadAloud}

If the characters examine the bench, add:

\begin{DndReadAloud}
An inscription carved into the bench reads, in Common, "Sit here and reflect on the Moon Weaver."
\end{DndReadAloud}

The waterfalls, flora, and moonlight are illusions that can't be dispelled. If a character sits on the alabaster bench and uses a mirror or other reflective surface to deflect the moonlight so that it shines on the frieze of the Moon Weaver, a mote of pale white moonlight appears in the sculpture's cupped hands. This mote of light is the Fragment of Deception (see "Fragments of Suffering (p. 6)" earlier in the chapter).

\paragraph{Fragment of Deception}

When a character touches the mote of light in the Moon Weaver's cupped hands, it disappears and becomes part of the character's soul. When this happens, all the characters receive a fleeting vision (see the Apotheon Lore table earlier in the chapter).

The character who is carrying the Fragment of Deception gains the following benefit and drawback, and is aware of both:

\begin{DndSidebar}{}
\begin{description}
\item[Benefit.] When you take damage, you can use your reaction to turn \textit{invisible} and teleport up to 60 feet to an unoccupied space you can see. You remain \textit{invisible} until the start of your next turn or until you make a damage roll or cast a spell.
\item[Drawback.] You have disadvantage on Wisdom checks.
\end{description}
\end{DndSidebar}

Give the Fragment of Deception card (see appendix D (p. 11)) to the player whose character is carrying the fragment.

\subsubsection{N25: Fragment of Loathing}

Characters who pass through the waterfall at the east end of area N24 find themselves submerged in water once again:

\begin{DndReadAloud}
This enormous, submerged grotto has tall clumps of white sea grass sprouting from the sandy floor. Looming above you is a red moon pockmarked by craters. You sense great fury emanating from it.
Spaced around the perimeter of the moon, resting among the sea grass, are three stone altars carved with the symbols of Avandra the Change Bringer, Corellon the Arch Heart, and Sehanine the Moon Weaver. Each altar has an inscription carved into its top and a ruidium spear leaning against it.
\end{DndReadAloud}

This area represents the culmination of Alyxian's dreams: a world in which Ruidus can be destroyed. The moon, which hovers 20 feet above the floor, is called the Effigy of Ruidus. It is an immobile, Gargantuan object that has AC 17, 200 hit points, and immunity to poison and psychic damage. Reducing the Effigy of Ruidus to 0 hit points causes it to explode into ruidium dust and causes the moon in area N16 to crumble to dust as well.

When a creature moves within 20 feet of the Effigy of Ruidus, or when the effigy takes damage, read:

\begin{DndReadAloud}
Colors begin to swirl around the moon's surface, forming three layers of shimmering energy: an inner layer of bright vermilion, a middle layer of rich maroon, and an outer layer of crackling purple.
\end{DndReadAloud}

When the shield appears around the Effigy of Ruidus, have all creatures in the area roll initiative. On initiative count 20 and again on initiative count 10, the effigy makes two Mind Warp attacks (see below), targeting creatures in the area at random.

\begin{DndSidebar}{}
\paragraph{Mind Warp}

ms,rs +9 to hit, reach 5 ft. or range 120 ft., one creature. Hit: 14 (4d6) psychic damage.
\end{DndSidebar}

\paragraph{Shield Layers}

Each shield layer around the Effigy of Ruidus has its own property:

\begin{description}
\item[Purple Layer.] This layer crackles with magical energy that deals 14 (4d6) lightning damage to any creature that starts its turn within 10 feet of the Effigy of Ruidus.
\item[Maroon Layer.] This layer releases a pulse of necrotic energy every time the Effigy of Ruidus takes damage. The pulse deals 7 (2d6) necrotic damage to creatures in the chamber.
\item[Vermilion Layer.] This layer increases the moon's AC from 17 to 22.
\end{description}

\paragraph{Spears and Altars}

The ruidium spears that lean against the altars can be used as nonmagical weapons. Their primary purpose is to destroy the shield layers that protect the Effigy of Ruidus.

A character who wears gloves or gauntlets can handle any of the ruidium spears safely. Without such protection, a character gains 1 level of \textit{exhaustion} at the start of each of their turns for each ruidium spear they are holding; in addition, a character not already suffering from ruidium corruption becomes corrupted when it gains 1 or more levels of \textit{exhaustion} in this way.

To destroy a shield layer, a character must use an action to imagine the layer gone while grasping the spear, use the spear to damage the effigy, or hold the spear while casting a spell that deals damage to the effigy. The inscriptions on the altars are written in Celestial. The spear that leans against an altar is the one referred to in that altar's inscription:

\begin{description}
\item[Altar of Avandra (Northwest of the Effigy).] The inscription on this altar reads, "Grasp this spear and banish the purple shield."
\item[Altar of Corellon (Southwest of the Effigy).] The inscription on this altar reads, "Grasp this spear and use it to dispel the maroon shield."
\item[Altar of Sehanine (East of the Effigy).] The inscription on this altar reads, "Grasp this spear and imagine the red moon without its vermilion shield."
\end{description}

\paragraph{Fragment of Loathing}

When the effigy of Ruidus is destroyed, the Fragment of Loathing (see "Fragments of Suffering (p. 6)" earlier in the chapter), in the form of a tiny mote of white light, appears in its place and floats down until it's a few feet above the floor. When a character touches the mote of light, it disappears and becomes part of the character's soul. When this happens, all the characters receive a fleeting vision (see the Apotheon Lore table earlier in the chapter).

The character who is carrying the Fragment of Loathing gains the following benefit and drawback, and is aware of both:

\begin{DndSidebar}{}
\begin{description}
\item[Benefit.] When a creature damages you with a weapon attack or a spell, you can focus your hatred on that creature. Until the end of your next turn, you have advantage on attack rolls you make against the creature. You can focus your hatred on only one creature at a time.
\item[Drawback.] You have disadvantage on Charisma checks.
\end{description}
\end{DndSidebar}

Give the Fragment of Loathing card (see appendix D (p. 11)) to the player whose character is carrying the fragment.

\subsubsection{N26: Sanctuary of Despair}

\begin{DndReadAloud}
This immense cavern contains a spherical black mass resembling a giant organ or cyst that nearly fills the space. Measuring over one hundred feet in diameter, it has veins of ruidium crisscrossing its surface, which is covered with spiky protuberances that retract and elongate as the sphere pulsates.
\end{DndReadAloud}

This cavern is an appropriate location for the characters to encounter their rivals (see "Rivals in the Netherdeep" earlier in the chapter). There are three ways this scene can play out:

\begin{description}
\item[Friendly Rivals.] If the rivals have become the characters' friends or allies, they agree to help the characters in the final battle against Alyxian, assuming they've all acquired enough Fragments of Suffering to enter the Heart of Despair together. The characters might have to choose which of the rivals to take into this final confrontation and which ones to leave behind, based on who has a Fragment of Suffering and who doesn't. Friendly rivals are willing to transfer one or more of their Fragments of Suffering to characters who don't have one.
\item[Indifferent Rivals.] The rivals and the characters are both trying to reach the heart first. The two groups have deep-seated animosity and bonds of love or camaraderie in equal measure. The rivals might be willing to transfer Fragments of Suffering they've acquired to characters whom they view as better equipped to deal with Alyxian. Dermot and Maggie are the most helpful, while Irvan and Galsariad are the least helpful. Ultimately, Ayo makes the decisions.
\item[Hostile Rivals.] Ruidium corruption has made the rivals callous and selfish. They still need one or more Fragments of Suffering to enter the heart as a group, and they demand that the characters surrender any fragments they've obtained. If the characters refuse, the rivals attack them. If one or more characters enter the Heart of Despair during this confrontation, as many rivals pursue them as are able to do so.
\end{description}

\paragraph{Entering the Heart}

The surface of the Heart of Despair feels like mud to any creature carrying one or more of Fragments of Suffering. Such a creature can push its way through the outer surface and enter the heart's air-filled interior. To other creatures, the Heart of Despair feels like a crusty and impenetrable solid object.

If a character brings the spear from area N15 into the Heart of Despair, the golem-like creature called \textbf{Alyxian the Hunter} is destroyed instantly and permanently. If the characters have avoided Alyxian the Hunter so far, have it catch up to the party in this area, so that one or more characters can witness its destruction when the spear is brought into the Heart of Despair.

When one or more characters enter the Heart of Despair, read:

\begin{DndReadAloud}
The Fragments of Suffering shield you from harm as you are drawn into the fleshy, undulating Heart of Despair. The fragments then burn within you for a moment before they are torn from your soul.
\end{DndReadAloud}

The Fragments of Suffering disappear from the characters as they enter the Heart of Despair, which they can't leave until they confront the Apotheon (as described in the next chapter (p. 7)).

\chapter{The Heart of Despair}\label{ch:the-heart-of-despair-8-8}

The Heart of Despair is a prison of the Apotheon's own making. Nothing traps him in the Netherdeep except his belief that he doesn't deserve to leave. He has called the characters to his location so that they can deem him worthy of returning to the world. By the time they're ready to enter the Heart of Despair, the characters should have learned enough about the Apotheon to pass judgment on his future. Their options are as follows:
% Image placeholder: The threshold of the Betrayers' Rise is guarded by soldiers of the Aurora Watch to keep intruders out and demons in
% Image placeholder: The spectacular desert metropolis of Ank'Harel is the most precious jewel on the continent of Marquet
\section{Running This Chapter}

Read this chapter thoroughly before running it, paying particular attention to the stat blocks for Alyxian's forms and the "Lair Actions" section, which describes actions the Apotheon can take in any of his forms.
\section{Within the Heart}

Characters who enter the Heart of Despair find themselves standing in the middle of the area, just south of a statue of Alyxian (area H1). Describe the environment as follows:

\begin{DndReadAloud}
You pass through clammy darkness and tumble onto the glassy surface of a lake, which feels as solid under your feet as any floor. Rising out of the water are three rocky isles, upon which rest crumbling structures that look like ancient prayer sites.
Waterfalls tumble from the clear sky, giving off mist that blankets the area in shimmering silver. Directly in front of you, a life-sized statue of Alyxian armed with a spear and a shield stands on a white stone pedestal that rises inches above the water's surface. The statue's expression is serene, much like the water.
\end{DndReadAloud}

The interior of the Heart of Despair is a cylindrical space 200 feet high and over 200 feet in diameter. The water here is 20 feet deep.

Every visitor to the Heart and each of Alyxian's forms gains the benefit of a \textit{water walk} spell upon arrival; this effect lasts for 1 hour. A creature can end the effect on itself at any time (no action required), enabling it to sink below the water's surface. Once the effect ends on a creature, it can't be reactivated.

\subsection{Dealing with the Rivals}

Rivals who follow the characters into the Heart of Despair behave in one of two ways, depending on their attitude toward the characters:

\begin{description}
\item[Friendly or Indifferent Rivals.] The rivals allow the characters to deal with Alyxian in whatever manner they see fit, fighting alongside them if need be.
\item[Hostile Rivals.] The rivals attack the characters.
\end{description}

\subsection{Locations in the Heart}

The following locations are keyed to the Heart of Despair map.



\subsubsection{H1: Statue of Alyxian}

This painted stone statue depicts Alyxian as a young man. See "Alyxian Speaks" below for more information about the statue.

\subsubsection{H2--H4: Prayer Sites}

The ruins on the three islands are prayer sites dedicated to Avandra the Change Bringer (area H2), Corellon the Arch Heart (area H3), and Sehanine the Moon Weaver (area H4). The Apotheon draws on the power of these ruins when he takes his lair actions, as described in the "Lair Actions" section.

\subsection{Alyxian Speaks}

Before the characters begin exploring the Heart of Despair, read:

\begin{DndReadAloud}
The Apotheon's voice emanates from the statue and says, "After so long, we can finally speak to one another, and I am overjoyed. You have seen what I have done. What I have become. You, better than anyone, have the clarity of sight to judge me. May I leave?"
\end{DndReadAloud}

The Apotheon's inviting words should give the characters an uneasy, hesitant feeling. Alyxian believes they are the only beings alive who understand him well enough to judge him worthy of returning to the world. Whatever their decision, it must be unanimous (see "Deciding Alyxian's Fate" below).

The statue contains Alyxian's spirit. It is an object with AC 18, 50 hit points, and immunity to poison and psychic damage. If the characters destroy the statue, jump ahead to the "Battle for the Soul of the Apotheon" section.

\subsubsection{Deciding Alyxian's Fate}

If the characters grant the Apotheon his freedom, he gasps in surprise. This is the outcome he wants, but his own self-loathing did not prepare him for this answer, since deep down he knows he's in no condition to leave. He asks the characters if they are sure of their decision, whereupon any character who succeeds on a DC 16 Wisdom (Insight) check can sense the doubt that caused the Apotheon to question their decision. If the characters choose to free him at this point, proceed with the "Worst Ending: Unleashing Devastation" section.

Alyxian is willing to chat amicably with the characters and answer any questions they have---up to a point. Freedom is within his sight after eons of suffering. He is impatient, restless, and determined to end this chapter of his existence. He keeps the conversation going as long as he believes he's convincing the characters that he's a good person. If the characters don't come to a decision or start to question his assertions, frustration starts to creep into his tone, then anger.

If the characters ask about certain topics, you can read or paraphrase Alyxian's replies. Whenever he gives an answer, a character can make a DC 21 Wisdom (Insight) check. On a failed check, the character views him as sincere, sorrowful, and grateful. On a successful check, the character can tell he's deeply conflicted. On a check that succeeds by 5 or more, the character can also tell that he is growing impatient for the characters to judge him worthy. The following are the topics and Alyxian's replies:

\begin{description}
\item[Alyxian's Personality.] "I have been cursed by fate. By some cruel joke of the cosmos. If I am to suffer, was it not right that I should allow my suffering to spare others from the same fate?"
\item[Alyxian's Past.] "I was born as the Calamity raged around me. I grew up knowing fear, and I watched the Betrayer Gods turn my home into a barren wasteland. My only regrets are that I could not save others... and that my sacrifices were not enough to free me from my fate."
\item[Alyxian's Future.] "I have dreamed of my future for so long, but now that you ask, I find my dreams have abandoned me. I want to walk by myself along the streets of a city where no one has cause to fear or loathe me."
\item[Alyxian's Regret.] "You have felt my remorse and witnessed all the memories that I have prayed to be rid of. I am grateful for your actions in the retelling of those memories. I have watched them so many times, I wept to see that they could have turned out happily."
\item[Alyxian's Fury.] "I am filled with hatred. For the god who condemned me to this hell, and for the people who tear at my essence to satisfy their own greed. Does that scare you?" (The god Alyxian refers to is Gruumsh. The people Alyxian refers to are the factions who want to take ruidium from the Netherdeep and use it for their own ends.)
\item[Alyxian's Yearning.] "I have been imprisoned for so long, I dreamed of many things. Those dreams gave me hope, but---but now you are here! They needn't be fantasies any longer."
\item[The Heart of Despair.] "It defies belief that this beautiful place is the source of so much suffering. Let us not speak of this place any longer. I simply wish to be free of it."
\item[Ruidium.] "Do not speak to me of that wretched stuff. I would sooner slay anyone who takes it from my domain than see it used again. Let us be rid of this evil place and deal with ruidium later."
\end{description}

\subsubsection{Upsetting the Apotheon}

If a character points out that Alyxian seems angry during their conversation, he hastily retorts, "I am impatient. The world above sings to me, and you withhold judgment from me. What is your decision? Am I free to go? Choose!"

If any option other than unconditional release from his prison is suggested, if the characters take too long to decide, or if the characters doubt him openly, the Apotheon turns hostile. Read:

\begin{DndReadAloud}
The Heart of Despair rumbles. The water churns as if in anger, and the sounds of breaking glass and cracking stone pierce the space as the Apotheon speaks. "I have prayed for freedom. I have begged for release. I have screamed to an uncaring world and the impassive heavens for salvation. Yet even here, with you before me, I am denied! No more waiting. I will claim everything I have waited these long centuries for, and you will be the ones the world forgot."
\end{DndReadAloud}

Proceed with the next section.
\section{Battle for the Soul of the Apotheon}

Angering the Apotheon or denying him his release triggers a three-stage battle that begins as follows:

\begin{DndReadAloud}
The statue of Alyxian melts into a mass of pulsing black ichor. It fills your nose with a putrid scent, like a combination of iron and rotted meat. Thunder booms as crimson storm clouds blot out the sky. Bright red light flares from the ruined prayer sites, which are crawling with tendrils of ruidium that glow with malevolent power. Silence hangs over the lake for a moment.
Then the silence breaks. A milky-white monster with dozens of arms, too many blinking eyes, and a thick hide festooned with blades bursts out of the inky ooze. It rears back, then crashes down onto the lake as it howls in vengeance.
\end{DndReadAloud}

The battle with the Apotheon begins.

\subsection{How to Run the Encounter}

The following sections describe how the battle against the Apotheon is expected to play out.

\subsubsection{Multiple Stages}

The Apotheon has multiple forms that the characters encounter in sequence, each with its own stat block. Three of these forms---\textbf{Alyxian the Tormented}, \textbf{Alyxian the Callous}, and \textbf{Alyxian the Dispossessed}---are hostile. The Apotheon has a fourth form---\textbf{Alyxian the Absolved}---that poses no threat to the characters and appears only if the Apotheon is redeemed. The sequence is as follows:

Stage 1. The battle begins with the appearance of \textbf{Alyxian the Tormented}. When this form is reduced to 0 hit points, it disappears and is replaced by \textbf{Alyxian the Callous}.

Stage 2. When \textbf{Alyxian the Callous} is reduced to 0 hit points, this form of the Apotheon disappears and is replaced by \textbf{Alyxian the Dispossessed}.

Stage 3. When \textbf{Alyxian the Dispossessed} is reduced to 0 hit points, the Apotheon is well and truly dead. If \textbf{Alyxian the Dispossessed} is redeemed instead of killed, he transforms into \textbf{Alyxian the Absolved}. Either way, the battle ends.

\subsubsection{Emotional Healing}

While in the form of \textbf{Alyxian the Tormented}, \textbf{Alyxian the Callous}, or \textbf{Alyxian the Dispossessed}, the Apotheon taunts the characters, but an underlying sadness and regret hint that something humane still lives within him---a fragment of his original psyche that is desperate to curb his destructive tendencies yet unable to break free of the torment that drives his actions. By reaching out with words of empathy and compassion, the characters can help Alyxian achieve a kind of catharsis, leading to his redemption.

Have Alyxian speak to the characters often, to convey that this is not just a drawn-out combat encounter but an opportunity to rid Alyxian of his corruption. During the battle, the characters can improve Alyxian's emotional state by speaking to him with supportive words and succeeding on one or more Charisma (Persuasion) checks. With each successful check, they weaken his current form and push him toward assuming a less monstrous form. See the various stages of the battle, described below, for how to resolve these checks.

\paragraph{Placating the Apotheon}

Count the number of times the characters' actions in chapter 6 (p. 6) placated the Apotheon. Then share this number with the players. This number indicates how many times the characters can gain advantage on their Charisma (Persuasion) checks to improve Alyxian's emotional state.

% Image placeholder: In the middle of the Heart of Despair stands a statue of Alyxian the Apotheon in his most heroic form

\subsection{Lair Actions}

Rising from the lake are fallen temples to Sehanine the Moon Weaver, Avandra the Change Bringer, and Corellon the Arch Heart, imagined into being by the despairing and lonely Apotheon. These structures stand atop 20-foot-tall rocky isles. Regardless of his current form, Alyxian can draw on the power of these profane ruins to take lair actions. Each lair action is associated with a particular site; when a lair action is used, the corresponding ruin glows with bright crimson light that lingers until initiative count 20 of the next round.

On initiative count 20 (losing initiative ties), Alyxian can take one of the following lair actions; he can't take the same lair action two rounds in a row:

\begin{description}
\item[Avandra's Grasp.] Alyxian uses the power of Avandra's ruined temple (area H2) to create a watery tendril that rises out of the lake. One creature Alyxian can see that's not submerged in the lake must succeed on a DC 15 Dexterity saving throw or be \textit{restrained} by the tendril. The tendril is a Large object that has AC 10, 15 hit points, and immunity to poison and psychic damage. The tendril disappears when it is reduced to 0 hit points or the next time Alyxian takes a lair action.
\item[Corellon's Charm.] Alyxian causes a magical wisp of light to emerge from Corellon's ruined temple (area H3). The wisp enters the body of one creature Alyxian can see that isn't inside Corellon's ruined temple or atop the rocky island that supports it. The creature must succeed on a DC 15 Wisdom saving throw or be \textit{charmed} by Alyxian until initiative count 20 of the next round.
\item[Sehanine's Torment.] Alyxian uses the power of Sehanine's ruined temple (area H4) to create a watery form in the space of one creature that is either standing on the lake or submerged in it. The watery form assumes the vague likeness of the creature whose space it occupies and makes one melee weapon attack (+7 to hit) against it. On a hit, the attack deals 8 (1d8 + 4) bludgeoning damage. Whether it hits or misses, the watery form disappears after making its attack.
\end{description}

\paragraph{Prayers to the Gods}

A character can use an action to pray to Avandra, Corellon, or Sehanine in the deity's prayer site. The character must succeed on a DC 15 Wisdom (Religion) check, and the check has advantage if the character is a worshiper of that god. On a successful check, the profane ruins atop the island melt away, and Alyxian can no longer use the lair action associated with that prayer site. In addition, Alyxian takes 14 (4d6) psychic damage, regardless of his current form.

% Image placeholder: Alyxian's memories of his gods' prayer sites have been defiled by ruidium, granting him otherworldly powers

\subsection{Stage 1: Alyxian the Tormented}

Alyxian's first form is gigantic and monstrous, for it represents the most corrupted version of the Apotheon. All the pain that torments his soul is made manifest upon his flesh to such an extent that it eclipses the gentle soul within. In this form, he uses the accompanying \textbf{Alyxian the Tormented} stat block.

% Image placeholder: {@creature Alyxian the Tormented|CRCotN}

\subsubsection{Emotional State}

Devoid of virtually every attribute that once made him a hero, \textbf{Alyxian the Tormented} is nihilistic, despondent, and wrathful. He attaches the worst possible interpretation to what the characters say, believes them to be his enemies, rebuffs their attempts to help, and mocks them at every turn.

The following bits of dialogue are examples of what he might say in this emotional state:

\begin{itemize}
\item "You can't understand. No one can understand the pain of spending years beyond counting in darkness."
\item "No one remembers. No one cares. Even you!"
\item "Even you will be forgotten. Do you trust them? Your friends? Do not trust. Do not grow close."
\end{itemize}

\subsubsection{Comforting Alyxian the Tormented}

As an action, a character can speak to \textbf{Alyxian the Tormented} with empathy and compassion, after which the character can make a DC 15 Charisma (Persuasion) check. On a failed check, the character's words do not resonate with Alyxian. On a successful check, \textbf{Alyxian the Tormented} loses 20 hit points as he tries to shed his monstrous form, causing chunks of his body to slough off. As gruesome as this sight might seem, Alyxian's emotional state appears to improve as a result. The following bits of dialogue are examples of what he might say in response to the characters' helpful words:

\begin{itemize}
\item "How can I forgive them for this suffering?"
\item "Of course, you can say such things! You have had family, friends, allies, and each other."
\item "This monstrous form... you must destroy it. I can feel its hatred tearing at me. Please! I can feel my spirit inside straining to escape!"
\end{itemize}

\subsection{Stage 2: Alyxian the Callous}

When \textbf{Alyxian the Tormented} is reduced to 0 hit points, read or paraphrase the following text as his second form sprouts from the husk of his first one:

\begin{DndReadAloud}
The corpse splits open like a cocoon, and a radiant figure with angelic wings and wreathed in a halo of spears emerges from the husk. With a gleaming spear in his hand, he looks down on you as if in judgment. In a cold voice he says, "Your words mean nothing. I am the Apotheon, born of the blessing of three gods. Thousands died by my hand. You will be no different."
\end{DndReadAloud}

To spare himself the pain that comes with feeling, Alyxian has hidden his emotions beneath the expressionless mask of a killer. In this form, he uses the \textbf{Alyxian the Callous} stat block.

% Image placeholder: {@creature Alyxian the Callous|CRCotN}

\subsubsection{Emotional State}

\textbf{Alyxian the Callous} is cold and distant, disregarding words he doesn't care to hear. He sees himself as a perfect, "holy" being and believes that the world should be condemned for its crimes against him.

The following bits of dialogue are examples of what he might say in this emotional state:

\begin{itemize}
\item "What crime have I committed other than craving freedom? Why do you not condemn those who left me to rot?"
\item "It is as I thought: you are imperfect judges. You have not known my suffering. No one can know it; I am my only judge."
\item "I have lived a thousand lifetimes and felt a million sorrows. I am the judge this world deserves---and mine will be the final decree."
\end{itemize}

\subsubsection{Comforting Alyxian the Callous}

As an action, a character can speak to \textbf{Alyxian the Callous} with empathy and compassion, after which the character can make a DC 17 Charisma (Persuasion) check. On a failed check, the character's words do not resonate with Alyxian. On a successful check, \textbf{Alyxian the Callous} loses 25 hit points as his current form withers. As painful as this effect might seem, Alyxian's emotional state appears to improve as a result.

The following bits of dialogue are examples of what he might say in response to the characters' helpful words:

\begin{itemize}
\item "You do have a way with words, puny one."
\item "If the world will not pay for the hurt it caused me, then what? What happens?"
\item "How can this be righteousness? How is any of this allowed by fate, by gods that once fought by my side?"
\item "If there is no justice in this world, then the least you can do is grant me an honorable defeat. Come forth! Let us waste no more time on words."
\end{itemize}

\subsection{Stage 3: Alyxian the Dispossessed}

When \textbf{Alyxian the Callous} is reduced to 0 hit points, read or paraphrase the following text as his third form erupts from the remains of his second one:

\begin{DndReadAloud}
The Apotheon's angelic form falls, striking the ground in a golden starburst. Its gleaming feathers crumble to ash as Alyxian rises yet again. Before you, hunched and weary, is an old, haggard man. His gray hair falls over his face in loose strands, and he snarls as he readies his spear and his shield.
\end{DndReadAloud}

His heroic countenance shattered, Alyxian's radiant form sloughs off, revealing an old man. In this form, he uses the \textbf{Alyxian the Dispossessed} stat block.

% Image placeholder: {@creature Alyxian the Dispossessed|CRCotN}

\subsubsection{Emotional State}

\textbf{Alyxian the Dispossessed} is tired. The weariness that has built up in him over the centuries has begun to overcome his anger and bitterness. Now he needs to be guided back into the light.

The following bits of dialogue are examples of what he might say in this emotional state:

\begin{itemize}
\item "This world has moved beyond me. There is no place in it for a bitter man such as I."
\item "How could the gods ever forgive me for what I have done? I'm too old for redemption."
\item "Your naivete shows itself. The omens of Ruidus haunt me now as they did many moons ago."
\end{itemize}

\subsubsection{Comforting Alyxian the Dispossessed}

As an action, a character can speak to \textbf{Alyxian the Dispossessed} with empathy and compassion, after which the character can make a DC 19 Charisma (Persuasion) check. If a character can think of another way to remind Alyxian of the good man he could still be, have the character make a DC 19 ability check using whichever ability and skill you think is appropriate. On a failed check, the character's words do not resonate with Alyxian. On a successful check, years of negative emotions seem to drain away from him. After three successful checks, \textbf{Alyxian the Dispossessed} achieves catharsis and transforms into \textbf{Alyxian the Absolved} (see "Alyxian Redeemed" below).

The following bits of dialogue are examples of what he might say as \textbf{Alyxian the Absolved}:

\begin{itemize}
\item "I have survived for this long. Surely it was for some purpose."
\item "I have heard echoes of the most beautiful things from the surface. Laughter. Music. Joy. I want to see them. To know them again."
\item "What is left of the world that I once knew?"
\end{itemize}

\subsection{Ending the Battle}

The climactic battle ends with Alyxian victorious, defeated, or redeemed. These outcomes are discussed below.

\subsubsection{Alyxian Victorious}

If Alyxian defeats the characters, he disappears with a cackle. His victory triggers the worst possible outcome, as described in the "Worst Ending: Unleashing Devastation" section later in the chapter.

When Alyxian disappears, all creatures inside in the Heart of Desire---alive or dead---are transported to unoccupied spaces in area N26 (see chapter 6 (p. 6)), which is now an empty, flooded, spherical chamber.

\subsubsection{Alyxian Defeated}

If \textbf{Alyxian the Dispossessed} dies, any hope of the Apotheon's redemption dies with him. Read:

\begin{DndReadAloud}
Alyxian lets out a long, rattling sigh as ichor trickles from his wounds. He coughs, and a gout of the fluid spurts from his mouth. Ruidium veins creep up his neck, and he smiles ruefully. He drops to his knees, reduced to a shriveled, gaunt figure, and lets out a last whimper that is undeniably human.
"I have been so cold." The light leaves his eyes. "So cold and so alone." With that, he falls into the lake and sinks into the water.
\end{DndReadAloud}

After Alyxian dies, all creatures inside the Heart of Desire---alive or dead---are transported to unoccupied spaces in area N26, which is now an empty, flooded, spherical chamber. Proceed with the "Neutral Ending: Alyxian at Rest" section.

\subsubsection{Alyxian Redeemed}

Upon assuming the form of \textbf{Alyxian the Absolved} (see the accompanying stat block), the Apotheon no longer has any desire to fight the characters. Read:

\begin{DndReadAloud}
The Apotheon looks toward the sky. He takes in a long breath, then releases it, shuddering as tears run down his face. A smile of relief comes over his countenance.
"My soul feels so light," he whispers. He locks eyes with each of you, one by one, and says, "Please. Let us leave this cursed place."
\end{DndReadAloud}

If the characters unanimously grant permission for \textbf{Alyxian the Absolved} to leave his self-made prison, proceed with the "Best Ending: A World That Remembered" section. If they kill \textbf{Alyxian the Absolved}, the story takes the direction described in "Neutral Ending: Alyxian at Rest."

% Image placeholder: Rid of the corruption that had consumed him for ages, {@creature Alyxian the Absolved|CRCotN|Alyxian} walks the streets of Ank'Harel as a free man.
\section{Back into the Light}

The ending of this adventure tries to resolve any major questions that remain by providing explanations for what happened to the Apotheon, what happened to the rivals, and what happened to the factions. Other issues, including the fate of the characters and any nonplayer characters they grew attached to, are left for you and your players to resolve.

\subsection{Possible Endings}

How this story ends hinges on the outcome of the battle against Alyxian.

\subsubsection{Neutral Ending: Alyxian at Rest}

\textit{Alyxian is defeated instead of being redeemed.}

This ending applies if the characters, refusing to trust the Apotheon enough to allow him his freedom, destroy him during the final confrontation and thereby deprive the Netherdeep of its vital essence.

\paragraph{Ruidium's End}

Upon Alyxian's death, all ruidium on Exandria melts away, leaving no trace of the substance anywhere in the world. Creatures suffering from ruidium corruption are instantly rid of it. Magic items that contain ruidium transform as described in appendix B (p. 9).

\paragraph{Escape from the Netherdeep}

Without the power of the Apotheon to sustain it, the Netherdeep begins to collapse. Creatures have 30 minutes to escape from the Netherdeep as it crumbles around them; those who fail to get out are lost. If the characters have a clear path back to area N1 (see chapter 6 (p. 6)), escaping in this amount of time is doable. If the characters pass through one or more areas that contain hostile creatures, these creatures (which might include the rival adventurers) do their best to waylay the characters.

\paragraph{Parting Vision}

Characters who escape from the Netherdeep receive a vision of the Apotheon. The characters see him with his back turned to them as he receives the killing blow from Gruumsh. Then the characters see themselves, one after the other, striking the same blow. The Apotheon's blood seeps into the sand and disappears from sight. Time passes, and sandstorms rage, then clear, revealing that flowers have begun sprouting from the blood-drenched sand. Then buildings begin to emerge, forming the city of Ank'Harel---which is now at peace thanks to their actions.

\subsubsection{Worst Ending: Unleashing Devastation}

\textit{Alyxian is granted freedom without regaining compassion.}

If the characters take pity on the Apotheon and grant him his freedom without healing his emotional wounds, read:

\begin{DndReadAloud}
The statue of Alyxian explodes, spraying debris and black ichor. The prayer sites, too, explode into rubble as the Apotheon siphons off their last vestiges of power. His voice resounds from everywhere as he screams, "Never again will they forget my name!" A titanic mass of oily, slick shadow gushes past you and tears a hole in the edge of the Heart of Despair.
A wave of darkness comes over you. You are left gasping for breath, but that is the least of your worries.
Ruin is coming to Exandria.
\end{DndReadAloud}

After escaping from the Heart of Despair, the Apotheon races through the Netherdeep and erupts in Ank'Harel. Darkness fills the sky above the city, as if it were a moonless, starless night---and then Ruidus's light blazes abruptly, filling the inky sky with crimson ire.

After ruminating over his regret, fury, yearning, and despair for an age, the Apotheon condemns the world that allowed him to be imprisoned after a lifetime of selflessness. His vengeance is all-consuming, his pain unending. For the next year and a day, Ruidus remains a full moon, shining with unnatural brightness. During this time, Alyxian can appear anywhere in the world in the form of an oily black cloud that can reduce cities to rubble in a matter of days.

\paragraph{Escape from the Netherdeep}

Without the power of the Apotheon to sustain it, the Netherdeep begins to collapse (see "Neutral Ending: Alyxian at Rest" above). When the characters emerge from the Netherdeep, they find Ank'Harel ablaze, its people screaming and fleeing for safety. Alyxian's voice, distorted and hateful, booms out, "Soon, all shall remember!"

To continue a campaign in the midst of this devastation, you can devise a way to stop this wrathful avatar from razing all of Exandria, which could be the characters receiving blessings from other Prime Deities to compete against his power; raising an Exandrian army to face him and the occult worshipers who flock to him; or, for a party that is desperate or daring, seeking help from the Betrayer Gods (see "What of the Deities?" below).

\subsubsection{Best Ending: A World That Remembered}

\textit{Alyxian is shown kindness and reminded of love.}

When the Apotheon's corruption has been neutralized, the characters can speak to Alyxian's true self. They can remind him of the great deeds he has done and could still do, as well as express gratitude for his sacrifice on behalf of the world at large. As Alyxian sheds the last of his bitterness, pain, and fury, he becomes a humble man once more. In this form, he is truly ready to return to the world above.

\begin{DndSidebar}{Roleplaying Alyxian the Absolved}
If the characters save the Apotheon's tortured soul, \textbf{Alyxian the Dispossessed} fades away and is replaced by \textbf{Alyxian the Absolved}. Though he is not as he was before the horrors of the Netherdeep, this form of Alyxian has been relieved of his burdens. He has found freedom from years of anguish and regret. \textbf{Alyxian the Absolved} is neither an aberrant monster nor a divine avenger; he is, as he always was, a man seeking serenity.
\textbf{Alyxian the Absolved} can help the characters escape from the Netherdeep as it collapses, and he might even come to the party's aid in future adventures---though he would prefer to live out his remaining mortal days in peace. See "Best Ending: A World That Remembered" for more information.
\end{DndSidebar}

\paragraph{Ruidium's End}

The Apotheon who caused the spread of ruidium is dead, replaced by a version of the Apotheon who is at peace with himself and the world. The destruction of Alyxian's corrupted forms causes all ruidium throughout Exandria to melt away, leaving no trace of the substance anywhere in the world. Creatures suffering from ruidium corruption are instantly rid of it. Magic items that contain ruidium transform as described in appendix B (p. 9).

\paragraph{Escape from the Netherdeep}

The Netherdeep reacts violently when the characters heal the scars on the Apotheon's soul. Creatures have 30 minutes to escape from the Netherdeep as it crumbles around them; those who fail to get out are lost. If the characters have a clear path back to area N1 (see chapter 6 (p. 6)) and can breathe water, escaping in this length of time is a simple task. If the characters pass through one or more areas that contain hostile creatures, these creatures do their best to waylay the characters. \textbf{Alyxian the Absolved} casts \textit{water breathing} and fights by the characters' side in any such conflict. If the rivals are among these hostile creatures, they flee rather than face Alyxian in battle.

\paragraph{Finale in the Light}

No longer hampered by his fury, his memories, and the empty silence of the Netherdeep, Alyxian gapes in wonderment at Ank'Harel. The busy city that stretches out before him is filled with pleasant chatter, birdsong, and sunlight. He flinches, overwhelmed, and feels out of place in this new world. The characters can either help him settle into his new life or leave him to his own devices. Either way, in time, he finds peace and makes new friends.

At your discretion, one of the Prime Deities most closely tied to Alyxian (Avandra, Corellon, or Sehanine) might reward each character with a blessing of your choice (see "Supernatural Gifts, p. 7" in the Dungeon Master's Guide).

\subsection{What of the Factions?}

The nature of the characters' ongoing relationships with the factions of Ank'Harel depends on what becomes of Alyxian, what happens to ruidium, and what the characters do after emerging from the Netherdeep.

Characters who represented their faction vigorously and who provided resources (ruidium or otherwise) for their faction might be asked to assume positions of leadership. If either the story's neutral ending or best ending comes to pass, all ruidium is destroyed---an outcome that delights the Cobalt Soul, vexes the Allegiance of Allsight, and infuriates the Consortium of the Vermilion Dream.

% Image placeholder

\subsection{What of the Rivals?}

The characters' rivals have experienced their share of triumphs, tragedies, and trials. The following sections describe likely ending scenarios for the rivals, depending on their attitude toward the characters.

\subsubsection{Friendly Rivals}

If the rivals are friendly, their actions from here on depend on what happens to Alyxian:

\begin{description}
\item[Alyxian Is Killed.] The rivals help the heroes escape from the Netherdeep. Sometime later, they invite the characters to a tavern in Ank'Harel to share their memories of the Netherdeep, then try to put it all behind them.
\item[Alyxian Is Unleashed.] The rivals become staunch allies of the characters. The rivals agree to go wherever the party sends them to help stem the disaster.
\item[Alyxian Is Redeemed.] The rivals, recognizing and accepting that the characters have prevailed, honor the heroes during a night of celebration together. They might become fast friends of the characters in time.
\end{description}

\subsubsection{Indifferent Rivals}

If the rivals are indifferent, their actions from here on depend on what happens to Alyxian:

\begin{description}
\item[Alyxian Is Killed.] The rivals report what happened to their faction leaders, casting themselves as neither better nor worse than the characters. They take their leave in search of a place to build a new reputation for themselves.
\item[Alyxian Is Unleashed.] The rivals blame the characters for the cataclysm now facing the world but realize that their only hope of survival is to band together against their common enemy.
\item[Alyxian Is Redeemed.] The rivals thank the characters for engaging in the "race" with them, which didn't really seem to have a winner. Ayo remarks, "Next time there's something you want, we'll be there for you---maybe just to be a thorn in your side." Her parting words seem warm enough.
\end{description}

\subsubsection{Hostile Rivals}

If the rivals are hostile, their actions from here on depend on what happens to Alyxian:

\begin{description}
\item[Alyxian Is Killed.] The rivals report to their faction that the characters have destroyed their access to ruidium and killed a legendary hero. They sabotage the characters' relationship with their own faction as much as they can and angle to become viewed as the more accomplished adventurers.
\item[Alyxian Is Unleashed.] The rivals blame the characters for the devastation Alyxian causes. They complicate the characters' dealings with political leaders and nations, taking every opportunity to portray the characters as selfish, untrustworthy meddlers.
\item[Alyxian Is Redeemed.] The rivals, realizing that their further efforts here would be to no avail, set out for distant lands to find a new way to make their mark in the world.
\end{description}

\subsubsection{Heroes Unsung}

If the rivals died in the Netherdeep, the characters are haunted by dreams of them for a few weeks. In these dreams, they might see Maggie's solemn, steady gaze as she speaks, but not be able to hear what she is trying to tell them. They might see Ayo erupting in an outburst of anger, Galsariad fruitlessly trying to turn back time to change his party's fate, Dermot mourning his friends, and Irvan wondering if he ever did something great.

Eventually, the dreams subside, but on their final night, Ayo steps up to each character in turn and offers a salute. "We didn't do too badly, huh?" she says quietly. "Remember us. Maybe with a monument back in Jigow? That would be nice."

\subsection{What of the Deities?}

This adventure has chronicled Alyxian's relationship with three Prime Deities: Avandra, Corellon, and Sehanine. By the end of the adventure, the characters are heroes in their own right---and they have had opportunities to act in the name of the same deities Alyxian once heralded. If Alyxian threatens to bring doom to the world, these gods might send celestial envoys to entreat directly with the characters:

\begin{description}
\item[Avandra the Change Bringer.] Avandra might reach out to paladins, druids, clerics, wizards, the adventurous, and the bold. She encourages her faithful to grasp their fate, embrace change for the better, and rise against the threat posed by Alyxian. If the characters need her aid, her \textbf{solars} and \textbf{planetars} might deliver useful information about places they hope to visit, or Avandra might provide the occasional swing of fate in the characters' favor.
\item[Corellon the Arch Heart.] Corellon might reach out to elves, spellcasters, the artistic, and the clever. Although Corellon offers direct support less often than other deities, their servants or the strange visions Corellon provides often coax adventurers into searching for forgotten secrets and spells, empowering these heroes through the fruits of discovery and inspiration. Arcane tools of this sort can be useful in overcoming any future threat.
\item[Sehanine the Moon Weaver.] Sehanine might reach out to bards, rogues, druids, clerics, and the mischievous. Her childlike form appears to such individuals in dreams. If the characters need her aid, she might send the \textbf{deva} \textbf{Perigee} to offer them support or a place of refuge.
\end{description}

% Image placeholder: Sehanine the Moon Weaver visits a sleeping adventurer as they dream under the light of Exandria's two moons

\paragraph{Betrayer Gods}

The Betrayer Gods might tempt adventurers (either the same group who freed Alyxian or a new band of heroes) into forming an alliance against Alyxian if the Apotheon can't be subdued any other way. The following are boons the Betrayer Gods might offer to a party---for a price:

\textbf{Asmodeus} might provide a legion of devils.

\textbf{Gruumsh and Bane} might provide armies of marauders and warriors.

\textbf{Lolth} might offer to sow dissension among the mortals that have begun to worship Alyxian, fostering chaos and rebellion to stem that tide.

\textbf{Tiamat} might offer to send an ancient chromatic dragon---or five---to aid in an assault or a siege at a critical moment.

\textbf{Torog} might send a horde of demons and purple worms to assail the fortifications of the characters' enemies.

\textbf{Vecna} might offer previously forbidden arcane power, gifting the characters with powerful magic items that will eventually corrupt them.

\textbf{Zehir} might send assassins to their aid, or he might furnish the characters with a poison that weakens anyone imbued with divinity.

Although any of these alliances are potentially beneficial in the short term, receiving any of these benefits will require the characters to partake in actions or rituals that strengthen the Betrayer Gods and their hold over Exandria. Then, once Alyxian is dealt with, their cults' domination of the world can begin.

\subsection{What of Our Heroes?}

When all is said and done, the characters are credited as the ones who brought peace to a tormented soul or blamed for unleashing that soul upon an unsuspecting world.

If the characters achieve some measure of renown, their reputation precedes them if they travel to faraway lands. They are approached by eager folk with questions about what lies below Ank'Harel. "Tell us of Cael Morrow," they'll say. As time passes, the characters might become known more broadly as "the Redeemers," "the Prayer Bearers," "the Hope Bringers," "the Soul Healers," or any one of a hundred other names that bring to mind the mythic heroes who plunged into the Netherdeep and experienced both triumph and tragedy.

\chapter{Creatures}\label{ch:creatures-9-9}

This appendix provides descriptions and stat blocks for several creatures that appear in the adventure. Use them in adventures of your own creation, whether those tales are set in the world of Exandria or elsewhere in the D\&D multiverse.
The creatures are organized alphabetically.
\section{Using a Stat Block}

If you are unfamiliar with the monster stat block format, read the introduction, p. 0 of the Monster Manual before proceeding further. That book explains stat block terminology and gives rules for various monster traits---information that isn't repeated here.

\subsection{Unusual Attacks and Magic}

Some creatures have weapons that deal unusual damage types and spellcasting that functions in atypical ways. Such an exception is a special feature of a stat block and represents how the creature uses the weapon or casts its spells; the exception has no effect on how a weapon or spell functions for another creature.

\subsection{Meeting Magic Item Prerequisites}

If a stat block contains the name of a class in parentheses under the creature's name, the creature is considered a member of that class for the purpose of meeting prerequisites for magic items.
\section{Adventuring Rivals}

The following nonplayer characters, their personalities, and their motivations are detailed in the introduction:

\textbf{Ayo Jabe} is a water genasi and a natural-born leader who was born to orc parents during a fierce storm at sea.

\textbf{Dermot Wurder} is a goblin priest who worships the Luxon, a divine entity of light and rebirth.

\textbf{Galsariad Ardyth} is a drow mage from the Kryn Dynasty who studies the esoteric practice of dunamancy.

\textbf{Irvan Wastewalker} is a young man who has experienced a previous life as a bugbear of the Kryn Dynasty. He was reborn as a nomadic human through the power of the Luxon.

\textbf{Maggie Keeneyes} is an ogre fighter who has a sharp intellect and a mind for tactics that complements her brute strength.

\subsection{Rival Tiers}

Just as the characters advance in level and gain new equipment over the course of this adventure, so do the rivals increase in power. Each rival is represented by three stat blocks.

\subsection{Rivals' Battle Chatter}

The Battle Chatter table gives examples of things the rivals might say in combat, either to each other or to the characters. These lines of dialogue fit their bright-eyed outlook early in the adventure; you can make up new lines as they grow and change.

% Table: Battle Chatter
\begin{DndTable}[header={Battle Chatter}]{lX}
Rival & Sample Quotes \\
Ayo & "Okay team, they're off-balance. Rally!" "Have you had enough?" "Yeah, just give up. You're finished." \\
Dermot & "I'm here to make sure no one dies. Back off!" "The Luxon will protect me." "Ayo, Maggie, I could use a little help here!" \\
Galsariad & "You have much to learn." "And here I thought you possessed some talent." "With this next incantation, I shall pluck any hope of victory from your mind." \\
Irvan & "Heh, that was a good hit, wasn't it?" "Come on, come on! Eyes on me!" "Can I have this dance? No?" \\
Maggie & "That was quite a blunder." "You're lucky I'm feeling merciful." "Fight smarter---and harder." "People don't win fights by fighting fair." \\
\end{DndTable}

\subsection{Tier 1 Rivals}

% Image placeholder

The tier 1 rivals are fledgling adventurers who have gained a modicum of power through their individual exploits before the beginning of this story. They are seeking to write a grand tale of their own.

\begin{itemize}
\item \textbf{Ayo Jabe (Tier 1)}
\item \textbf{Dermot Wurder (Tier 1)}
\item \textbf{Galsariad Ardyth (Tier 1)}
\item \textbf{Irvan Wastewalker (Tier 1)}
\item \textbf{Maggie Keeneyes (Tier 1)}
\end{itemize}

\subsection{Tier 2 Rivals}

The tier 2 rivals are learning that their adventures take them into horrific places that will test their mettle, but they face these challenges with heads held high.

\begin{DndReadAloud}
"I never thought there could be places so different from Jigow in the world. Cites as beautiful as Ank'Harel. Places that make my heart hurt as much as Bazzoxan."
\end{DndReadAloud}

\begin{itemize}
\item \textbf{Ayo Jabe (Tier 2)}
\item \textbf{Dermot Wurder (Tier 2)}
\item \textbf{Galsariad Ardyth (Tier 2)}
\item \textbf{Irvan Wastewalker (Tier 2)}
\item \textbf{Maggie Keeneyes (Tier 2)}
\end{itemize}

\subsection{Tier 3 Rivals}

Their journey has forced the tier 3 rivals to make decisions, sometimes out of grim necessity, that they might have not made earlier. All five suffer from ruidium corruption (see "Ruidium" in the introduction (p. 0)). When the characters encounter them, assume that the rivals have removed all levels of \textit{exhaustion} from themselves through rest or magic, but the other effects of ruidium corruption remain.

% Image placeholder: Tier 3 Rivals (Left to Right): Ayo Jabe, Dermot Wurder, Maggie Keeneyes, Irvan Wastewalker, and Galsariad Ardyth

If instead you want to weaken the rivals to make them less of a physical threat or reinforce the danger of ruidium corruption, you can give one or more of the rivals 1d4 levels of \textit{exhaustion} each. In this case, you'll need to keep track of each rival's \textit{exhaustion} level and the condition's corresponding effects (see "\textit{Exhaustion}" in the Player's Handbook).

\paragraph{Useful Magic}

To facilitate their efforts, Galsariad has learned the \textit{water breathing} spell. Additionally, all the rivals have acquired ruidium equipment to protect them from the effects of underwater pressure.

\paragraph{Irvan's Prosthetic Arm}

During a misadventure that unfolded while the rivals were apart from the characters, Irvan Wastewalker lost an arm and gained an arcane prosthetic. This magic item functions for no one else.

\begin{itemize}
\item \textbf{Ayo Jabe (Tier 3)}
\item \textbf{Dermot Wurder (Tier 3)}
\item \textbf{Galsariad Ardyth (Tier 3)}
\item \textbf{Irvan Wastewalker (Tier 3)}
\item \textbf{Maggie Keeneyes (Tier 3)}
\end{itemize}

\begin{DndReadAloud}
"I'm far from home. far from the Luxon beacons. far from rebirth. there's no room for mistakes. not anymore."
\end{DndReadAloud}
\section{Creatures A-Z}

\begin{itemize}
\item \textbf{Alyxian Aboleth}
\item \textbf{Bristled Moorbounder}
\item \textbf{Corrupted Giant Shark}
\item \textbf{Death Embrace}
\item \textbf{Gloomstalker}
\item \textbf{Horizonback Tortoise}
\item \textbf{Light Devourer}
\item \textbf{Monastic High Curator}
\item \textbf{Monastic Infiltrator}
\item \textbf{Monastic Operative}
\item \textbf{Moorbounder}
\item \textbf{Occult Extollant}
\item \textbf{Occult Initiate}
\item \textbf{Occult Silvertongue}
\item \textbf{Quipper}
\item \textbf{Scholarly Agent}
\item \textbf{Scholarly Excavator}
\item \textbf{Scholarly Mastermind}
\item \textbf{Scuttling Serpentmaw}
\item \textbf{Slithering Bloodfin}
\item \textbf{Sorrowfish}
\item \textbf{Swarm of Sorrowfish}
\end{itemize}

\chapter{Magic Items}\label{ch:magic-items-10-10}

Items imbued with magic are the stuff of song and legend across Exandria. Many of the magic items that the characters can find in this adventure are described in this appendix. Some of these items are medals that the characters can win during the Festival of Merit (see chapter 1), while others are obtainable only by joining a faction (see chapter 4 (p. 4)).
\section{Ruidium Items}

Ruidium is linked to the Apotheon, channeling not just Alyxian's distorted emotions but also the curse of misfortune bestowed upon him by the moon Ruidus. Though it is a source of magical power, ruidium corrupts everything it touches. Even magic items can be corrupted by ruidium, and using such items comes with a risk.

Weapons and armor can be transformed into ruidium items by infusing them with powdered ruidium, which gives these items a rusty-red coloration. Other items can be transformed into ruidium items after prolonged contact with the mineral; such items have ruidium crystals embedded in them or veins of ruidium running through them.
\section{Magic Item Descriptions}

The magic items in this section are presented in alphabetical order. For the rules on magic items, p. 7, see the Dungeon Master's Guide.

\begin{DndSidebar}{Vestiges of Divergence}
When the great wars of the Calamity rolled across Exandria, heroes of divinity accepted legendary relics from their patrons and gods. At the same time, archmages wove dangerous arcane power into items of immense magical might. When the war ended, most of these magic items were buried with their wielders beneath ash and dust. Those items that remained accessible were passed down as heirlooms. These remnants of a lost era are known as the Vestiges of Divergence.
Each Vestige of Divergence is a magic item that grows in power over time. When it is first encountered, the item is usually in its Dormant State. Either on its own or in response to the heroic deeds of its bearer, a Vestige of Divergence can enter its Awakened State. In time, it can reach its ultimate potential and enter its Exalted State, usually in response to a momentous event or some heroic achievement on the part of its bearer. The properties gained at each new state are cumulative with the properties of a previous state, unless otherwise noted.
\end{DndSidebar}

\begin{itemize}
\item \textit{Breathing Bubble}
\item \textit{Jewel of Three Prayers}
\item \textit{Ring of Free Action}
\item \textit{Ring of Red Fury}
\item \textit{Ruidium Armor}
\item \textit{Ruidium Shield}
\item \textit{Ruidium Weapon}
\end{itemize}

\chapter{Medals of Merit}\label{ch:medals-of-merit-11-11}

% Image placeholder: An aboleth mutated by ruidium haunts the dark waters of Cael Morrow
\section{Medals of Merit}
\section{Printable Image}

% Image placeholder

\chapter{Fragments of Suffering}\label{ch:fragments-of-suffering-12-12}

% Image placeholder: Adventurers are transported to the Netherdeep, an underwater domain warped by alien magic and the torment of the Apotheon
\section{Fragments of Suffering}
\section{Printable Image}

% Image placeholder

\chapter{Story Concept Art}\label{ch:story-concept-art-13-13}

The concept art in this appendix was created to inspire the writers and to help the artists envision some of the new characters, locations, and concepts that figure prominently in \textit{Critical Role: Call of the Netherdeep}.
\begin{itemize}
\item Allow the Apotheon to go free without first healing his emotional wounds (and in doing so, cause his corruption to wreak havoc throughout Exandria)
\item Deem the Apotheon beyond redemption and destroy him
\item Heal the Apotheon's emotional wounds and bring him peace before granting him his freedom
\end{itemize}
Left: The Ancient towers of Bazzoxan look like only a god could have crafted them.
Below: Ruidium festers and grows if left unchecked, just like the Apotheon's anguish. Stronger than iron, it grows in gruesome organic patterns.
% Image placeholder: The Apotheon's second form, blessed with an angelic countenance, rises out of the slain corpse of his first monstrous incarnation
A shade of maroon unifies the rivals' outfits. Even though they're just starting out, they look like a team
\subsection{Using the Cards}

The cards on this page are meant to be photocopied and given to players whose characters win Medals of Merit in chapter 1 (see that chapter for contest rules).

The cards in appendix D (p. 11) are meant to be given to players whose characters gain Fragments of Suffering in chapter 6 (p. 6) (see that chapter for more information).

\begin{itemize}
\item \textit{Medal of Muscle}
\item \textit{Medal of the Conch}
\item \textit{Medal of the Horizonback}
\item \textit{Medal of the Maze}
\item \textit{Medal of the Meat Pie}
\item \textit{Medal of the Wetlands}
\item \textit{Medal of Wit}
\end{itemize}
The people of Ank'Harel wear fashionable clothes that reflect the abundance of one of Exandria's greatest cities.
\begin{itemize}
\item Fragment of Abhorrence
\item Fragment of Attachment
\item Fragment of Deception
\item Fragment of Despondence
\item Fragment of Intransigence
\item Fragment of Loathing
\item Fragment of Melancholy
\item Fragment of Pity
\item Fragment of Rancor
\end{itemize}
The armor of the Aurora Watch has an insectile appearance and chirps like a cricket as the wind blows through it.
% Image placeholder
The leaders of the Consortium of the Vermilion Dream operate in the shadows. They don't all feature in this adventure, but they could be villains in your ongoing campaign.
% Image placeholder
In his human form, Alyxian bears tokens of the three gods who blessed him: Avandra's brooch, Corellon's dagger, and Sehanine's shield.
% Image placeholder
Alyxian has two inhuman forms that make him strong but prevent anyone from healing his trauma. The first is a scarred monster with ancient weapons stuck in its hide (with a tiny figure of a human below to show scale).
% Image placeholder
The second is an angelic form incapable of compassion.
% Image placeholder

\chapter{Poster Map}\label{ch:poster-map-14-14}

% Image placeholder
% Image placeholder: A map of the great city of Ank'Harel, the caverns of the last city of Cael Morrow, and the mysterious realm beneath—as explored by the Allegiance of Allsight. 836 PD

\chapter{Credits}\label{ch:credits-15-15}

\begin{description}
\item[\mbox{}] \begin{description}
\item[Project Leads.] James J. Haeck, Matthew Mercer, Christopher Perkins
\item[Art Director.] Kate Irwin
\item[Writers.] James J. Haeck, Makenzie De Armas, LaTia Jacquise, Cassandra Khaw, Sadie Lowry
\item[Additional Writing.] Dan Dillon, Taymoor Rehman
\item[Rules Developers.] Jeremy Crawford, Ben Petrisor
\item[Editors.] Judy Bauer, Kim Mohan, Christopher Perkins, Hannah Rose
\item[Senior Graphic Designer.] Trish Yochum
\item[Graphic Designer.] Matt Cole
\item[Cover Illustrator.] Minttu Hynninen
\item[Cartographers.] Stacey Allan, Will Doyle, Deven Rue
\item[Interior Illustrators.] Eric Belisle, Hunter S. Bonyun, Zoltan Boros, David René Christensen, CoupleOfKooks, Kent Davis, Nikki Dawes, Axel Defois, Max Dunbar, Isabel Gibney, Minttu Hynninen, Julian Kok, Linda Lithén, Jessica Mahon, Andrew Mar, Marcela Medeiros, Robson Michel, Goñi Montes, Jim Nelson, Nguyen Hieu, Jessica Nguyen, Irina Nordsol, Caio E Santos, David Sladek, Crystal Sully, Brian Valeza, Anna Veltkamp, Johannes Voss, Tyler Walpole, Lauren Walsh, Shawn Wood, Zuzanna Wužyk, Kieran Yanner, Anna Zee, Maria Zolotukhina
\item[Concept Art.] Claudio Pozas, Richard Whitters, Shawn Wood
\item[Cultural Consultant.] Basheer Ghouse
\item[Project Engineer.] Cynda Callaway
\item[Imaging Technician.] Daniel Corona
\item[Prepress Specialist.] Jefferson Dunlap
\end{description}
\item[\mbox{}] \subsection{D\&D Studio}

\begin{description}
\item[Executive Producer.] Ray Winninger
\item[Game Architects.] Jeremy Crawford, Christopher Perkins
\item[Design Manager.] Steve Scott
\item[Design Department.] Sydney Adams, Judy Bauer, Makenzie De Armas, Dan Dillon, Amanda Hamon, Ari Levitch, Ben Petrisor, Taymoor Rehman, F. Wesley Schneider, James Wyatt
\item[Art Team Manager.] Richard Whitters
\item[Art Department.] Matt Cole, Trystan Falcone, Bree Heiss, Kate Irwin, Bob Jordan, Emi Tanji, Shawn Wood, Trish Yochum
\item[Senior Producers.] Lisa Ohanian, Dan Tovar
\item[Producers.] Bill Benham, Robert Hawkey, Lea Heleotis
\item[Director of Product Management.] Liz Schuh
\item[Product Managers.] Natalie Egan, Chris Lindsay, Hilary Ross, Chris Tulach
\end{description}
\item[\mbox{}] \subsection{D\&D Marketing}

\begin{description}
\item[Director of Global Brand Marketing.] Brian Perry
\item[Global Brand Manager.] Shelly Mazzanoble
\item[Senior Marketing Communications Manager.] Greg Tito
\item[Community Manager.] Brandy Camel
\end{description}
\item[\mbox{}] \subsection{Critical Role Team}

\begin{description}
\item[Creator of Exandria.] Matthew Mercer
\item[Critical Role Cast.] Laura Bailey, Taliesin Jaffe, Ashley Johnson, Liam O'Brien, Marisha Ray, Sam Riegel, Travis Willingham
\item[Story Ideas and Feedback.] Dani Carr, Basheer Ghouse, Mark Hulmes, Erika Ishii, Aabria Iyengar, Matthew Key, Brennan Lee Mulligan, Eduardo Lopez, Rachel Romero, Ben Van Der Fluit, Brittany Walloch
\end{description}
\end{description}
% Image placeholder: On the Cover: Minttu Hynninen illustrates five adventurers who aim to release the Apotheon, a forgotten hero of Exandria, from his prison. On the back cover, a slithering bloodfin hunts in the sunken abyss known as the Netherdeep.

\end{document}